\documentclass{../../templates/lkx_pset}

\title{Astron 140 Problem Set 10}
\author{Lev Kruglyak}
\due{November 26, 2024}

\usepackage{mdframed}


\usepackage[T1]{fontenc}
\RequirePackage{mlmodern}


\mdfdefinestyle{answer}{%
	linecolor=black,
	outerlinewidth=2pt,
	%roundcorner=20pt,
	innertopmargin=4pt,
	innerbottommargin=4pt,
	innerrightmargin=4pt,
	innerleftmargin=4pt,
	leftmargin = 4pt,
	rightmargin = 4pt
	%backgroundcolor=gray!50!white}
}

\newenvironment{answerbox}{
	\begin{mdframed}[style=answer,nobreak=true,userdefinedwidth=30em]}{\end{mdframed}}

\usepackage{siunitx}
\providecommand{\unitsi}[1]{\qty[per-mode = symbol]{#1}{}}
\providecommand{\mpssi}[1]{\qty[per-mode = symbol]{#1}{\m\per\s}}
\providecommand{\mpsssi}[1]{\qty[per-mode = symbol]{#1}{\m\per\s^2}}
\providecommand{\Gsi}[1]{\qty[per-mode = symbol]{#1}{\m^3\per\s^2\kg}}
\providecommand{\kBsi}[1]{\qty[per-mode = symbol]{#1}{\m^2\s^{-2}\K^{-1}\kg}}
\providecommand{\hsi}[1]{\qty[per-mode = symbol]{#1}{\J\cdot s}}
\providecommand{\kgsi}[1]{\qty[per-mode = symbol]{#1}{\kg}}
\providecommand{\ssi}[1]{\qty[per-mode = symbol]{#1}{\s}}
\providecommand{\psssi}[1]{\qty[per-mode = symbol]{#1}{\s^{-2}}}
\providecommand{\msi}[1]{\qty[per-mode = symbol]{#1}{\m}}
\providecommand{\desi}[1]{\qty[per-mode = symbol]{#1}{\J\per \m^3}}

\renewcommand{\O}{\mathrm{O}}
\providecommand{\Aff}{\mathrm{Aff}}
\providecommand{\SO}{\mathrm{SO}}

\providecommand{\Frame}{\mathrm{Fr}}

\providecommand{\A}{\mathbb{A}}

\providecommand{\definefunction}[5]{
	\begin{array}{rcl}
		#1 : #2 & \xrightarrow{\phantom{---}} & #3 \\
		#4      & \xmapsto{\phantom{---}}     & #5
	\end{array}
}


\providecommand{\pp}[2]{\frac{\partial #1}{\partial #2}}


\begin{document}
\maketitle

\begin{problem}{1}
  Following the lectures,
\end{problem}

\begin{parts}
  \begin{part}{(a)}
    Show how $\overline{h}_{\mu\nu}$, the trace-reverse of the metric pertubation, transforms under the gauge transformation $x^\mu \mapsto x^\mu_{\textrm{new}} = x^\mu + \xi^\mu(x)$.
  \end{part}

  Under the assumption of a weak field limit, we can write the metric of a solution to the Einstein equations as a small perturbation of the flat Minkowski metric:
  \[
    g_{\mu\nu} = \eta_{\mu\nu} + h_{\mu\nu}\quad\textrm{where}\quad |h_{\mu\nu}| \ll 1.
  \]
  Given a gauge transformation $x^{\mu} \mapsto x^\mu_{\textrm{new}} = x^\mu + \xi^\mu(x)$, we get
  \[
    \begin{aligned}
      g_{\mu\nu} \mapsto g_{\mu\nu}^{\textrm{new}} 
      &= \pp{x^\alpha}{x^\mu_{\textrm{new}}}\pp{x^\beta}{x^{\nu}_{\textrm{new}}} g_{\alpha\beta}\\
      &= 
      \left( \pp{x^\alpha_{\textrm{new}}}{x^{\mu}_{\textrm{new}}} - \pp{\xi^\alpha}{x^{\mu}_{\textrm{new}}} \right)
      \left( \pp{x^\beta_{\textrm{new}}}{x^{\nu}_{\textrm{new}}} - \pp{\xi^\beta}{x^{\nu}_{\textrm{new}}} \right) (\eta_{\alpha\beta} + h_{\alpha\beta})\\
      &= \left( \delta^{\alpha}_\mu - \partial_\mu \xi^\alpha\right)
      \left( \delta^\beta_\nu -\partial_\nu \xi^\beta\right)(\eta_{\alpha\beta} + h_{\alpha\beta})\\
      &\approx \eta_{\mu\nu}-\partial_\mu \xi_\nu - \partial_\nu \xi_\mu + h_{\mu\nu}
    \end{aligned}
  \]
  where the last equality only holds up to first order. This implies that $h_{\mu\nu}$ undergoes the gauge transformation
  \[
    h_{\mu\nu} \mapsto h_{\mu\nu}^{\textrm{new}} = h_{\mu\nu} - \partial_\mu \xi_\nu - \partial_\nu \xi_\mu.
  \]
  It is however useful to consider the trace reverse of $h_{\mu\nu}$, namely \[\overline{h}_{\mu\nu} = h_{\mu\nu}- \frac{1}{2}\eta_{\mu\nu} h\quad\textrm{where}\quad h = h^{\alpha}_\alpha = \eta^{\alpha\beta}h_{\beta\alpha}.\]
  This trace reversed perturbation undergoes the gauge transformation
  \[
    \begin{aligned}
    \overline{h}_{\mu\nu} \mapsto 
    \overline{h}^{\textrm{new}}_{\mu\nu} 
    &= h_{\mu\nu}^{\textrm{new}} - \frac{1}{2}\eta_{\mu\nu} h^{\textrm{new}}\\
    &= h_{\mu\nu} - \partial_\mu \xi_\nu - \partial_\nu \xi_\mu - \frac{1}{2}\eta_{\mu\nu}(\eta^{\alpha\beta} (h_{\beta\alpha} - \partial_{\beta}\xi_\alpha - \partial_{\alpha}\xi_\beta))\\
    &= h^{\mu\nu} - \frac{1}{2}\eta_{\mu\nu}h - \partial_\mu \xi_\nu - \partial_\nu \xi_\mu + \frac{1}{2}\eta_{\mu\nu} \eta^{\alpha\beta}\partial_\alpha\xi_\beta\\
    &= \overline{h}^{\mu\nu} - \partial_\mu \xi_\nu - \partial_\nu \xi_\mu + \eta_{\mu\nu}\partial^\alpha\xi_\alpha.
    \end{aligned}
  \]
  Thus, the gauge transformation for the trace reversed perturbation is 
  \[
    \overline{h}_{\mu\nu} \mapsto \overline{h}_{\mu\nu}^{\textrm{new}}  
  = \overline{h}^{\mu\nu} - \partial_\mu \xi_\nu - \partial_\nu \xi_\mu + \eta_{\mu\nu}\partial^\alpha\xi_\alpha.\]
  \begin{part}{(b)}
    Explain why the Lorentz gauge $\partial^\mu \overline{h}_{\mu\nu}=0$ restricts almost all the degrees of freedom in this gauge transformation; but still leaves some residual gauge degrees of freedom.
  \end{part}

  The Lorentz gauge $\partial^\mu \overline{h}_{\mu\nu} = 0$ has the following effect:
  \[
    \partial^\mu \overline{h}_{\mu\nu}^{\textrm{new}} = \partial^\mu \overline{h}_{\mu\nu} - \square \xi_\nu - \partial_\nu\partial^\mu \xi_\mu + \partial_\nu \partial^\alpha \xi_\alpha = 0
  \]
  Since $\partial^\mu \overline{h}^\textrm{new}_{\mu\nu} = 0$ as well, and the last two terms cancel, we get $\square \xi_\nu = 0$. This is a massive restriction on the space of possible perturbations of $\xi_\nu$, and makes it finite dimensional.

  \begin{part}{(c)}
    What are these residual gauge degrees of freedom?
  \end{part}
  The restriction we derived is a wave equation, with solutions $\xi_\mu = B_\mu e^{ik_\alpha x^\alpha}$, where $k^\mu k_\mu = 0$. This gives us $4$ degrees of freedom in $B_\mu$, and $k_\alpha$ shared with the wave solutions to the Einstein equation.

  \begin{part}{(d)}
    What other conditions need to be introduced to fix them, and then why are they fixed?
  \end{part}

  Earlier in the lectures, we derived that the Einstein equation gives us $\square \overline{h}_{\mu\nu} = 0$, another wave equation with solutions $\overline{h}_{\mu\nu}= A_{\mu\nu} e^{ik_\alpha x^\alpha}$ with $k_0^2 = k_i^2$. Combined with e Lorentz gauge, we get $k^\mu A_{\mu\nu} =0$ which leaves 6 degrees of freedom. Note that
  \[
    A_{\mu\nu} \mapsto A^{\textrm{new}}_{\mu\nu} = A_{\mu\nu} - ik_\mu B_\nu - ik_\nu B_\mu + i\eta_{\mu\nu} k^\alpha b_\alpha.
  \]
  Thus, adding the traceless condition $\eta^{\mu\nu} A_{\mu\nu}=0$ removes $1$ degree of freedom leaving us with $5$, and adding a transversality condition $A_{0\mu}=0$ gives us $3$ conditions (one redundant with the Lorenz gauge) so we are left with $2$ degrees of freedom. In particular, all of the $B_{\mu}$ are set to zero since they correspond to non-physical transformations. The remaining degrees of freedom correspond to polarizations of gravitational waves in the periodic solutions $\overline{h}_{\mu\nu}$.
\end{parts}

\end{document}
