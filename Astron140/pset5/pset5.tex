\documentclass{../../templates/lkx_pset}

\title{Astron 140 Problem Set 6}
\author{Lev Kruglyak}
\due{October 15, 2024}

\usepackage{siunitx}
\usepackage{mdframed}

\mdfdefinestyle{answer}{%
	linecolor=black,
	outerlinewidth=2pt,
	%roundcorner=20pt,
	innertopmargin=4pt,
	innerbottommargin=4pt,
	innerrightmargin=4pt,
	innerleftmargin=4pt,
	leftmargin = 4pt,
	rightmargin = 4pt
	%backgroundcolor=gray!50!white}
}

\newenvironment{answerbox}{
	\begin{mdframed}[style=answer,nobreak=true,userdefinedwidth=30em]}{\end{mdframed}}

\providecommand{\mpssi}[1]{\qty[per-mode = symbol]{#1}{\m\per\s}}
\providecommand{\ssi}[1]{\qty[per-mode = symbol]{#1}{\s}}
\providecommand{\msi}[1]{\qty[per-mode = symbol]{#1}{\m}}


\begin{document}
\maketitle

\begin{problem}{1}
Prove the local flatness theorem stated in the lecture.
\end{problem}

\begin{solution}
	Let $M$ be a Lorentzian $n$-manifold with metric $g$, and let $p\in M$ be a point.
	The local flatness theorem states that there is a coordinate chart $x^{\mu}$ at $p$ such that
	\[
		g_{\mu\nu}(x) = \eta_{\mu\nu} + O(x^2)
	\]
	where $\eta_{\mu\nu}$ is the standard Minkowski metric.

	To construct such a chart, let's start with any local coordinates $x^{\mu}$ about $p\in M$ such that $p$ corresponds to the origin. Then, by Taylor expanding the metric we get
	\[
		g_{\mu\nu}(x) = g_{\mu\nu}(0) + \frac{\partial g_{\mu\nu}(0)}{\partial x^{\lambda}} x^{\lambda} + \frac{1}{2}\frac{\partial^2 g_{\mu\nu}(0)}{\partial x^{\lambda} \partial x^{\sigma}} x^{\lambda} x^{\sigma} + O(x^3).
	\]
	Now let's see what happens under a change of coordinates to some other coordinate chart $y^{\mu}$. By the tensor transformation law, we have
	\[
		g'_{\alpha\beta}(y) = \pp{x^{\mu}}{y^\alpha}\pp{x^\nu}{y^\beta} g_{\mu\nu}(x).
	\]
	For brevity, let $\Omega^\mu_\alpha = \partial x^\mu / \partial y^{\alpha}|_{y=0}$, $\Omega^\mu_{\alpha\beta} = \partial^2 x^{\mu} / \partial y^{\alpha}\partial y^{\beta}|_{y=0}$, and so on for higher order derivatives. By Taylor's theorem, we can then write
	\[
		\frac{\partial x^\mu}{\partial y^\alpha} = \Omega^\mu_{\alpha} + \Omega^\mu_{\alpha\beta}y^\beta + O(y^2).
	\]
	We can then rewrite the tensor transformation law for the metric as
	\[
		g'_{\alpha\beta}(y) = (\Omega^\mu_{\alpha} + \Omega^\mu_{\alpha\gamma}y^\gamma)(\Omega^\nu_\beta + \Omega^\nu_{\beta\delta}y^\delta)\left(g_{\mu\nu}(0) + \frac{\partial g_{\mu\nu}(0)}{\partial x^{\lambda}}x^\lambda(y) + \frac{1}{2}\frac{\partial^2 g_{\mu\nu}(0)}{\partial x^{\lambda} \partial x^{\sigma}} x^\lambda(y)x^{\sigma}(y) \right) + O(y^3).
	\]

	Let's go through these terms degree by degree. In the zero degree case, we have the system of equations
	\[
		\Omega^\mu_\alpha\Omega^\nu_\beta g_{\mu\nu}(0) = \eta_{\mu\nu}.
	\]
	Since the metrics ar
	e symmetric, there are $n(n+1)/2$ independent equations, with $n^2$ free variables. (the $\Omega^\mu_\alpha$ terms) We know that $n^2 \geq n(n+1)/2$, so we can find solutions $\Omega^\mu_\alpha$.

	Next, in the linear case, we have the system of equations
	\[
		\left(\Omega^\mu_{\alpha\gamma}y^\gamma + \Omega^\nu_{\beta\delta} y^\delta\right) g_{\mu\nu}(0) + \left(
		\Omega^\mu_\alpha + \Omega^\nu_\beta
		\right) \pp{g_{\mu\nu}(0)}{x^\lambda} x^\lambda(y)
		= 0.
	\]
	Since we've already solved for $\Omega^\mu_\alpha$, the only independent variables in these equations are $\Omega^\mu_{\alpha\beta}$. Note that there are $n\times n(n+1)/2= n^2(n+1)/2$ independent equations (by symmetry of the metric tensor) and we have $n\times n(n+1)/2 = n^2(n+1)/2$ variables (since $\Omega^\mu_{\alpha\beta} = \Omega^\mu_{\beta\alpha}$). Thus, we can solve for $\Omega^\mu_{\alpha\beta}$ as well.

	These two observations alone are enough to prove the local flatness theorem -- and the number of degrees of freedom in dimension $n=4$ is $n(n+1)/2 - n^2=6$, which is exactly the dimension of the Lie group of Lorentz transformations. This makes sense, since Lorentz transformations leave the metric invariant in the flat spacetime.

	We can go a step further to show that the quadratic terms cannot be eliminated by coordinate transformations alone. If we try to solve
	\[
		\left(\Omega^\mu_{\alpha\gamma\kappa}y^{\kappa}y^{\gamma} + \Omega^\nu_{\beta\delta\epsilon}y^{\epsilon}y^{\delta}\right)g_{\mu\nu}(0) + \left(\Omega^\mu_{\alpha\gamma}y^\gamma + \Omega^\nu_{\beta\delta}y^{\delta}\right)\pp{g_{\mu\nu}(0)}{x^\lambda} x^\lambda(y) + \left(\Omega^\mu_\alpha + \Omega^nu_\beta\right)\frac{1}{2}\frac{\partial^2 g_{\mu\nu}(0)}{\partial x^{\lambda} \partial x^{\sigma}} x^\lambda(y)x^{\sigma}(y) = 0,
	\]
	we see that there are $(n(n+1)/2)^2 = n^2(n+1)^2/4$ equations, but $n^2(n+1)(n+2)/6$ parameters. The number of equations is higher than the number of parameters, so these equations cannot be solved in general. Furthermore, in the dimension $n=4$ case, the number of additional equations is
	\[
		\frac{n^2(n+1)^2}{4} - \frac{n^2(n+1)(n+2)}{6} = 100 - 80 = 20.
	\]
	Thus, the second order curvature of a Lorentzian $4$-manifold can be entirely expressed in 20 equations. This comes up later in the problem set.

	\begin{part}{}Explain the physical significance of this theorem.
	\end{part}

	Under the physical interpretation of gravity as the curvature of a Lorentzian manifold, what this theorem states is that gravity can be transformed away by a coordinate transformation in a small enough region (technically infinitesimally) around a point. This is Einstein's happiest thought -- the idea that a free falling frame experiences the same laws of physics as an inertial frame under no gravitational potential.
\end{solution}

\begin{problem}{2}
Covariant derivatives and Christoffel symbols.
\end{problem}

\begin{parts}
	\begin{part}{}
		Following the lecture notes, show that $\nabla_\mu A_\nu = e_\nu\cdot \partial_\mu A$ is a tensor.
	\end{part}

	Let's write $A = A_\nu e^\nu$. Then, note that
	\[
		\begin{aligned}
			\nabla_{\nu'} A_{\mu'} =  e_{\nu'}\cdot (\partial_{\mu'} A)
			 & = e_{\nu'}\cdot \frac{\partial x^{\mu}}{\partial x^{\mu'}}\partial_\mu A                                             \\
			 & = \pp{x^\mu}{x^{\mu'}}\pp{x^\nu}{x^{\nu'}}e_{\nu}\cdot A = \pp{x^\mu}{x^{\mu'}}\pp{x^\nu}{x^{\nu'}}\nabla_\nu A^\mu.
		\end{aligned}
	\]
	This is exactly the required tensor transformation law.

	\begin{part}{}
		Recall that we defined the Christoffel symbol $\Gamma^\lambda_{\mu\nu}$ through the basis vectors $e_\mu$ and the inverse basis vectors using $\nabla_\mu A^\nu = e^\nu \cdot \partial_\mu A$. Show that according to this definition,
		\[ \nabla_\mu A_\nu = \partial_\mu A_\nu - \Gamma^\lambda_{\mu\nu}A_\lambda.\]
	\end{part}

	Next, let's expand the definition to get
	\[
		\begin{aligned}
			\nabla_{\mu} A_{\nu}
			&= e_\nu\cdot \partial_\mu (A_\lambda e^\lambda)\\
			&= e_\nu\cdot e^\lambda(\partial_\mu A_\lambda) + A_\lambda e_\nu \cdot (\partial_\mu e^\lambda)\\
			&= \delta^\lambda_\nu (\partial_\mu A_\lambda) + A_\lambda e_\nu\cdot (\partial_\mu e^\lambda)\\
			&= \partial_\mu A_\nu - A_\lambda e^\nu\cdot (\partial_\mu e_\lambda) = \partial_\mu A_\nu - \Gamma^\lambda_{\mu\nu} A_\lambda.
		\end{aligned}
	\]
	Here, the equality $e_\nu\cdot (\partial_\mu e^\lambda) = -e^\nu\cdot (\partial_\mu e_\lambda)$ follows from the Leibniz rule.
\end{parts}

\begin{problem}{3}
In lecture, we showed $[\nabla_\mu, \nabla_\nu]\phi = 0$, where $\phi$ is a scalar. Using this relation, show that the Christoffel symbol satisfies the so-called ``torsion-free'' condition: $\Gamma^\lambda_{\mu\nu} = \Gamma^\lambda_{\nu\mu}$.
\end{problem}

\begin{solution}
	By expanding the commutator, we get that it is equal to
	\[
		\begin{aligned}
			[\nabla_\mu, \nabla_\nu]\phi = (\nabla_\mu\nabla_\nu - \nabla_\nu\nabla_\mu)\phi
			 & = \partial_\mu\partial_\nu \phi - \Gamma^\lambda_{\mu\nu}\partial_\lambda\phi - \partial_\nu\partial_\mu\phi - \Gamma^\lambda_{\nu\mu}\partial_\lambda \phi \\
			 & = (\Gamma^\lambda_{\nu\mu} - \Gamma^\lambda_{\mu\nu})\partial_\lambda \phi.
		\end{aligned}
	\]
\end{solution}
Since this commutator is zero, it immediately follows that $\Gamma^\lambda_{\nu\mu} = \Gamma^\lambda_{\mu\nu}$.

\begin{problem}{4}
Show that the covariant derivative defined above satisfies the following properties:
\end{problem}

\begin{parts}
	\begin{part}{(a)}
		Linearity: $\nabla_\mu (\mathbf{T} + \mathbf{S}) = \nabla_\mu \mathbf{T} + \nabla_\mu\mathbf{S}$.
	\end{part}
	Linearity follows by definition:
	\[
		\nabla_\mu (\mathbf{T} + \mathbf{S})^\nu = \partial_\mu (\mathbf{T} +\mathbf{S})^\nu +\Gamma^\lambda_{\mu\nu} (\mathbf{T}+\mathbf{S})^\lambda =
		\partial_\mu\mathbf{T}^\nu + \Gamma^\nu_{\mu\lambda}\mathbf{T}^\lambda +
		\partial_\mu\mathbf{S}^\nu + \Gamma^\nu_{\mu\lambda}\mathbf{S}^\lambda
		= \nabla_\mu \mathbf{T}^\nu + \nabla_\mu\mathbf{S}^\nu.
	\]

	\begin{part}{(b)}
		Leibniz rule: $\nabla_\mu (\mathbf{T}\mathbf{S}) = (\nabla_\mu \mathbf{T})\mathbf{S} + \mathbf{T}(\nabla_\mu\mathbf{S})$.
	\end{part}
	The Leibniz rule follows from the Leibniz rule for $\partial_\mu$:
	\[
		\begin{aligned}
			\nabla_\mu \mathbf{T}^\nu\mathbf{S}^\sigma
			 & =
			\partial_\mu(\mathbf{T}^\nu\mathbf{S}^\sigma) +
			\Gamma^\nu_{\mu\lambda} \mathbf{T}^\lambda\mathbf{S}^\sigma
			+\Gamma^\sigma_{\mu\lambda} \mathbf{T}^\nu\mathbf{S}^\lambda                                     \\
			 & =
			\partial_\mu\mathbf{T}^\nu\mathbf{S}^\sigma + \mathbf{T}^\nu\partial_\mu\mathbf{S}^\sigma+
			\Gamma^\nu_{\mu\lambda} \mathbf{T}^\lambda\mathbf{S}^\sigma
			+\Gamma^\sigma_{\mu\lambda} \mathbf{T}^\nu\mathbf{S}^\lambda                                     \\
			 & = (\nabla_\mu \mathbf{T}^\nu)\mathbf{S}^\sigma + \mathbf{T}^\nu(\nabla_\mu\mathbf{S}^\sigma).
		\end{aligned}
	\]

	\begin{part}{(c)}
		Commutes with contractions: $\nabla_\mu \delta_\nu^\lambda = 0$.
	\end{part}

	For a $(1,1)$-tensor, the covariant derivative has the form
	\[
		\nabla_\mu T^\lambda_\nu = \partial_\mu T^\lambda_\nu + \Gamma^\lambda_{\mu\sigma} T^\sigma_\nu - \Gamma^\sigma_{\mu\nu} T^\lambda_\sigma.
	\]
	Since the delta is constant, the derivative vanishes and we are left with
	\[
		\nabla_\mu \delta^\lambda_\nu = \Gamma^\lambda_{\mu\sigma} \delta^\sigma_\nu - \Gamma^\sigma_{\mu\nu}\delta^\lambda_\sigma = \Gamma^\lambda_{\mu\nu} - \Gamma^\lambda_{\mu\nu} = 0.
	\]

	\begin{part}{(d)}
		If $\nabla_\rho g_{\mu\nu}=0$, then $\nabla_\rho g^{\mu\nu}=0$.
	\end{part}
	Recall that $g_{\mu\lambda}g^{\lambda\nu}=\delta_\mu^\nu$. Taking the covariant derivative of both sides and using the Leibniz rule and the contraction commutation rule, we get
	\[
		(\nabla_\rho g_{\mu\lambda})g^{\lambda\nu} + g_{\mu\lambda}(\nabla_\rho g^{\lambda\nu}) = 0.
	\]
	Since $\nabla_\rho g_{\mu\lambda}=0$, this simplifies to $g_{\mu\lambda}(\nabla_\rho g^{\lambda\nu})=0$. Multiplying both sides by the tensor $g^{\lambda\mu}$ then gives us the desired $\nabla_\rho g^{\lambda\nu}=0$ since $g^{\lambda\mu}g_{\mu\lambda}=1$.
\end{parts}

\begin{problem}{5}
We'll do some computations on a 2-dimensional sphere of radius $R$. Use the ``cylindrical coordinate system'', $\{\rho,\phi\}$. Here, $\phi$ is the azimuthal angle and $\rho=R\sin \theta$ is the perpendicular distance to the $z$-axis.
\end{problem}

\begin{parts}
	\begin{part}{(a)}
		Show that the infinitesimal distance between the two points, that are separated by $d\rho$ and $d\phi$ is
		\[
			ds^2 = \frac{R^2\,d\rho^2}{R^2-\rho^2} + \rho^2\,d\phi^2.
		\]
		This specifies the metric.
	\end{part}

	If we embed the sphere by $\iota : S^2 \to \A^3_{(x,y,z)}$, the inclusion map takes the form
	\[
		\iota(\rho, \phi) = \left(\rho\cos \phi, \rho\sin \phi, \sqrt{R^2 - \rho^2}\right).
	\]
	The metric on $S^2$ is then the pullback by $\iota$ of the canonical metric $dx^2+dy^2+dz^2$ on $\A^3$. Computing the differentials $dx,dy,dz$, we get
	\[
		\begin{aligned}
			dx & = \cos\phi\,d\rho-\rho\sin\phi\,d\phi,    \\
			dy & = \sin\phi\,d\rho + \rho\cos\phi\,d\phi,  \\
			dz & = \frac{-\rho}{\sqrt{R^2-\rho^2}}\,d\rho.
		\end{aligned}
	\]
	Using this, the pullback of the metric becomes
	\[
		\begin{aligned}
			ds^2 = \iota_*(dx^2+dy^2+dz^2)
			 & = (\cos\phi\,d\rho - \rho \sin\phi\,d\phi)^2 + (\sin\phi\,d\rho + \rho\cos\phi\,d\phi)^2 + \rho^2/(R^2-\rho^2)\,d\rho^2 \\
			 & = d\rho^2 + \rho^2\,d\phi^2 + \rho^2/(R^2-\rho^2)\,d\rho^2                                                              \\
			 & = \frac{R^2\,d\rho^2}{R^2-\rho^2} + \rho^2\,d\phi^2.
		\end{aligned}
	\]
	This is the desired metric.

	\begin{part}{(b)}
		Write down all components of the metric, work out the inverse metric, and use them to compute all components of the Christoffel symbols.
	\end{part}

	Since there are no cross terms, the metric is diagonal so the metric and inverse metric have the form
	\[
		g_{ij} = \begin{pmatrix}\frac{R^2}{R^2-\rho^2}&0\\ 0 &\rho^2\end{pmatrix}
		\quad\textrm{and}\quad
		g^{ij} = \begin{pmatrix}\frac{R^2-\rho^2}{R^2}&0\\ 0 &\frac{1}{\rho^2}\end{pmatrix}
	\]
	The Christoffel symbols are given in terms of the metric by
	\[
		\Gamma^i_{j k} = \frac{1}{2} g^{im}(\partial_j g_{km} + \partial_k g_{jm} - \partial_m g_{j k}).
	\]
	Computing the derivatives of the metric, we have
	\[
		\partial_\rho g_{\rho\rho} = \frac{2R^2\rho}{(R^2-\rho^2)^2},\quad\partial_\phi g_{\rho\rho}=0,\quad \partial_\rho g_{\phi\phi} = 2\rho, \quad \partial_\phi g_{\phi\phi} = 0.
	\]
	Plugging these into the formula for the Christoffel symbol, we get the symbols
	\[
		\Gamma^\rho_{\rho\rho} = \frac{\rho}{R^2-\rho^2},\quad \Gamma^\rho_{\phi\phi} = -\rho\left(\frac{R^2-\rho^2}{R^2}\right),\quad \Gamma^\phi_{\rho\phi} = \Gamma^\phi_{\phi\rho}=\frac{1}{\rho}.
	\]

	\begin{part}{(c)}
		Write down the geodesic equations. Show that the geodesic on the surface is a great circle.
	\end{part}

	In general, the geodesic equation is
	\[
		\frac{d^2x^\mu}{d\lambda^2} + \Gamma^\mu_{\nu\rho} \frac{dx^\nu}{d\lambda}\frac{dx^\rho}{d\lambda} = 0.
	\]
	In our case, this gives the system of equations
	\[
		\ddot{x}^\rho + \frac{\rho}{R^2-\rho^2}(\dot{x}^\rho)^2 - \rho\left(\frac{R^2-\rho^2}{R^2}\right)(\dot{x}^\phi)^2 = 0,\quad \ddot{x}^\phi + \left(\frac{2}{\rho}\right)\dot{x}^\phi\dot{x}^\rho = 0.
	\]
	Let's show that the longitude lines are geodesics. Such a line would have $\dot{x}^\phi=0$, so the geodesic equations become:
	\[
		\ddot{x}^\rho + \frac{\rho}{R^2-\rho^2}(\dot{x}^\rho)^2=0
	\]
	Let $x^\rho(s) = R\sin(f(s))$ for any linear function $f(s)$ with slope $m$. Then
	\[
		\dot{x}^\rho = mR\cos(f(s))\quad\textrm{and}\quad \ddot{x}^\rho = -m^2R\sin(f(s)).
	\]
	Plugging this into the differential equation, we get
	\[
		-m^2R\sin(f(s)) + \frac{R\sin(f(s))}{R^2(1-\sin^2(f(s)))}m^2R^2\cos^2(f(s))=0.
	\]
	Thus, the longitude lines satisfy the geodesic equations. Since the metric is invariant under rotations of the sphere, this means that all geodesics are great circles.

	\begin{part}{(d)}
		Maybe a naive creature living on the surface of the sphere (which is now not at the north pole)
		thinks that, if he starts with a velocity that initially points to the direction in which the latitude
		remains unchanged, the straightest path should follow the path that has constant latitude. Show
		that this path does not satisfy the geodesic equations
	\end{part}

	For any $0<\rho < R$, consider the curve $x(s) = (\rho, s)$. This is a latitude line as described in the problem which is not the equator. The derivatives of this curve are
	\[
		\dot{x}^\rho = 0, \quad\ddot{x}^\rho =0,\quad \dot{x}^\phi = 1,\quad\ddot{x}^\phi = 0.
	\]
	Plugging these into the geodesic equation, the first one gives us
	\[
		-\rho\left(\frac{R^2-\rho^2}{R^2}\right) = 0.
	\]
	This means that either $\rho = 0$ or $\rho = R$, both of which were unallowed by assumption. Thus, this path is not a geodesic.
\end{parts}

\begin{problem}{6}
Show that
\[
	[\nabla_\alpha,\nabla_\beta]A^\mu = R^\mu_{\lambda\alpha\beta}A^\lambda,
\]
with the Riemann tensor defined in the lecture.
\end{problem}

\begin{solution}
	Recall that the Riemann tensor is defined as
	\[
		R^\mu_{\lambda\alpha\beta} = \partial_\alpha\Gamma^\mu_{\beta\lambda} - \partial_\beta\Gamma^\mu_{\alpha\lambda} + \Gamma^\mu_{\alpha\sigma}\Gamma^\sigma_{\beta\lambda} - \Gamma^{\mu}_{\beta\sigma}\Gamma^{\sigma}_{\alpha\lambda}.
	\]
	On the other hand, the commutator can be expressed as
	\[
		\begin{aligned}
			[\nabla_\alpha, \nabla_\beta] A^\mu = (\nabla_\alpha\nabla_\beta - \nabla_\beta\nabla_\alpha) A^\mu.
			%  & = (\nabla_\alpha\nabla_\beta - \nabla_\beta\nabla_\alpha) A^\mu                                                                                                               \\
			%  & = \nabla_\alpha (\partial_\beta A^\mu + \Gamma^\mu_{\beta\lambda}A^\lambda)
			% - \nabla_\beta (\partial_\alpha A^\mu + \Gamma^\mu_{\alpha\lambda}A^\lambda)                                                                                                     \\
			%  & = \nabla_\alpha\partial_\beta A^\mu + \Gamma^\mu_{\beta\lambda}\nabla_\alpha A^\lambda - \nabla_\beta\partial_\alpha A^\mu - \Gamma^\mu_{\alpha\lambda}\nabla_\beta A^\lambda\\
			%  &= \nabla_\alpha\partial_\beta A^\mu + \Gamma^\mu_{\beta\lambda} (\partial_\alpha A^\lambda + \Gamma^\lambda_{\alpha \sigma}A^\sigma)
			%  - \nabla_\beta\partial_\alpha A^\mu - \Gamma^\mu_{\alpha\lambda} (\partial_\beta A^\lambda + \Gamma^\lambda_{\beta \sigma}A^\sigma)\\
			%  &= (\nabla_\alpha\partial_\beta + \Gamma^\lambda_{\beta\mu}\partial_\alpha + \Gamma^\sigma_{\beta\lambda}\Gamma^\lambda_{\alpha\mu}
		\end{aligned}
	\]
	Using the covariant differentiation formula $\nabla_\mu A^\nu = \partial_\mu A^\nu + \Gamma^\nu_{\mu\lambda} A^\lambda$, we can expand the terms on one side of this commutator as
	\[
		\begin{aligned}
			\nabla_\alpha\nabla_\beta A^\mu
			 & = \partial_\alpha(\nabla_\beta A^\mu) + \Gamma^\mu_{\alpha\lambda}(\nabla_\beta A^\lambda)                                                                                                                                                                         \\
			 & = \partial_\alpha(\partial_\beta A^\mu + \Gamma^\mu_{\beta\sigma}A^\sigma)
			+\Gamma^\mu_{\alpha\lambda}(\partial_\beta A^\lambda + \Gamma^\lambda_{\beta\sigma}A^\sigma)                                                                                                                                                                          \\
			 & =\partial_\alpha\partial_\beta A^\mu+ (\partial_\alpha\Gamma^\mu_{\beta\sigma})A^\sigma + \Gamma^\mu_{\beta\sigma}\partial_\alpha A^\sigma + \Gamma^\mu_{\alpha\lambda}\partial_\beta A^\lambda + \Gamma^\mu_{\alpha\lambda}\Gamma^\lambda_{\beta\sigma} A^\sigma.
		\end{aligned}
	\]
	By swapping $\alpha$ and $\beta$ in the above expression, we get
	\[
		\nabla_\beta\nabla_\alpha A^\mu
		=\partial_\beta\partial_\alpha A^\mu + (\partial_\beta\Gamma^\mu_{\alpha\sigma})A^\sigma + \Gamma^\mu_{\alpha\sigma}\partial_\beta A^\sigma + \Gamma^\mu_{\beta\lambda}\partial_\alpha A^\lambda + \Gamma^\mu_{\beta\lambda}\Gamma^\lambda_{\alpha\sigma} A^\sigma.
	\]
	Many terms will cancel when we calculate the actual commutator. Namely, we are left with
	\[
		[\nabla_\alpha, \nabla_\beta]A^\mu = (\nabla_\alpha\nabla_\beta - \nabla_\beta\nabla_\alpha)A^\mu = (\partial_\alpha\Gamma^\mu_{\beta\sigma} - \partial_\beta \Gamma^\mu_{\alpha\sigma} + \Gamma^\mu_{\alpha\lambda}\Gamma^\lambda_{\beta\sigma} - \Gamma^\mu_{\beta\lambda}\Gamma^\lambda_{\alpha\sigma}) A^\sigma = R_{\sigma\alpha\beta}^{\mu} A^\sigma.
	\]
	This is exactly the identity we set out to prove.
\end{solution}

\begin{problem}{7}
Show the following symmetry properties of the Riemann tensor.
\end{problem}
\begin{solution}
	It suffices to check these properties infinitesimally, so let's work in a local inertial frame at a point $p\in M$. The Christoffel symbols vanish, so the Riemann curvature tensor takes the form
	\[
		\begin{aligned}
			R^{\mu}_{\nu\alpha\beta}(p)
			 & = \partial_\alpha \Gamma^\mu_{\beta\lambda}(p) - \partial_\beta \Gamma^\mu_{\alpha\lambda}(p)                                                                                                                                                      \\
			 & = \frac{1}{2}g^{\mu\lambda}(p)\left( \partial_\nu \partial_\alpha g_{\beta\lambda}(p) - \partial_\nu\partial\beta g_{\alpha\lambda}(p) - \partial_\lambda\partial_\alpha g_{\beta\nu}(p) + \partial_\lambda\partial_\beta g_{\alpha\nu}(p)\right).
		\end{aligned}
	\]
	Thus, after lowering the upper index, we get the infinitesimal form
	\[
		R_{\mu\nu\alpha\beta}(p) =
		\frac{1}{2}\left((\partial_\nu\partial_\alpha g_{\beta\mu}(p) -\partial_\nu\partial_\beta g_{\alpha\mu}(p)) - (\partial_\mu\partial_\alpha g_{\beta\nu}(p) - \partial_\mu\partial_\beta g_{\alpha\nu}(p))\right).
	\]

	% \begin{part}{}
	% \end{part}
	%
	% Recall that $R^\mu_{\nu\alpha\beta}A^\lambda = [\nabla_\alpha, \nabla_\beta] A^\mu$. Since $[\nabla_{\alpha}, \nabla_\beta] = -[\nabla_\beta, \nabla_\alpha]$, we get $R^\mu_{\nu\alpha\beta} = -R^\mu_{\nu\beta\alpha}$ and thus \[R_{\mu\nu\alpha\beta} = g^{\mu\sigma}R^\sigma_{\nu\alpha\beta} = -g^{\mu\sigma}R^\sigma_{\nu\beta\alpha} = -R_{\mu\nu\beta\alpha}.\]

	\begin{part}{}
		$R_{\mu\nu\alpha\beta} = R_{\alpha\beta\mu\nu}$,
		$R_{\mu\nu\alpha\beta} = -R_{\nu\mu\alpha\beta}$, and
		$R_{\mu\nu\alpha\beta}=-R_{\mu\nu\beta\alpha}$
	\end{part}

	These formulas follow immediately from our derived expression for $R_{\mu\nu\alpha\beta}(p)$ -- swapping $\alpha\leftrightarrow \beta$ or $\mu\leftrightarrow\nu$ adds a minus sign, while swapping $\alpha,\beta \leftrightarrow \mu,\nu$ does not affect the overall sign.

	% Looking at the formul
	%
	% \begin{part}{}
	% \end{part}
	%
	% This follows immediately from the first two properties, namely
	% \[
	% 	R_{\mu\nu\alpha\beta} = R_{\alpha\beta\mu\nu} = -R_{\alpha\beta\nu\mu} = -R_{\nu\mu\alpha\beta}
	% \]

	\begin{part}{}
		$R_{\mu\nu\alpha\beta}+R_{\mu\beta\nu\alpha}+R_{\mu\alpha\beta\nu}=0$
	\end{part}
	Writing out these expressions at $p$, we get
	\[
		\begin{aligned}
			R_{\mu\nu\alpha\beta}(p)+R_{\mu\beta\nu\alpha}(p)+R_{\mu\alpha\beta\nu}(p)
			 & =
			\frac{1}{2}\left(\partial_\nu\partial_\alpha g_{\beta\mu}(p) -\partial_\nu\partial_\beta g_{\alpha\mu}(p) - \partial_\mu\partial_\alpha g_{\beta\nu}(p) + \partial_\mu\partial_\beta g_{\alpha\nu}(p)\right)        \\
			 & +
			\frac{1}{2}\left(\partial_\beta\partial_\nu g_{\alpha\mu}(p) -\partial_\beta\partial_\alpha g_{\nu\mu}(p) - \partial_\mu\partial_\nu g_{\alpha\beta}(p) + \partial_\mu\partial_\alpha g_{\nu\beta}(p)\right)        \\
			 & +\frac{1}{2}\left(\partial_\alpha\partial_\beta g_{\nu\mu}(p) -\partial_\alpha\partial_\nu g_{\beta\mu}(p) - \partial_\mu\partial_\beta g_{\nu\alpha}(p) + \partial_\mu\partial_\nu g_{\beta\alpha}(p)\right)=0, \\
		\end{aligned}
	\]
	by making use of the commutation of partials and symmetry of the metric tensor to cancel like terms.
\end{solution}

\begin{problem}{8}
Number of components in the Riemann curvature tensor.
\end{problem}

\begin{parts}
	\begin{part}{}
		Use the above symmetry properties to show that in $(3+1)$-dimensional spacetime, the number of independent components of $R_{\mu\nu\alpha\beta}$ is $20$.
	\end{part}

	A general tensor with $4$ indices has $4^4 = 256$ components. The antisymmetry conditions mean that there are $6$ independent combinations for $\mu\nu$ and $\alpha\beta$, so we have $36$ combinations. The further symmetry condition when swapping $\mu\nu$ and $\alpha\beta$ gives us $6\times 7/2=21$ components. Finally, the Bianchi identity reduces the number of independent components by $1$ so we are left with $20$ components.

	\begin{part}{}
		Recall that in the proof of the local flatness theorem, we counted $100$ components of $\partial_\alpha\partial_\beta g_{\mu\nu}$, and there are 80 parameters in the coordinate transformation of $\partial_\alpha\partial_\beta g_{\mu\nu}$. Can you relate these and spell out an interpretation of the number $20$?
	\end{part}

	This means that there are $20$ components of $\partial_\alpha\partial_\beta g_{\mu\nu}$ which cannot be transformed away under coordinate transformations, and these correspond to the $20$ independent components of the Riemann curvature tensor. In other words, the Riemann curvature tensor captures \emph{all} of the intrinsic second order curvature data of the Lorentzian manifold.
\end{parts}

\end{document}
