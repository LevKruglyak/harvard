\documentclass{../../templates/lkx_pset}

\title{Astron 140 Problem Set 6}
\author{Lev Kruglyak}
\due{November 5, 2024}

\usepackage{mdframed}


\usepackage[T1]{fontenc}
\RequirePackage{mlmodern}


\mdfdefinestyle{answer}{%
	linecolor=black,
	outerlinewidth=2pt,
	%roundcorner=20pt,
	innertopmargin=4pt,
	innerbottommargin=4pt,
	innerrightmargin=4pt,
	innerleftmargin=4pt,
	leftmargin = 4pt,
	rightmargin = 4pt
	%backgroundcolor=gray!50!white}
}

\newenvironment{answerbox}{
	\begin{mdframed}[style=answer,nobreak=true,userdefinedwidth=30em]}{\end{mdframed}}

\usepackage{siunitx}
\providecommand{\unitsi}[1]{\qty[per-mode = symbol]{#1}{}}
\providecommand{\mpssi}[1]{\qty[per-mode = symbol]{#1}{\m\per\s}}
\providecommand{\mpsssi}[1]{\qty[per-mode = symbol]{#1}{\m\per\s^2}}
\providecommand{\Gsi}[1]{\qty[per-mode = symbol]{#1}{\m^3\per\s^2\kg}}
\providecommand{\kBsi}[1]{\qty[per-mode = symbol]{#1}{\m^2\s^{-2}\K^{-1}\kg}}
\providecommand{\hsi}[1]{\qty[per-mode = symbol]{#1}{\J\cdot s}}
\providecommand{\kgsi}[1]{\qty[per-mode = symbol]{#1}{\kg}}
\providecommand{\ssi}[1]{\qty[per-mode = symbol]{#1}{\s}}
\providecommand{\psssi}[1]{\qty[per-mode = symbol]{#1}{\s^{-2}}}
\providecommand{\msi}[1]{\qty[per-mode = symbol]{#1}{\m}}
\providecommand{\desi}[1]{\qty[per-mode = symbol]{#1}{\J\per \m^3}}

\renewcommand{\O}{\mathrm{O}}
\providecommand{\Aff}{\mathrm{Aff}}
\providecommand{\SO}{\mathrm{SO}}

\providecommand{\Frame}{\mathrm{Fr}}

\providecommand{\A}{\mathbb{A}}

\providecommand{\definefunction}[5]{
	\begin{array}{rcl}
		#1 : #2 & \xrightarrow{\phantom{---}} & #3 \\
		#4      & \xmapsto{\phantom{---}}     & #5
	\end{array}
}


\providecommand{\pp}[2]{\frac{\partial #1}{\partial #2}}


\begin{document}
\maketitle

\begin{problem}{1}
In the analyses of the Mercury perihelion precession, we compared two cases, namely the case with or without the GR correction term, respectively. Schematically review the final solutions $\delta r(\phi)$ for these two cases (no derivation is necessary), and point out what is the key difference between the two solutions that leads to the precession.
\end{problem}


\begin{problem}{2}
Using the result derived in the lecture, compute the rate of perihelion precession of the Earth orbit.
\end{problem}

\begin{problem}{3}
Consider the problem of gravitational deflection of light. Following the lecture, derive the expression of the effective potential, $V_{\text{eff}}(r)$. Use the effective potential to list all possible kinds of trajectories of the light in the Schwarzschild metric. (Recall the method we used when analyzing the perihelion precession.) Sketch the shape of $V_{\text{eff}}(r)$, and illustrate each case on the $V_{\text{eff}}$ plot and each trajectory with a 2D sketch.
\end{problem}

\begin{problem}{4}
Using the formula of the deflection angle in the Schwarzschild metric, $\Delta \phi_{\text{deflection}} = 4GM/c^2 b$, estimate the deflection angle of a light ray passing through the edge of Jupiter.
\end{problem}

\end{document}
