\documentclass[expanded]{lkx_pset}

\title{CS181 Problem Set 6}
\author{Lev Kruglyak}
\due{April 26, 2024}

\usepackage{pgfplots}
\pgfplotsset{compat=1.14}
\usepackage[outputdir=build]{minted}
\usepackage{graphicx}

\usepackage{amsmath}
\usepackage{amssymb}
\usepackage{graphicx}
\usepackage{tikz}
\usetikzlibrary{patterns}
\usepackage{subfig}
\usepackage{comment}

\newcommand{\boldA}{\mathbf{A}}
\newcommand{\boldB}{\mathbf{B}}
\newcommand{\boldC}{\mathbf{C}}
\newcommand{\boldD}{\mathbf{D}}
\newcommand{\boldE}{\mathbf{E}}
\newcommand{\boldF}{\mathbf{F}}
\newcommand{\boldG}{\mathbf{G}}
\newcommand{\boldH}{\mathbf{H}}
\newcommand{\boldI}{\mathbf{I}}
\newcommand{\boldJ}{\mathbf{J}}
\newcommand{\boldK}{\mathbf{K}}
\newcommand{\boldL}{\mathbf{L}}
\newcommand{\boldM}{\mathbf{M}}
\newcommand{\boldN}{\mathbf{N}}
\newcommand{\boldO}{\mathbf{O}}
\newcommand{\boldP}{\mathbf{P}}
\newcommand{\boldQ}{\mathbf{Q}}
\newcommand{\boldR}{\mathbf{R}}
\newcommand{\boldS}{\mathbf{S}}
\newcommand{\boldT}{\mathbf{T}}
\newcommand{\boldU}{\mathbf{U}}
\newcommand{\boldV}{\mathbf{V}}
\newcommand{\boldW}{\mathbf{W}}
\newcommand{\boldX}{\mathbf{X}}
\newcommand{\boldY}{\mathbf{Y}}
\newcommand{\boldZ}{\mathbf{Z}}
\newcommand{\bolda}{\mathbf{a}}
\newcommand{\boldb}{\mathbf{b}}
\newcommand{\boldc}{\mathbf{c}}
\newcommand{\boldd}{\mathbf{d}}
\newcommand{\bolde}{\mathbf{e}}
\newcommand{\boldf}{\mathbf{f}}
\newcommand{\boldg}{\mathbf{g}}
\newcommand{\boldh}{\mathbf{h}}
\newcommand{\boldi}{\mathbf{i}}
\newcommand{\boldj}{\mathbf{j}}
\newcommand{\boldk}{\mathbf{k}}
\newcommand{\boldl}{\mathbf{l}}
\newcommand{\boldm}{\mathbf{m}}
\newcommand{\boldn}{\mathbf{n}}
\newcommand{\boldo}{\mathbf{o}}
\newcommand{\boldp}{\mathbf{p}}
\newcommand{\boldq}{\mathbf{q}}
\newcommand{\boldr}{\mathbf{r}}
\newcommand{\bolds}{\mathbf{s}}
\newcommand{\boldt}{\mathbf{t}}
\newcommand{\boldu}{\mathbf{u}}
\newcommand{\boldv}{\mathbf{v}}
\newcommand{\boldw}{\mathbf{w}}
\newcommand{\boldx}{\mathbf{x}}
\newcommand{\boldy}{\mathbf{y}}
\newcommand{\boldz}{\mathbf{z}}

\newcommand{\mcA}{\mathcal{A}}
\newcommand{\mcB}{\mathcal{B}}
\newcommand{\mcC}{\mathcal{C}}
\newcommand{\mcD}{\mathcal{D}}
\newcommand{\mcE}{\mathcal{E}}
\newcommand{\mcF}{\mathcal{F}}
\newcommand{\mcG}{\mathcal{G}}
\newcommand{\mcH}{\mathcal{H}}
\newcommand{\mcI}{\mathcal{I}}
\newcommand{\mcJ}{\mathcal{J}}
\newcommand{\mcK}{\mathcal{K}}
\newcommand{\mcL}{\mathcal{L}}
\newcommand{\mcM}{\mathcal{M}}
\newcommand{\mcN}{\mathcal{N}}
\newcommand{\mcO}{\mathcal{O}}
\newcommand{\mcP}{\mathcal{P}}
\newcommand{\mcQ}{\mathcal{Q}}
\newcommand{\mcR}{\mathcal{R}}
\newcommand{\mcS}{\mathcal{S}}
\newcommand{\mcT}{\mathcal{T}}
\newcommand{\mcU}{\mathcal{U}}
\newcommand{\mcV}{\mathcal{V}}
\newcommand{\mcW}{\mathcal{W}}
\newcommand{\mcX}{\mathcal{X}}
\newcommand{\mcY}{\mathcal{Y}}
\newcommand{\mcZ}{\mathcal{Z}}

\newcommand{\reals}{\ensuremath{\mathbb{R}}}
\newcommand{\integers}{\ensuremath{\mathbb{Z}}}
\newcommand{\rationals}{\ensuremath{\mathbb{Q}}}
\newcommand{\naturals}{\ensuremath{\mathbb{N}}}
\newcommand{\trans}{\ensuremath{\mathsf{T}}}
\newcommand{\ident}{\mathbf{I}}
\newcommand{\bzero}{\mathbf{0}}

\newcommand{\balpha}{\mathbf{\alpha}}
\newcommand{\bbeta}{\mathbf{\beta}}
\newcommand{\bdelta}{\mathbf{\delta}}
\newcommand{\boldeta}{\mathbf{\eta}}
\newcommand{\bkappa}{\mathbf{\kappa}}
\newcommand{\bgamma}{\mathbf{\gamma}}
\newcommand{\bmu}{\boldsymbol{\mu}}
\newcommand{\bphi}{\mathbf{\phi}}
\newcommand{\bpi}{\boldsymbol{\pi}}
\newcommand{\bpsi}{\mathbf{\psi}}
\newcommand{\bsigma}{\mathbf{\sigma}}
\newcommand{\btheta}{\mathbf{\theta}}
\newcommand{\bxi}{\mathbf{\xi}}
\newcommand{\bGamma}{\mathbf{\Gamma}}
\newcommand{\bLambda}{\mathbf{\Lambda}}
\newcommand{\bOmega}{\mathbf{\Omega}}
\newcommand{\bPhi}{\mathbf{\Phi}}
\newcommand{\bPi}{\mathbf{\Pi}}
\newcommand{\bPsi}{\mathbf{\Psi}}
\newcommand{\bSigma}{\mathbf{\Sigma}}
\newcommand{\bTheta}{\mathbf{\Theta}}
\newcommand{\bUpsilon}{\mathbf{\Upsilon}}
\newcommand{\bXi}{\mathbf{\Xi}}
\newcommand{\bepsilon}{\mathbf{\epsilon}}

\def\argmin{\operatornamewithlimits{arg\,min}}

\newcommand{\given}{\,|\,}
\newcommand{\distNorm}{\mathcal{N}}

\newcommand{\mueps}{\mu_{\epsilon}}
\newcommand{\sigeps}{\sigma_{\epsilon}}
\newcommand{\mugam}{\mu_{\gamma}}
\newcommand{\siggam}{\sigma_{\gamma}}
\newcommand{\muzp}{\mu_{p}}
\newcommand{\sigzp}{\sigma_{p}}
\newcommand{\gauss}[3]{\frac{1}{2\pi#3}e^{-\frac{(#1-#2)^2}{2#3}}}


% \collaborator{Artemas Radik}
\collaborator{AJ LaMotta}
\collaborator{Leonardo Kaplan}
\collaborator{GPT-4 (for debugging help)}

\begin{document}
\maketitle

\begin{problem}{1}[Hidden Markov Models]
In this problem, you will be working with one-dimensional Kalman filters, which are \textit{continuous-state} Hidden Markov Models. Let $z_0, z_1, \cdots , z_T$ be the hidden states of the system and $x_0, x_1, \cdots, x_T$ be the observations produced. Then, state transitions and emissions of observations work as follows:
\begin{eqnarray*}
	z_{t+1} &= z_{t} + \epsilon_{t} \\
	x_{t} & = z_{t} + \gamma_{t}
\end{eqnarray*}
where $\epsilon_t \sim N(0,\sigeps^2)$ and $\gamma_t \sim N(0,\siggam^2)$. The value of the first hidden state follows the distribution $z_0 \sim N(\muzp,\sigzp^2)$.
\end{problem}
\begin{parts}
	\begin{part}{1} Draw the graphical model corresponding to the one-dimensional Kalman filter.
	\end{part}
	\begin{part}{2} In this part we will walk through the derivation of the conditional distribution of $z_t|(x_0, \cdots, x_{t})$.
	\end{part}
	\begin{parts}
		\begin{part}{a} How does the quantity $p(z_t| x_0, \cdots, x_{t})$ relate to $\alpha_t(z_t)$ and $\beta_t(z_t)$ from the forward-backward algorithm for HMMs?  What is the operation we are performing called?
		\end{part}
		\begin{part}{b} The above quantity $p(z_t|x_0, \cdots, x_t)$ is the PDF for a Normal distribution with mean $\mu_t$ and variance $\sigma_t^2$. We start our derivation of $\mu_t$ and $\sigma_t^2$ by writing:
			\begin{align*}
				p(z_t|x_0, \cdots, x_t) \propto p(x_t|z_t)p(z_t|x_0, \cdots x_{t-1})
			\end{align*}
			What is $p(x_t|z_t)$ equal to?
		\end{part}
		\begin{part}{c} Suppose we are given the mean and variance of the conditional distribution $z_{t-1}|(x_0, \cdots, x_{t-1})$ as $\mu_{t-1}$, $\sigma^2_{t-1}$. What is $p(z_t|x_0, \cdots x_{t-1})$ equal to?
		\end{part}

		\textbf{Hint 1}: Start by marginalizing out over $z_{t-1}$.

		\textbf{Hint 2}: You may cite the fact that
		\[\int N(y-x ; \mu_a, \sigma^2_a)N(x ; \mu_b, \sigma^2_b)dx = N(y ; (\mu_a + \mu_b), (\sigma^2_a + \sigma^2_b))\]
		\begin{part}{d} Combine your answers from parts (b) and (c) to get a final expression for $p(z_t|x_0, \cdots, x_t)$. Report the mean $\mu_t$ and variance $\sigma_t^2$ of this Normal.
		\end{part}

		\textbf{Hint 1}: Rewrite $N(x_t; z_t, \siggam^2)$ as $N(z_t; x_t, \siggam^2)$.

		\textbf{Hint 2}: You may cite the fact that
		\[N(x; \mu_a, \sigma^2_a)N(x; \mu_b, \sigma^2_b) = N\left(x; \frac{\sigma^2_b}{\sigma^2_a+\sigma^2_b}\mu_a + \frac{\sigma^2_a}{\sigma^2_a+\sigma^2_b}\mu_b, \ \left(\frac{1}{\sigma^2_a} + \frac{1}{\sigma^2_b}\right)^{-1}\right)\]
	\end{parts}
	\begin{part}{3}Interpret $\mu_t$ in terms of how it combines observations from the past with the current observation.
	\end{part}
\end{parts}

\end{document}
