\documentclass{../../templates/lkx_pset}

\title{Ling 105 Problem Set 1}
\author{Lev Kruglyak}
\due{September 13, 2024}

\usepackage{tipa}

%
% \collaborator{AJ LaMotta}
% \collaborator{Jarell Cheong}

\begin{document}
\maketitle

\begin{problem}{1}
Google rtMRI IPA and you should see the USC Span website. Observe \textipa{[\textturnr]} as pronounced by Esling, Gordon, and Byrd (newer version only, not the 2005 version).
Who curls and who bunches? You can pause the video if you’re unsure. Now observe
the same three speakers saying \textipa{[l]}. Does it look more like curling or bunching?
\end{problem}

\begin{solution}
	\bfseries It looks like Esling and Byrd curl their tongue for \textipa{[\textturnr]}, while Gordon bunches. However, they all seem to curl their tongue for \textipa{[l]}.
\end{solution}

\begin{problem}{2}
Label each of the following IPA symbols or narrow places of articulation according
to its broad place of articulation: labial, coronal, dorsal, or “other” if none of these
three. If more than one place applies, give both.
\end{problem}

\begin{solution}
	\begin{center}
		\begin{tabular}{|c|c|c|c|}
			\hline
			Velar         & \textbf{Dorsal}  & \textipa{[tS]} & \textbf{Coronal}           \\
			\hline
			\textipa{[f]} & \textbf{Labial}  & Glottal        & \textbf{Other}             \\
			\hline
			\textipa{[T]} & \textbf{Coronal} & \textipa{[w]}  & \textbf{Labial and Dorsal} \\
			\hline
			Alveolar      & \textbf{Coronal} & Palatal        & \textbf{Dorsal}            \\
			\hline
			Uvular        & \textbf{Dorsal}  & Retroflex      & \textbf{Coronal}           \\
			\hline
		\end{tabular}
	\end{center}
\end{solution}

\begin{problem}{3}
For each of the following two midsagittals, give the place of articulation, manner,
and nasality, along with a matching IPA symbol (more than one symbol may be
compatible, but you need give only one). In case it is difficult to tell from the pic-
ture, the closure at the rear of the tongue is complete in (b).
\end{problem}

\begin{solution}
	\bfseries
	For (a), it appears to be a dental nasal strop -- for example \textipa{[n]}. In (b), the sound is definitely velar, perhaps fricative -- for instance \textipa{[x]}.
\end{solution}

\begin{problem}{4}
Give an IPA symbol (or pair of symbols) for each of the following.
\end{problem}

\begin{solution}
	\begin{center}
		\begin{tabular}{|c|c|c|c|}
			\hline
			Voiced alveolar sibilant     & \textbf{\textipa{[z]}}
			                             & Voiced palato-alveolar affricate & \textbf{\textipa{[dJ]}}
			\\ \hline
			Velar nasal stop             & \textbf{\textipa{[N]}}
			                             & Glottal fricative                & \textbf{\textipa{[h]}, \textipa{[H]}}
			\\ \hline
			Coronal trill                & \textbf{\textipa{[r]}}
			                             & Voiceless palatal stop           & \textbf{\textipa{[c]}}
			\\ \hline
			Voiceless bilabial plosive   & \textbf{\textipa{[p]}}
			                             & Voiced velar fricative           & \textbf{\textipa{[G]}}
			\\ \hline
			Voiced interdental fricative & \textbf{\textipa{[D]}}
			                             & Palatal glide                     & \textbf{\textipa{[J]}}
			\\ \hline
		\end{tabular}
	\end{center}
\end{solution}

\pagebreak
\begin{problem}{5}
  During speech, is the velum (more specifically, velo-pharyngeal port) usually open or closed? When not speaking, is the velum typically open or closed? Why are nasals such as \textipa{[m, n, N]} considered stops?
\end{problem}

\begin{solution}
  \bfseries
Unless someone is making nasal sounds, the velum is typically open. When not speaking, it's open to allow for breathing through the nose. The nasals are considered stops because airflow is redirected through the nasal cavity instead of the mouth.
\end{solution}

\begin{problem}{6}
  A plosive is possible at the glottis, namely, \textipa{[P]}. But there is no corresponding nasal stop at that place. Why is a glottal nasal impossible?
\end{problem}

\begin{solution}
  \bfseries
  The nasal cavity is above the glottis, so it's impossible for airflow to reach the nasal cavity if it's being blocked at the glottis.
\end{solution}

\begin{problem}{7}
  Rank the following in terms of aperture (openness): glide, fricative, vowel, plosive. At which of these four apertures do phonemic voicing contrasts occur in English?
\end{problem}

\begin{solution}
  \bfseries
  In order of increasing aperture, we have: plosive, fricative, glide, vowel. In English, phonemic voicing contrasts occur for plosive and fricative apertures.
\end{solution}

\begin{problem}{8}
  Does English have a phonemic trill? Give the symbol for a bilabial trill. Can you make a bilabial trill?
\end{problem}

\begin{solution}
  \bfseries 
  There are no trills in English. The symbol for a bilabial trill is \textipa{[\;B]}. I can produce this sound; it sounds like a ``voiced rasberry''.
\end{solution}

\begin{problem}{9}
  Is the major place of \textipa{[M]} considered labial, coronal or both? Why? \textipa{[M]} is found as an allophone of /m/ in English; give an example of an English word in which it might be encountered. \textipa{/M/} is claimed to be phonemic in one language, which you can find online. Name the language and where it is spoken and give an example of a word in that language that includes it.
\end{problem}

\begin{solution}
  \bfseries
  This sound is labial, and found in English in the word ``symphony'' for instance. The only language which in which \textipa{[M]} is phonemic is the Kukuya language, spoken by the Teke people in the Congolese plateau. A word in this language which contains this phoneme is \textipa{/k\`i-M\`a\`al\`a/}, which means ``to laugh at''.
\end{solution}

\pagebreak
\begin{problem}{10}
  Optional survey.
\end{problem}
\begin{solution}
  \bfseries Answers:
  \begin{enumerate}[(a)]
    \item I'm a native speaker of English and Russian.
    \item I have studied English, Russian, and Hebrew.
    \item Same.
    \item Different.
    \item Same.
    \item Same.
    \item Both rhyme.
    \item Water fountain.
    \item Soda.
    \item Three syllables.
    \item Two syllables.
    \item Don't rhyme.
  \end{enumerate}
\end{solution}


\end{document}
