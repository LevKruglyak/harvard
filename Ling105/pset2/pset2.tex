\documentclass{../../templates/lkx_pset}

\title{Ling 105 Problem Set 2}
\author{Lev Kruglyak}
\due{September 20, 2024}

\usepackage{tipa}

%
% \collaborator{AJ LaMotta}
% \collaborator{Jarell Cheong}

\begin{document}
\maketitle

\begin{problem}{1}
Would the plosive in each of the following English words normally be aspirated or unaspirated? Aspiration, as discussed in class, is not an arbitrary function of individual words, but follows a rule.
\end{problem}

\begin{solution}
	\begin{center}
		\begin{tabular}{|c|c|c|c|}
			\hline
			attain   & \textbf{Aspirated}   & art     & \textbf{Aspirated}   \\
			\hline
			tell     & \textbf{Aspirated}   & vesper  & \textbf{Unaspirated}   \\
			\hline
			space    & \textbf{Unaspirated} & liquor  & \textbf{Unaspirated} \\
			\hline
			cancel   & \textbf{Aspirated} & starve  & \textbf{Unaspirated} \\
			\hline
			subpoena & \textbf{Aspirated}   & impinge & \textbf{Aspirated}   \\
			\hline
		\end{tabular}
	\end{center}
\end{solution}

\begin{problem}{2}
Which English words do the following narrow IPA transcriptions represent?
\end{problem}

\begin{solution}
	\begin{center}
		\begin{tabular}{|c|c|c|c|}
			\hline
			\textipa{["fi\textbarl]} & \textbf{Feel} & \textipa{["k\super{h}Ist]} & \textbf{Kissed} \\
			\hline
			\textipa{["Os\s{m}]} & \textbf{Awesome} & \textipa{["fiv\textrhookschwa]} & \textbf{Fever} \\
			\hline
			\textipa{["k\r*waI\*r]} & \textbf{Choir} & \textipa{["s\~a\~Ind]} & \textbf{Signed} \\
			\hline
			\textipa{["naIR@d]} & \textbf{Knighted} & \textipa{["soU\~IN]} & \textbf{Sewing} \\
			\hline
			\textipa{["t\super{h}\~2N]} & \textbf{Tongue} & \textipa{["hE\*rd]} & \textbf{Herd} \\
			\hline
			\textipa{["\r*d\~a\~Un]} & \textbf{Dawn} & \textipa{["k\super{h}\~o\~Um]} & \textbf{Comb} \\
			\hline
			\textipa{["DaI]} & \textbf{Thy} & \textipa{["Z\~On\*r@]} & \textbf{Genre} \\
			\hline
			\textipa{["k\super{h}\textrhookrevepsilon\textglotstop\s{n}]} & \textbf{Curtain} & \textipa{["k\r{\j}u]} & \textbf{Queue} \\
			\hline
			\textipa{["eIZ\s{n}]} & \textbf{Asian} & \textipa{["2v]} & \textbf{Of} \\
			\hline
			\textipa{["k\r*{\*r}\~2m]} & \textbf{Crumb} & \textipa{["k\r*w\~\i n]} & \textbf{Queen} \\
			\hline
		\end{tabular}
	\end{center}
\end{solution}

\begin{problem}{3}
Transcribe the following words into narrow IPA, as they would usually be pronounced.
\end{problem}

\begin{solution}
	\begin{center}
		\begin{tabular}{|c|c|c|c|}
			\hline
			Heat & \textbf{\textipa{["hit]}} & Clipped & \textbf{\textipa{["k\r*lIpt]}} \\
			\hline
			Alarm & \textbf{\textipa{[@"la\*rm]}} & Lapses & \textbf{\textipa{["l\ae psiz]}} \\
			\hline
			Carry & \textbf{\textipa{["k\super{h}E\*ri]}} & Nasal & \textbf{\textipa{["neIz\s{\textbarl}]}} \\
			\hline
			Stomach & \textbf{\textipa{["st\~2mIk]}} & Banner & \textbf{\textipa{["\r*b\~\ae\~R\textrhookschwa]}} \\
			\hline
			Rug & \textbf{\textipa{["\*r2g]}} & Streamed & \textbf{\textipa{["st\*r\~\i md]}} \\
			\hline
			Bird & \textbf{\textipa{[\r*b\textrhookrevepsilon d]}} & Prison & \textbf{\textipa{["p\r*{\*r}Iz\s{n}]}} \\
			\hline
			Bones & \textbf{\textipa{["\r*b\~o\~Unz]}} & Allow & \textbf{\textipa{[@"laU]}} \\
			\hline
			Item & \textbf{\textipa{["aIR\s{m}]}} & Reasoned & \textbf{\textipa{[\*riz\s{n}d]}} \\
			\hline
			These & \textbf{\textipa{["Diz]}} & Funnel & \textbf{\textipa{["f\~2\~R\s{\textbarl}]}} \\
			\hline
			Knowledege & \textbf{\textipa{["nAlIdZ]}} & We'll & \textbf{\textipa{["wI\textbarl]}} \\
			\hline
		\end{tabular}
	\end{center}
\end{solution}

\begin{problem}{4}
  Write the appropriate IPA symbol (with a diacritic if needed).
\end{problem}

\begin{solution}
  \begin{center}
		\begin{tabular}{|c|c|c|c|}
			\hline
			Voiceless lateral fricative & \textbf{\textipa{[\textbeltl]}} & Voiceless uvular fricative & \textbf{\textipa{[X]}} \\
			\hline
			  Voiced interdental fricative & \textbf{\textipa{[D]}} & Voiceless dental plosive & \textbf{\textipa{[\|[t]}} \\
			\hline
			  Devoiced velar nasal & \textbf{\textipa{[\r{N}]}} & Voiceless palatal stop & \textbf{\textipa{[c]}} \\
			\hline
			Alveolar tap & \textbf{\textipa{[R]}} & Tense mid back rounded vowel & \textbf{\textipa{[O]}} \\
			\hline
			Glottal stop & \textbf{\textipa{[P]}} & Low back vowel & \textbf{\textipa{[A]}} \\
			\hline
			Voiceless glottal fricative & \textbf{\textipa{[h]}} & High back lax vowel & \textbf{\textipa{[U]}} \\
			\hline
			Voiceless uvular plosive & \textbf{\textipa{[\;G]}} & Nasalized schwa & \textbf{\textipa{[\~@]}} \\
			\hline
			Labiovelar glide & \textbf{\textipa{[w]}} &``ash'' & \textbf{\textipa{[\ae]}} \\
			\hline
			Voiced alveolar fricative & \textbf{\textipa{[z]}} & ``engma'' & \textbf{\textipa{[N]}} \\
			\hline
		Uvular trill & \textbf{\textipa{[\;R]}} & Untrilled alveolar rhotic & \textbf{\textipa{[R]}, \textipa{[\*r]}} \\
			\hline
		\end{tabular}
  \end{center}
\end{solution}

\begin{problem}{5}
  At what places of articulation is a rhotic possible? At what places is a lateral possible but not a rhotic?
\end{problem}

\begin{solution}
  \bfseries 
  Rhotics are possible in the alveolar, postalveolar, retroflex, and uvular places of articulation. At the dental, palatal, and velar places, lateral sounds are possible but not rhotics.
\end{solution}

\begin{problem}{6}
  Two waveforms and spectrograms are show. For each say whever VOT is positive, negative, or approximately zero, whether each is voiced or voiceless, and whether each is aspirated or unasiprated.
\end{problem}

\begin{solution}
  \bfseries 
  For the first, VOT is positive, the sound is voiceless and aspirated. For the second, the VOT is approximately zero, and the sound is voiceless and unaspirated.
\end{solution}

\begin{problem}{7}
  What is the difference in articulation between English ``b'' and Hindi/French ``b'' in initial position?
\end{problem}

\begin{solution}
  \bfseries 
  In Hindi/French, the VOT is negative so the sound is voiced while in English, the VOT is approximately zero so the sound is voiceless.
\end{solution}

\begin{problem}{8}
  The IPA has simple symbols for voiceless \textipa{[p]} and voiced \textipa{[b]}, but not for aspirated \textipa{[p\super{h}]}. Why might this be the case? Furthermore, why is the symbol \textipa{[\r*{b}]} sometimes used when the IPA already has \textipa{[p]}?
\end{problem}

\begin{solution}
  \bfseries 
  There is no need for a separate symbol for \textipa{[p\super{h}]} since we already have the symbols \textipa{[p]} and \textipa{h} -- this avoids bloating the alphabet. The natural question might be why we need the symbol \textipa{[p]} then, since it is quite similar to \textipa{[\r*{b}]}. While these are pronounced essentially the same, the phonemic distinction between \textipa{/b/} and \textipa{/p/} is useful, for example in English.
\end{solution}

\end{document}
