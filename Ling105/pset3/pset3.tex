\documentclass{../../templates/lkx_pset}

\title{Ling 105 Problem Set 3}
\author{Lev Kruglyak}
\due{September 27, 2024}

\usepackage{tipa}

%
% \collaborator{AJ LaMotta}
% \collaborator{Jarell Cheong}

\begin{document}
\maketitle

\begin{problem}{1}
Which English words do the following narrow IPA transcriptions represent? Keep in mind that words might include affixes.
\end{problem}

\begin{enumerate}[(a)]
  \item \textbf{Iron}
  \item \textbf{Fowl}
  \item \textbf{Use}
  \item \textbf{Thus }
  \item \textbf{Kale}
  \item \textbf{Noun}
  \item \textbf{Bared}
  \item \textbf{Castle}
  \item \textbf{Money}
  \item \textbf{Written}
  \item \textbf{Autumn}
  \item \textbf{Fuse}
  \item \textbf{Cushion}
  \item \textbf{Wrong}
  \item \textbf{Animated}
  \item \textbf{Tree}
  \item \textbf{Tenant}
  \item \textbf{Turkey}
  \item \textbf{Washed}
  \item \textbf{Dessert}
\end{enumerate}

\pagebreak
\begin{problem}{2}
Transcribe the following words into narrow IPA.
\end{problem}

\begin{enumerate}[(a)]
  \item \textbf{\textipa{["l2vd]}}
  \item \textbf{\textipa{["w\textrhookrevepsilon dz]}}
  \item \textbf{\textipa{["sO\textbarl t\r*Id]}}
  \item \textbf{\textipa{["p\super{h}\~o\~Inj\s{n}t]}}
  \item \textbf{\textipa{[\r{g}\textrhookschwa"aZ]}}
  \item \textbf{\textipa{["Of\s{n}]}}
  \item \textbf{\textipa{["dZioUd]}}
  \item \textbf{\textipa{[""p\super{h}A\*rm@"Z\~An]}}
  \item \textbf{\textipa{["miRi\textrhookschwa]}}
  \item \textbf{\textipa{["luz]}}
  \item \textbf{\textipa{["t\r*w\~IndZ]}}
  \item \textbf{\textipa{[\r{@}"k\r*waI\textrhookschwa z]}}
  \item \textbf{\textipa{["spaIR\textrhookschwa]}}
  \item \textbf{\textipa{["akt\r*Iv]}}
  \item \textbf{\textipa{["k\super{h}\~e\~Imb\*rIdZ]}}
  \item \textbf{\textipa{[""m\ae s@"tSusIts]}}
  \item \textbf{\textipa{[@"lEkS\s{n}]}}
  \item \textbf{\textipa{["tSa\*r\textbarl z]}}
  \item \textbf{\textipa{["k\super{h}\~ant\r*{@}\~{R}\s{n}t]}}
\end{enumerate}

\pagebreak
\begin{problem}{3}
  Write the appropriate IPA symbol.
\end{problem}

\begin{enumerate}[(a)]
  \item \textbf{\textipa{[y:]}}
  \item \textbf{\textipa{[\;R]}}
  \item \textbf{\textipa{[\s{\textbarl}]}}
  \item \textbf{\textipa{[\|[{\textbarl}]}}
  \item \textbf{\textipa{[\~R]}}
  \item \textbf{\textipa{[\|[pf]}}
  \item \textbf{\textipa{[\c{c}]}}
  \item \textbf{\textipa{[]}}
  \item \textbf{\textipa{[\|[n]}}
  \item \textbf{\textipa{[\:l]}}
  \item \textbf{\textipa{[\ae]}}
  \item 
  \item \textbf{\textipa{[\textlyoghlig]}}
  \item \textbf{\textipa{[H]}}
  \item \textbf{\textipa{[$\mathbf{\emptyset}$]}}
  \item \textbf{\textipa{[V]}}
  \item \textbf{\textipa{[U]}}
  \item \textbf{\textipa{[2]}}
  \item \textbf{\textipa{[i\textrhoticity]}}
  \item \textbf{\textipa{[X]}}
\end{enumerate}

\pagebreak
\begin{problem}{4}
  Recall that German unlaut involves fronting a vowel while keeping its other features intact. The following singular German nouns form their plurals by umlauting the first vowel. In each case, give the IPA of the first vowel of the plural form.
\end{problem}

  \begin{enumerate}[(a)]
    \item \textbf{\textipa{[\oe]}}
    \item \textbf{\textipa{[$\widetilde\emptyset$:]}}
    \item \textbf{\textipa{[Y:]}}
    \item \textbf{\textipa{[\~E]}}
    \item \textbf{\textipa{[Y]}}
  \end{enumerate}

\begin{problem}{5}
  When we say that distinct allophones of a phoneme are in complementary distribution, what does the word ``complementary'' mean? When we say that vowel shortening in closed syllables is a compensatory process, what does ``compensatory'' mean?
\end{problem}

\begin{solution}
  \bfseries
  Complimentary here means that the distinct allophones never occur in the same phonetic environments and are mutually exclusive. Compensatory on the other hand means that phonetic features balance each other to maintain a level of phonological equilibrium or ease of articulation. For example, vowel shortening is compensatory since it is response to the increased difficulty of articulation following a consonant cluster.
\end{solution}

\begin{problem}{6}
  What is the wavelength (in ms) of an 80 Hz tone? What frequency (in Hz) corresponds to a wavelength of 15 ms?
\end{problem}

\begin{solution}
  \bfseries
  The wavelength is $\mathbf{1\textrm{ s} / 80 = 1000 \textrm{ ms} / 80 = 12.5\textrm{ ms}}$. 

  Next, the frequency is $\mathbf{1/(15\textrm{ ms} / 1000 \textrm{ms}) \approx 66.6\textrm{ Hz}}$.
\end{solution}

\begin{problem}{7}
    Just noticeable difference.
\end{problem}

\begin{solution}
  \bfseries
  Q1: Lower

  Q2: Lower

  Q3: Lower

  Q4: Lower

  Q5: Lower

  Q6: Higher 

  Q7: Higher 

  Q8: Higher 

  Q9: Higher 

  Q10: Higher 
\end{solution}

\begin{problem}{8}
  Two tones are played at 100 decibels. One is 100 Hz and one is 1000 Hz. Which sounds louder?
\end{problem}

\begin{solution}
  \bfseries
  1000 Hz.
\end{solution}

\begin{problem}{9}
  If a 400 Hz tone and a 600 Hz tone are played simultaneously, what fundamental frequency do humans perceive?
\end{problem}

\begin{solution}
  \bfseries
  The perceived fundamental frequency is $\textrm{gcd}\mathbf{(400,600)=200}$ Hz.
\end{solution}

\begin{problem}{10}
  Thai has five tones (low, mid, high, falling, rising), but the full range of tonal contrasts is only possible on syllables containing a long vowel or a short vowel followed by a voiced consonant. If a syllable contains a short vowel followed by a voiceless consonant -- traditionally termed a ``checked'' or ``dead'' syllable -- only two tones (high and low) are possible. Why might fewer tones be licensed on checked syllables?
\end{problem}

\begin{solution}
  \bfseries
  First of all, the shorted duration of checked syllables limits the time available for expressing rising or falling tones. Also, voiceless consonants following a checked syllable prevents sustained vocal fold vibration which is essential for maintaining distinctions between the tones.
\end{solution}
\end{document}
