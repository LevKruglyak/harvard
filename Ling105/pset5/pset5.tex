\documentclass{../../templates/lkx_pset}

\title{Ling 105 Problem Set 5}
\author{Lev Kruglyak}
\due{October 11, 2024}

\input{../ling105.sty}

\begin{document}

\maketitle

\begin{problem}{1}
\end{problem}
\begin{solution}
	The spectrograms in order depict \textipa{[i]}, \textipa{[u]}, and \textipa{[A]}.
\end{solution}

\begin{problem}{2}
\end{problem}
\begin{solution}
	The spectrograms in order depict \textipa{[k\ae]}, \textipa{[t\ae]}, and \textipa{[p\ae]}.
\end{solution}

\begin{problem}{3}
\end{problem}
\begin{solution}
	\textipa{[i]} involves a larger pharyngeal cavity than \textipa{[A]}, and this is associated with a lower $f_1$.
\end{solution}

\begin{problem}{4}
\end{problem}
\begin{solution}
	In order of $f_2$ from lowest to highest, the vowels are \textipa{[u]}, \textipa{[A]}, \textipa{[\ae]}, \textipa{[E]}, \textipa{[i]}.
\end{solution}

\begin{problem}{5}
\end{problem}
\begin{solution}
	The airstream mechanisms and directions are:
	\begin{enumerate}[(a)]
		\item pulmonic egressive
		\item velaric ingressive
		\item glottalic egressive
		\item pulmonic egressive 
		\item glottalic ingressive
		\item pulmonic egressive
		\item pulmonic egressive
		\item velaric ingressive
		\item glottalic egressive
		\item glottalic ingressive
	\end{enumerate}
\end{solution}

\begin{problem}{6}
\end{problem}
\begin{solution}
	The first waveform shows a breathy voice, and the second shows a creaky voice. The glottis is more open for the breathy voice, since the creakyness comes from partial/complete closures of the glottis.
\end{solution}

\begin{problem}{7}
\end{problem}

\begin{solution}
	Ejective clicks are possible, since clicking is velaric and independent of the larynx. For example, \textipa{[k!']} is used in Xhosa.
\end{solution}

\begin{problem}{8}
\end{problem}

\begin{enumerate}[(a)]
	\item Voiced lateral fricatives are transcribed as ``dl''.
	\item Engmas are transcribed as ``ng''.
	\item Voiceless aspirated coronal stops are transcribed as ``th''.
	\item Devoiced aspirated alveolar clicks are transcribed as ``gq''.
	\item Voiceless velar fricatives are transcribed as ``rh''.
\end{enumerate}

\begin{problem}{9}
\end{problem}

\begin{solution}
  The minimum of the $f_0$ is approximately $85$ Hz, the vowel ends at a $f_0$ of $124$ Hz and starts at a $f_0$ at around $100$ Hz.
\end{solution}

\begin{problem}{10}
\end{problem}

\begin{enumerate}[(a)]
	\item Voiced bilabial click is \textipa{[g\!o]}
	\item Uvular post-alveolar click is \textipa{[q!]}
	\item Uvular ejective palatal click is \textipa{[q\textdoublebarpipe']}
	\item Aspirated dental click is \textipa{[k|\super{h}]}
	\item Breathy-voiced nasal dental click is \textipa{[\"N|]}
\end{enumerate}

\end{document}
