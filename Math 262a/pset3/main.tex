\documentclass{pset}

\title{Math 262a Problem Set 3}
\author{Lev Kruglyak}
\due{September 29, 2023}

\providecommand{\VF}{\mathfrak{X}}
\providecommand{\E}{\mathbb{E}}
\providecommand{\Symp}{\mathrm{Symp}}
\providecommand{\Spin}{\mathrm{Spin}}
\providecommand{\Pin}{\mathrm{Pin}}
\providecommand{\gr}{\mathrm{gr}}
\providecommand{\Sym}{\mathrm{Sym}}
\providecommand{\sgn}{\mathrm{sgn}}
\providecommand{\Ext}{\mathrm{Ext}}
\providecommand{\U}{\mathrm{U}}
\providecommand{\O}{\mathrm{O}}
\providecommand{\lie}{\mathcal{L}}
\providecommand{\con}{\iota}
\renewcommand{\H}{\mathcal{H}}
\providecommand{\Cl}{\mathrm{Cl}}
\providecommand{\curl}{\nabla\times}
\renewcommand{\div}{\nabla\cdot}
\providecommand{\grad}{\nabla}
\renewcommand{\bigwedge}{\scalebox{1.5}{$\wedge$}}

\providecommand{\SL}{\mathrm{SL}}
\renewcommand{\O}{\mathrm{O}}
\providecommand{\SO}{\mathrm{SO}}
\providecommand{\Sp}{\mathrm{Sp}}
\renewcommand{\sl}{\mathfrak{sl}}
\providecommand{\so}{\mathfrak{so}}
\renewcommand{\sp}{\mathfrak{sp}}
\providecommand{\gl}{\mathfrak{gl}}


\theoremstyle{plain}
\newtheorem*{regtheorem}{Theorem}

\begin{document}
\maketitle

\begin{problem}{1}
  Consider a classical particle moving on the Euclidean line $\E^1$ with standard metric and coordinate $x$. Fix $a>0$ and set 
  \[V(x) = \frac{1}{2}(x^2-a^2)^2.\]
\end{problem}

\begin{parts}
  \begin{part}{a}
    What are the classical trajectories of least energy?
  \end{part}

  Solving $\ddot{x} = V'(x)$ is tricky since it's a nonlinear second order differential equation. Thus, we must approximate solutions. The potential has two approximately quadratic wells at $x=\pm a$. If the particle has initial velocity small enough and starts in one of these wells, it will behave like a harmonic oscillator, otherwise it will go back and forth between the two wells.

  \begin{part}{b}
    What about for a classical particle moving in \emph{inverted} potential $-V$?
  \end{part}

  In the inverted double well potential, there is only one approximately quadratic well at $x=0$, and the potential goes down to $-\infty$ outside the region $(-a,a)$. Thus, the system behaves like a harmonic oscillator when the initial position and velocity are small enough, otherwise all particles accelerate towards $\pm\infty$.

  \begin{part}{c}
    For $T > 0$ is there a classical trajectory $\gamma : [- T /2, T /2] \to \E^1$ with initial position $\gamma(-T / 2) = -a$ and final position $\gamma(T /2)=a$, and if so give a formula or characterize it.
  \end{part}

  It can be shown that the minimum velocity at $-T/2$ is $3a^4/2$, otherwise the path $\gamma$ will never pass $x=a$. It's possible to solve for $T$ such that $\gamma(-T/2) = -a$, $\gamma'(-T/2)=3a^4/2$, and $\gamma(T/2)=a$. This path will then be $2T$-periodic since the potential is symmetric. If we want a higher $T$, we cannot slow down the particle since it is already in the ``ground energy'' state, so we must increase its initial speed by some factor so that it frequency decreases.

  \begin{part}{d}
    What happens if $T\to \infty$?
  \end{part}

  As $T\to \infty$, the particle's motion becomes increasingly faster and its range also grows.
\end{parts}

\begin{problem}{2}
  Let $\H$ be a $2$-dimensional complex inner product space. Consider the function $p :\mathbb{P}\H \times \mathbb{P}\H \to [0,1]$ given by $p(L, K) = |\langle \psi, \xi\rangle|^2$, where $\psi\in L$ and $\xi\in K$ are nonzero.
\end{problem}

\begin{parts}
  \begin{part}{a}
    Prove that $p=\cos^2(d/2)$, where $d$ is the geodesic distance on the $2$-sphere of unit radius.
  \end{part}

  Recall that there is a map $\mathbb{P}\mathcal{H} \cong \CP^1 \mapsto S^2$ which is a degree $2$ covering map. This is where $d/2$ comes from, since $d$ is just the angle between two points on a sphere. Then $|\langle \psi, \xi\rangle|$ is $\cos(d/2)$ so $p(L,K) = |\langle \psi, \xi \rangle|^2 = \cos^2(d/2)$.
\end{parts}

\begin{problem}{3}
  This concerns the particle on a ring $M = \E^1 / 2\pi L\Z$ of length $2\pi L$, both classical and quantum.
\end{problem}

\begin{parts}
  \begin{part}{a}
    Let $m>0$ be the mass of the particle. We define a $1$-parameter family of theories parametrized by $\theta\in \R$. The Lagrangian is
    \[
      L = \left(\frac{m}{2}\dot{x}^2+\frac{\theta}{2\pi L}\dot{x}\right)\;dt
    \]
    The first term is the usual kinetic energy; the potential energy vanishes. The second term is locally exact, but not globally exact, since there is no global function $x$ on the circle. Identify the space $N$ of classical solutions, and show that:
    \begin{itemize}
      \item The derived Hamiltonian system is independent of $\theta$.
      \item The principal $\R$-bundle over $N$ with connection changes with $\theta$.
    \end{itemize}
  \end{part}

  First, let's use the Euler-Lagrange equations to identify the space $N$. Let $\psi : \R \to M$ be a path on $M$. Since $\mathcal{L}(x,\dot{x}, t) = m\dot{x}^2/2+\theta \dot{x}/2\pi L$, we have the minimization problem
  \[\frac{\partial \mathcal{L}}{\partial x}-\frac{d}{dt}\left(\frac{\partial \mathcal{L}}{\partial \dot{x}}\right) = \frac{d}{dt}\left(m\dot{x}+\frac{\theta}{2\pi L}\right) =m\ddot{x}= 0.\]
  This means that $N$ is the space of constant velocity motions on the ring of length $L$. As usual, a path is uniquely and smoothly determined by its initial position and velocity, so topologically, $N = TM = S^1\times \R$ is a cylinder. Notice that there is no dependence on $\theta$ here.

  Now, let's derive the connection and symplectic form. Recall that $L\in \Omega^{0,1}(N\times \R)$, so we calculate $\delta L\in \Omega^{1,1}_N$:
  \[\delta L = \left(m\dot{x}\;\delta \dot{x} + \theta / 2\pi \right)\wedge dt = -(m\dot{x} + \theta t/2\pi)\;dt\wedge \delta\dot{x}.\]
  Then, the connection $\gamma\in \Omega^{1,0}(N\times \R)$ satisfies $\delta L+d\gamma = 0$, so we see that
  \[\gamma = (mx+\theta t/2\pi)\;\delta\dot{x}.\]
  Furthermore, the symplectic form is $\delta \gamma = m\;\delta x\wedge \delta\dot{x}$, which is independent of $\theta$. Thus, the whole Hamiltonian system is independent of $\theta$ as well. However, for any choice of time $t$, the connection $\gamma(t)$ \emph{does} change with $\theta$.

  \begin{part}{b}
    The quantum theory has Hilbert space $\H = L^2(M; \C)$ and the Hamiltonian is the unbounded self-adjoint operator
    \[
      H = \frac{1}{2}\left(i\frac{\partial}{\partial x} + \frac{\theta}{2\pi L}\right)^2.
    \]
    Diagonalize $H$, and in particular compute its spectrum. Is there periodicity in $\theta$? How does this compare to the classical system?
  \end{part}
  First, we write $H=T^2$, where 
  \[
    T = \frac{1}{\sqrt{2}}\left(i\frac{\partial}{\partial x} + \frac{\theta}{2\pi L}\right)
  \]
  Note that for any $\psi_n = e^{inx/L}$, we have
  \[
    T\psi_n = \frac{1}{\sqrt{2}}\left(\frac{\theta}{2\pi L} -\frac{n}{L}\right) \implies H\psi_n = \frac{1}{2}\left(\frac{\theta}{2\pi L} - \frac{n}{L}\right)^2\psi_n.
  \]
  Since $\psi_n$ must be a function from $M$, we get the condition $\psi_n(t) = \psi_n(t+2\pi L)$. Thus, $n\in \Z$. So our diagonalization becomes
  \[
    H \xi = \frac{1}{2}\sum_{n \in \Z} \left(\frac{\theta}{2\pi L} - \frac{n}{L}\right)^2 \langle \xi, \psi_n\rangle \cdot \psi_n
  \]
  There are a few cases for the spectrum of $H$. 

  \begin{itemize}
    \item \textbf{\underline{Case $\theta\in 2\pi L\Z$:}} We have an eigenvalue $0$ with multiplicity $1$, and eigenvalues $n^2/2$ for all $n\in \Z_{>0}$ with multiplicity $2$. Thus, the system is \emph{gapless}.

    \item \textbf{\underline{Case $\theta\in \pi + 2\pi L\Z$:}} We have eigenvalues $(2n+1)^2/4$ for all $n\in \Z_{>0}$ with multiplicity $2$. This is a \emph{gapped} system.

    \item \textbf{\underline{Other cases:}} Eigenvalues are all discrete, have multiplicities $1$, and the system is gapped. Furthermore, they are periodic with $\theta \mapsto \theta + 2\pi L$.
  \end{itemize}
  
  There are a few big differences with the classical case. Firstly, this system is quantized, and this quantization depends on $\theta$ periodicially. In the quantum case, the dynamics and allowed energies of the system are not affected by $\theta$, certainly not periodically.
\end{parts}

\begin{problem}{4}
  Let $\H$ be a separable complex Hilbert space. A continuous linear operator $T : \H \to \H$ is \emph{compact} if the closure of the image $T(D)\subset \H$ of the closed unit disk $D\subset \H$ is compact. Prove the following:
  \begin{regtheorem}
    Let $T : \H \to \H$ be a positive self-adjoint compact operator. Prove that there exists a complete orthonormal basis $\{\psi_n\}^\infty_{n=1}$ and complex numbers $\mu_1 \geq \mu_2\geq\cdots$ such that:
    \begin{enumerate}[(i)]
      \item $T\psi_n = \mu_n \psi_n$.
      \item For any $c>0$ there is a finite number of $\mu_n > c$. 
      \item $\lim_{n\to \infty} \mu_n = 0$.
    \end{enumerate}
  \end{regtheorem}
  % The completeness means that the closure of the span of the set $\{e_n\}$ equals $\H$. One approach is to consider the quadratic form $p |-> <p, Tp>$ on D. Since T is compact, this achieves max on e_1. Show that T preserves the orthogonal complomenet Ce_1 and iterate
\end{problem}

\end{document}
