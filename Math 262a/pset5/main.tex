\documentclass{pset}

\title{Math 262a Problem Set 5}
\author{Lev Kruglyak}
\due{October 6, 2023}

\providecommand{\VF}{\mathfrak{X}}
\providecommand{\E}{\mathbb{E}}
\providecommand{\Symp}{\mathrm{Symp}}
\providecommand{\Spin}{\mathrm{Spin}}
\providecommand{\Pin}{\mathrm{Pin}}
\providecommand{\gr}{\mathrm{gr}}
\providecommand{\Sym}{\mathrm{Sym}}
\providecommand{\sgn}{\mathrm{sgn}}
\providecommand{\Ext}{\mathrm{Ext}}
\providecommand{\U}{\mathrm{U}}
\providecommand{\O}{\mathrm{O}}
\providecommand{\lie}{\mathcal{L}}
\providecommand{\con}{\iota}
\renewcommand{\H}{\mathcal{H}}
\providecommand{\Cl}{\mathrm{Cl}}
\providecommand{\curl}{\nabla\times}
\renewcommand{\div}{\nabla\cdot}
\providecommand{\grad}{\nabla}
\renewcommand{\bigwedge}{\scalebox{1.5}{$\wedge$}}

\providecommand{\SL}{\mathrm{SL}}
\renewcommand{\O}{\mathrm{O}}
\providecommand{\SO}{\mathrm{SO}}
\providecommand{\Sp}{\mathrm{Sp}}
\renewcommand{\sl}{\mathfrak{sl}}
\providecommand{\so}{\mathfrak{so}}
\renewcommand{\sp}{\mathfrak{sp}}
\providecommand{\gl}{\mathfrak{gl}}


\begin{document}
\maketitle

% \begin{problem}{1}
%   Recall the $2$-component Lie group $PQ$ constructed in lecture. Given a right $PQ$-torsor and a Hamiltonian $H$, construct the data of a quantum mechanical system. Can you generalize to a parametrized family of data? To a piece of data with symmetry? Can you combine these?
% \end{problem}
%
% \begin{solution}
%
% \end{solution}
%
\begin{problem}{2}
  For $n\in \Z^{\geq 0}$, the standard $\Cl_{+n}$ is the unital algebra over $\R$ generated $e_1,\ldots, e_n$ subject to the relation 
  \[e_i e_j + e_je_i = 2\delta_{ij},\quad 1\leq i, j \leq n.\]
  $\Cl_{-n}$ is similar, but the relation $e_ie_j + e_je_i = -2\delta_{ij}$. The complex version, for which the sign is irrelevant, but we take the plus sign for definiteness, is denoted $\Cl_n^\C$. The abstract version for a vector space $U$ with a symmetric bilinear form $B$ is the free unital algebra $\Cl(U, B)$ generated by $U$ subject to 
  \[u_1u_2 + u_2u_1 = B(u_1, u_2), \quad u_1, u_2\in U.\]
  These Clifford algebras are superalgebras.
\end{problem}

\begin{parts}
  \begin{part}{a}
    Construct an isomorphism of superalgebras $\Cl^\C_2 \to \End(\C^{1|1})$.
  \end{part}
  
  \begin{part}{b}
    Define the tensor product $A_1\otimes A_2$ of superalgebras $A_1, A_2$. Mind your Koszul sign rule!
  \end{part}
\end{parts}

\begin{problem}{3}
  For computations in superalgebra, say over $\R$, it is wise to extend scalrs from $\R$ to the commutative superalgebra $\R[\eta_1,\ldots, \eta_N]$ for arbirary $N$. Multiply odd elements by a variable $\eta_i$ to obtain an even element, and then use the even elements to compute, thereby eliminating signs until the end of the computation. For example, if $Q_1$ and $Q_2$ are odd elements of a super Lie algebra, then compute the sign rule for the Lie bracket as follows:
  \[
    [\eta_1 Q_1, \eta_2 Q_2] = -[\eta_2 Q_2, \eta_1 Q_1] = +\eta_2\eta_1 [Q_2, Q_1] = -\eta_1\eta_2[Q_2, Q_1].
  \]
  On the other hand,
  \[
    [\eta_1 Q_1, \eta_2 Q_2] = -\eta_1\eta_2[Q_1, Q_2],
  \]
  and by comparing we deduce the correct rule for the Lie bracket:
  \[
    [Q_1, Q_2] = [Q_2, Q_1].
  \]
\end{problem}

\begin{parts}
  \begin{part}{a}
    Revisit part (b) of the previous problem and apply these ``even rules''.
  \end{part}

  \begin{part}{b}
    Deduce the proper signs for commutativity in the symmetric algebra $\Sym^\bullet(V)$ of a super vector space $V$.
  \end{part}

  \begin{part}{c}
    Deduce the Jacobi identity for a super $\mathfrak{g} = \mathfrak{g}^0\oplus \mathfrak{g}^1$. Verify that
    \begin{enumerate}[(i)]
      \item $\mathfrak{g}^0$ is an (ungraded) Lie algebra,
      \item $\mathfrak{g}^1$ is a $\mathfrak{g}^0$-module,
      \item the Lie bracket induces a symmetric pairing $\mathfrak{g}^1\times \mathfrak{g}^1 \to \mathfrak{g}^0$ of $\mathfrak{g}^0$-modules, and
      \item for $x\in \mathfrak{g}^1$ we have $[x,[x,x]] = 0$.
    \end{enumerate}
    Conversely, show that the data (i)-(iv) determines a super Lie algebra.
  \end{part}

  \begin{part}{d}
    Define the opposite superalgebra to a superalgebra. What is the opposite superalgebra to $\Cl(U, B)$?
  \end{part}
\end{parts}

\begin{problem}{4}
  This problem identifies the double cover of low dimensional orthogonal groups. Let $V$ be a complex vector space.
\end{problem}

\begin{solution}
  Recall that the $k$-th exterior power $\bigwedge^k V$ can be defined as the subspace
  \[
    \bigwedge^k V = \left\{ \frac{1}{|S_k|}\sum_{\sigma\in S_k} \sgn(\sigma)\cdot v_{\sigma(1)}\otimes \cdots \otimes v_{\sigma(k)} : \{v_1,\ldots, v_k\} \subset V\right\} \subset V^{\otimes k}.
  \]
  We have a canonical duality isomorphism $\Phi : (V^*)^{\otimes k} \to \left(V^{\otimes k}\right)^*$ which sends $\xi^1\otimes \cdots \otimes \xi^k$ to the map sending $v_1\otimes \cdots \otimes v_k \mapsto \xi^1(v_1)\cdots\xi^k(v_k)$. Since the exterior power is a subspace of the tensor power, this restricts to a map $\Phi : \bigwedge^k V^* \to (\bigwedge^k V)^*$. This gives us a bilinear form
  \[
    \langle\cdot,\cdot\rangle :\bigwedge^k V^* \otimes \bigwedge^k V \to \C
  \]
  which sends $v\otimes w \mapsto \Phi(v)(w)$.

  Now for the rest of the problem, suppose $V$ is a $4$-dimensional vector space, $\mu\in \bigwedge^4 V^*$ is a volume form, and consider the bilinear form $b$ on $\bigwedge^2 V$ given by $b(\alpha,\beta) = \langle \mu, \alpha\wedge \beta\rangle$.

  \begin{part}{a}
    Choose a basis $\{e_1, e_2, e_3, e_4\}$ of $V$ and induced basis $\{e_i \wedge e_j : i < j\}$ of $\bigwedge^2 V$. Then choose $\mu = e^1\wedge e^2\wedge e^3\wedge e^4$, where $\{e^1,e^2,e^3,e^4\}$ is the dual basis. Show that $b$ is nondegenerate.
  \end{part}

  Let's order the basis for $\bigwedge^2 V$ by $\{e_1\wedge e_2, e_1\wedge e_3, e_1\wedge e_4, e_2\wedge e_3, e_2\wedge e_4, e_3\wedge e_4\}$. Then the matrix $B$ corresponding to the bilinear form $b$ is anti-diagonal, with nonzero signature. Then $B$ has determinant $1$, and the form is nondegenerate and symmetric. 

  \begin{part}{b}
    Define the homomorphism
    \[
      \pi : \Aut(V, \mu) \to \Aut\left(\bigwedge^2 V, b\right)
    \]
    which maps a volume-preserving automorphism of $V$ to a bilinear form-preserving automorphism of $\bigwedge^2 V$. Write the corresponding map of Lie algebras. Prove that the latter is an isomorphism.
  \end{part}

  Let $f \in \Aut(V)$ be an automorphism. Let's define $\pi(f)(v\wedge w) = f(v)\wedge f(w)$ and extend linearly. We need to show that if $f$ preserves $\mu$ then $\pi(f)$ preserves the bilinear form. $f$ preserves $\eta$ exactly if 
  \[
    (f^*\Phi(\mu))(v_1\wedge\ldots\wedge v_4) = \Phi(\mu)(f(v_1)\wedge\ldots\wedge f(v_4)) = \Phi(\mu)(v_1\wedge \ldots\wedge v_4).
  \]
  Note that here we are using $\Phi(\mu)$ to treat $\mu$ as a functional on $\bigwedge^4 V$. Now suppose $v_1\wedge v_2, v_3\wedge v_4 \in \bigwedge^2 V$ are some pure exterior powers. We need to show that $b(\pi(f)(v_1\wedge v_2), \pi(f)(v_3\wedge v_4)) = b(v_1\wedge v_2, v_3\wedge v_4)$. Expanding, we see that
  \[
    \begin{aligned}
      b(\pi(f)(v_1\wedge v_2), \pi(f)(v_3\wedge v_4)) &= \langle \mu, f(v_1)\wedge f(v_2)\wedge f(v_3)\wedge f(v_4)\rangle\\
                                                      &= \Phi(\mu)(f(v_1)\wedge f(v_2)\wedge f(v_3)\wedge f(v_4))\\
                                                      &= \Phi(\mu)(v_1\wedge v_2\wedge v_3 \wedge v_4)\\
                                                      &= b(v_1\wedge v_2, v_3\wedge v_4).
    \end{aligned}
  \]
  Now let's look at the corresponding map of Lie algebras. 
  Note that $\Aut(V,\mu) \cong \SL_4(\C)$ as Lie groups and $\Aut(\bigwedge^2 V, b)\cong \O_6(\C)$. Thus, the associated Lie algebra to $\Aut(V, \mu)$ is $\sl_4(\C)$ and the associated Lie algebra to $\Aut(\bigwedge^2 V, b)$ is $\so_6(\C)$. Let's call the associated map of Lie algebras
  \[
    \widetilde{\pi} : \sl_4(\C) \to \so_6(\C).
  \]
  Since both of these Lie algebras are $15$-dimensional, $\widetilde{\pi}$ is an isomorphism if it is injective. To show that $\widetilde{\pi}$ is injective, suppose $X\in \sl_4(\C)$. Then $\widetilde{\pi}(X)(v\wedge w) = Xv\wedge w + v\wedge Xw$. If $\widetilde{\pi}(X)=0$, this means that $X(v)\wedge w=X(w)\wedge v$ for all $v,w\in V$. Clearly, $X$ must be zero, so $\widetilde{\pi}$ is injective.

  \begin{part}{c}
    Deduce that the image of $\pi$ is open and a subgroup, where $\pi$ maps onto the identity component of $\Aut(\bigwedge^2 V, b)$. Show that the latter group has two components.
  \end{part}

  We know that $\widetilde{\pi} : T_1 \SL_4(\C) \to T_1 \O_6(\C)$ is an isomorphism, so by the inverse function theorem the image of $\pi$ should be an open neighborhood of the identity, and a subgroup because it is the image of a Lie group homomorphism. Indeed, this latter group has two components, $\pm \SO_6(\C)$ and $\pi$ maps into the identity component.

  \begin{part}{d}
    Prove that the kernel of $\pi$ is $\{\pm 1\}$, so that $\pi$ is a 2:1 covering. Deduce that $\SL_4(\C)$ is a double cover of $\SO_6(\C)$.
  \end{part}

  If we restrict $\pi$ to its image, it sends $\SL_4(\C)$ to $\SO_6(\C)$, both of which are now connected. Since $\widetilde{\pi}$ is an isomorphism, it follows that $\pi$ is a covering map of degree $|\ker \pi|$. Now suppose that $\pi(f)$ acts by $1$ on $\bigwedge^2 V$. This means that $f(v)\wedge f(w) = v\wedge w$ for all $v,w\in \bigwedge^2 V$. The only way this can happen is if $f=\pm 1$, since $(-v)\wedge (-w)=v\wedge w$. Thus, $\SL_4(\C)$ is a double cover of $\SO_6(\C)$.

  \begin{part}{e}
    Now choose $\omega\in \bigwedge^2 V^*$ with $(\omega\wedge \omega)/2 = \eta$. For example take $\omega=e^1\wedge e^2 + e^3\wedge e^4$. Let $W\subset \bigwedge^2 V$ be the annihilator of $\omega$. Then use the restriction of the map $\pi$ above to define
    \[
      \pi : \Aut(V, \omega) \to \Aut(W, b).
    \]
    As before, prove that $\pi$ is a 2:1 covering map. Deduce that $\Sp_4(\C)$ is a double cover of $\SO_5(\C)$.
  \end{part}

  \begin{part}{f}
    Write $V=U_1\oplus U_2$ as the direct sum of $2$-dimensional subspaces. Choose nonzero $\omega_1\in \bigwedge^2 U^*_1$ and $\omega_2\in \bigwedge^2 U^*_2$. Construct a decomposition \[\bigwedge^2 V^* \cong \bigwedge^2 U^*_1\oplus \bigwedge^2 U_2^*\oplus (U_1^*\otimes U_2^*).\]
    Then, following the ideas in previous parts, construct a 2:1 covering
    \[
    \pi : \Aut(U_1, \omega_1)\times \Aut(U_2,\omega_2) \to \Aut(U_1^*\otimes U_2^*, b),
    \]
    where we use $\eta = \omega_1\wedge \omega_2$ in the definition of $b$. Deduce that $\SL_2(\C)\times \SL_2(\C)$ is a double cover of $\SO_4(\C)$.
  \end{part}

  % Let's prove the decomposition of $\bigwedge^2 V^*$ at the level of pure tensors. Suppose $v\wedge w\in \bigwedge^2 V^*$. Note that $v=v_1+v_2$ and $w=w_1+w_2$ where $v_1, w_1\in U_1^*$ and $v_2, w_2\in U_2^*$. Then,
  % \[
  %   v\wedge w = v_1\wedge w_1 + v_2\wedge w_1 + v_1\wedge w_2 + v_2\wedge w_2 = \underbrace{v_1\wedge w_1}_{\wedge^2 U_1^*} + \underbrace{v_2\wedge w_2}_{\wedge^2 U_2^*} + \underbrace{v_1\wedge w_2 - w_1\wedge v_2}_{U_1^*\otimes U_2^*}
  % \]

  \begin{part}{g}
    Construct a double cover $\pi : \SL_2(\C) \to \SO_3(\C)$.
  \end{part}

  \begin{part}{h}
    From the beginning, introduce a hermitian inner product on $V$ and restrict to the subgroups that preserve this inner product. What do you learn?
  \end{part}
\end{solution}

\end{document}
