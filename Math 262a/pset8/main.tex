\documentclass{pset}

\title{Math 262a Problem Set 8}
\author{Lev Kruglyak}
\due{November 10, 2023}

\providecommand{\VF}{\mathfrak{X}}
\providecommand{\E}{\mathbb{E}}
\providecommand{\Symp}{\mathrm{Symp}}
\providecommand{\Spin}{\mathrm{Spin}}
\providecommand{\Pin}{\mathrm{Pin}}
\providecommand{\gr}{\mathrm{gr}}
\providecommand{\Sym}{\mathrm{Sym}}
\providecommand{\sgn}{\mathrm{sgn}}
\providecommand{\Ext}{\mathrm{Ext}}
\providecommand{\U}{\mathrm{U}}
\providecommand{\O}{\mathrm{O}}
\providecommand{\lie}{\mathcal{L}}
\providecommand{\con}{\iota}
\renewcommand{\H}{\mathcal{H}}
\providecommand{\Cl}{\mathrm{Cl}}
\providecommand{\curl}{\nabla\times}
\renewcommand{\div}{\nabla\cdot}
\providecommand{\grad}{\nabla}
\renewcommand{\bigwedge}{\scalebox{1.5}{$\wedge$}}

\providecommand{\SL}{\mathrm{SL}}
\renewcommand{\O}{\mathrm{O}}
\providecommand{\SO}{\mathrm{SO}}
\providecommand{\Sp}{\mathrm{Sp}}
\renewcommand{\sl}{\mathfrak{sl}}
\providecommand{\so}{\mathfrak{so}}
\renewcommand{\sp}{\mathfrak{sp}}
\providecommand{\gl}{\mathfrak{gl}}


\begin{document}
\maketitle

\begin{problem}{1}[Representations of $\SL_2(\R)$]
\end{problem}

\begin{parts}
  \begin{part}{a}
    Construct a finite dimensional representation of the Lie group $\SL_2(\R)$.
  \end{part}

  The Lie group $\SL_2(\R)$ can be thought of as the space of linear transformations on $\R^2$ with determinant $1$, so there is a natural irreducible representation on $\R^2$.

  \begin{part}{b}
    Prove that a \emph{unitary} irreducible representation of $\SL_2(\R)$ is either the trivial representation or is infinite dimensional.
  \end{part}

  Let's prove this without any advanced machinery. Consider the matrix $A(t)\in \SL_2(\R)$ given by
  \[
    A(t) = \begin{pmatrix}1&t\\ 0 & 1\end{pmatrix}
  \]
  for all $t\in \R$. This has the nice property that $A(\alpha\cdot t) = A(t)^\alpha$ for integral $\alpha$. Another nice family of matrices $B(t)\in \SL_2(\R)$ is given by 
  \[
    B(t) = \begin{pmatrix}t&0\\0&1/t\end{pmatrix}
  \]
  for nonzero $t\in \R$. These families have the following nice commutation relation:
  \[
    B(\sqrt{m})A(t)B(\sqrt{m})^{-1} = A(m^2\cdot t) = A(t)^{m}
  \]
  for any $m\in \Z$. Now suppose we had a finite dimensional unitary representation $\varphi : \SL_2(\R) \to \U(n)$ for some $n$. The commutation relation of $A$ and $B$ thus implies that
  \[
    \varphi(B(\sqrt{m}))\varphi(A(t))\varphi(B(\sqrt{m}))^{-1} = \varphi(A(t))^{m}
  \]
  This means that $\varphi(A(t))$ is similar to $\varphi(A(t))^m$, and thus has the same eigenvalues. Letting $\{\lambda_1,\ldots, \lambda_n\}$ be the eigenvalues of $\varphi(A(t))$, it follows that $\{\lambda_1^m,\ldots, \lambda_n^m\}$ is some permutation of these eigenvalues. Since $m$ is arbitrary, this shows that $\varphi(A(t))$ must be the identity matrix.

  Since the normal subgroup generated by $\{A(t) : t\in \R \}$ is the whole of $\SL_2(\R)$, it follows that $\varphi$ must send every element to the identity and thus is the trivial representation. This same trick does not work in infinite dimensions, since the permutation of the eigenvalues of an infinite matrix can go on forever.

  \begin{part}{c}
    Conclude the same for a representation of the Lorentz group $\Spin_{1, n-1}$ for $n\geq 3$.
  \end{part}

  We can prove this topologically. Suppose $\mu : \Spin_{1,n-1} \to U(m)$ be some non-trivial finite dimensional representation of the Lorentz group. The kernel of this map must be normal in $\Spin_{1,n-1}$, but since $\Spin_{1,n-1}$ is a simple Lie group, this kernel must be trivial. Thus, $\mu$ is faithful and so a diffemomorphism onto its image. But $\Spin_{1,n-1}$ is not compact for $n\geq 3$, yet $U(m)$ is compact, a contradiction.
\end{parts}

\begin{problem}{2}
  A spin group $\Spin_{p,q}$ has a \emph{vector representation} on $V=\R^{p,q}$ that acts via the quotient $SO_{p,q}$. Let $S$ be a \emph{real} spin representation, which in dimensions $3-6$ are the real representations underlying the defining representation. Is there a $\Spin_{p,q}$-invariant nonzero symmetric bilinear map
  \[
    \Gamma : S \times S \to V.
  \]
\end{problem}
\end{document}
