\documentclass[11pt,letterpaper]{article}

\input{../../../../.config/latex/preamble_v1.tex}

\lightmode

\title{\textbf{Math 114 Problem Set 1}}

\begin{document}
\maketitle

\begin{problem}
    The Cantor set can also be described in terms of ternary expansions:
    \begin{enumerate}[(a)]
        \item Every number in $[0,1]$ has the ternary expansion
        \[
            x = \sum^\infty_{k=1}a_k3^{-k}, \quad \textrm{ where } a_k=0,1,\textrm{ or }2
        .\] 
        Prove that $x\in \mathcal{C}$ if and only if $x$ has a representation as above with $a_k=0$ or $2$. 
        \item The \emph{Cantor-Lebesgue function} is defined on $\mathcal{C}$ by
        \[
            F(x)=\sum^\infty_{k=1}\frac{b_k}{2^k}\quad\textrm{if }x=\sum^\infty_{k=1}a_k3^{-k}, \textrm{ where }b_k=a_k /2
        .\] 
        Here we choose the expansion from (a).

        Show that $F$ is well defined and continuous on $\mathcal{C}$, and moreover $F(0)=0$ and $F(1)=1$.
        \item Prove that $F: \mathcal{C} \to [0,1]$ is surjective.
        \item Extend $F$ to a continuous function on $[0,1]$.
    \end{enumerate}
\end{problem}

\begin{solution}
    \textbf{(a)} Let's define the Cantor set recursively as follows. Define $\mathcal{C} = \bigcap_{k\geq 0} C_k$ where
    \[
        C_k = \begin{cases}
            [0,1] & k = 0\\
            C_{k-1}/3 + (C_{k-1}+2)/3 & k > 0
        \end{cases}
    .\]
    For any $x\in \mathcal{C}$, define $a_k$ to be $0$ if $x\in C_{k-1} /3$ and $2$ if $x\in (C_{k-1}+2) /3$. Note that by induction, since $\sum^N_{k=1}a_k /3^k\in C_N$ and $x\in C_N$, we have  
    \[
        \left|x - \sum^N_{k=1}a_k /3^k\right| < \frac{1}{3^{N+1}} \implies x = \sum^\infty_{k=1}a_k /3^k
    .\] 
    Clearly, such a representation is unique since the sets $C_k/3$ and $(C_k+2) /3$ are disjoint, and $C_{k+1}\subset C_{k}$. This proves the backward direction. For the forward direction, we'll prove the contrapositive. Suppose $x$ only has representations which have some $a_k=1$. Then $\sum^{k-1}_{i=1}a_i /3^i \in C_{k-1}$, yet $\sum^k_{i=1}a_i /3^i \in (C_{k-1}+1) /3 \not\subset \mathcal{C}$. This concludes the proof.  

    \textbf{(b)} First of all, this function is well defined by the uniqueness proof in (a). Similarly, its clear that $F(0)=0$ and $F(1)=1$ because $0 = 0 / 3^1 + 0 / 3^2 +\cdots$ and $1 = 2 / 3^1 + 2 / 3^3 + \cdots$. To prove continuity, first consider the metric space of binary sequences $\{0,1\}^\N$ with the metric 
    \[
        d_3(a,b) = \sum^\infty_{k=1}\frac{2|a_k - b_k|}{3^k}
    .\] 
    This can be easily checked to be a metric. Note that there is a canonical homeomorphism
    \[
        \mu : (\{0,1\}^\N, d_3) \to \mathcal{C}
    \]
    which sends $a$ to $\sum^\infty_{k=1}a_k /3^k$. (Here the metric $d_3$ coincides with the standard Euclidean metric on $\mathcal{C}$.) Open sets in $(\{0,1\}^\N, d_3)$ are generated by ``open intervals'' of the form $[a_1,\ldots,a_k, *, *, \cdots]$ for some fixed $[a_1,\ldots,a_k]$. Next, recall that we have the basis of $[0,1]$ given by intervals of the form $(\alpha-1 /2^N, \alpha+1 /2^N)$ where $\alpha=\sum^{N-1}_{k=1}\alpha_k /2^k$. The preimage of this under $F$ is $\mu([\alpha_1,\ldots,\alpha_{N-1},*, *, \cdots])$. This is exactly the open set $(\beta, \beta+2 / 3^N+\epsilon)\cap \mathcal{C}$ where $\beta=\sum^{N-1}_{k=1}2\alpha_k /3^k$ so $F$ is continuous.

    \textbf{(c)} This follows immediately from the fact that every number has at least one binary expansion.

    \textbf{(d)} Consider the function $\widetilde{F} : [0,1] \to [0,1]$ given by
    \[
        \widetilde{F}(x) = \begin{cases}
            F(x) & x\in \mathcal{C} \\
            \sup_{y\leq x, y\in \mathcal{C}} F(x) & \textrm{otherwise}
        \end{cases}
    .\]
    We claim this is continuous. To prove this, note that $F(x)$ is clearly a (non-strictly) monotonically increasing function, so by definition $\widetilde{F}$ is as well. In fact, $\widetilde{F}(x)$ must be the unique monotonically increasing function extending $F$. Since $\widetilde{F}(x)$ is surjective and increasing, it must be continuous.
\end{solution}

\begin{problem}{\textbf{(The Borel-Cantelli Lemma)}}
    Suppose $\{E_k\}^\infty_{k=1}$ is a countable family of measurable subsets of $\R^d$ and that
    \[
        \sum^\infty_{k=1} m(E_k) < \infty
    .\] 
Let 
\[
    \begin{aligned}
        E &= \{x\in \R^d : x\in E_k, \textrm{ for infinitely many }k\}\\
        &= \limsup_{k\to \infty}(E_k).
    \end{aligned}
\] 
\begin{enumerate}[(a)]
    \item Show that $E$ is measurable.
    \item Prove $m(E)=0$.
\end{enumerate}
\end{problem}

\begin{solution}
    \textbf{(a)} Observe that $E=\bigcap^\infty_{k=1}\bigcup_{n\geq k}E_n$. Since all of the $E_k$ were assumed measurable, it follows that $E$ is as well since it is a countable union/intersection of measurable sets.

    \textbf{(b)} We have
    \[
        m(E)=m\left(\bigcap^\infty_{k=1}\bigcup_{n\geq k}E_n\right)\leq m\left(\bigcup_{n\geq k}E_n\right)\leq \sum_{n\geq k}m(E_n)
    .\] 
    Since $\sum^\infty_{k=1}m(E_k)$ converges, it follows that this upper bound must approach zero as $k\to\infty$, and so $m(E)=0$. 
\end{solution}

\begin{problem}
    Let $\{f_n\}$ be a sequence of measurable functions on $[0,1]$ with $|f_n(x)|<\infty$ for a.e. $x$. Show that there exists a sequence $c_n$ of positive real numbers such that
    \[
        \frac{f_n(x)}{c_n} \to 0\quad\textrm{ a.e. }x
    .\]
\end{problem}

\begin{solution}
    First consider the function:
    \[
        \lambda_n(c) = m\left(f_n^{-1}\left(\overline{\R} \setminus \left[-c /n, c /n\right]\right)\right)\quad \forall c\in \overline{\R}
    .\] 
    Since $f_n$ are all measurable functions, this is well defined since $\overline{\R}\setminus [-c /n, c /n]$ is a Borel set. Note that $\lambda_n(0)\leq m([0,1])=1$. Next we claim that $\lim_{c\to \infty} \lambda(c)=0$. Note that $\{\pm\infty\}\subset \overline{\R}$ is a measurable set, and by assumption $m(f_n^{-1}(\{\pm\infty\}))=0$. Since $\overline{\R}=\{\pm\infty\}\cup\bigcup_c [-c /n, c /n]$, it follows that $\lim_{c\to\infty}m(f^{-1}_n([-c /n, c /n]))=1$. Then
    \[
        \lim_{c\to\infty} \lambda(c) = \lim_{c\to\infty} m(f^{-1}_n(\overline{\R}\setminus [-c /n, c /n])) = \lim_{c\to\infty}\left(1 - m(f^{-1}_n([-c /n, c /n]))\right)=0
    \]
    as desired. Now for each $n$, choose some $c_n$ such that $\lambda(c_n)<2^{-n}$. Let $E_k=f^{-1}_n(\overline{\R}\setminus [-c_n /n, c_n /n])$ so that $m(E_k)=\lambda(c_n)<2^{-n}$. Note that 
    \[
        \sum^\infty_{k=1}m(E_k)\leq \sum^\infty_{k=1}2^{-k}=1<\infty
    \]
    so we can apply the Borel-Cantelli lemma. Let $E=\limsup_{k\to \infty} E_k$. Then $x\in E$ if $|f_n(x) /c_n|>1 /n$ for infinitely many $n$ so $x\in [0,1]\setminus E$ if $|f_n(x) / c_n|\leq 1 /n$ for infinitely many $n$. This means that $f_n(x) /c_n \to 0$ for every point $x\in [0,1]\setminus E$. Since $m(E)=0$ by the Borel-Cantelli lemma we have $m([0,1]\setminus E)=1$, completing the proof. 
\end{solution}

\end{document}