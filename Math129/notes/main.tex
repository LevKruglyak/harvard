\documentclass[11pt,letterpaper]{article}

% Math stuff
\usepackage{amsmath, amsfonts, mathtools, amsthm, amssymb}
% Fancy script capitals
\usepackage{mathrsfs}
\usepackage{cancel}
% Bold math
\usepackage{bm}
\usepackage{pgfplots}
\pgfplotsset{compat=1.17}
\usepackage{tikz}
\usepackage{quiver}
% Geometry
\usepackage[letterpaper, portrait, margin=1.25in, includefoot]{geometry}


\providecommand{\bE}{\mathbf{E}}
\providecommand{\bB}{\mathbf{B}}
\providecommand{\bJ}{\mathbf{J}}
\providecommand{\bj}{\mathbf{j}}
\providecommand{\bff}{\mathbf{f}}
\providecommand{\VF}{\mathfrak{X}}

\providecommand{\R}{\mathbb{R}}
\providecommand{\C}{\mathbb{C}}
\providecommand{\Z}{\mathbb{Z}}
\providecommand{\RP}{\mathbb{RP}}
\providecommand{\Hom}{\mathrm{Hom}}
\providecommand{\CC}{\mathscr{C}}
\providecommand{\Eq}{\mathrm{Eq}}
\providecommand{\Coeq}{\mathrm{Coeq}}
\providecommand{\hCW}{\mathbf{hCW}}
\providecommand{\Set}{\mathbf{Set}}
\providecommand{\colim}{\mathrm{colim}}
\providecommand{\Th}{\mathrm{Th}}

\newcommand\defn[1]{\textbf{#1}}
\newcommand\todo[1]{{\color{red}\textbf{#1}}}

\theoremstyle{definition}
\newtheorem{definition}{Definition}[subsection]
\newtheorem{theorem}[definition]{Theorem}
\newtheorem{remark}[definition]{Remark}
\newtheorem{proposition}[definition]{Proposition}
\newtheorem{claim}[definition]{Claim}
\newtheorem{lemma}[definition]{Lemma}
\newtheorem{example}[definition]{Example}
\newtheorem{corollary}[definition]{Corollary}

% Restriction
\newcommand\restr[2]{{
  \left.\kern-\nulldelimiterspace
  #1
  \vphantom{\big|}
  \right|_{#2}
}}

\edef\restoreparindent{\parindent=\the\parindent\relax}
\usepackage{parskip}
\restoreparindent

\usepackage[shortlabels]{enumitem}
\setlist[enumerate]{topsep=1ex,itemsep=1ex,partopsep=1ex,parsep=1ex}
\setlist[itemize]{topsep=1ex,itemsep=1ex,partopsep=1ex,parsep=1ex}

\renewcommand{\abstractname}{Summary}    % clear the title


\title{\textbf{Introduction to Algebraic Number Theory}}
\author{Lev Kruglyak}
\date{}

\begin{document}
\maketitle

\tableofcontents

\section{Motivation}
\subsection{Fermat's last theorem}
\subsection{Primes of the form $x^2+ny^2$}

\section{Number Fields}
\subsection{Norm and trace}
\subsection{Discriminants}
\subsection{Cyclotomic number fields}

\section{Prime Decomposition}
\subsection{Galois theory applied to prime decompositions}
\subsection{Decomposition and inertia groups}

\section{The ideal class group}

\begin{appendices}
\section{Galois theory}


\subsection{Field extensions}
\begin{definition}
    A \emph{field extension} is a pair of fields $K\subset L$, and denoted $L /K$. The \emph{degree}, or \emph{index} of the field extension, denoted $[L : K]$ is defined as the dimension of $L$ as a vector space over $K$. A field extension is said to be \emph{finite} if it has finite degree, and said to be \emph{infinite} otherwise.
\end{definition}

For example, $\C$ is a degree two extension of $\R$, and an infinite extension of $\Q$. An important class of field extensions we are usually interested are those defined by polynomial equations; for instance $\C$ could be considered as the smallest field containing $\R$ which has a solution to the polynomial equation $x^2+1=0$. In general, a simple way to generate field extensions is to start with a base field and construct extension fields which contain roots to some irreducible polynomial in the base field. This is motivated by the following theorem:

\begin{theorem}
    Given some field $K$ and irreducible polynomial $p(x)\in K[x]$, there exists a field extension of $K$ which contains some root of $p(x)$.
\end{theorem}
\begin{proof}
    Consider the ring $L=K[x] /(p(x))$. This is a field because $p(x)$ is irreducible and so $(p(x))$ is maximal. Then $p(\theta)=0$ in $L$, where $\theta=x\mod p(x)$. Note that there is an isomorphic copy of $K$ in $L$ so this is a field extension.
\end{proof}

To better understand this extension field, we can try writing out all of its elements explicitly.

\begin{theorem}
    Let $p(x)\in K[x]$ be an irreducible polynomial of degree $n$, and let $L$ be the field $F[x]/(p(x))$. Let $\theta=x\mod p(x)$. Then $1,\theta,\theta^2,\ldots,\theta^{n-1}$ are a basis for $L$ as a $K$-vector space, so $[L : K]=n$. 
\end{theorem}

For example if $K=\R$, $L=\R[x]/(x^2+1)\cong \C$. If we were to replace $\R$ with $\Q$, then $L$ would be $\Q[x]/(x^2+1)\cong \Q(i)$, i.e. the field of fractions of the Gaussian integers. 

\begin{definition}
    Let $L$ be an extension of $K$ and let $\alpha_1,\alpha_2,\ldots\in L$ be some elements. Then the smallest subfield of $L$ containing both $K$ and the elements $\alpha_1,\alpha_2,\ldots$, denoted $K(\alpha_1,\alpha_2,\ldots)$ is called the field \emph{generated by $\alpha_1,\alpha_2,\ldots$ over $K$.}
\end{definition}

\begin{definition}
    If the field $L$ is generated by a single element $\alpha$ over $K$, i.e. $L=K(\alpha)$, then $L$ is said to be a \emph{simple} extension of $K$ and the element $\alpha$ is called a \emph{primitive element} for the extension. 
\end{definition}

\begin{theorem}
    Let $K$ be a field and $p(x)\in K[x]$ a irreducible polynomial. Suppose $L$ is an extension field of $K$ containing a root of $p(x)$. Then $K(\alpha)\cong K[x]/(p(x))$.
\end{theorem}
\begin{proof}
    Use the natural evaluation homomorphism $\varphi : K[x] \to K(\alpha)$.
\end{proof}

\begin{theorem}
    Let $\varphi : K \to K'$ be a field isomorphism. Let $p(x)\in K[x]$ and $p'(x)\in K'[x]$ be the irreducible polynomial obtained by applying $\varphi$ to the coefficients of $p(x)$. Let $\alpha$ be some root of $p(x)$ in some extension and $\beta$ be some root of $p'(x)$. Then there is a natural isomorphism $\sigma : K(\alpha) \to K'(\beta)$ which sends $\alpha$ to $\beta$.
\end{theorem}

\subsubsection{Algebraic extensions}

\begin{definition}
    An element $\alpha\in L$ is said to be \emph{algebraic} over $K$ if $\alpha$ is the root of a nonzero polynomial $p(x)\in K[x]$. Otherwise $\alpha$ is \emph{transcendental} over $K$. The extension $L/K$ is said to be \emph{algebraic} if every element of $L$ is algebraic over $K$.
\end{definition}

\subsubsection{Splitting Fields}
\end{appendices}

\end{document}