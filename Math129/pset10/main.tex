\documentclass[11pt,letterpaper]{article}

\input{../../../../.config/latex/preamble_v1.tex}
\lightmode

\title{\textbf{Math 129 Problem Set 10}}

\begin{document}
\maketitle

\begin{cproblem}{6.11}
    Show that if $\mathcal{O}_K=\Z$ and $m$ is any nonzero integer, then $G^+_{(m)}$ is isomorphic to $\Z_m^\times$.
\end{cproblem}

\begin{solution}
    First let's describe the equivalence relation $\sim^+_{(m)}$. Two ideals $I,J\subset \mathcal{O}_K$ are equivalent under this relationship if there exist $\alpha I=\beta J$ where $\alpha\equiv \beta\equiv 1\mod m$. Since $\mathcal{O}_K = \Z$ is a PID, ideals are of the form $I=(a), J=(b)$ for some $a,b\in \Z$. Since $(\alpha a) = (\beta b)$ is equivalent to $\alpha a = \pm\beta b$, and this implies that $a\equiv b\mod m$, there is a one to one correspondence between elements of $\Z_m^\times$ and equivalence classes under $\sim^+_{(m)}$. This correspondence behaves well with respect to multiplication so $G_{(m)}^+\cong \Z^\times_m$.
\end{solution}

\begin{cproblem}{6.12}
    Let $U_M^+$ denote the group of totally positive units in $\mathcal{O}_K$ satisfying $u\equiv 1\mod M$. Show that $U^+_M$ is a free abelian group of rank $r+s-1$. % See 6.5c
\end{cproblem}

\begin{solution}
    In Problem~6.5, it is proved that if the field $K$ has at least one real embedding, then $U^+$, the group of all totally positive units is a free abelian group of rank $r+s-1$. Let $u_1,\ldots,u_{r+s-1}$ be some basis for $U^+$. For every $u_i$, there is some $k_i\in \Z$ such that $u_i^{k_i}\equiv 1\mod M$. Then the free module generated by $u_i^{k_i}$ has rank $r+s-1$, and since $U_M^+\subset U_+$ which has rank $r+s-1$, it follows that $U_M^+$ also has rank $r+s-1$.
\end{solution}

\begin{cproblem}{6.13}
    Modify the proof of Theorem~39 to yield the following improvement: If $C$ is any ray class (equivalence under $\sim^+_M $), then (with the obvious notation) we have
    \[
        i_C(t) = \kappa^+_M t + \varepsilon_C(t)
    \]
    where $\kappa^+_M$ is independent of $C$ and $\varepsilon_C(t)$ is $O(t^{1-1/n})$.
\end{cproblem}

\begin{solution}
    The proof is essentially unchanged, we know that $U^+_M$ is a free abelian group of rank $r+s-1$, and we get the same lattice behavior and properties as for the non ray class group. The only crucial difference is that $\kappa^+_M$ would be smaller since ray classes are smaller than ordinary ideal classes.
\end{solution}

\begin{cproblem}{6.14}
    Let $u_1,\ldots,u_{r+s-1}$ be any $r+s-1$ units in a number ring $\mathcal{O}_K$ and let $G$ be the subgroup of $U=\mathcal{O}_K^\times$ generated by all $u_i$ and all roots of unity in $\mathcal{O}_K$. Let $\Lambda_G$ be the sublattice of $\Lambda_U$ consisting of the log vectors of units in $G$.
    \begin{enumerate}[(a)]
        \item Prove that the factor groups $U / G$ and ${\Lambda_U}/{\Lambda_G}$ are isomorphic.
        \item Prove that the log vectors of the $u_i$ are linearly independent over $\R$ iff $U / G$ is finite.
        \item Define the regulator $\textrm{reg}(u_1,\ldots,u_{r+s-1})$ to be the absolute value of the determinant formed from the log vectors of the $u_i$ along with any vector having coordinate sum $1$. % The lemma for theorem 41 shows that any such vector resutls in the same value.
            Show that $U /G$ is finite iff $\textrm{reg}(u_1,\ldots,u_{r+s-1})\neq 0$.
        \item Assuming that $\textrm{reg}(u_1,\ldots,u_{r+s-1})\neq 0$, prove that
        \[
            \textrm{reg}(u_1,\ldots,u_{r+s-1})=|U / G|\cdot \textrm{reg}(R)  
        .\] 
        % See exercise 3, chapter 5, and theorem 41
    \end{enumerate}
\end{cproblem}

\begin{solution}
    \textbf{(a)} Recall that we have an (injective) embedding of $K \to \R^{r+2s}$ which restricts to an embedding $\lambda : \mathcal{O}_K \to \Lambda_K$, where $\Lambda_K$ is the fundamental lattice of the number field. This embedding is also an additive homomorphism, so we have a surjective map $U \to \Lambda_U / \Lambda_G$ which takes an $\alpha$ and maps it to $\lambda(\alpha) + \Lambda_G$. The kernel of this map is the set of $\alpha$ such that $\lambda(\alpha)\in \Lambda_G$, which is exactly $G$. So the first isomorphism theorem gives us a canonical isomorphism $U /G \to \Lambda_U / \Lambda_G$.
    
    \textbf{(b)} Suppose first that the log vectors of the $u_i$ are linearly independent over $\R$. Then letting $\log : \Lambda_K \to \R^{r+s}$ be the standard log map, it follows that $\log(u_1,\ldots,u_{r+s-1})$ is a $r+s-1$ dimensional sublattice of $\log \Lambda_K$. Thus the quotient is finite. If the quotient is infinite, we get a clear contradiction which shows that the log vectors aren't linearly independent. 
\end{solution}

\begin{cproblem}{7.1}
    Fill in details in the proof of Theorem~42:
    \begin{enumerate}[(a)]
        \item Show that
        \[
            1-\frac{1}{2^s}+\frac{1}{3^s}-\frac{1}{4^s}+\cdots=(1-2^{1-s})\zeta(s)
        \] 
        for $s>1$.
        \item Verify that $1-2^{1-s}$ has a simple zero at $s=1$.
    \end{enumerate}
\end{cproblem}

\begin{solution}
    \textbf{(a)} Observe that
    \[
        (1-2^{1-s})\zeta(s)=\sum^\infty_{n=1}\frac{1}{n^s}-2\sum^\infty_{n=1}\frac{1}{(2n)^s} = 1-\frac{1}{2^s}+\frac{1}{3^s}-\frac{1}{4^s}+\cdots
    .\] 

    \textbf{(b)} First we'll expand $1-2^{1-s}$ as a power series. Note that
    \[
        2^z = e^{z\log 2} = \sum_{n=0}^\infty \frac{(z\log 2)^n}{n!}
    \] 
    so $1-2^{1-s}$ has the power series
    \[
        1-2^{1-s} = 1-\sum_{n=0}^\infty \frac{(1-s)^n \log^n 2}{n!}= -\sum_{n=1}^\infty \frac{(1-s)^n\log^n 2}{n!} = (s-1)\left(\sum_{n=1}^\infty \frac{(1-s)^{n-1}\log^n 2}{n!}\right)
    .\]
    The second part of the factor has simple form, and the power of $(s-1)$ is one, so the zero at $1$ is simple.
\end{solution}

\begin{cproblem}{7.3}
    Let $A$ and $B$ be disjoint sets of primes in a number field. Show that 
    \[
        \delta(A\cup B) = \delta(A) + \delta(B)
    \]
    if all of these polar densities exist, and that if any two of them exist, then so does the third.
\end{cproblem}

\begin{solution}
    First suppose all the polar densities exist. By definition, we have $\zeta_{K,A}(s)=(s-1)^{n\delta(A)}g_A(s)$ for some analytic function $g_A(s)$ which is defined and nonzero at $s=1$. We have the same for $\zeta_{K,B}(s)$. Then,
    \[
        \zeta_{K, A\cup B}(s) = \prod_{\mathfrak{p}\in A\cup B}\left(1-\frac{1}{\norm{\mathfrak{p}}^s}\right)^{-1} = \prod_{\mathfrak{p}\in A}\left(1-\frac{1}{\norm{\mathfrak{p}}^s}\right)^{-1}\prod_{\mathfrak{p}\in B}\left(1-\frac{1}{\norm{\mathfrak{p}}^s}\right)^{-1} 
    \]
    which can be extended to a function $(s-1)^{n(\delta(A)+\delta(B)}g_A(s)g_B(s)$, and so $\delta(A\cup B)=\delta(A)+\delta(B)$. Now suppose that two of them exist; there are two cases to consider. If both $\delta(A)$ and $\delta(B)$ exist, then the above argument shows that $\delta(A\cup B)$ must also exist. To address the other case, suppose without loss of generality that $\delta(A)$ and $\delta(A\cup B)$ exist. Then
    \[
        \zeta_{K, B}(s) = \frac{\zeta_{K, A\cup B}(s)}{\zeta_{K, A}(s)}
    \]
    so any extension of the two that exist gives an extension to $\zeta_{K, B}(s)$ giving a well defined density.

\end{solution}

\begin{cproblem}{7.6}
    Let $H$ be a proper subgroup of $\Z^*_m$. Give an elementary proof, using nothing more that the Chinese remainder theorem that there are infinitely many primes $p\in \Z$ such that $\overline{p}\not\in H$.
\end{cproblem}

\begin{solution}
    Suppose for the sake of contradiction that there were a finite number of primes $p_1,\ldots,p_k$ with $\overline{p_i}\not\in H$. Let $x\in \Z$ be an integer congruent to $1\mod p_i$ for all $i$ (This works by Chinese remainder theorem) and $\overline{x}\not\in H$. We can ensure the second part since all $x+np_1\cdots p_k$ are solutions for some fundamental solution $x<p_1\cdots p_k$. Then 
\end{solution}

\end{document}
