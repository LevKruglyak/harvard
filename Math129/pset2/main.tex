\documentclass[11pt,letterpaper]{article}

\input{../../../../.config/latex/preamble_v1.tex}
\lightmode

\title{\textbf{Math 129 Problem Set 2}}

\begin{document}
\maketitle

% Problem 1
\begin{cproblem}{1.21}
    Show that every element of $\Q[\omega]$ is uniquely expressible in the form
    \[
        a_0+a_1\omega+a_2\omega^2+\cdots+a_{p-2}\omega^{p-2}, \quad a_i\in \Q\;\;\forall i
    \] 
    by showing showing that $\omega$ is the root of the polynomial 
    \[
        f(t)=t^{p-1}+t^{p-2}+\cdots+t+1
    \]
    and that $f(t)$ is irreducible over $\Q$. 
\end{cproblem}

%(Hint: It is enough to show that $f(t+1)$ is irreducible, which can be established by Eisenstein's criterion (appendix A). It helps to notice that $f(t+1)=((t+1)^p-1)/t$.)

\begin{solution}
    Clearly $\omega$ is a root of $f(t)=t^{p-1}+t^{p-2}+\cdots+t+1$ since $t^p-1=(t-1)f(t)$ and $\omega\neq 1$. So using the substitution $\omega^{p-1}=-(\omega^{p-2}+\cdots + \omega+1)$, for any $\alpha\in \Q[\omega]$ we can reduce any $a_0'+a_1'\omega+\cdots+a_n'\omega^n$ to get an expression:
    \[
        \alpha = a_0+a_1\omega+a_2\omega^2+\cdots+a_{p-2}\omega^{p-2},\quad a_i\in \Q\;\;\forall i
    .\]   
    To prove that this expression is unique, we first must show that $f(t)$ is irreducible in $\Z[t]$. Note that
    \[
        f(t+1)=\frac{(t+1)^p-1}{t}=\binom{p}{p}x^{p-1} + \binom{p}{p-1}x^{p-2}+\cdots + \binom{p}{2}x+\binom{p}{1}
    .\]  
    Then $p\mid \binom{p}{k}$ for all $2\leq k\leq p-1$, $p\nmid \binom{p}{p}=1$, and $p^2\nmid \binom{p}{1}=p$. Thus Eisenstein's criterion implies that $f(t+1)$ and hence $f(t)$ is irreducible. Now to prove uniqueness, suppose we had
    \[
        a_0+a_1\omega+a_2\omega^2+\cdots+a_{p-2}\omega^{p-2}=b_0+b_1\omega+b_2\omega^2+\cdots+b_{p-2}\omega^{p-2}
    .\]    
    Assuming for the sake of contradiction that $a_i\neq b_i$ for some $i$, then $(a_0-b_0)+(a_1-b_1)\omega+\cdots+(a_{p-2}-b_{p-2})\omega^{p-2}=0$, so $\omega$ is a root of an at most $p-2$ degree polynomial. But $\omega$ is the root of $f$ which is minimal because it is an irreducible polynomial of degree $p-1$, so we have a contradiction. Thus $a_i=b_i$ for all $i$ and the expression is unique.
\end{solution}

% Problem 2
\begin{cproblem}{1.22}
    Use Problem~1.21 to show that if $\alpha\in \Z[\omega]$ and $p\mid \alpha$, then (writing $\alpha=a_0+a_1\omega+\cdots+a_{p-2}\omega^{p-2}$, $a_i\in \Z$) all $a_i$ are divisible by $p$. Define congruence mod p for $\beta, \gamma \in \Z[\omega]$ as follows: 
    \[
        \beta\equiv \gamma \mod{p}\quad\Leftrightarrow\quad \beta-\gamma = \delta p,\; \delta\in \Z[\omega]
    .\] 
    (Equivalently, this is congruence mod the principal ideal $p\Z[\omega]$.)
\end{cproblem}

\begin{solution}
    Suppose $\alpha = p\beta$ for some $\beta\in \Z[\omega]$. Then
    \[
        \alpha = a_0+a_1\omega+a_2\omega^2+\cdots+a_{p-2}\omega^{p-2}=pb_0+pb_1\omega+pb_2\omega^2+\cdots+pb_{p-2}\omega^{p-2} = p\beta
    \]
    for some integers $a_i, b_i$. By Problem~1.21, these expressions must be equal at a coefficient level, so $a_i=pb_i$ for all $i$. Thus $p\mid a_i$ for all $i$. 
\end{solution}

% Problem 3
\begin{cproblem}{1.23}
    Show that if $\beta\equiv \gamma \mod{p}$, then $\overline{\beta}\equiv \overline{\gamma} \mod{p}$, where the bar denotes complex conjugation. 
\end{cproblem}

\begin{solution}
    First, note that if $\beta=\beta_0+\beta_1\omega+\beta_2\omega^2+\cdots+\beta_{p-2}\omega^{p-2}$, we have
    \[
        \begin{aligned}
            \overline{\beta}&=\beta_0+\beta_1\overline{\omega}+\beta_2\overline{\omega^2}+\cdots+\beta_{p-2}\overline{\omega^{p-2}}\\
            &=\beta_0+\beta_1\omega^{p-1}+\beta_2\omega^{p-2}+\cdots+\beta_{p-2}\omega^2\\
            &=\beta_0+\beta_1(-\omega^{p-2}-\omega^{p-3}+\cdots-\omega-1)+\beta_2\omega^{p-2}+\cdots+\beta_{p-2}\omega^2\\
            &=(\beta_0-\beta_1)-\beta_1\omega+(\beta_2-\beta_1)\omega^{p-2}+\cdots+(\beta_{p-2}-\beta_1)\omega^2.
        \end{aligned}
    \] 
    Now since $\beta\equiv \gamma\mod p$, it follows from Problem~1.22 that $\beta_i\equiv \gamma_i\mod p$. This implies that $\overline{\beta}\equiv \overline{\gamma}\mod p$ by the above expression, since $\beta_i-\beta_1\equiv \gamma_i-\gamma_1\mod p$ for $i\neq 1$ and $\beta_1\equiv \gamma_1\mod p$.
\end{solution}

% Problem 4
\begin{cproblem}{1.24}
    Show that $(\beta+\gamma)^p\equiv \beta^p+\gamma^p\mod{p}$, and generalize this to arbitrary numbers of terms by induction. 
\end{cproblem}

\begin{solution}
    By the binomial theorem,
    \[
        (\beta+\gamma)^p=\sum^p_{k=0}\binom{p}{k}\beta^k\gamma^{p-k}\equiv \beta^p+\gamma^p\mod{p}
    .\] 
    because of the basic fact about binomial coefficients which says that $p\mid \binom{p}{k}$ if and only if $1\geq k\leq p-1$. The induction argument easily follows from this, giving us the more general claim that $(\beta_1+\cdots+\beta_n)^p\equiv \beta_1^p+\cdots+\beta_n^p\mod{p}$.
\end{solution}

% Problem 5
\begin{cproblem}{1.25}
    Show that $\forall \alpha\in\Z[\omega]$, $\alpha^p$ is congruent mod $p$ to some $a\in \Z$.    
\end{cproblem}

\begin{solution}
    Using Problem~1.21, write $\alpha=a_0+a_1\omega+\cdots+a_{p-2}\omega^{p-2}$. Then by Problem~1.24,
    \[
        \alpha^p=(a_0+a_1\omega+\cdots+a_{p-2}\omega^{p-2})^p\equiv a_0^p+a_1^p\omega^p+\cdots+a_{p-2}^p\omega^{p(p-2)}\equiv a_0^p+a_1^p+\cdots+a_{p-2}^p\mod p
    .\]  
    Clearly $a_0^p+a_1^p+\cdots+a_{p-2}^p\in \Z$, so we are done. 
\end{solution}

%(Hint: Write $\alpha$ in terms of $\omega$ and use 1.24)

% Problem 6
\begin{cproblem}{2.12}
    Now we can prove Kummer's lemma on units in the $p$th cyclotomic field, as stated before Problem~1.26: Let $\omega=e^{2\pi i / p}$, $p$ an odd prime, and suppose $u$ is a unit in $\Z[\omega]$.
    \begin{enumerate}
        \item Show that $u /\overline{u}$ is a root of $1$. Use Problem~2.11(c) and observe that complex conjugation is a member of the Galois group of $\Q[\omega]$ over $\Q$. Conclude that $u / \overline{u}=\pm \omega k$ for some $k$.  
        \item Show that the + sign holds: Assuming $u /\overline{u}=-\omega^k$, we have $u^p=-\overline{u^p}$; show that this implies that $u^p$ is divisible by $p$ in $\Z[\omega]$. But this is impossible because $u$ is a unit.
    \end{enumerate} 
\end{cproblem}

\begin{solution}
    \textbf{(a)} Note that $\overline{\omega}=\omega^{-1}$, so complex conjugation is a member of $\Gal(\Q[\omega]/\Q)$. However $\Gal(\Q[\omega]/\Q)\cong\Z^\times_p$, which is an abelian group so conjugation commutes with any automorphism. So for any $\sigma\in \Gal(\Q[\omega]/\Q)$, we have
    \[
        \left|\sigma(u /\overline{u})\right|=\left|\sigma(u) /\overline{\sigma(u)}\right|=\left|\sigma(u)\right| /\left|\sigma(u)\right|=1
    .\] 
    Thus all of the Galois conjugates of $u /\overline{u}$ have norm $1$. It follows from Problem~2.11(c) that $u /\overline{u}$ is a root of unity. By Corollary~2.3, the only roots of unity in $\Z[\omega]$ are $2p$-th roots of unity, so $u /\overline{u}=\pm\omega^k$ for some $k\in \Z$.   
    
    \textbf{(b)} Now assume that $u /\overline{u}=-\omega^k$, so $u^p=-\overline{u^p}$. By Problem~1.23, we know that $u^p\equiv \overline{u^p}$, so $2u^p\equiv 0\mod p$. Since $p$ is an odd prime, we can divide both sides by $2$ to get $u^p\equiv 0\mod p$. This means that $u^p=p\alpha$ for some $\alpha\in \Z[\omega]$, a contradiction because $u^p$ is a unit.  
\end{solution}

% Problem 7
\begin{cproblem}{2.13}
    Show that 1 and $-1$ are the only units in the ring $\mathbb{A}\cap \Q[\sqrt{m}]$, $m$ squarefree, $m<0,m\neq -1,-3$. What if $m=-1$ or $-3$?    
\end{cproblem}

By Corollary~2 to Theorem~1 we know that for squarefree $m$,
\[
    \mathbb{A}\cap \Q[\sqrt{m}] = \begin{cases}
        \Z[\sqrt{m}]&\textrm{if }m\equiv 2,3\mod 4\\
\Z\left[\frac{1+\sqrt{m}}{2}\right]&\textrm{if }m\equiv 1\mod 4
    \end{cases}
.\] 

If $m\equiv 2,3\mod 4$, then a unit in $A\cap \Q[\sqrt{m}]$ is some $a+b\sqrt{m}$ with $a^2-mb^2=1$, Since $a^2$ and $-mb^2$ are both nonnegative and $m\neq -1$, the only $a+b\sqrt{m}$ which are units are are $a=\pm 1$. If $m\equiv 1\mod 4$, then a unit in $A\cap \Q[\sqrt{m}]$ is some $\frac{a+b\sqrt{m}}{2}$ with $a^2-mb^2=4$. Then since $a^2$ and $-mb^2$ are both nonnegative and $m\neq -3$, the only solution to this are $a=\pm 1$.

Now if $m=-1$, then units satisfy $a^2+b^2=1$ so the only units are $\pm 1, \pm i$. If $m=-3$, then units satisfy $a^2+3b^2=4$, so the only units are $\pm 1, \frac{1\pm \sqrt{-3}}{2}, \frac{-1\pm \sqrt{-3}}{2}$.   

% Problem 8
\begin{cproblem}{2.14}
    Show that $1+\sqrt{2}$ is a unit in $\Z[\sqrt{2}]$, but not a root of 1. Use the powers of $1+\sqrt{2}$ to generate infinitely many solutions to the Diophantine equation $a-2b^2=\pm 1$.   
\end{cproblem}

\begin{solution}
    Clearly $1+\sqrt{2}$ is a unit because $\mathcal{N}(1+\sqrt{2})=-1$. Now suppose $a+b\sqrt{2}$ is a unit. Then $(a+b\sqrt{2})(1+\sqrt{2})=(a+2b)+(a+b)\sqrt{2}$ is also a unit. This gives a way to generate infinitely many solutions to $a^2-2b^2=\pm 1$, i.e. given a solution $(a,b)$, there is a larger solution $(a+2b, a+b)$.    
\end{solution}

% Problem 9
\begin{cproblem}{2.15}\noindent
    \begin{enumerate}[(a)]
        \item Show that $\Z[\sqrt{-5}]$ contains no elements whose norm is $2$ or $3$.
        \item Verify that $2\cdot 3=(1+\sqrt{-5})(1-\sqrt{-5})$ is an example of non unique factorization in the ring $\Z[\sqrt{-5}]$. 
    \end{enumerate}
\end{cproblem}

\begin{solution}
    \textbf{(a)} Suppose $a+b\sqrt{-5} \in \Z[\sqrt{-5}]$ has norm $2$ or $3$. Then $a^2+5b^2=2,3$, which is clearly impossible since the smallest value of $a^2+5b^2$ which is greater than 1 is $2^2+5\cdot 0^2=4$.   
    
    \textbf{(b)} Since there is no element of norm $2$ or $3$, $2$ and $3$ are irreducible in $\Z[\sqrt{-5}]$. (Any non unit and non zero $a\in \Z$ is irreducible in $\Z[\sqrt{m}]$ if and only if $a=\mathcal{N}(\alpha)$ for some $\alpha\in \Z[\sqrt{m}]$.) Similarly, $1\pm \sqrt{-5}$ is irreducible because $\mathcal{N}(1\pm\sqrt{-5})=6$, so any factorization $\alpha\beta=1\pm\sqrt{-5}$ would have $\mathcal{N}(\alpha)=2$ and $\mathcal{N}(\beta)=3$, however (a) implies that no such elements exist. So $2\cdot 3$ and $(1+\sqrt{-5})(1-\sqrt{-5})$ are two distinct irreducible factorizations of $6$. 
\end{solution}

% Problem 10
\begin{cproblem}{2.17}
    Here is another interpretation of the trace and norm: Let $K\subset L$ and fix $\alpha\in L$; multiplication by $\alpha$ gives a linear mapping of $L$ to itself, considering $L$ as a $K$-vector space. Let $A$ denote the matrix of this mapping with respect to the basis $\{\alpha_1, \alpha_2,\ldots\}$ for $L$ over $K$. (Thus the $j$th column of $A$ consists of the coordinates of $\alpha \alpha_j$ with respect to the $\alpha_i$.) Show that $T^L_K(\alpha)$ and $N^L_K(\alpha)$ are the trace and determinant of this matrix.    
\end{cproblem}

%(Hint: It is well known that the trace and determinant are independent of the particular basis chosen; thus it is sufficient to calculate them for any convenient basis. Fix a basis {β1, β2, . . . } for L over K [α] and multiply by powers of α to obtain a basis for L over K. Finally, use Theorem 4′.)

\begin{solution}
    Suppose $L$ has degree $n$ over $K$, and $\alpha$ has degree $d$. Let $\{\beta_1,\beta_2,\ldots,\beta_c\}$ be a basis for $L$ over $K[\alpha]$ where $c=n /d$. The basis for $L$ over $K$ can then be written as \[B=\{\beta_1, \alpha\beta_1, \ldots, \alpha^{d-1}\beta_1,\ldots\beta_c,\alpha\beta_c,\ldots,\alpha^{d-1}\beta_c\}.\]
    Now let's consider what multiplication by $\alpha$ does to this basis. If $\alpha^d+a_{d-1}\alpha^{d-1}+a_{d-2}\alpha^{d-2}+\cdots + a_1\alpha+a_0$, then
    \[
        \alpha\cdot \alpha^k\beta_i = \begin{cases}
            \alpha^{k+1}\beta_i&\textrm{if }k<d-1\\
            -a_0\beta_i-a_1\alpha\beta_i-\cdots-a_{d-1}\alpha^{d-1}\beta_i&\textrm{if }k=d-1
        \end{cases}
    .\] 
    This gives us a matrix of the form:
    \[
        M_B(\alpha)=\begin{bmatrix}
            M_0 & 0 &\cdots & 0\\
            0 & M_1 & \cdots & 0\\
            \vdots & \vdots & \ddots & \vdots\\
            0 & 0 & \cdots & M_c
        \end{bmatrix}, \textrm{ where }
        M_j=
        \begin{bmatrix}
            0 & 1 & 0 & \cdots & 0\\
            0 & 0 & 1 & \cdots & 0\\
            \vdots & \vdots & \vdots & \ddots &\vdots\\
            0 & 0 & 0 & \cdots & 1\\
            -c_0 & -c_1 & -c_2 & \cdots & -c_{d-1}
        \end{bmatrix}
    \] 
    Observe that $\textrm{tr}(M_B(\alpha))=-\frac{n}{d}c_{d-1}$, and $\det(M_B(\alpha))=\left((-1)^{n}c_0\right)^{n /d}$. By Vieta's theorem and Theorem~2.4', these are exactly equal to $T^L_K(\alpha)$ and $N^L_K(\alpha)$ respectively.  
\end{solution}

% Problem 11
\begin{cproblem}{2.24}
    Let $G$ be a free abelian group of rank $n$ and let $H$ be a subgroup. Without loss of generality we take $G=\Z\oplus \cdots \oplus \Z$ ($n$ times). We will show by induction that $H$ is a free abelian group of rank $\leq n$. First we prove it for $n=1$. Then, assuming the result holds for $n-1$, let $\pi : G \to \Z$ denote the obvious projection of $G$ onto the first factor (so that an $n$-tuple of integers gets sent to its first component). Let $K$ denote the kernel of $\pi$.  
    \begin{enumerate}
        \item Show that $H\cap K$ is a free abelian group of rank $\leq n-1$.
        \item The image $\pi(H)\subset \Z$ is either $\{0\}$ or infinite cyclic. If it is $\{0\}$, then $H=H\cap K$; otherwise fix $h\in H$ such that $\pi(h)$ generates $\pi(H)$ and show that $H$ is the direct sum of its subgroups $\Z h$ and $H\cap K$.
    \end{enumerate}
\end{cproblem}

\begin{solution}
    \textbf{(a)} Since $K=\{0\}\oplus \Z^{\oplus n-1}$ is a free abelian group of rank $n-1$, and $H\cap K$ is a subgroup of $K$, the inductive assumption implies that $H\cap K$ is a free abelian subgroup of $K$.
    
    \textbf{(b)} If $\pi(H)=\{0\}$ we are done so suppose $\pi(H)$ is infinite cyclic, generated by $\pi(h)$ for some $h\in H$ (assume without loss of generality that the other components of $h$ are zero). To prove that $H=(H\cap K)\oplus h\Z$, we first need to show that the two subspaces have trivial intersection. Suppose $ah\in H\cap K$ for some $a\in \Z$. This means that $\pi(ah)=0$, so $\pi(h)=0$, which is a contradiction unless $a=0$. Thus $0$ is the only element of $(H\cap K)\cap h\Z$. 
    
    Next, we must show that every element of $H$ can be expressed as a sum of elements in $H\cap K$ and $h\Z$. Let $g\in H$. Then $g=(g-\pi(g))+ah$, where $\pi(g)=\pi(ah)$.   
\end{solution}

\end{document}