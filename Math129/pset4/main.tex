\documentclass[11pt,letterpaper]{article}

\input{../../../../.config/latex/preamble_v1.tex}
\lightmode

\title{\textbf{Math 129 Problem Set 4}}

\begin{document}
\maketitle

\textit{For any number field $K$, we'll use $\mathcal{O}_K$ to denote the ring of integers of $K$, i.e. $\mathcal{O}_K = \mathbb{A}\cap K$. Also let $\Delta(\cdots)$ be the discriminant. We'll use $\Delta_K$ to mean the discriminant of a number field $K$.}

\begin{cproblem}{2.40}
    In the notation of Theorem~2.13, establish the formula
    \[
        \Delta(\alpha)=(d_1d_2\cdots d_{n-1})^2\Delta_K 
    .\] 
\end{cproblem}

\begin{solution}
    Notice that by Theorem~2.13, we have
    \[
        \Delta_K = \Delta\left(1, \frac{f_1(\alpha)}{d_1}, \ldots, \frac{f_{n-1}(\alpha)}{d_{n-1}}\right).
    \]  
    By properties of discriminant and determinant, multiplying the matrix by $d_1d_2\cdots d_{n-1}$ gives us $\Delta(1,f_1(\alpha),\ldots,f_{n-1}(\alpha))=(d_1d_2\cdots d_{n-1})^2\Delta_K$. We next claim that $1,f_1(\alpha),\ldots,f_{n-1}(\alpha)$ is an integral basis for $\Z[\alpha]$. This follows by induction and because $f_i$ are all monic of degree $i$, so we can show that $\alpha^i$ can be generated by $1,f_1(\alpha),\ldots,f_i(\alpha)$. Thus $\Delta(\alpha)=\Delta(1,f_1(\alpha),\ldots,f_{n-1}(\alpha))$, completing the proof.    
\end{solution}

\begin{cproblem}{2.43}
    Let $f(x)=x^5+ax+b$ where $a,b\in \Z$ and assume that $f(x)$ is irreducible in $\Q[x]$. Let $\alpha$ be a root of $f(x)$.
    \begin{enumerate}[(a)]
        \item Show that $\Delta(\alpha)=4^4a^5+5^5b^4$.
        \item Suppose $\alpha^5=\alpha+1$. Show that $\mathcal{O}_{\Q[\alpha]}=\Z[\alpha]$. 
    \end{enumerate}
\end{cproblem}

\begin{solution}
    \textbf{(a)} First note that $\alpha^5+a\alpha+b=0$ so $\alpha^4=\frac{a\alpha+b}{-\alpha}$. Thus $f'(\alpha)=\frac{5a\alpha+5b}{-\alpha}+a\alpha=\frac{4a\alpha+5b}{-\alpha}$. By Theorem~2.8, and because $d\equiv 1\mod 4$ we have
    \[
        \Delta(\alpha)=\nrm^{\Q[\alpha]}(f'(\alpha))=\frac{\nrm^{\Q[\alpha]}(4a\alpha+5b)}{\nrm^{\Q[\alpha]}(-\alpha)}
    .\]  
    Note that $4a\alpha+5b$ is a root of the irreducible polynomial $g_1(x)=\left(\frac{x-5b}{4a}\right)^5+a\left(\frac{x-5b}{4a}\right)+b$, so by Theorem~2.4 and Vieta's formulas,
    \[
        \begin{aligned}
            \nrm^{\Q[\alpha]}(4a\alpha+5b)&=(-1)^5\left(\frac{x^0\textrm{ coefficient of }g_1}{x^5\textrm{ coefficient of }g_1}\right)=\frac{\left(-\frac{5b}{4a}\right)^5-\frac{5b}{4}+b}{\left(\frac{1}{4a}\right)^5}\\
            &=-5^5b^5-5\cdot 4^4a^5b+4^5a^5b=4^4a^5b+5^5b^5.
        \end{aligned}
    \]  
    Similarly, $-\alpha$ is the roof of the irreducible polynomial $g_2(x)=x^5+ax-b$ so $\nrm^{\Q[\alpha]}(-\alpha)=b$. Thus $\Delta(\alpha)=4^4a^5+5^5b^4$.

    \textbf{(b)} First we'll show that $f(x)=x^5-x-1$ is irreducible, since $\alpha$ is a root. Clearly if $f(x)$ were reducible, it would not have any linear factors because $f(x)\equiv 1\mod 2$ so it has no integral roots. So it must have one quadratic factor and one cubic factor. Let $g(x)$ be the irreducible quadratic factor of $f(x)$. Then $\F_5[x]/(g(x))\cong \F_5[\alpha]\cong \F_{25}$. However in $\F_{25}$, $\alpha^25=\alpha$, yet $\alpha^25=(\alpha^5)^5=(\alpha+1)^5=\alpha^5+1=\alpha+2$, so $\alpha+2=\alpha$. This is impossible, so $f(x)$ must be irreducible.  
    
    Then by (a), $\Delta(\alpha)=2869$ which is squarefree, so by Theorem~2.9, $\{1,\alpha,\alpha^2,\alpha^3,\alpha^4\}$ is an integral basis for $\Q[\alpha]$. Thus $\mathcal{O}_{\Q[\alpha]}=\Z[\alpha]$.  
\end{solution}

\begin{cproblem}{3.1}
    For any integral domain $R$, prove that the following conditions are equivalent:
    \begin{enumerate}
        \item Every ideal is finitely generated.
        \item Every increasing sequence of ideals $I_1\subset I_2\subset \cdots$ is eventually constant.
        \item Every non-empty set $S$ of ideals of $R$ has a maximal member; i.e. $\exists M \in S$ such that $M\subset I\in S$ implies that $M=I$. 
    \end{enumerate}
\end{cproblem}

\begin{solution}
    $(1) \implies (2)$: Suppose $I_1\subset I_2\subset \cdots$ is an increasing chain of ideals of $R$. Then $\bigcup_i I_i$ is an ideal in $R$, so it must be finitely generated by (1), say $\bigcup_i I_i=(r_1,\ldots, r_n)$. We can assume without loss of generality that all of the inclusions are proper. Say $r_1\in I_1$. Then $I_2$ must contain one of the other $r_i$ or else $I_1=I_2$, say $r_2\in I_2$. Then by induction $I_n=(r_1,\ldots, r_n)$ so $I_m=I_n$ for all $m > n$. So the sequence is eventually constant.
    
    $(2)\implies (3)$: $S$ can be given the structure of a partially ordered set, and (2) implies that every chain has an upper bound, so by Zorn's lemma there must be some maximal element satisfying the conditions of (3).
    
    $(3)\implies (1)$: Let $I$ be an ideal in $R$. Consider the family of ideals $\{(S)\}_{S\textrm{ finite subset of } I}$ where $(S)$ is the ideal generated by the set $S\subset I$. By (3), there must be some maximal member of this family, say $M=(r_1,\ldots,r_n)$. Then for any element $r\in I$, we have $M\subset (r_1,\ldots,r_n, r)$ so $r\in M$. This means that $I=M$ so $I$ is finitely generated.
\end{solution}

\begin{cproblem}{3.2}
    Prove that every finite integral domain is a field.
\end{cproblem}

\begin{solution}
    Let $K$ be a finite integral domain and let $\alpha\in K$ be a nonzero element. Consider the set $S_\alpha=\{1,\alpha, \alpha^2, \ldots\}\subset K$. Since $K$ is an integral domain, $0\not\in S_\alpha$. So by the pigeonhole principle there must be some $n>m$ such that $\alpha^n=\alpha^m\neq 0$. Then $\alpha^{n-m}=1$ and so $\alpha^{n-m-1}$ is a multiplicative inverse for $\alpha$. Thus $K$ is a field.      
\end{solution}

\begin{cproblem}{3.7}
    Show that if $I,J$ are ideals in a commutative ring such that $1\in I+J$, then $1\in I^n+J^m$ for all $m,n$.
\end{cproblem}

\begin{solution}
    Since $1\in I+J$, there is an $\alpha\in I$ and $\beta\in J$ such that $1=\alpha+\beta$. Then
    \[
        \begin{aligned}
            1=(\alpha+\beta)^{n+m}&=\sum^{n+m}_{k=0}\binom{n+m}{k}\alpha^k\beta^{n+m-k}\\
            &=\underbrace{\sum^n_{k=0}\binom{n+m}{n-k}\alpha^{n-k}\beta^{m+k}}_{J^m}+\underbrace{\sum^m_{k=1}\binom{n+m}{k+n}\alpha^{k+n}\beta^{m-k}}_{I^n}.
        \end{aligned}
    \]  
    Thus $1\in I^n+J^m$.
\end{solution}

\begin{cproblem}{3.9}
    Let $K\subset L$ be number fields
    \begin{enumerate}[(a)]
        \item Let $I,J\subset \mathcal{O}_K$ be ideals and suppose $I\cdot\mathcal{O}_L \mid J\cdot\mathcal{O}_L$. Show that $I \mid J$.
        \item Show that for each ideal $I$ in $\mathcal{O}_K$, we have $I=I\cdot\mathcal{O}_L\cap \mathcal{O}_K$.
        \item Characterize those ideals $I$ of $\mathcal{O}_L$ such that $I=(I\cap \mathcal{O}_K)\cdot\mathcal{O}_L$.   
    \end{enumerate}
\end{cproblem}

\begin{solution}
    \textbf{(a)} Factor $I=\prod_i\mathfrak{p}^{e_i}_i$ and $J=\prod_i\mathfrak{p}^{r_i}_i$ where only a finite number of the $e_i, r_j$ are nonzero. Then we have $I\cdot \mathcal{O}_L=\prod_i\left(\mathfrak{p}_i\cdot \mathcal{O}_L\right)^{e_i}$ and $J\cdot \mathcal{O}_L=\prod_i \left(\mathfrak{p}_i\cdot \mathcal{O}_L \right)^{r_i}$. However by Theorem~3.20, each $\mathfrak{p}_i\cdot \mathcal{O}_L=\prod_{j\in S_i}\mathfrak{P}_j^{s_j}$ where $\mathfrak{P}_j$ is a prime in $\mathcal{O}_L$ and $S_i\cap S_n=\emptyset$ for $i\neq n$. Thus 
    \[
        I\cdot \mathcal{O}_L=\prod_i\left(\mathfrak{p}_i\cdot \mathcal{O}_L\right)^{e_i}=\prod_i \prod_{j\in S_i}\mathfrak{P}_j^{e_is_j}
    \]  
    and likewise for $J$. Thus if $I\cdot \mathcal{O}_L \mid J\cdot \mathcal{O}_L$ then $e_is_j\leq r_is_j$ for all $i$ and $j\in S_i$. Since $S_i$ are nonempty, this means that $e_i\leq r_i$ for all $i$. However this implies that $I\mid J$ so we are done.
    
    \textbf{(b)} Again factor $I=\prod_i\mathfrak{p}^{e_i}_i$, setting $I\cdot \mathcal{O}_L=\prod_i\prod_{j\in S_i}\mathfrak{P}^{e_is_j}_j$. Then note that by Theorem~3.19 $\mathfrak{P}_j\cap \mathcal{O}_K=\mathfrak{p}_i$ whenever $j\in S_i$. This means that $\left(\prod_{j\in S_i}\mathfrak{P}^{s_j}_j\right)\cap \mathcal{O}_K=\mathfrak{p}_i$ and so by extension $I\cdot \mathcal{O}_L\cap \mathcal{O}_K=I$.    
    
    \textbf{(c)} We claim that this is only true if $I=J\cdot \mathcal{O}_L$ for some ideal $J\subset\mathcal{O}_K$. Indeed if $J\subset \mathcal{O}_K$ is an ideal then by (b), $(J\cdot \mathcal{O}_K\cap \mathcal{O}_K)\cdot \OO_L=J\cdot \OO_L=I$. Conversely, if $I$ is some ideal in $\OO_L$ satisfying $I=(I\cap \OO_K )\cdot \OO_L$ then $J=(I\cap \OO_K)$. This completes the proof.  
\end{solution}

\end{document}