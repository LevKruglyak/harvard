\documentclass[11pt,letterpaper]{article}

\input{../../../../.config/latex/preamble_v1.tex}
\lightmode

\title{\textbf{Math 129 Problem Set 5}}

\begin{document}
\maketitle

\textit{For any number field $K$, we'll use $\mathcal{O}_K$ to denote the ring of integers of $K$, i.e. $\mathcal{O}_K = \mathbb{A}\cap K$. Also let $\Delta(\cdots)$ be the discriminant. We'll use $\Delta_K$ to mean the discriminant of a number field $K$.}

\textit{I collaborated with Ignasi Segura Vicente on this problem set.}

\begin{cproblem}{2.8}\noindent
    \begin{enumerate}[(a)]
        \item Let $p$ be an odd prime and $\zeta_p=e^{2\pi i /p}$. Show that \[
            \sqrt{(-1)^{\frac{p-1}{2}}p}\in \Q[\zeta_p]
        .\] 
        Express $\sqrt{-3}$ and $\sqrt{5}$ in the appropriate $\Q[\zeta_p]$.
        \item Show that the $8^{\textrm{th}}$ cyclotomic field contains $\sqrt{2}$.
        \item Show that every quadratic number field $K$ is contained in $\Q[\zeta_d]$ where $d=|\Delta_K|$.
    \end{enumerate}
\end{cproblem}

\begin{solution}
    \textbf{(a)} Recall that for any odd prime $p$ we have
    \[
        \Delta_{\Q[\zeta_p]} = (-1)^{\frac{p-1}{2}}p^{p-2}
    .\] 
    However by definition of discriminant, we know that $\Delta_{\Q[\zeta_p]}=\alpha^2$ where $\alpha\in \Q[\zeta_p]$ is the determinant of the discriminant matrix. So 
    \[
        (-1)^{\frac{p-1}{2}}p=\frac{\Delta_{\Q[\zeta_p]}}{p^{p-3}}=\frac{\alpha^2}{p^{p-3}}=\left(\frac{\alpha}{p^{(p-3) /2}}\right)^2
    .\]  
    Thus $\sqrt{(-1)^{\frac{p-1}{2}}p}\in \Q[\zeta_p]$ as desired. To find $\sqrt{-3}$, we let $p=3$, $\zeta=\zeta_3$ and use the derived formula for $\sqrt{-3}$, i.e.
    \[
        \sqrt{-3}=\frac{1}{3^{(3-3) /2}}\begin{vmatrix}
            \sigma_1(\zeta)&\sigma_1(\zeta^2)\\
            \sigma_2(\zeta)&\sigma_2(\zeta^2)\\
        \end{vmatrix}=\begin{vmatrix}
            \zeta&\zeta^2\\
            \zeta^2&\zeta\\
        \end{vmatrix}= \zeta^2-\zeta^4=\zeta^2-\zeta
    \]  
    Where $\sigma_i\in \Gal(\Q[\zeta_p]/\Q)$ is the automorphism sending $z$ to $z^i$. Checking the square $(\zeta^2-\zeta)^2=\zeta-2+\zeta^2=-3$ confirms the formula. Next for $\sqrt{5}$, set $p=5$ and $\zeta=\zeta_5$. Then 
    \[
        \sqrt{5}=\frac{1}{5^{(5-3) /2}}\begin{vmatrix}
            \sigma_1(\zeta)&\sigma_1(\zeta^2)&\sigma_1(\zeta^3)&\sigma_1(\zeta^4)\\
            \sigma_2(\zeta)&\sigma_2(\zeta^2)&\sigma_2(\zeta^3)&\sigma_2(\zeta^4)\\
            \sigma_3(\zeta)&\sigma_3(\zeta^2)&\sigma_3(\zeta^3)&\sigma_3(\zeta^4)\\
            \sigma_4(\zeta)&\sigma_4(\zeta^2)&\sigma_4(\zeta^3)&\sigma_4(\zeta^4)\\
        \end{vmatrix}=
        \frac{1}{5}
        \begin{vmatrix}
            \zeta&\zeta^2&\zeta^3&\zeta^4\\
            \zeta^2&\zeta^4&\zeta&\zeta^3\\
            \zeta^3&\zeta&\zeta^4&\zeta^2\\
            \zeta^4&\zeta^3&\zeta^2&\zeta
        \end{vmatrix}
        =2\zeta^3+2\zeta^2+1
    .\]
    As before, a simple check confirms that $(2\zeta^3+2\zeta^2+1)^2=5$. 
    
    \textbf{(b)} Let $\zeta=\zeta_8$ and consider $\zeta+\zeta^{-1}$. Then $(\zeta+\zeta^{-1})^2=\zeta^2+2+\zeta^{-2}=2$, so $\sqrt{2}\in \Q[\zeta_8]$.
    
    \textbf{(c)} For brevity, we'll write $\zeta_n=\zeta_{|n|}$. We know that every quadratic number field is of the form $\Q[\sqrt{m}]$ for some squarefree integer $m$ with $m\neq 0,1$. Let's prime factorize $m=p_1p_2\cdots p_k$ where $p_i$ are distinct primes since $m$ is squarefree. (Note that we don't need a $\pm$ sign since $p_i$ are allowed to be negative primes.) Recall that the discriminant of a quadratic number field is
    \[
        \Delta_{\Q[\sqrt{m}]}=\begin{cases}
            4m&m\equiv 2,3\mod 4\\
            m&m\equiv 1\mod 4
        \end{cases}
    \] 
    First suppose $m\equiv 1\mod 4$. Then $m$ can be factored as $m=(\ell_1\cdots \ell_r)(p_1q_1)\cdots (p_sq_s)$ where $p_j, q_j>0$ are positive primes and $\ell_i$ are (possibly negative) primes satisfying $\ell_i\equiv 1\mod 4$ and $p_j, q_j\equiv 3\mod 4$. Then
    \[
        \sqrt{m}=\left(\sqrt{\ell_1}\cdots \sqrt{\ell_r}\right)\left(\sqrt{-p_1}\sqrt{-q_1}\right)\cdots\left(\sqrt{-p_s}\sqrt{-q_s}\right)
    .\] 
    Note that by (a), $\sqrt{\ell_i}\in \Q[\zeta_{\ell_i}]$ and $\sqrt{-p_j}\in \Q[\zeta_{p_j}]$. (resp $q_j$) Since $\Q[\zeta_a]\subset \Q[\zeta_b]$ for any $a\mid b$, it follows that $\sqrt{m}\in \Q[\zeta_{m}]$ since $\ell_i, p_j, q_j \mid m$. So any squarefree $m\equiv 1\mod 4$ satisfies $\sqrt{m}\in \Q[\zeta_{m}]=\Q[\zeta_{\Delta_K}]$. 
    
    Next, suppose that $m\equiv 3\mod 4$. This means that $m=pn$ where $p\equiv 3\mod 4$ and $n\equiv 1\mod n$ is some squarefree integer. There are now two cases. Without loss of generality, we can assume that $p=-1$ or $p$ is a prime, since $n$ can absorb all $1\mod 4$ factors out of $p$. If $p=-1$, then $\sqrt{-1}\in \Q[\zeta_4]$ and by the earlier argument $\sqrt{n}\in \Q[\zeta_n]$. Combining this gives $\sqrt{n}=\sqrt{-1}\cdot\sqrt{n}\in \Q[\zeta_{4n}]=\Q[\zeta_{\Delta_K}]$ as desired. Now if $p$ is a prime, then by (a) we know that $\sqrt{-p}\in \Q[\zeta_p]$, however since $\sqrt{-1}\in \Q[\zeta_4]$, we have $\sqrt{p}\in \Q[\zeta_{p}, \zeta_4]\subset \Q[\zeta_{4p}]$. Thus since $\sqrt{m}=\sqrt{p}\sqrt{n}$ we have $\sqrt{m}\in \Q[\zeta_{4pn}]=\Q[\zeta_{4m}]=\Q[\zeta_{\Delta_K}]$. 

    Our last case to consider is when $m\equiv 2\mod 4$. Such $m$ can be expressed as $m=2n$ for some $n\equiv 1,3\mod 4$. By the results of the preceding paragraphs, $\sqrt{n}\in \Q[\zeta_{4n}]$. (If $n\equiv 1\mod 4$, $\sqrt{n}\in \Q[\zeta_{n}]\subset \Q[\zeta_{4n}]$) Then by (b), $\sqrt{2}\in \Q[\zeta_8]$ so $\sqrt{m}=\sqrt{2}\sqrt{n}\in \Q[\zeta_{4\cdot 2n}]=\Q[\zeta_K]$. This completes the proof.   
\end{solution}

\begin{cproblem}{3.8}\noindent
    \begin{enumerate}[(a)]
        \item Show that the ideal $(2,x)$ in $\Z[x]$ is not principal.
        \item Let $f,g\in \Z[x]$ and let $m,n$ be the gcd's of the coefficients of $f$ and $g$, respectively. Prove Gauss' Lemma: $mn$ is the gcd of the coefficients of $fg$. % Reduce to the case in which m=n=1 and argue as in the lemma for Theorem 1.
        \item Show that if $f\in \Z[x]$ and $f$ is irreducible over $\Z$, then $f$ is irreducible over $\Q$.
        \item Suppose $f$ is irreducible over $\Z$ and the gcd of its coefficients is $1$. Show that if $f \mid gh$ in $\Z[x]$, then $f \mid g$ or $f \mid h$.
        \item Show that $\Z[x]$ is a UFD, the irreducible elements being the polynomials $f$ as in (d), along with the primes $p\in \Z$.
    \end{enumerate}
\end{cproblem}

\begin{solution}
    \textbf{(a)} Suppose for the sake of contradiction that $(2,x)$ is principal in $\Z[x]$. Then $(2,x)=(\alpha(x))$ for some $\alpha(x)\in \Z[x]$. Thus $2=\alpha(x)\beta(x)$ for some $\beta(x)$, however this implies that $\alpha(x)$ has degree zero, and since $2$ is prime it implies that $\alpha(x)=1$ or $2$. It clearly cannot be $1$ since $1\not\in (2,x)$, so $\alpha(x)=2$. But then $x=2\beta(x)$ for some $\beta(x)\in\Z[x]$ which is impossible. So $(2,x)$ is not principal.    
    
    \textbf{(b)} Say a polynomial $f(x)\in \Z[x]$ is \emph{primitive} if the gcd of its coefficients is $1$. Clearly any polynomial $f(x)\in \Z[x]$ can be uniquely expressed as $f(x)=du(x)$ where $d$ is the gcd of all of the coefficients of $f$ and $u(x)$ is primitive. If we can prove that the product of two primitive polynomials is primitive, we'll have proved the claim since for $f,g\in \Z[x]$ and $f=d_1u_1$, $g=d_2u_2$ the product $fg=d_1d_2u_1u_2$ is the unique expression, so the gcd of $fg$ is the product of the gcd of $f$ and the gcd of $g$.
    
    Now suppose $f,g$ are primitive polynomials, and suppose for the sake of contradiction that $p\mid fg$ for some $d>0$, assume without loss of generality that $p$ is prime. Write $f(x)=a_0+a_1x+\cdots+a_rx^r$ and $g(x)=b_0+b_1x+\cdots+b_sx^s$. Let $a_i$ be the the first coefficient of $f$ not divisible by $p$ and let $b_j$ be the first coefficient of $g$ not divisible by $p$. Then the coefficient of $x^{i+j}$ in $fg$ is of the form $a_0b^{i+j}+a_1b^{i+j-1}+\cdots+a^ib^j+\cdots+a_{i+j}b_0$. This must be divisible by $p$ since it is a coefficient of $fg$, however every term except for $a^ib^j$ is divisible by $p$. This is a contradiction so $p=1$ and $fg$ is primitive.
    
    \textbf{(c)} Suppose $f$ were reducible over $\Q$, say $f(x)=\frac{a}{b}\alpha(x)\beta(x)$ where $\alpha(x),\beta(x)\in \Z[x]$ are primitive and $a,b$ are coprime. Then $b f(x)=a\alpha(x)\beta(x)$. The gcd of the left side is $b$ and the gcd of the right side is $a$ so $b=a$ and so $f(x)=\alpha(x)\beta(x)$. Thus $f(x)$ is reducible. This is the contrapositive of the required statement. 
    
    \textbf{(d)} Consider these as polynomials in $\Q[x]$. Then since $\Q[x]$ is a UFD we know that if $f\mid gh$ we have $f\mid g$ or $f\mid h$ in $\Q[x]$. Assume without loss of generality that $f\mid g$. This means that $g(x)=f(x)q(x)$ for some $q(x)\in \Q[x]$. We would like to show that $q(x)\in \Z[x]$ since this would imply that $f\mid g$ in $\Z[x]$. We can write $q(x)=\frac{a}{b}q'(x)$ for some $\frac{a}{b}\in \Q$ and primitive $q'(x)\in \Z[x]$. Then $g(x)=\frac{a}{b}f(x)q'(x)$. Since $f(x)q'(x)$ is primitive by (a), $\frac{a}{b}$ must be an integer, so $q(x)\in \Z[x]$ and we are done.   
    
    \textbf{(e)} Let $f(x)\in \Z[x]$ be some arbitrary polynomial. We can clearly decompose $f(x)$ into irreducibles $q_1(x)\cdots q_r(x)$. We can factor out the maximal $d\in \Z$ so that $f(x)=du_1(x)\cdots u_r(x)$ where the $u_i(x)$ are all primitive irreducibles. We can assume without loss of generality that the leading term of $u_i(x)$ are all positive. Then $f(x)=\pm p_1\cdots p_su_1(x)\cdots u_r(x)$ where $p_i\in \Z$ are positive primes. To prove uniqueness, suppose $f(x)=\pm q_1\cdots q_{s'} w_1(x)\cdots w_{r'}(x)$ for some different primes and primitive irreducibles. Since $\pm p_1\cdots p_s$ and $\pm q_1\cdots q_{s'}$ are the gcd's of the coefficients of $f(x)$, we know that $p_1\cdots p_s=q_1\cdots q_{s'}$ so $s=s'$ and $p_i$ and $q_i$ are the same primes, just reordered because $\Z$ is a UFD. So $u_1(x)\cdots u_r(x)=w_1(x)\cdots w_{r'}(x)$. We know that $u_1(x)\mid w_1(x)\cdots w_{r'}(x)$ so by (d), $u_1(x)=w_i(x)$ for some $i$. Assume without loss of generality that $i=1$ so $u_1(x)=w_1(x)$. Then by induction we can show that $r=r'$ and $u_i(x)=w_i(x)$. So the expression is unique up to some reordering of the primes.    
\end{solution}

\begin{cproblem}{3.11}
    Let $K$ be a number field, and $I$ a nonzero ideal in $\mathcal{O}_K$. Prove that $\|I\|$ divides $\nrm^K(\alpha)$ for all $\alpha\in I$, and equality holds iff $I=(\alpha)$. 
\end{cproblem}

\begin{solution}
    Recall from Theorem~3.22c that we have $\|(\alpha)\|=\nrm^K(\alpha)$. Now for any $\alpha\in I$, we have $(\alpha)\subset I$, so by the third ring isomorphism theorem we have $\|I\|=|\mathcal{O}_K /I|$ divides $\|(\alpha)\|=\mathcal{O}_K /(\alpha)|=\nrm^K(\alpha)$, completing the proof. 
\end{solution}

\begin{cproblem}{3.12}\noindent
    \begin{enumerate}[(a)]
        \item Verify that $5S=(5,\alpha+2)(5,\alpha^2+3\alpha-1)$ in the ring $S=\Z[\sqrt[3]{2}], \alpha=\sqrt[3]{2}$.
        \item Show that there is a ring isomorphism
        \[
            \Z[x]/(5,x^2+3x-1) \to \Z_5[x]/(x^2+3x-1)    
        .\] 
        \item Show that there is a ring homomorphism
        \[
            \Z[x]/(5,x^2+3x-1) \to S/(5,\alpha^2+3\alpha-1)
        .\] 
        \item Conclude that either $S /(5,\alpha^2+3\alpha-1)$ is a field of order $25$ or else $(5,\alpha^2+3\alpha-1)=S$.
        \item Show that $(5,\alpha^2+3\alpha-1)\neq S$ by considering (a). 
    \end{enumerate}
\end{cproblem}

\begin{solution}
    \textbf{(a)} Note that $I=(5,\alpha+2)(5,\alpha^2+3\alpha-1)=(25, 5(\alpha+2), 5(\alpha^2+3\alpha-1), (\alpha+2)(\alpha^2+3\alpha-1))$. However $(\alpha+2)(\alpha^2+3\alpha-1)=5\alpha^2+5\alpha$. So $5(\alpha+1)\in I$ and $5(\alpha+2)\in I$ so $5\alpha\in I$ and $5\in I$. So $5S=I$ since $1\not\in I$. 
    
    \textbf{(b)} This is true by the third isomorphism theorem for rings; let $R=\Z[x]$, $J=(5)$, $I=(5,x^2+3x-1)$. Then the third isomorphism theorem states that 
    \[
        \frac{R/J}{I/J}\cong \frac{R}{I}
    .\]
    Note that $I/J=(5,x^2+3x-1) / (5) = (x^2+3x-1)\Z_5[x]$. Thus $\Z[x]/(5,x^2+3x-1)\cong (\Z[x] /(5)) / (x^2+3x-1) = \Z_5[x]/(x^2+3x-1)$.
    
    \textbf{(c)} There is a surjective homomorphism $\varphi : \Z[x] \to S$ given by $f(x) \to f(\alpha)$. For any ideal $I\subset \Z[x]$, this induces a surjective homomorphism $\widetilde{\varphi} : \Z[x]/I \to S/\varphi(I)$ given by $f(x)+I \mapsto f(\alpha)+\varphi(I)$.

    \textbf{(d)} First, note that $\Z_5[x]/(x^2+3x-1)$ is a field of 25 elements because $x^2+3x-1$ is irreducible in $\Z_5[x]$. Using the map from (c), we thus know that $\Ima(\widetilde{\varphi}) = S/(5,\alpha^2+3\alpha-1)$ is a finite field of size 25 or size 1. (Field homomorphisms can either be injective or the zero map.) If it's a finite field of size 25, we are done. Otherwise, $S/(5,\alpha^2+3\alpha-1)=\{0\}$ so $(5,\alpha^2+3\alpha-1)$.
    
    \textbf{(e)} If $(5,\alpha^2+3\alpha-1)=S$, then by (a) we have $5S=(5,\alpha+2)S$ which is a contradiction because $\alpha+2\not\in 5S$. 

    This means that $(5)$ doesn't ramify in $\mathcal{O}_{\Q[\sqrt[3]{2}]}$. Checking LMFDB, we can actually see that $\mathcal{O}_{\Q[\sqrt[3]{2}]}=\Z[\sqrt[3]{2}]$. We also can show that $\Q[\sqrt[3]{2}]=\Q[\sqrt[3]{4}]$ and $\mathcal{O}_{\Q[\sqrt[3]{2}]}=\mathcal{O}_{\Q[\sqrt[3]{4}]}$, so $(5)$ doesn't ramify in $\mathcal{O}_{\Q[\sqrt[3]{4}]}$ either. 
\end{solution}

\begin{cproblem}{3.16}
    Let $K\subset L$ be number fields. Denote by $G(\mathcal{O}_K)$ and $G(\mathcal{O}_L)$ the ideal class groups of $K$ and $L$ respectively.
    \begin{enumerate}[(a)]
        \item Show that there is a homomorphism $G(\mathcal{O}_L) \to G(\mathcal{O}_K)$ defined by sending $[I]$ to $[\nrm^L_K(I)]$. 
        \item Let $\mathfrak{q}$ be a prime of $\mathcal{O}_L$ lying over a prime $\mathfrak{p}$ of $\mathcal{O}_K$. Let $d_{\mathfrak{q}}$ denote the order of the class containing $\mathfrak{q}$ in $G(\mathcal{O}_L)$, $d_{\mathfrak{p}}$ the order of the class containing $\mathfrak{p}$ in $G(\mathcal{O}_K)$. Prove that
        \[
            d_{\mathfrak{p}} \mid d_{\mathfrak{q}}f(\mathfrak{q}\mid \mathfrak{p})    
        .\]     
    \end{enumerate}
\end{cproblem}

\begin{solution}
    \textbf{(a)} Recall that if $I \subset \mathcal{O}_L$, with prime factorization $I=\mathfrak{q}_1\cdots \mathfrak{q}_n$, we define the norm of $I$ as
    \[
        \nrm^L_K(I)=\prod_{i=1}^n \mathfrak{p}_i^{f(\mathfrak{q}_i \mid \mathfrak{p}_i)}
    \]
    where $\mathfrak{q}_i$ lies above the prime $\mathfrak{p}_i\subset \mathcal{O}_K$. Now let $\varphi : G(\mathcal{O}_L) \to G(\mathcal{O}_K)$ be the map defined in the problem. First we have to show that it is a well defined map, so suppose $I,J\subset \mathcal{O}_L$ and $I\sim J$ i.e. there are $\alpha,\beta\in \mathcal{O}_L$ such that $\alpha I = \beta J$. This is equivalent to saying that $(\alpha)I=(\beta)J$ so $\nrm^L_K(\alpha)\nrm^L_K(I)=\nrm^L_K(\beta)\nrm^L_K(J)$ so $\nrm^L_K(I)\sim \nrm^L_K(J)$. It's clearly a homomorphism because $$\varphi([I][J])=[\nrm^L_K(IJ)]=[\nrm^L_K(I)][\nrm^L_K(J)]=\varphi([I])\varphi([J]).$$ 
    
    \textbf{(b)} Let $[\mathfrak{q}]$ be the class of $\mathfrak{q}\in G(\mathcal{O}_L)$ and $[\mathfrak{p}]$ be the class of $\mathfrak{p}\in G(\mathcal{O}_K)$. Then $\varphi([\mathfrak{q}])=[\mathfrak{p}]^{f(\mathfrak{q}\mid\mathfrak{p})}$. By Lagrange's theorem, $[\mathfrak{q}]^{d_{\mathfrak{q}}}=e_{G(\mathcal{O}_L)}$ so $\varphi([\mathfrak{q}]^{d_{\mathfrak{q}}})=[\mathfrak{p}]^{d_{\mathfrak{q}}f(\mathfrak{p}\mid\mathfrak{q})}=e_{\mathcal{O}_K}$. So again by Lagrange's theorem, we have $d_{\mathfrak{p}} \mid d_{\mathfrak{q}}f(\mathfrak{q} \mid\mathfrak{p})$ as desired. 
\end{solution}

\begin{cproblem}{3.19} Let $K\subset L$ be number fields. Let $\mathfrak{p}$ be a prime of $\mathcal{O}_K$.
    \begin{enumerate}[(a)]
        \item Show that if $\alpha\in \mathcal{O}_L$ and $\beta\in \mathcal{O}_K$, and $\alpha\beta\in \mathfrak{p} \mathcal{O}_L$, then either $\alpha\in \mathfrak{p} \mathcal{O}_L$ or $\beta\in \mathfrak{p}$.
        % Recall that OL / pOL is a vector space over OK / p.
        \item Let $\alpha, \alpha_1, \ldots, \alpha_n\in \mathcal{O}_L$; $\beta,\beta_1,\ldots,\beta_n\in \mathcal{O}_K$, and $\alpha\not\in \mathfrak{p}\mathcal{O}_L$. Suppose $\alpha\beta=\alpha_1\beta_1+\cdots+\alpha_n\beta_n$. Prove that there exists $\gamma\in K$ such that $\beta\gamma$ and all of the $\beta_i\gamma$ are in $\mathcal{O}_K$ and the $\beta_i\gamma$ are not all in $\mathfrak{p}$. 
        % Hint: see proof of Theorem 22b 
        \item Prove the following generalization of Theorem~3.24: Let $\alpha_1,\ldots,\alpha_n$ be a basis for $L$ over $K$ consisting entirely of members of $\mathcal{O}_L$, and let $\mathfrak{p}$ be a prime of $\mathcal{O}_K$ which is ramified in $\mathcal{O}_L$. Then $\disc^L_K(\alpha_1,\ldots,\alpha_n)\in \mathfrak{p}$. 
        % See exercise 23 chapter 2 for the definition and properties of relative discriminant
    \end{enumerate}
\end{cproblem}

\begin{solution}
    \textbf{(a)} Recall that $\mathcal{O}_L / \mathfrak{p} \mathcal{O}_L$ is a $K_{\mathfrak{p}}$-vector space, where $K_{\mathfrak{q}}$ is the residue field of $\mathfrak{q}$. Then if $\alpha\beta\in \mathfrak{p} \mathcal{O}_L$, this means that $\alpha\beta = 0 \in \mathcal{O}_L / \mathfrak{p} \mathcal{O}_L$, so by the properties of a vector space either $\alpha = 0\in \mathcal{O}_L / \mathfrak{p} \mathcal{O}_L$ or $\beta = 0\in K_{\mathfrak{p}}$. These conditions are equivalent to $\alpha\in \mathfrak{p} \mathcal{O}_L$ and $\beta\in \mathfrak{p}$ as desired.  
    
    \textbf{(b)} We can assume that all of the $\beta_i$ are in $\mathfrak{p}$ otherwise $\gamma=1$ would work. Then since $\alpha\beta\in \mathfrak{p}\mathcal{O}_L$ yet $\alpha\not\in \mathfrak{p}\mathcal{O}_L$ so by (a), $\beta\in \mathfrak{p}$. Recall the lemma from the proof of Theorem 3.22(b):
    \begin{ilemma}
        Let $A$ and $B$ be nonzero ideals in a Dedekind domain $R$, with $B \subset A$ and $A \neq R$. Then there exists $\gamma\in K$ such that $\gamma B\subset R$, $\gamma B \not\subset A$.
    \end{ilemma}
    Letting $A=\mathfrak{p}$ and $B=(\beta, \beta_1,\ldots,\beta_n)$ so that $B\subset A$, and $A\neq \mathcal{O}_K$ so we can apply the lemma to get a $\gamma\in K$ with $\gamma B \subset \mathcal{O}_K$ and $\gamma B\not\subset \mathfrak{p}$. Suppose for the sake of contradiction that $\gamma \beta_i\in \mathfrak{p}$. Then $\gamma B\subset \mathfrak{p}$, so $\alpha(\gamma \beta)=\alpha_1(\gamma \beta_1)+\cdots+\alpha_n (\gamma \beta_n)\in \mathfrak{p} \mathcal{O}_L$. However $\alpha\not\in \mathfrak{p} \mathcal{O}_L$, so $\gamma \beta\in \mathfrak{p}$ by (a). However $\gamma\beta\not\in \mathfrak{p}$ since $\gamma B\not\subset \mathfrak{p}$. This is a contradiction, so one of the $\gamma\beta_i\in \mathfrak{p}$.
    
    \textbf{(c)} Pick some prime $\mathfrak{q}$ lying over $\mathfrak{p}$ satisfying $e(\mathfrak{q}\mid\mathfrak{p})>1$. Then $\mathfrak{p}\mathcal{O}_L=\mathfrak{q}I$ for some ideal $I\subset \mathcal{O}_L$. Finally, let's pick some $\alpha\in I-\mathfrak{p}\mathcal{O}_L$ so since $I\subset \mathfrak{q}$, $\alpha$ is in every prime of $\mathcal{O}_L$ lying over $\mathfrak{p}$ but $\alpha\not\in \mathfrak{p}\mathcal{O}_L$. Write $\alpha=m_1\alpha_1+\cdots+m_n\alpha_n$ for some $m_i\in K$. Then there is some $\beta\in \mathcal{O}_K$ such that $\beta m_i\in \mathcal{O}_K$. Then 
    \[
        \alpha\beta= (m_1\beta)\alpha_1+(m_2\beta)\alpha_2+\cdots+(m_n\beta)\alpha_n
    .\] 
    Then since $\alpha\not\in \mathfrak{p}\mathcal{O}_L$, by (b) there is some $\gamma\in K$ such that $\beta\gamma$ and all of the $m_i\beta\gamma\in \mathcal{O}_K$ and the $m_i\beta\gamma$ are not all in $\mathfrak{p}$. Let $\kappa=\beta\gamma$ and $\kappa_i=m_i\beta\gamma$ so that
    \[
        \alpha\kappa = \alpha_1\kappa_1+\alpha_2\kappa_2+\cdots+\alpha_n\kappa_n
    .\]    
    By definition, not all of the $\kappa_i\in \mathfrak{p}$, so say without loss of generality that $\kappa_1\not\in \mathfrak{p}$. Then by a set of column operation and using the fact that $\kappa_i\in \mathcal{O}_K$, we get 
    \[
        \disc^L_K(\alpha,\alpha_2,\ldots,\alpha_n) = \kappa_1^2\disc^L_K(\alpha_1,\ldots,\alpha_n)
    .\] 
    So since $\kappa_1\not\in \mathfrak{p}$, it suffices to show that $\disc^L_K(\alpha, \alpha_2,\ldots,\alpha_n)\in \mathfrak{p}$.  Let $M$ be some extension of $L$ which is normal over $K$, and let $\sigma_1,\ldots,\sigma_n$ be the embeddings of $L$ in $\C$ fixing $K$. Recall that these can be extended to embeddings $\overline{\sigma_1},\ldots,\overline{\sigma_n} : M \to \C$ which fix $K$ and agree with $\sigma_i$ on $L$. Let $\mathfrak{P}$ be some prime in $\mathcal{O}_M$ lying over $\mathfrak{p}$ with $e(\mathfrak{P}\mid \mathfrak{p})>1$. (Picking any prime lying over $\mathfrak{q}$ works) Then $\mathfrak{p}\mathcal{O}_M=\mathfrak{P}J$ for some ideal $J\subset \mathfrak{P}$. We claim that $\overline{\sigma_i}(\alpha)\in \mathfrak{P}$ for all $i$. Note that $(\overline{\sigma_i})^{-1}(\mathfrak{P})$ is a prime of $\mathcal{O}_M$ lying over $\mathfrak{p}$, so $\alpha\in (\overline{\sigma_i})^{-1}(\mathfrak{P})$ and thus $\overline{\sigma_i}(\alpha)=\sigma_i(\alpha)\in \mathfrak{P}$. This implies that $\disc^L_K(\alpha,\alpha_2,\ldots,\alpha_n)\in \mathfrak{P}$. However since $\disc^L_K(\alpha,\alpha_2,\ldots,\alpha_n)\in \mathcal{O}_K$, it follows that $\disc^L_K(\alpha,\alpha_2,\ldots,\alpha_n)\in \mathfrak{p}$ since $\mathfrak{P}\cap \mathcal{O}_K = \mathfrak{p}$. 
\end{solution}

\end{document}