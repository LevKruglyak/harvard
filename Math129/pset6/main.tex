\documentclass[11pt,letterpaper]{article}

\input{../../../../.config/latex/preamble_v1.tex}
\lightmode

\title{\textbf{Math 129 Problem Set 6}}

\begin{document}
\maketitle

\begin{center}
    \textit{I collaborated with Ignasi Vicente for this problem set.}
\end{center}

\begin{cproblem}{}[Spec]\noindent
    \begin{enumerate}[(a)]
        \item Show that if $f : R \to S$ is any ring homomorphism (assuming $f(1)=1$), there is an induced map of sets $\widetilde{f} : \Spec(S) \to \Spec(R)$.
        \item Find an example of a ring homomorphism that isn't an isomorphism of rings, but induces a bijection of spectrums.
        \item Describe $\widetilde{f} : \Spec(S) \to \Spec(R)$ when $R=\C[t]$ and $S=\C[t,s] /(s^2-t)$, where $f : \C[t] \to \C[t,s]/(s^2-t)$ is the natural inclusion. 
    \end{enumerate}
\end{cproblem}

\begin{solution}
    \textbf{(a)} Let $\mathfrak{q}\subset S$ be a prime ideal, and let $\mathfrak{p}=f^{-1}(\mathfrak{q})$. We claim that $\mathfrak{p}$ is a prime ideal of $R$. To see this, let $ab\in \mathfrak{p}$. This means that $f(ab)=f(a)f(b)\in \mathfrak{q}$ so $f(a)\in \mathfrak{q}$ or $f(b)\in \mathfrak{q}$. This means that $a\in \mathfrak{p}$ or $b\in \mathfrak{p}$, so $\mathfrak{p}$ is a prime ideal. Thus, we can define $\widetilde{f} : \mathfrak{q} \mapsto f^{-1}(\mathfrak{q})$.
    
    \textbf{(b)} Consider the natural reduction map from $\Z /4\Z$ to $\Z/2\Z$. The only prime ideal of $\Z /4\Z$ is $(2)$ and the only prime ideal of $\Z/2\Z$ is $(0)$. Thus the reduction map induces a bijection between $\{(0)\}$ and $\{(2)\}$.  

    \textbf{(c)} First we'll calculate $\Spec(\C[t])$. Note that $\C$ is a field so $\C[t]$ is a principal ideal domain. Thus every ideal is of the form $(f(t))$ for some polynomial $f(t)\in \C[t]$. Thus, the prime ideals in $\C[t]$ are $(x-a)$ for $a\in \C$. 
    
    \begin{claim}
        Let $R$ be a ring and $I$ an ideal in $R$. Let $f : R \to R/I$ be the natural surjection. Then $\widetilde{f} : \Spec(R/I) \to \Spec(R)$ is an inclusion mapping prime ideals in $\Spec(R /A)$ to prime ideals in $\Spec(R)$ containing $I$. 
    \end{claim}
    \begin{proof}
        This follows from the correspondence theorem and the third isomorphism; note that $\mathfrak{p}\subset R / I$ then $f^{-1}(\mathfrak{p})$ is an ideal of $R$ containing $I$. Since $(R/I) /\mathfrak{p}$ is an integral domain, so is $R /f^{-1}(\mathfrak{p}) = R / (If^{-1}(\mathfrak{p}))$. 
    \end{proof}
    
    Since $\Spec(\C[t,s])$ consists of ideals $(t-a, s-b)$ for $a,b\in \C$, the claim implies that the spectrum $\Spec(\C[t,s] /(s^2-t))$ consists of the prime ideals of the form $(s-a)$ for $a\in \C$. Note that $f^{-1}((s-a))=(t-a^2)$, so $\widetilde{f}$ sends $(t-a)$ to $(t-a^2)$ and obviously $(0)$ is sent to $(0)$.
\end{solution}

\begin{cproblem}{3.28}
    Let $f(x)=x^n+a_{n-1}x^{n-1}+\cdots+a_0$, all $a_i\in \Z$, and let $p$ be a prime divisor of $a_0$. Let $p^r$ be the exact power of $p$ dividing $a_0$, and suppose all $a_i$ are all divisible by $p^r$. Assume moreover that $f$ is irreducible over $\Q$ (which is automatic if $r=1$) and let $\alpha$ be a root of $f$. Let $K=\Q[\alpha]$.
    \begin{enumerate}[(a)]
        \item Prove that $(p^r)=p^r \mathcal{O}_K$ is the $n^{\text{th}}$ power of an ideal in $\mathcal{O}_K  $. %\textit{First show that $\alpha^n=p^r\beta$ with $(\beta)$ relatively prime to $(p)$}
        \item Show that if $r$ is relatively prime to $n$, then $(p)$ is the $n^{\text{th}}$ power of an ideal in $R$. Conclude that in this case $p$ is totally ramified in $\mathcal{O}_K$.
        \item Show that if $r$ relatively prime to $n$, then $\Delta_{K}$ is divisible by $p^{n-1}$. What can you prove if $(n,r)=m>1$? %See problem 21
    \end{enumerate}
\end{cproblem}

\begin{solution}
    \textbf{(a)} Since $f(\alpha)=0$, we can write
    \[
        \alpha^n=-a_{n-1}\alpha^{n-1}-\cdots-a_1\alpha-a_0=p^r\left(-\frac{a_{n-1}}{p^r}\alpha^{n-1}-\cdots-\frac{a_1}{p^r}\alpha-\frac{a_0}{p^r}\right)
    .\]   
    Let's call this last term $\beta$ so that $\alpha^n=p^r\beta$. Note that all of the terms $a_{i}/p^r$ are integers, and $p\nmid a_0/p^r$. Let $\beta_i = -a_i /p^r$, so that $\beta = \beta_{n-1}\alpha^{n-1}+\cdots+\beta_1\alpha+\beta_0$. Note that $p\nmid \beta$, since otherwise we would have some polynomial $g(x)\in \Z[x]$ with $g(\alpha)=0$ and $\deg g < \deg f$, a contradiction to the irreducibility of $f$. So $(p^r)$ is coprime to $(\beta)$. Then since $(\alpha)^n=(p^r)(\beta)$, it follows that $(p^r)=I^n$ for some ideal $I\subset \mathcal{O}_K$. 
    
    \textbf{(b)} Write $(p)=\mathfrak{p}^{e_1}_1\cdots \mathfrak{p}^{e_k}_k$ for $\mathfrak{p}_i$ prime. Then $(p^r)=(p)^r=\mathfrak{p}^{re_1}_1\cdots \mathfrak{p}^{re_k}_k$. Since $(p^r)=I^n$ for some ideal by (a), it follows that $n\mid re_i$ for all $i$. Since $(n,r)=1$, we have $n\mid e_i$ for all $i$. Thus $(p)$ is an $n$-th power of an ideal of $\mathcal{O}_K$. By the decomposition equation $\sum_{i=1}^k e_if_i=n$ yet $n\mid e_i$ so $e_i\geq n$. This means that $k=1$, $e_1=n$, and $f_1=1$. Thus $p\mathcal{O}_K=\mathfrak{p}_1^n$ so $p$ is totally ramified.
    
    \textbf{(c)} We'll address the case when $r$ is relatively prime to $n$ first. By (b), $p\mathcal{O}_K=\mathfrak{p}^n$ for a prime $\mathfrak{p}\subset \mathcal{O}_K$. By the decomposition equation we have $f(\mathfrak{p}\mid p)=1$. By Problem~3.21b, $\Delta_K$ is divisible by $p^k$ for $k=n-f(\mathfrak{q}\mid p) = n-1$. So if $r$ is relatively prime to $n$ then $p^{n-1}\mid \Delta_K$.
    
    In the case when $(n,r)=m>1$, let $p\mathcal{O}_K=\mathfrak{p}^{e_1}_1\cdots \mathfrak{p}_k^{e_k}$. Then $n\mid e_i r$ so $e_i$ is a multiple of $n /m$. By the decomposition equation $e_1f_1+\cdots+e_kf_k=n$. Then $\sum_i f_i \leq m$, and this maximum is achieved when all $e_i=n /m$. Then by Problem~3.21b, we have $p^{n-m}\mid \Delta_K$. 
\end{solution}

\begin{cproblem}{4.1}
    Show that $E(\mathfrak{q}\mid \mathfrak{p})$ is a normal subgroup of $D(\mathfrak{q}\mid \mathfrak{p})$ directly from the definition of these groups.
\end{cproblem}

\begin{solution}
    Let $\sigma\in E(\mathfrak{q}\mid \mathfrak{p})$ be some automorphism. By definition of the inertia group we have $\sigma(\alpha)-\alpha\in \mathfrak{q}$ for all $\alpha\in \mathcal{O}_L$. Then for any $\zeta\in D(\mathfrak{q}\mid \mathfrak{p})$ since $\zeta^{-1}\in \Gal(L /K)$, it follows that $\zeta^{-1}(\alpha)\in \mathcal{O}_L$ so $\zeta(\sigma^{-1}(\alpha))-\sigma^{-1}(\alpha)\in \mathfrak{q}$. Since $\zeta$ preserves the prime $\mathfrak{q}$, we have $\zeta(\sigma(\zeta^{-1}(\alpha))-\sigma^{-1}(\alpha))=\zeta\sigma\zeta^{-1}(\alpha)-\alpha$. Thus $\zeta\sigma\zeta^{-1}\in E(\mathfrak{q}\mid \mathfrak{p})$. This proves the normality of $E(\mathfrak{q}\mid \mathfrak{p})$ in $D(\mathfrak{q}\mid \mathfrak{p})$.
\end{solution}

\begin{cproblem}{4.2}
    Suppose $D(\mathfrak{q}\mid \mathfrak{p})$ is a normal subgroup of $\Gal(L/K)$. Then $\mathfrak{p}$ splits into $r$ distinct primes in $L_{D(\mathfrak{q}\mid \mathfrak{p})}$. If $E(\mathfrak{q}\mid \mathfrak{p})$ is also normal in $\Gal(L /K)$, then each of them remains prime (is ``inert'') in $L_{E(\mathfrak{q}\mid \mathfrak{p})}$ . Finally, each one becomes an $e^{\text{th}}$ power in $L$.
\end{cproblem}

\begin{solution}
    If $D(\mathfrak{q}\mid \mathfrak{p})$ is normal in $\Gal(L /K)$, then by the fundamental theorem of Galois theory, $L_{D(\mathfrak{q}\mid \mathfrak{p})}$ is a normal extension of $K$. We know that $\mathfrak{q}_{D(\mathfrak{q}\mid \mathfrak{p})}$ has ramification index and inertial degree $1$ over $\mathfrak{p}$, hence so does every prime $\mathfrak{p}'$ in $L_{D(\mathfrak{q}\mid \mathfrak{p})}$ lying over $\mathfrak{p}$. So there must be exactly $r$ such primes. It follows that there are exactly $r$ primes in $L_{E(\mathfrak{q}\mid \mathfrak{p})}$ lying over $\mathfrak{p}$ since this is true in both $L_{D(\mathfrak{q}\mid \mathfrak{p})}$ and $L$. This implies that each $\mathfrak{p}$ lies under a unique primes $\mathfrak{p}''$ in $L_{E(\mathfrak{q}\mid \mathfrak{p})}$; however $\mathfrak{p}''$ might be ramified over $\mathfrak{p}'$. If $E(\mathfrak{q}\mid \mathfrak{p})$ is normal in $\Gal(L /K)$, then $e(\mathfrak{p}''\mid \mathfrak{p})=e(\mathfrak{q}_{E(\mathfrak{q}\mid \mathfrak{p})})=1$ hence $e(\mathfrak{p}''\mid \mathfrak{p}')=1$. This proves that $\mathfrak{p}'$ is inert in $L_{E(\mathfrak{q}\mid \mathfrak{p})}$, i.e. $\mathfrak{p}'' = \mathfrak{p}'(\mathcal{O}_L)_{E(\mathfrak{q}\mid \mathfrak{p})}$. 
    
    We claim that $\mathfrak{p}''$ becomes an $e^{\text{th}}$ power in $L$. Let $\mathfrak{q}''$ be a prime of $L$ lying over $\mathfrak{p}''$. By transitivity, $\mathfrak{q}''$ lies over $\mathfrak{p}$ so we have $e=e(\mathfrak{q}''\mid \mathfrak{p})=e(\mathfrak{q}''\mid \mathfrak{p}'')e(\mathfrak{p}''\mid \mathfrak{p}')e(\mathfrak{p}'\mid \mathfrak{p})$. Earlier, we showed that $e(\mathfrak{p}''\mid \mathfrak{p}')=e(\mathfrak{p}'\mid \mathfrak{p})=1$; thus $e=e(\mathfrak{q}''\mid \mathfrak{p}'')$. So $\mathfrak{p}''\mathcal{O}_L = (\mathfrak{q}_1\cdots \mathfrak{q}_k)^e$ where $\mathfrak{q}_i$ are the primes of $L$ lying over $\mathfrak{p}''$. Hence $\mathfrak{p}''$ is an $e^{\text{th}}$ power in $L$.     
\end{solution}

\begin{cproblem}{4.10}
    Let $K$ be a number field, and let $L$ and $M$ be two finite extensions of $K$. Assume that $M$ is normal over $K$. Then the composite field $LM$ is normal over $L$ and the Galois group $\Gal(LM/L)$ is embedded in $\Gal(M /K)$ by restricting automorphisms to $M$. Let $\mathfrak{p}\subset \mathcal{O}_K, \mathfrak{n}\subset \mathcal{O}_L, \mathfrak{m}\subset \mathcal{O}_M,$ and $\mathfrak{q}\subset \mathcal{O}_{LM}$ be primes such that $\mathfrak{n}$ lies over $\mathfrak{q}$ and $\mathfrak{m}$ and $\mathfrak{q}$ and $\mathfrak{m}$ lie over $p$.
    \begin{enumerate}[(a)]
        \item Prove that $D(\mathfrak{q}\mid \mathfrak{n})$ is embedded in $D(\mathfrak{m}\mid \mathfrak{p})$ by restricting automorphisms.
        \item Prove that $E(\mathfrak{q}\mid \mathfrak{n})$ is embedded in $E(\mathfrak{m}\mid \mathfrak{p})$ by restricting automorphisms.
        \item Prove that if $\mathfrak{p}$ is unramified in $M$, then every prime of $L$ lying over $\mathfrak{p}$ is unramified in $LM$.
    \end{enumerate}
\end{cproblem}

\begin{solution}
    \textbf{(a)} Firstly if $\sigma\in D(\mathfrak{q}\mid \mathfrak{n})$, then by definition $\sigma(\mathfrak{q}) = \mathfrak{q}$. Let $\overline{\sigma}\in \Gal(M /K)$ be the restriction of $\sigma$ to $M\subset LM$. Then $\overline{\sigma}(\mathfrak{q}\cap M)=\mathfrak{q}\cap M$ however $\mathfrak{q}\cap M=\mathfrak{m}$ so $\overline{\sigma}(\mathfrak{m})=\mathfrak{m}$. This gives us a well defined map $D(\mathfrak{q}\mid \mathfrak{n}) \to D(\mathfrak{m}\mid \mathfrak{p})$. To prove that this map is injective, suppose $\overline{\sigma}$ is the identity on $M$, then $\sigma$ is the identity on $M$. Similarly, $\sigma$ is the identity on $L$ so it must be the identity on the composite field $LM$. Thus there is an imbedding $D(\mathfrak{q}\mid \mathfrak{n})\hookrightarrow D(\mathfrak{m}\mid \mathfrak{p})$.        

    \textbf{(b)} By (a), the natural restriction map $\Gal(LM /L)$ to $\Gal(M / K)$ is an injective homomorphism, so it suffices to show that the image of $E(\mathfrak{q}\mid \mathfrak{n})$ under this map is $E(\mathfrak{m}\mid \mathfrak{p})$. Let $\sigma\in E(\mathfrak{q}\mid \mathfrak{n})$. Then if $\sigma(\alpha)-\alpha\in \mathfrak{q}$ for all $\alpha\in \mathcal{O}_{LM}$, then for all $\alpha\in \mathcal{O}_M\subset \mathcal{O}_{LM}$ we have $\sigma(\alpha)-\alpha\in \mathfrak{q}\cap M = \mathfrak{m}$. Thus $\overline{\sigma}\in E(\mathfrak{m}\mid \mathfrak{p})$, and we have our embedding.  

    \textbf{(c)} Suppose $\mathfrak{p}$ is unramified in $M$. Let $\mathfrak{n}$ be a prime of $\mathcal{O}_L$ lying over $\mathfrak{p}$, and let $\mathfrak{q}$ be a prime of $\mathcal{O}_{LM}$ lying over $\mathfrak{n}$. Since $M$ is Galois, Theorem~4.28 gives us the degree relation $e(\mathfrak{m}\mid \mathfrak{p})=e(\mathfrak{m}\mid \mathfrak{m}_{E(\mathfrak{m}\mid \mathfrak{p})}) = [M : M_{E(\mathfrak{m}\mid \mathfrak{p})}]$ where $\mathfrak{m}=\mathfrak{q}\cap \mathcal{O}_M$. Since $\mathfrak{p}$ is unramified, $e(\mathfrak{m}\mid \mathfrak{p})=1$ so $[M : M_{E(\mathfrak{m}\mid \mathfrak{p})}]=|E(\mathfrak{m}\mid \mathfrak{p})|=1$. By (b), $|E(\mathfrak{q}\mid \mathfrak{n})|\leq |E(\mathfrak{m}\mid \mathfrak{p})|=1$ so $e(\mathfrak{q}\mid \mathfrak{n})=1$. Since $LM$ is Galois over $L$, applying Theorem~4.28 gives us $e(\mathfrak{q}\mid \mathfrak{n})=[LM : (LM)_{E(\mathfrak{q}\mid \mathfrak{n})}]=|E(\mathfrak{q}\mid \mathfrak{n})|=1$. This proves that $\mathfrak{n}$ is unramified in $LM$.   
\end{solution}

\end{document}
