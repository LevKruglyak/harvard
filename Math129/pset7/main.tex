\documentclass[11pt,letterpaper]{article}

\input{../../../../.config/latex/preamble_v1.tex}
\lightmode

\title{\textbf{Math 129 Problem Set 7}}

\begin{document}
\maketitle

\begin{cproblem}{5.2}
    Let $\Lambda$ be an $n$-dimensional lattice in $\R^n$ and let $\{v_1,\ldots,v_n\}$ and $\{w_1,\ldots,w_n\}$ be any two $\Z$-bases for $\Lambda$. Prove that the absolute value of the determinant formed by taking the $v_i$ as the rows is equal to the one formed from the $w_i$. This shows that $\textrm{vol}(\R^n /\Lambda)$ can be defined unambiguously.  
\end{cproblem}

\begin{solution}
    Let $T_v : \R^n \to \R^n$ be the invertible linear transformation which takes the unit vector $e_i$ to $v_i$. Similarly construct $T_w : \R^n \to \R^n$. Then the determinants in question are equal to $|\det T_v|$ and $|\det T_w|$. Since $w,v$ are $\Z$-bases for $\Lambda$, we know that $T_v(\Z^n)=T_w(\Z^n)=\Lambda$. Then $T_w\circ T_v^{-1} : \Lambda \to \Lambda$ is an invertible $\Z$-linear map, so $|\det T_w\circ T_v^{-1}| = 1$. However by elementary properties of the determinant, $|\det T_w\circ T_v^{-1}| = |\det T_w|/ |\det T_v|$ so $|\det T_v|=|\det T_w|$ as desired. 
\end{solution}

\begin{cproblem}{5.3}
    Let $\Lambda$ be as in the previous exercise and let $M$ be any $n$-dimensional sublattice of $\Lambda$. Prove that
    \[
        \textrm{vol}(\R^n / M) = |\Lambda / M|\cdot \textrm{vol}(\R^n/\Lambda)
    .\] 
\end{cproblem}
\begin{solution}
    Let $T$ be an invertible linear transformation $T : \R^n \to \R^n$ with $T(\Z^n) = \Lambda$. Similarly let $H : \R^n \to \R^n$ be an invertible linear transformation with $H(\Lambda) = M$. Then $|\det H\circ T| =|\det H||\det T|$, however $(H\circ T)(\Z^n)=M$ so $|\det(H\circ T)| =\textrm{vol}(\R^n /M)$. Similarly $|\det T| = \textrm{vol}(R^n /\Lambda)$, so it suffices to show that $|\det H| = | \Lambda / M|$. Note that by Problem~2.27b, there is a basis $\beta_1,\ldots,\beta_n$ of $\Lambda$ such that $d_1\beta_1,\ldots,d_n\beta_n$ is a basis for $M$. Then clearly the determinant of $H$ is equal to $d_1\cdots d_n$. Similarly, $\Lambda / M = (\Z/d_1\Z)\oplus\cdots (\Z /d_n\Z)$ so $|\Lambda /M|=d_1\cdots d_n$. This concludes the proof. 
\end{solution}

\begin{cproblem}{5.4}
    Prove that the subset of $S\subset \R^n$ defined by the inequalities
    \[
        |x_1|+\cdots+|x_r| + 2\left(\sqrt{x^2_{r+1} + x^2_{r+2}}+\cdots +\sqrt{x^2_{n-1} + x^2_n}\right)\leq n
    \] 
    is convex.
\end{cproblem}
\begin{solution}
    First we'll show that $S$ is midpoint convex. Suppose $x,y\in S$. We claim that $\frac{x+y}{2}\in S$. First, note that by the triangle inequality on $\R$ we have $\left|\frac{x_i+y_i}{2}\right|\leq \frac{|x_i|+|y_i|}{2}$. Similarly, using the triangle inequality on $\R^2$, we have
    \[
        \sqrt{\left(\frac{x_i+y_i}{2}\right)^2 + \left(\frac{x_{i+1}+y_{i+1}}{2}\right)^2} \leq \frac{1}{{2}}\sqrt{x_i^2+x_{i+1}^2} + \frac{1}{{2}}\sqrt{y_i^2+y_{i+1}^2}
    .\]  
    Adding the inequalities for $x$ and $y$ together, and using the triangle inequalities, we thus get,
    \[
        \begin{aligned}
            \sum^r_{i=1}\left|\frac{x_i+y_i}{2}\right| + 2\sum^{n}_{\substack{i=r+1\\ j = r+2}}\sqrt{\left(\frac{x_i+y_i}{2}\right)^2+\left(\frac{x_j+y_j}{2}\right)^2} \leq\\
            \frac{1}{2}\left(\sum^r_{i=1} |x_i|+|y_i| + 2\sum^n_{\substack{i=r+1\\ j = r+2}}\sqrt{x_i^2+x_j^2}+\sqrt{y_i^2+y_j^2}\right) \leq n
        \end{aligned}
    \] 
    So $S$ is midpoint convex. Now suppose $\theta = t x + (1-t)y$ is some convex combination for $t\in [0,1]$. By taking successive midpoints of $x,y$, we can construct a sequence of elements of $S$ which converges to $\theta$. Since $S$ is a closed set, $\theta$ must thus be in $S$ as well. So $S$ is convex. 
\end{solution}

\begin{cproblem}{5.5}
    Prove by induction that
    \[
        \frac{n^n}{n!}\geq 2^{n-1}
    .\] 
    Use this to show that $|\Delta_K| \geq 4^{r-1}\pi^{2s}$, and that $|\Delta_K| > 1$ whenever $K\neq \Q$.
\end{cproblem}

\begin{solution}
    The base case of $n=1$ is clear, since $1\geq 1$. Now suppose the inequality works for $n-1$ for some $n\geq 2$. Then
    \[
        2^{n-1} \leq n\cdot 2^{n-2} \leq \frac{n(n-1)^{(n-1)}}{(n-1)!} \leq \frac{n\cdot n^{(n-1)}}{(n-1)!} \leq \frac{n^n}{n!}
    .\] 
    Then by Corollary~5.2 we have
    \[
        \sqrt{|\Delta_K|} \geq \frac{n^n}{n!} \left(\frac{\pi}{4}\right)^s \geq 2^{r+2s-1} \frac{\pi^s}{2^{2s}} = 2^{r-1}\pi^s
    .\]
    Thus $|\Delta_K|\geq 4^{r-1}\pi^{2s}$. Note that for integers $r,s\geq 0$, $4^{r-1}\pi^{2s}\geq 1$, with equality occurring only if $(r,s)=(1,0)$. This means that $n=1$, so the only number field satsfying this is $K=\Q$. Thus if $K\neq \Q$, we have $|\Delta_K|>1$.
\end{solution}

\begin{cproblem}{5.6}
    Show that $\mathcal{O}_{\Q[\sqrt{m}]}$ is a principal ideal domain when $m=2,3,5,6,7,173,293,$ or $437$.
\end{cproblem}

\begin{solution}
    Recall that the discriminant of a quadratic number field $\Q[\sqrt{m}]$ is given by
    \[
        \Delta_{\Q[\sqrt{m}]}=\begin{cases}
            4m&m\equiv 2,3\mod 4\\
            m&m\equiv 1\mod 4
        \end{cases}
    \] 
    Also for every ideal class of $\mathcal{O}_{\Q[\sqrt{m}]}$, there is some ideal $J$ with
    \[
        \|J\| \leq \frac{n!}{n^n}\left(\frac{4}{\pi}\right)^s\sqrt{|\Delta_{\Q[\sqrt{m}]}|}=\frac{1}{2}\sqrt{|\Delta_{\Q[\sqrt{m}]}|}=\lambda(m)
    .\] 
    Calculating this Minkowski bound for $m=2,3,5$, we get $\lambda(2)\approx 1.41$, $\lambda(3)\approx 1.73$, $\lambda(5)\approx 1.12$. In all these cases, every ideal class in $\mathcal{O}_{\Q[\sqrt{m}]}$ contains an ideal of norm $1$, so every ideal class is principal. For $m=7$, we have $\lambda(7)\approx 2.29$. All ideal classes containing $\|J\|=1$ are principal so we only need to consider ideal classes containing $\|J\|=2$. It suffices to only look at prime ideals with norm less than or equal to $2$. 

    Note that $2\mathcal{O}_{\Q[\sqrt{7}]}=(2,1+\sqrt{7})$ since $x^2-7\equiv x^2+1\equiv (x+1)^2\mod 2$. Then $\|(2,1+\sqrt{7})\|=2$ and the ideal is prime. However we also have the factorization $2=(3+\sqrt{7})(3-\sqrt{7})$. Note that
    \[
        \frac{3+\sqrt{7}}{3-\sqrt{7}} = \frac{(3+\sqrt{7})^2}{2} = 8+3\sqrt{7}\in \mathcal{O}_{\Q[\sqrt{7}]}^\times
    \]
    So $(3+\sqrt{7})=(3-\sqrt{7})=\mathfrak{p}$ and hence every ideal is principal.
\end{solution}

\begin{cproblem}{}[Proof Explanation]
    Our goal is to prove the following correspondence for rational primes $p$:
    \[
        \{\textrm{ideals }p\mathcal{O}_K\textrm{ which split in }\mathcal{O}_K\} \Longleftrightarrow \{p \mid \Delta_K\}
    .\] 
\end{cproblem}

\begin{solution}
    The proof starts by describing the determinant $\Delta_K$ as the determinant $|\textrm{T}^K_{\Q}(\alpha_i\alpha_j)|$ where $\{\alpha_i\}$ is an integral basis for $\mathcal{O}_K$. Let's consider this determinant over the field $\F_p$, so it is zero since $p|\Delta_K$. So the rows must be linearly dependent over $\F_p$. This means that there are integers $m_1,\ldots,m_n\in \Z$ not all divisible by $p$ such that
    \[
        \sum^n_{i=1} m_i \textrm{T}^K_\Q(\alpha_i\alpha_j)\equiv 0\mod p
    \]
    for all $j$. Letting $\alpha = \sum m_i\alpha_i$, the above equality is equivalent to $p \mid \textrm{T}^K_\Q (\alpha \alpha_j)$ for each $j$. So $\textrm{T}^K_{\Q}(\alpha \mathcal{O}_K) \subset p\Z$. Since not all $m_i$ are divisible by $p$, it follows that $\alpha\not\in p\mathcal{O}_K$. Suppose for the sake of contradiction that $p$ is unramified in $\mathcal{O}_K$. Let $p\mathcal{O}_K=\mathfrak{p}_1\cdots \mathfrak{p}_k$. Then $\alpha\not\in \mathfrak{p}$ for some $\mathfrak{p}=\mathfrak{p}_i$. 
    
    For this next step, we'll pass up to the normal closure $L$ of $K$, prove that the trace $\textrm{T}^L(\alpha \mathcal{O}_L)\in p\Z$, and finally use the Galois theory of prime decompositions to prove that we get a sum of distinct automorphisms summing to zero, a contradiction.
    
    Let $L$ be the normal closure of $K$ over $\Q$. Since $p$ is unramified in $K$, it must also be unramified in the normal closure $L$ by the corollary to Theorem~4.31. Let $\mathfrak{q}$ be some prime lying over $\mathfrak{p}$ in $\mathcal{O}_L$. We also have $\alpha\not\in \mathfrak{q}$ because $\mathfrak{q}\cap \mathcal{O}_K = \mathfrak{p}$. 

    Now use the Chinese remainder theorem to get an element $\beta\in \mathcal{O}_L$ which isn't in $\mathfrak{q}$ but is in all of the other primes of $\mathcal{O}_L$ lying over $p$. Then we have the following:
    \begin{enumerate}
        \item $\textrm{T}^L(\alpha\beta\mathcal{O}_L)\subset \mathfrak{q}$ 
        \item $\sigma(\alpha\mathcal{O}_L)\subset \mathfrak{q}$ for each $\sigma\in \Gal(L /\Q) - D(\mathfrak{q}\mid p)$.
    \end{enumerate}
    The first statement follows immediately since we've shown that $\textrm{T}^L(\alpha \mathcal{O}_L)\subset p\Z \subset \mathfrak{q}$. For the second statement, $\beta\in \sigma^{-1}(\mathfrak{q})$ since $\sigma^{-1}\mathfrak{q}$ is distinct from $\mathfrak{q}$. (Otherwise $\sigma\in D(\mathfrak{q}\mid p)$). Thus $\sigma(\beta)\in \mathfrak{q}$, hence implying the second statement. Now combining the two results together:
    \[
        \sum_{\sigma\in D(\mathfrak{q}\mid p)}\sigma(\alpha\beta \mathcal{O}_L) \subset \mathfrak{q}
    .\]    
    Here the sum is interpreted to run over all $\mathcal{O}_L$. Now recall that members of $D(\mathfrak{q} \mid p)$ induce automorphisms for $L_{\mathfrak{q}} = \mathcal{O}_L / \mathfrak{q}$. Let's reduce everything mod $\mathfrak{q}$, including the automorphisms. Then
    \[
        0=\sum_{\sigma\in D(\mathfrak{q}\mid p)} \widetilde{\sigma}(\alpha\beta L_{\mathfrak{q}}) = \sum_{\sigma\in D(\mathfrak{q}\mid p)} \widetilde{\sigma}(L_{\mathfrak{q}})
    .\]  
    Since the intertia group $E(\mathfrak{q}\mid p)$ is trivial since $p$ is unramified in $L$, the automorphism $\widetilde{\sigma}$ are all distinct. However automorphisms are linearly independent over the base field, so we have a contradiction and we are done.
    
\end{solution}

\end{document}