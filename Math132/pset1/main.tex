\documentclass[11pt,letterpaper]{article}

\input{../../../../.config/latex/preamble_v1.tex}
\def\H{\mathcal{H}}
\def\L{\mathcal{L}}
\def\S{\mathcal{S}}
\def\M{\mathcal{M}}
\def\E{\mathbb{E}}
\def\Re{\mathrm{Re}}
\def\Im{\mathrm{Im}}
\def\id{\mathrm{id}}
\def\ceq{\vcentcolon=}

\lightmode

\title{\textbf{Math 132 Problem Set 1}}
\date{\textbf{Due:} February 2, 2023}

\begin{document}
\maketitle

\begin{problem}
    Suppose that $f : X \to Y$ is a diffeomorphism. Prove that at each $x\in X$, the derivative
    \[
        df_x : T_x X \to T_{f(x)}Y
    \]
    is an isomorphism of tangent spaces. 
\end{problem}

\begin{solution}
    \quad Since $f$ is a diffeomorphism, it must have some smooth inverse $g : Y \to X$. For any $x\in X$, consider the linear map $dg_{f(x)} : T_{f(x)} Y \to T_x Y$. Since $g\circ f=\id_X$, and $d(\id_X)_x = \id_{T_x X}$, we have \[dg_{f(x)}\circ df_{x} = \id_{T_x X}\] by the chain rule. A similar argument shows that \[df_{x}\circ dg_{f(x)}=\id_{T_{f(x)} Y}.\] Since it has a linear inverse, it follows that $df_{x}$ is an isomorphism. 
\end{solution}

\begin{problem}
    Write elements of $\R^{2n}$ as $n\times 2$ matrices, which you should think of as pairs of column vectors $[v_1, v_2]$. With this in mind, consider the set $V\subset \R^{2n}$ of orthonormal pairs $[v_1,v_2]$. By definition, this means the pairs $[v_1,v_2]$ satisfying $v_1\cdot v_2=0, v_1\cdot v_1 = 1, v_2\cdot v_2 = 1$. This turns out to be a smooth manifold known as a \emph{Stiefel manifold}. Can you guess the dimension of this manifold?
    \medskip 

    More generally, there is a Stiefel manifold of orthonormal $k$-tuples $[v_1,\ldots,v_k]$ of vectors $v_i\in \R^n$. It is naturally a subspace of $\R^{nk}$. Can you guess the dimension of this manifold? 
\end{problem}

\begin{solution}
    \quad We'll do the general case. Let's denote the Stiefel manifold of orthonormal $k$-tuples in $\R^n$ as $S_{n,k}$, and for now we can give this the subspace topology as a subspace of $\R^{nk}$. We can then consider $S_{n,k}$ as the locus of the following system of (nonlinear) equations:
    \[
        \begin{cases}
            v_{i,1}^2 + \cdots + v_{i,n}^2 = 1,& 0\leq i < k,\\
            v_{i,1}v_{j,1} + \cdots + v_{i,n}v_{j,n} = 0,& 0\leq i < j < k.
        \end{cases}
    \] 
    The first equation carves out a manifold homeomorphic to $S^{n-1}\times \R^{nk-n}$, notably one with codimension $1$ in $\R^{nk}$. Similarly, the second equation carves out some more complicated space, but also of codimension one, since $v_{i,1}$ can be expressed in terms of the other variables. Since we have $\binom{k+1}{2}$ of these ``independent'' equations, and assuming their locuses intersect nicely, we can guess that:
    \[
        \boxed{\dim S_{n,k} = nk - \binom{k+1}{2}} \implies \dim S_{n,2} = 2n-3
    \] 
    \quad Just as a sanity check, we note that this agrees with our intuition in the case when $k=1$, since $S_{n,1}$ is homeomorphic to $S^{n-1}$. Similarly, we expect a symmetry $\dim S_{n,k} = \dim S_{n,n-k}$ since orthonormal $k$-tuples can be put into correspondence with their orthogonal complements. This also holds of the guessed formula.   
\end{solution}

\begin{problem}
    Smooth functions.
\end{problem}

\begin{solution}
    This problem involves the definition of smooth functions.
    \begin{partproblem}{a}
        Suppose that $f : M \to N$ is a function between smooth manifolds. Show that $f$ is smooth if and only if for each smooth $g: N \to \R$ the composition $g\circ f$ is smooth.
    \end{partproblem}

    \begin{solution}
        \quad The forward direction is clear, since the composition of two smooth functions is once again smooth. In the reverse direction, suppose that for each smooth $g : N \to \R$, the composition $g\circ f$ is smooth. To show this, we'll first prove the case when $M=\R^m$ and $N=\R^n$. 

        \quad Let's write $f=(f_1,\ldots,f_n)$. For any $0\leq i < n$, let $g_i : \R^n \to \R$ be the function that selects the $i$th coordinate. This is clearly a smooth function, so by assumption $f_i = g_i\circ f$ should also be a smooth function from $M\to \R$. This means that the partial derivatives $\partial f_i / \partial x_j$ exist for all $0\leq j < m$. Since we can do this for all $0\leq i < n$, it follows that all partial derivatives of $f$ exist, hence it is smooth.
        
        \quad This immediately implies the claim when $M$ is an open subset of $\R^m$ since smoothness is preserved by restriction. (i.e. a local property) To complete the proof, we can now work in the full generality and let $M$ and $N$ be $k$ and $\ell$ dimensional smooth manifolds in $\R^m$ and $\R^n$ respectively. To show that $f : M \to N$ is smooth, let $x\in M$ and let $U\subset \R^m$ be some open neighborhood of $x$ in $\R^m$. Yet $\restr{f}{U} : U \to \R^n$ must be smooth by the earlier argument since the ``composition with $g$'' condition is preserved by function restriction so we are done.
    \end{solution}

    \begin{partproblem}{b}
        Suppose that $M$ is a smooth manifold, $\{U_i\}$ is a covering of $M$ by open subsets. Show that a function $f : M \to \R$ is smooth if and only if the restriction of $f$ to each $U_i$ is smooth.
    \end{partproblem}

    \begin{solution}
        \quad The forward direction is clear since smoothness is preserved by restriction. Now conversely suppose that the restrictions $\restr{f}{U_i}$ are smooth for all $U_i$. Then for any point $x\in M$, we can choose some $U_i$ containing $x$, and since $\restr{f}{U_i}$ is smooth, the function $f$ is smooth at $x$ so we are done.
    \end{solution}

    \begin{partproblem}{c}
        The above result is not true if the condition that the $U_i$ be open is dropped. Can you find a counterexample? % S0P3
    \end{partproblem}

    \begin{solution}
        \quad Let $M=\R$ and consider the covering of it by $(-\infty,0), \{0\}, (0,\infty)$. Now let $f : M \to \R$ be given by $f(x)=x^{1 /3}$. This is clearly smooth on $(0,\infty)$ and $(-\infty,0)$. At $\{0\}$, it is trivially smooth, since we can take any smooth function passing through $0$ to extend $f$ locally. However as a whole, $f$ isn't smooth because it has undefined first derivative at $x=0$. 
    \end{solution}
\end{solution}

\begin{problem}
    GP, Section~2, Problem~11.
\end{problem}

\begin{solution}
    Tangent spaces.
    \begin{partproblem}{a}
        Suppose that $f : X \to Y$ is a smooth map, and let $F : X \to X\times Y$ be $F(x)=(x, f(x))$. Show that
        \[
            dF_x(v) = (v, df_x(v))
        .\] 
    \end{partproblem}

    \begin{solution}
        \quad Clearly $F$ is a smooth map because it is a product of smooth maps. (S1P14) By S2P9d, we have for any smooth $f : X\to A$ and $g : X\to B$ the relation
        \[
            d(f\times g)_x = df_x\times dg_x
        .\] 
        Thus we have $dF_x = d(\id)_x\times df_x$ and so $dF_x(v)=(v,df_x(v))$. 
    \end{solution}

    \begin{partproblem}{b}
        Prove that the tangent space to graph of $f$ at the point $(x,f(x))$ is the graph of $df_x : T_x(X) \to T_{f(x)}(Y)$.
    \end{partproblem}

    \begin{solution}
        \quad By S2P9a, we have an natural equality $T_{(x,y)}X\times Y = T_xX\times T_yY$ with similarly induced maps. We have an induced map of tangent spaces
        \[
            dF_x : T_xX \to T_{(x,f(x))}X\times Y = T_xX \times T_{f(x)}Y
        .\]
        Since this map is trivial on it's first component, and the tangent space at the graph of $f$ at the point $(x,f(x))$ is exactly the image of this map, we have the desired result.
    \end{solution}
\end{solution}

\begin{problem}
    A \emph{curve} in a manifold $X$ is a smooth map $t \to c(t)$ of an interval of $\R$ into $X$. The \emph{velocity vector} of the curve $c$ at time $t_0$ (denoted $dc /dt(t_0)$) is defined to be the vector $dc_{t_0}(1)\in T_{x_0(X)}$, where $x_0=c(t_0)$, and $dc_{t_0} : \R \to T_{x_0}X.$ In the case when $X=\R^k$ and $c(t)=(c_1(t),\ldots,c_k(t))$ in coordinates, check that
    \[
        \frac{dc}{dt}(t_0) = (c'_1(t_0),\ldots,c'_k(t_0))
    .\]  
    Prove that every vector in $T_xX$ is the velocity vector of some curve in $X$, and conversely.
\end{problem}

\begin{solution}
    \quad In the first part, we have (without loss of generality) a smooth map $c : [0,1] \to \R^k$. Thus by definition of derivative we get $dc /dt (t) = dc_t(1) = \left(\partial c_1 /\partial t (t), \ldots, \partial c_k /\partial t(t)\right)$ which is exactly what we want. For the second part, let's fix some point $x$, and chart $\psi_x : \R^k \to U_x\subset X$ around $x$, where $\psi_x$ is a diffeomorphism and $U_x$ is an open neighborhood of $x$ in $X$. 

    \quad Recall that the tangent space $T_x X$ is defined as the image of $d(\iota \circ \psi_x)_x$ where $\iota : M \to \R^n$ is the canonical inclusion. Now for any vector $v\in T_x X$, let $v'\in \R^k$ be the preimage under this map, so that $d(\iota\circ \psi_x)_x(v')=v$. Now consider the curve $c_v : [-1,1] \to M$ given by $c_v(t)=\psi_x(tv')$. Then $d(c_v)_0(1) = v$ by construction.
\end{solution}

\end{document}