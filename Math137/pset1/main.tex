\documentclass[11pt,letterpaper]{article}

\input{../../../../.config/latex/preamble_v1.tex}
\lightmode

\title{\textbf{Math 137 Problem Set 1}}

\begin{document}
\maketitle

% Problem 1
\begin{problem}\noindent
  Let $K$ be a field and let $X$ be a set of $m$ points in $K^n$.
  \begin{enumerate}[(a)]
    \item Show that there is a set $S\subseteq K[X_1,\dots,X_n]$ of size at most $n^m$ such that $X=\mathcal V(S)$.
    \item Assuming that $K=\mathbb R$, show that there is a polynomial $f\in K[X_1,\dots,X_n]$ such that $X=\mathcal V(f)$.
    \item \emph{(bonus)} Assuming that the field $K$ is finite, show that there is a polynomial $f \in K[X_1,\dots,X_n]$ such that $X=\mathcal V(f)$. (Hint: Use Fermat's little theorem / Lagrange's theorem.)
    \item \emph{(bonus)} Assuming that the field $K$ is infinite, show that there is a set $S\subseteq K[X_1,\dots,X_n]$ of size at most $n+1$ such that $X=\mathcal V(S)$.
  \end{enumerate}
\end{problem}

\begin{changemargin}{1em}{1em}
  Let $X=\{x_1, x_2, \ldots , x_m\}$ be the set of points, with coordinates $x_i = (x_{i1}, x_{i2}, \ldots , x_{in})$.

  \textbf{(a)} Recall that $\mathcal{V}(I)\cup \mathcal{V}(J)=\mathcal{V}(I\cdot J)$. Given any point $x_i\in X$, consider the set $S_i=\{X_1-x_{i 1}, X_2-x_{i 2}, \ldots, X_n - x_{i n}\}$. Then clearly $\mathcal{V}(S_i)=\{x_i\}$, so letting $S=S_1\cdot S_2\cdots S_m$ we have $\mathcal{V}(S)= X$. Since $S$ consists of all products of $m$ elements chosen from sets of $n$ elements, we have $|S|=n^m$.     
  
  \textbf{(b)} More generally, let $K$ be any ordered field. We can construct $f$ as 
  \[
    f(X_1, \ldots , X_n) = \prod\limits_{i=1}^{m}\left((X_1-x_{i1})^2 + \cdots + (X_n - x_{in})^2\right).
  \]  
  Because $K$ is an ordered field, each of these factors $(X_1-x_{i 1})^2+\cdots+(X_n-x_{i n})^2$ only vanishes at $x_i$. Hence when we multiply them all together, we get a polynomial which vanishes at $X$ only.  
  
  \textbf{(c)} Let $K$ be a field of order $p^k$. Then Lagrange's theorem tells us that 
  \[
    a^{p^k-1}=
    \begin{cases}
        0& \mathrm{if}\; a=0\\
        1& \mathrm{if}\; a\neq 0
    \end{cases}
    , \quad a\in K
  \]

  This means that for any point $x_j$, the term $1-(X_i-x_{ji})^{p^k-1}$ will only equal zero if $X_i\neq x_{ji}$, and $1$ otherwise. Then the product $\prod_i (1-(X_i-x_{ji})^{p^k-1})$ will only be equal to $1$ if $X_i=x_{ji}$ for all $i$. Thus, subtracting this function from $1$ gives us the desired function:
  \[
      f_i(X_1, \ldots, X_n) = 1-\prod_i (1-(X_i-x_{ji})^{p^k-1})
  .\]  
  Since this vanishes exactly on $x_j$, we can just take $f=f_1f_2\cdots f_m$ and we are done.  

  \textbf{(d)} To start, add the $n$ polynomials $f_i(X_1, \ldots, X_n)=(X_i-x_{1 i})(X_i-x_{2 i})\cdots (X_i-x_{m i})$. This gives us a grid of points consisting of all $X_1,X_2,\ldots,X_n$ combinations between the points. Next, we'll use the property that in any field, $A_1+A_2+\cdots+A_m=A_j$ if all the other terms are zero. Constructing this polynomial is becoming quite difficult to describe at this hour of the night, so I will do an example and hope this suffices to convince the reader on the correctness of my construction. 
  
  Suppose we had points $(1,1), (1,2), (2,3)$. Then the first $n$ polynomials vanish on the points $(1,1), (1,2), (1,3), (2,1), (2,2), (2,3)$. The last polynomial eliminates the extraneous points, taking the form
  \[
      f(x,y)=(x-1)\left((y-3)(y-1)+(y-3)(y-2)\right)+(x-2)\left((y-1)(y-2)\right)
  .\]   
  In this manner, by isolating each line and weaving through the points, we can hit every required point on the grid.
  
\end{changemargin}

% Problem 2
\begin{problem}
  Show that $A=\{(t,\sin(t)) \mid t\in \mathbb R\}$ is not an algebraic subset of $\mathbb R^2$.
\end{problem}

\begin{changemargin}{1em}{1em}
  Suppose for the sake of contradiction that $A$ is algebraic set. Consider $\R$ as a subspace of $\R^2$, embedded along the line $y=0$. Then $A\cap \R=\frac{\pi}{2}\Z$ is an algebraic set in $\R$, since the Zariski topology agrees with subspaces. However the only algebraic sets in $\R$ are finite sets of points and $\R$ itself, so we have a contradiction. Hence $A$ is not algebraic.         
\end{changemargin}

% Problem 3
\begin{problem}
  Consider the \emph{one-sheet hyperboloid}
  \[
    V = \{(x,y,z)\in\mathbb R^3 \mid x^2 + y^2 = z^2 + 1\} \subseteq \mathbb R^3.
  \]
  Prove that every point $P\in V$ lies on exactly two (straight) lines $l_1,l_2\subseteq V$.
\end{problem}

\begin{changemargin}{1em}{1em}
  To prove that there are exactly two lines in $V$ going through a point $(x,y,z)$, suppose 
  \[
    \begin{bmatrix}
      x^\prime \\y^\prime \\z^\prime 
    \end{bmatrix} = 
    \begin{bmatrix}
      x+at\\ y+bt\\ z+ct
    \end{bmatrix}
  \]
  is some line in $V$ parametrized by $t$, determined by the slopes $a,b,c\in \R$. Then for every $t$, we would have 
  \[
    \begin{aligned}
      (x+at)^2 + (y+bt)^2 - (z+ct)^2 - 1&= 0\\
      (x^2 + y^2 - z^2 - 1) + 2t(ax+by-cz) + t^2(a^2 + b^2 - c^2) &= 0\\
      2(ax+by-cz) + t(a^2 + b^2 - c^2) &= 0. \quad (t\neq 0)
    \end{aligned}
  \]
  Since this is true for all nonzero $t$, we set $t=1$ and $t=-1$ and add the two equations together to get
  \[
    \begin{cases}
      a^2 + b^2 = c^2\\
      ax+by=cz.
    \end{cases}
  \]  
  To show that there are two lines on the hyperboloid, it suffices to show that the solution set to this equation consists of two intersecting lines. Note that the transformation $(a,b,c)\to (\lambda a, \lambda b, \lambda c)$ doesn't affect the equation (i.e. it is homogenous), so we can reduce a dimension by performing the variable change $A=\frac{a}{c}$ and $B=\frac{b}{c}$ to get the system 
  \[
    \begin{cases}
      A^2+B^2=1\\
      Ax+By=z.
    \end{cases}
  \] 
  Now to show that there are two lines on the hyperboloid it suffices to show that there are two solutions to this system for all $x,y,z$ satisfying $x^2+y^2=z^2+1$. Solving for $B$ in the second equation we get
  \[
    B=\frac{z}{y}-\left(\frac{x}{y}\right)A
  .\]
  We can assume without loss of generality that $y\neq 0$, since both $x$ and $y$ can't be zero. Then note that,   
  \[
    \left(\frac{z}{y}\right)^2+\left(\frac{1}{y}\right)^2=\left(\frac{x}{y}\right)^2+1
  .\]       
  This justifies the variable transformation $X=\frac{z}{y}$, $Y=\frac{1}{y}$, and $Z=\frac{x}{y}$, so in these coordinates $B=X-ZA$. Substituting this into the first equation, we get 
  \[
    (1-Z^2)A^2+(-2XZ)A+(X^2-1)=0
  \]
  Using the above identities, the discriminant of this quadratic simplifies to $\Delta = 4Y^2$, which is clearly nonzero, so we have exactly two solutions. So we are done.
\end{changemargin}

% Problem 4
\begin{problem}
  For every $n\geq1$, show that the ideal $I=(X,Y)^n$ of $K[X,Y]$ is not generated by $n$ of its elements.
\end{problem}

\begin{solution}
    It is clear that $(X,Y)^n=(X^n, X^{n-1}Y, \ldots, XY^{n-1}, Y^n)$. Suppose for the sake of contradiction that there were generators $g_1,g_2,\ldots,g_n$. We can assume without loss of generality that all of these generators have smallest degree term greater than or equal to $n$. Consider the $K$-vector space spanned by formal symbols $X^n, X^{n-1}Y, \ldots, XY^{n-1}, Y^n$. Let $v$ be any vector in this space. Then $v\in (X,Y)^n$ so there are some constants $c_1,c_2,\ldots,c_n\in K$ such that $v=c_1 g_1+\cdots+c_n g_n$. Since $v$ has degree less than or equal to $n$, we can assume that $c_i$ all must be constant. So $g_1,g_2,\ldots,g_n$ span the vector space, a contradiction because it is $n+1$ dimensional.  
\end{solution}

% Problem 5
\begin{problem}
  Let $K$ be any field and let $A$ be any subset of $K^n$. Show that $\mathcal V(\mathcal I(A))$ is the closure of $A$ with respect to the Zariski topology. (This is called the Zariski closure of $A$.) 
\end{problem}

\begin{changemargin}{1em}{1em}
  Clearly $A\subset \mathcal{V}(\mathcal{I}(A))$ so $\overline{A} \subset \overline{\mathcal{V}(\mathcal{I}(A))}$, however since $\mathcal{V}(\mathcal{I}(A))$ is closed, it follows that $\overline{A} \subset \mathcal{V}(\mathcal{I}(A))$. To show that this inclusion is an equality, it suffices to show that any closed set $\mathcal{V}(C)$ containing $A$ must also contain $\mathcal{V}(\mathcal{I}(A))$. Since $\mathcal{V}(C)$ contains $A$, every polynomial $f\in C$ must vanish on all of $A$, so $f\in \mathcal{I}(A)$ and so $C\subset \mathcal{I}(A)$. Since $\mathcal{V}$ reverses inclusions, $\mathcal{V}(C)\supset \mathcal{V}(\mathcal{I}(A))$ and so we are done.   
\end{changemargin}

\end{document}