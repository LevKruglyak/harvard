\documentclass[11pt,letterpaper]{article}

\input{../../../../.config/latex/preamble_v1.tex}
\lightmode

\title{\textbf{Math 137 Problem Set 2}}

\begin{document}
\maketitle

\hr
\begin{center}
    \textit{I collaborated with AJ LaMotta and Eliot Hodges for this problem set.}
\end{center}
\hr

% Problem 1
\begin{problem}
    Let $A$ be an algebraic subset of $K^n$ and let $B$ be an algebraic subset of $K^m$. Show that the cartesian product $A\times B$ is an algebraic subset of $K^n \times K^m = K^{n+m}$.
\end{problem}

\begin{solution}
    Suppose $A$ satisfies $f_1(x)=\cdots=f_n(x)=0$ for $x=x_1,x_2,\ldots,x_n$ and $B$ satisfies $g_1(y)=\cdots=g_m(y)=0$  for $y=y_1,y_2,\ldots,y_m$. Define $\overline{f_i}(x,y)=f_i(x)$ and $\overline{g_i}(x,y)=g_i(y)$. Then clearly,
    \[
        A\times B = \{(x,y)\in K^{n+m} \mid \overline{f_1}(x,y)=\cdots=\overline{f_n}(x,y)=\overline{g_1}(x,y)=\cdots=\overline{g_m}(x,y)=0\}
    .\]      
    Thus $A\times B$ is an algebraic subset of $K^{n+m}$.
\end{solution}

% Problem 2
\begin{problem}
    Show that $X=\{(t,e^t)\mid t\in\mathbb R\}$ is not an algebraic subset of~$\mathbb R^2$. 
\end{problem}

\begin{solution}
    Suppose for the sake of contradiction that $X$ were an algebraic set. Then there must be some finite set of nonzero polynomials $f_i\in \R[x,y]$ such that $X=\mathcal{V}(f_1,f_2,\ldots,f_n)$, i.e. $f_i(t, e^t)=0$ for all $t\in \R$. Pick any of the $f_i$, say $f_i=f$ and consider the function $f(t,e^t)$ as a polynomial in the formal symbol $e^t$ with coefficients in $\R[t]$, so
    \[
        f(t,e^t)=\sum_{k=0}^N g_k(t)(e^t)^k,\quad g_k(t)\in \R[t]
    .\]        
    We'll now induct on the degree $N$ to show that such a polynomial cannot exist. For the base case of $N=0$, this means that $f(t,e^t)=g_0(t)=0$ for all $t\in \R$, a contradiction since a nonzero real valued polynomial cannot vanish on all of $\R$. 
    
    Now suppose there cannot exist such a polynomial for all degrees less than $N$, and that $f(t,e^t)$ is of degree $N$. If $g_0(t)=0$, since $e^t\neq 0$, we can factor out $e^t$ from $f$ and be left with a degree $N-1$ polynomial which still vanishes everywhere, completing the induction step. Otherwise, let $M=\deg g_0(t)$ be the degree of the nonzero constant term of $f$.  Then
    \[
        \begin{aligned}
            f^{(M+1)}(t,e^t)&=\sum^N_{k=1}\left(g_k(t)(e^t)^k\right)^{(M+1)}+g^{(M+1)}_0(t)\\
            &=\sum^N_{k=1}\sum^{M+1}_{j=0}\binom{M+1}{j}g^{(M+1-j)}_k(t)k^j(e^t)^k\\
            &=\sum^N_{k=1}h_k(t)(e^t)^k
        \end{aligned}
    \]
    where $f^{(n)}$ denotes the $n$-th derivative with respect to $t$. Observe that $h_k(t)$ are nonzero if $g_k(t)$ was nonzero, and $f^{(M+1)}(t,e^t)$ has no constant term, so we can again factor out $e^t$ to get a polynomial of degree $N-1$ which vanishes on $\R$. This completes the induction.      
\end{solution}

% Problem 3
\begin{problem}
    For each of the following ideals $I$ of $\mathbb C[X,Y]$, is $1\in I$? If so, show how to write $1$ as a linear combination of the given generators.
    \begin{enumerate}[label=\alph*)]
        \item $I=(X-Y, X^2 + XY - 2Y^2, X + Y - 2)$
        \item $I=(X^2 + Y^2 - 1, X + Y - 1, X - Y)$
    \end{enumerate}
\end{problem}

\begin{solution}
    \textbf{(a)} Note that $(1,1)$ is a root of every element of $I$, so $\mathcal{V}(I)\neq \emptyset$. However if $1\in I$, then $I=\C[X,Y]$. This is a contradiction since the weak Nullstellensatz implies $\mathcal{V}(\C[X,Y])=\emptyset$, so $1\not\in I$.
    
    First we'll show that $I=(X-1, Y-1)$. Clearly $(X-1, Y-1)\subset I$ because $X-1=\frac{1}{2}(X-Y)+\frac{1}{2}(X+Y-2)$. Similarly, $Y-1=(X+Y-2)-(X-1)$. To prove the converse, note that 
    \[
        \begin{aligned}
            X-Y&=(X-1)-(Y-1)\\
            X^2+XY-2Y^2&=(X+Y+1)(X-1)-(2Y+1)(Y-1)\\
            X+Y-2&=(X-1)+(Y-1).
        \end{aligned}
    \]   
    So $I=(X-1,Y-1)$. Next we claim that $1\not\in I$. Suppose $a(X,Y)(X-1)+b(X,Y)(Y-1)=1$ for some $a,b\in \C[X,Y]$. Then substituting $X=Y$, we get $\left(a(X,Y)+b(X,Y)\right)(X-1)=1$, however this is impossible because the only way $\left(a(X,Y)+b(X,Y)\right)(X-1)$ can have no non constant terms is if $a(X,Y)+b(X,Y)=0$, which would also violate the equation. So $1\not\in I$. 
    
    \textbf{(b)} We claim that $I=\C[X,Y]$. This is very easy to show; consider the linear combination:
    \[
        \begin{aligned}
            -2\cdot(X^2+Y^2-1)+(1+2X)\cdot(X+Y-1)+(1-2Y)\cdot (X-Y)=1.
        \end{aligned}
    \] 
    Since $1\in I$, it follows that $I=\C[X,Y]$. 
\end{solution}

% Problem 4
\begin{problem}
    Let $I$ be an ideal of a polynomial ring $K[X_1,\dots,X_n]$ over a field $K$. Let $J=\sqrt I$ be its radical. Show that $J^n \subseteq I$ for some $n\geq1$.
\end{problem}

\begin{solution}
    Let $I$ be an ideal in $K[X_1,\ldots, X_n]$. Then $\sqrt{I}$ is an ideal in $K[X_1,\ldots,X_n]$ so it is finitely generated by Hilbert's basis theorem, say $\sqrt{I}=(f_1,\ldots,f_m)$. For each of these generators $f_i^{e_i}\in I$ for some $e_i$. So write $\sqrt{I}=(f_1)+\cdots+(f_m)$. Then letting $e=e_1+\cdots+e_m$, 
    \[
        \left(\sqrt{I}\right)^{e}=\left((f_1)+\cdots+(f_m)\right)^e=\sum_{b_1+\cdots+b_m=e} (f_1^{b_1})\cdots (f_m^{b_m})
    .\]
    Since for every choice of partition $b_i$, there will always be a term in the product such that $b_i\geq e_i$, it follows that $(\sqrt{I})^e\subset I$.        
\end{solution}

% Problem 5
\begin{problem}
    Let $K$ be any field and let $A$ and $B$ be algebraic subsets of~$K^n$. Show that there exists an integer $m\geq n$ and an algebraic subset $C$ of $K^m$ such that the image of $C$ under the projection $K^m\rightarrow K^n$ sending $(x_1,\dots,x_m)$ to $(x_1,\dots,x_n)$ is the set difference $A - B$.
\end{problem}

\begin{solution}
    Suppose $A=\mathcal{V}(f_1,\ldots,f_a)$ and $B=\mathcal{V}(g_1,\ldots,g_b)$. We claim that the space $K^{n+b}$ suffices. Construct polynomials
    \[
        \begin{aligned}
            \overline{g_i}(x_1,\ldots, x_n, t_1,\ldots,t_b) &= g_i(x_1,\ldots,x_n)t_i-1,\\
            \overline{f_i}(x_1,\ldots, x_n, t_1,\ldots,t_b) &= f_i(x_1,\ldots,x_n).
        \end{aligned}
    \]  
    Let $\pi : K^{n+b} \to K^n$ be the projection map. Then $\pi(\mathcal{V}(\overline{g_i}))=K^n-\mathcal{V}(g_i)$, since the only way $\overline{g_i}$ could be zero for a given point $x_1,\ldots,x_n$ was if there exists some $t_i$ such that $t_i=1 / g_i(x_1,\ldots,x_n)$, so $\pi(\mathcal{V}(\overline{g_i}))$ is precisely the set of points for which $g_i$ is nonzero. Clearly $\pi(\mathcal{V}(\overline{f_i}))=\mathcal{V}(f_i)$, so if $C=\mathcal{V}(\overline{f_1},\ldots,\overline{f_a},\overline{g_1}\overline{g_2}\cdots \overline{g_b})$, $$\pi(C)=\bigcup_i(K^n-\mathcal{V}(g_i))\cap \bigcap_i\mathcal{V}(f_i)=(K^n-B)\cap A= A-B.$$  
    This concludes the proof.
    
\end{solution}

% Problem 6
\begin{problem}
    Give an example of an algebraic field extension $L$ of a field $K$ that is not module-finite (i.e.{} not a finite-dimensional vector space)
\end{problem}

\begin{solution}
    Let $K=\F_p$ and $L=\overline{\F_p}$ be its algebraic closure. This is an algebraic extension by definition, however it isn't a finite extension because
    \[
        \overline{\F_p}=\bigcup_{n\geq 1} \F_{p^n}
    .\]  
\end{solution}

% Problem 7
\begin{problem}
    Let $K$ be an infinite field and let $P_1,\dots,P_m$ be $m$ distinct nonzero points in $K^n$. Show that there is an invertible linear map $f:K^n\rightarrow K^n$ such that the $n\cdot m$ coordinates of the $m$ points $f(P_1),\dots,f(P_m)$ are distinct.
\end{problem}

\begin{solution}
    Consider the linear map $f$ as a point in $K^{n^2}$. Then $\det(f)$ is a polynomial in $K[x_1,\ldots,x_{n^2}]$, so the set of invertible matrices is the complement of an algebraic set, namely $\det(f)=0$. Now suppose we have a set of nonzero distinct points $P_1,\ldots,P_m$. Consider the polynomial $g\in K[x_1,\ldots,x_{n^2}]$ defined by
    \[
        g(f)=\prod_{i< j}\prod_{a< b}(f(P_a)_i - f(P_b)_j)
    .\] 
    Since each $P_i$ has at least one nonzero coordinate, each $f(P_a)_i-f(P_b)_j$ is nonzero for some $f\in K^{n^2}$. So to find a linear map satisfying the conditions of the problem, it suffices to find some $f\in K^{n^2}$ such that $g(f)\neq 0$ and $\det(f)\neq 0$. If no such $f$ exists, then $g\cdot \det$ vanishes on all of $K^{n^2}$, which is impossible by the Nichtnullstellensatz since $K$ is infinite.    
\end{solution}

\end{document}