\documentclass[11pt,letterpaper]{article}

\input{../../../../.config/latex/preamble_v1.tex}
\lightmode

\title{\textbf{Math 137 Problem Set 5}}

\providecommand{\lm}{\mathrm{lm}}
\providecommand{\lt}{\mathrm{lt}}

\begin{document}
\maketitle

Throughout, $K$ is assumed to be an algebraically closed field. I collaborated with AJ LaMotta. 

\begin{problem}
    Let $f\in K[X_1,\dots,X_n]$. Show that i
    \[
    \mathcal V\left(f,\frac{\partial f}{\partial X_1},\dots,\frac{\partial f}{\partial X_n}\right) = \emptyset,
    \]
    then the polynomial $f$ is squarefree.
\end{problem}

\begin{solution}
    Suppose $f=g^2h$ for $g,h\in K[X_1, \ldots, X_n]$. Then
    \[
        \begin{aligned}
            \frac{\partial f}{\partial x_i} = \frac{\partial g^2}{\partial x_i}h+g^2\frac{\partial h}{\partial x_i} = 2g\frac{\partial g}{\partial x_i}h + g^2\frac{\partial h}{\partial x_i}=g\left(2\frac{\partial g}{x_i}h + g^2\frac{\partial h}{\partial x_i}\right)=gh_i.
        \end{aligned}
    \] 
    for some $h_i\in K[X_1,\ldots,X_n]$. We can assume that $\frac{\partial f}{\partial x_i}\neq 0$ without loss of generality, so we know that $\mathcal V(g^2h, gh_1, \ldots, gh_n)=\emptyset$. However this means that $g$ has no roots, yet since $K$ is algebraically closed, $g$ must be a constant polynomial. This proves that $f$ is squarefree.
\end{solution}

\begin{problem}
    Show that every monomial order $\leq$ on $\mathcal S(X_1,\dots,X_n)$ is a well-order.
\end{problem}

\begin{solution}
    Let $\leq$ is a monomial order, suppose for the sake of contradiction that $\leq$ isn't a well order. By definition this means that there is an infinitely descending chain of monomials, i.e. monomials $M_1,M_2,\ldots$ with $M_1 > M_2 > \cdots$. Now letting $I_n = (M_1,\ldots,M_n)$ be the ideal generated by the first $n$ ideals, we know that $M_{n+1}\not\in I_n$ because otherwise $M_{n+1}=c_1M_1+\cdots+c_nM_n$ so $M_{n+1}$ would be divisible by a greater monomial in the order $M_i$. Then $M_{n+1}=RM_i$ for some monomial $R$, which implies that $M_{n+1}< RM_i=M_{n+1}$ which is clearly contradictory. However this contradiction gives rise to an infinite increasing chain of ideals $I_1\subset I_2\subset \cdots$, which is impossible since $K[X_1,\ldots,X_n]$ is a Noetherian ring.
\end{solution}

\begin{problem}
    Let $G$ be a Gr\"{o}bner basis of $I\subseteq K[X_1,\dots,X_n]$. Show that the set $\mathcal V(I)$ is finite if and only if for all $1\leq i\leq n$, there is an element $g\in G$ such that $\lm(g)=X_i^t$ for some $t\geq0$.
\end{problem}

\begin{solution}
    To prove the forward direction, suppose $\mathcal{V}(I)$ is finite. Then we have 
    \[
        \sqrt{I} \ni (X_i-a_1)\cdots (X_i-a_m)
    \]
    where $a_1,\ldots, a_m$ are the $i$-th coordinates of points in $\mathcal{V}(I)$. Then the leading monomial of this polynomial is $X_i^m$. Recall that there is some $t$ such that $I\supset(\sqrt{I})^t$ so there is a polynomial in $I$ with leading monomial $X_i^{tm}$. Since $G$ is a Gr\"obner basis, there must be a $g$ with $\lm(g)=X_i^{tm}$. We can do this for every $X_i$, so we are done.

    To prove the backward direction, suppose that $\lm(g_i)=X^{a_i}_i$ for a collection of Gr\"obner bases $g_1,\ldots,g_n$. Then the standard monomials of $I$ are of the form $X^{b_i}_i$ with $b_i < a_i$. This is a finite collection, so by Corollary 8.3 the vanishing locus of $I$ must be finite as well.
\end{solution}

\begin{problem}\noindent
    \begin{enumerate}[label=\alph*)]
        \item Let $a,b>0$ and consider the set
        \[
        V=\{(x,y)\in\mathbb Z^2: x,y\geq0\textnormal{ and }ax+by\leq1\}.
        \]
        Fix any monomial ordering. Show that $X^rY^s$ is a standard monomial for $\mathcal I(V)$ if and only if $(r,s)\in V$.
        \item What is the smallest (total) degree $d$ of a nonzero polynomial $f\in\mathcal I(V)$?
    \end{enumerate}
\end{problem}

\begin{solution}
    \textbf{(a)} First, let $(r,s)\in V$. Let $A=\{0,1,\ldots, r\}$, and $B=\{0,1,\ldots, s\}$. Applying the combinatorial Nullstellensatz to $A\times B$, we conclude that $X^rY^s$ is a standard monomial for $\mathcal{I}(A\times B)$. However note that $\mathcal{I}(V)\subset \mathcal{I}(A\times B)$ since $A\times B\subset V$, so a standard monomial for $\mathcal{I}(A\times B)$ is also a standard monomial for $\mathcal{I}(V)$. Since $V$ is finite of size $(1+\lfloor1/a\rfloor)(1+\lfloor1/b\rfloor)$ so there are $|V|$ standard monomials in $ \mathcal{I} (V)$, and they are all exactly of the form $X^rY^s$.

    \textbf{(b)} We claim that $d=\min(\floor{1/a}, \floor{1/b})+1$. We've seen in $A$ that the leading monomial of any nonzero $f\in \mathcal{I}(V)$ must be of the form $X^rY^s$ with $(r,s)\not\in V$, this means that $ar+bs>1$. The smallest degree (with respect to lexicographic ordering) is $d=\min\{x+y\mid ax+by>1\}$. However this quantity is straightforwardwardly proven to be $\min(\floor{1/a}, \floor{1/b})+1$, as desired.
\end{solution}

\begin{problem}
Let $\leq$ be any monomial order on $K[X_1,\dots,X_n]$ and let $0\neq f,g\in K[X_1,\dots,X_n]$. Show that if $\gcd(\lm(f),\lm(g)) = 1$, then $0$ is a reduction of
\[
S(f,g) = \frac{M}{\lt(f)}\cdot f - \frac{M}{\lt(g)}\cdot g
\]
with respect to $\{f,g\}$, where $M = \lcm(\lm(f),\lm(g))=\lm(fg)$.
\end{problem}

\begin{solution}
    Let's express the polynomials $f$ and $g$ as $f=\sum_{i=1}^k a_i M_i$ and $\sum_{i=1}^\ell b_iN_i$ for some nonzero coefficients. Let's further assume without loss of generality that $M_1 < M_2 < \cdots < M_k$ and $N_1 < N_2 < \cdots < N_\ell$, so that $\lm(f)=M_k$ and $\lm(g)=N_\ell$. Then we have
    \[
        S(f,g)=\left(\sum^k_{i=1} \frac{a_i}{a_k}N_\ell M_i\right)-\left(\sum^\ell_{i=1} \frac{b_i}{b_\ell} M_k N_i\right)
    .\] 
    Now note that if $N_\ell M_i = M_k N_j$ for some $j < \ell$ then we have $N_{\ell} | M_k N_j$, and since $N_\ell$ is coprime to $M_k$ we have $N_\ell | N_j$. This is a contradiction since $j<\ell$. We get the same thing if we consider the $M$ side, so the terms coming from $M$ all cancel. Next, consider the identity
    \[
        S(f,g)=\left(\frac{\textrm{lt}(g)-g}{\textrm{lc}(f)\textrm{lc}(g)}\right)f + \left(\frac{f-\textrm{lt}(f)}{\textrm{lc}(f)\textrm{lc}(g)}\right)g
    .\] 
    Notice then that if the first summand is nonzero, we have
    \[
        \lm\left(\left(\frac{\textrm{lt}(g)-g}{\textrm{lc}(f)\textrm{lc}(g)}\right)f\right) = \lm(\textrm{lt}(g)-g)\lm(f)\leq \lm(S(f,g))
    .\] 
    However $S(f,g)=0$, so repeating the same argument for the second summand, we have that $0$ is a reduction of $S(f,g)$ with respect to $\{f, g\}$.
\end{solution}


% \begin{problem}[bonus]
% Let $\leq$ be a monomial order on $\mathcal S(X_1,\dots,X_n,Y_1,\dots,Y_m)$ such that $A<B$ whenever $A$ is contained in $\mathcal S(X_1,\dots,X_n)$ but $B$ isn't contained in $\mathcal S(X_1,\dots,X_n)$.

% Let $G$ be a Gröbner basis of an ideal $I\subseteq K[X_1,\dots,X_n,Y_1,\dots,Y_m]$.
% \begin{enumerate}[label=\alph*)]
% \item Show that $G\cap K[X_1,\dots,X_n]$ is a Gröbner basis of the ideal $I'=I\cap K[X_1,\dots,X_n]$ of $K[X_1,\dots,X_n]$.
% \item Show that the Zariski closure of the image of $\mathcal V(I)\subseteq K^{n+m}$ under the projection to $K^n$ (projecting to the first $n$ coordinates) is $\mathcal V(I')\subseteq K^n$.
% \end{enumerate}
% \end{problem}

\end{document}
