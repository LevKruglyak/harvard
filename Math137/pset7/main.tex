\documentclass[11pt,letterpaper]{article}

\input{../../../../.config/latex/preamble_v1.tex}
\lightmode

\title{\textbf{Math 137 Problem Set 7}}

\begin{document}
\maketitle

\begin{center}
    \textit{I collaborated with AJ LaMotta for this problem set.}
\end{center}

\begin{problem}
    Which of the following morphisms are finite? (Say for $K=\C$.)
    \begin{enumerate}[(a)]
        \item The morphism $\varphi : K^2\rightarrow K$ sending $(x,y)$ to $x^3 y + x y^3 + 3x + 1$.
        \item The morphism $\varphi: K\rightarrow K^2$ sending $x$ to $(x^2,x^3)$.
    \end{enumerate}
\end{problem}
\begin{solution}
    \textbf{(a)} Notice that $x^3y+xy^3+3x+1=1$ has infinitely many solutions; an infinite family can be given by $(0,y)$ for any $y\in \R$. Thus $\varphi$ cannot be finite since a point in $K$ has infinitely many preimages. 
   
    \textbf{(b)} Note that $\Gamma(K)=K[t]$, and $\varphi^*(\Gamma(K^2))=\varphi^*(K[x,y])=K[t^2, t^3]=K[t^2]$. Then $K[t]$ has a natural $K[t^2]$-module structure, and is generated by $\{1,t\}$. Since $\Gamma(K)$ is a finitely generated $\varphi^*(\Gamma(K^2))$-module, the morphism $\varphi$ is finite.  
\end{solution}
    
\begin{problem}\noindent
    \begin{enumerate}[(a)]
        \item Let $\varphi:V\rightarrow W$ be a morphism. Show that if $V$ is the union of algebraic subsets $V_1,\dots,V_n$ and each restriction $\varphi:V_i\rightarrow W$ is a finite morphism, then $\varphi$ is a finite morphism.
        \item Let $V\subseteq K^n$ be a finite set and let $W$ be any algebraic set. Show that every map $\varphi:V\rightarrow W$ is a finite morphism.
    \end{enumerate}
\end{problem}

\begin{solution}
    \textbf{(a)} By induction, we can assume without loss of generality that $n=2$, so let $V=A\cup B$. We'll show that $\Gamma(A\cup B)$ is integral over $\varphi^*(\Gamma(W))$. Let $f\in \Gamma(A\cup B)$ be some function. Then $\restr{f}{A}$ and $\restr{f}{B}$ are integral over $\varphi^*(\Gamma(W))$. We thus have monic polynomials $\alpha,\beta\in \varphi^*(\Gamma(W))[x]$ with $\alpha(f)=0$ in $\Gamma(A)$ and $\beta(f)=0$ in $\Gamma(B)$. Then $(\alpha\cdot\beta)(f)=0$ on $\Gamma(A\cup B)=\Gamma(V)$ so $f$ is integral, which then implies that $\varphi$ is finite.
    
    \textbf{(b)} Again by induction and the argument in (a), we can assume that $V$ is a single point. Then $\varphi^* : \Gamma(W) \to K$ is surjective and so $\varphi$ is finite. 
\end{solution}
    
\begin{problem}
    Let $\varphi:V\rightarrow W$ be a dominant morphism between irreducible algebraic sets. Assume $\Gamma(V)$ is generated by $n$ elements as a $\varphi^*(\Gamma(W))$-module. Show that the preimage of any point $Q\in W$ has size at most~$n$.
\end{problem}

\begin{solution}
    Since $\Gamma(V)$ is a finite $\varphi^*(\Gamma(W))$ extension with a generating set of size $n$, we can choose some generating set $f_1,\ldots, f_n\in \Gamma(V)$ such that any element $f\in \Gamma(V)$ can be written as
    \[
        f = (g_1\circ \varphi)\cdot f_1+\cdots + (g_n\circ \varphi)\cdot f_n
    \] 
    for some $g_i \in \Gamma(W)$. Note that the restriction map $\Gamma(V) \to \Gamma(\varphi^{-1}(Q))$ gives a spanning set for $\Gamma(\varphi^{-1}(Q)$ of elements of the form $\restr{f_i}{\varphi^{-1}(Q)}$ and we can choose arbitrary $K$-coefficients since $g_i(Q)$ can take on any value. Since $\Gamma(\varphi^{-1}(Q))\cong K^{|\varphi^{-1}(Q)|}$, we thus have that $|\varphi^{-1}(Q)|\leq n$ by basic linear algebra.
\end{solution}
    
\begin{problem}
    Let $L$ be a finitely generated field extension of $K$ with $n=\textrm{trdeg}(L/K)$ and let $R\subseteq L$ be a finitely generated ring extension of $K$ whose field of fractions is $L$. Show that there are elements $a_1,\dots,a_n$ of $R$ such that $R$ is an integral ring extension of $K[a_1,\dots,a_n]$.
\end{problem}
    
\begin{solution}
    Recall the Noether normalization lemma:
    \begin{ilemma}[Noether normalization]
        If $V$ is irreducible algebraic, there is a finite morphism $V\to K^{\dim V}$.
    \end{ilemma}

    We will show that this lemma implies our result. Note that since $R$ is a finitely generated ring extension of $K$, it must be a finitely generated $K$-algebra so it is isomorphic to some $K[x_1,\ldots,x_m]/I$ for an ideal $I$. Since $R$ must be an integral domain, (otherwise we wouldn't be able to take the field of fractions) it follows that $I$ must be a prime ideal. However by lecture we can always find an irreducible algebraic set $V\subset K^m$ with $\mathcal{I}(V)=I$. Then since $\textrm{trdeg}(\textrm{Frac}(R) / K)=n$, $V$ is $n$ dimensional so by Noether's normalization lemma there is some finite morphism $\varphi : V \to K^n$. Equivalently, $R=\Gamma(V)$ is integral over $\varphi^*(K[t_1,\ldots,t_n])$. This second $K$-algebra can be identified with $K[a_1,\ldots, a_n]$ where $a_i=\varphi_i\in \Gamma(V)$ is the $i$-th coordinate function. This is what we were looking for.
\end{solution}    

\begin{problem}
    Say $K=\C$. Construct a surjective but nonfinite morphism $\varphi:V\rightarrow W$ between irreducible algebraic sets such that every $P\in W$ has only finitely many preimages.
\end{problem}

\begin{solution}
    Let $V=\mathcal{V}(x^2y+z-x)\subset\C^3$ and let $W=\C^2$. Construct the morphism $\varphi : V \to W$ as taking a point $(x,y,z)$ and sending it to $(y,z)$. First, let's show that every point in $W$ has a finite, nonempty preimage, this will show that $\varphi$ is surjective. Notice that when $y=0$ we have $\varphi(-z,0,z)=(0,z)$, so we have a single preimage. When $y\neq 0$, we have
    \[
        \varphi\left(\frac{1\pm \sqrt{1-4yz}}{2y}, y,z\right) = (z,y)
    \] 
    so we have two preimages in this case. Now recall from lecture that finite morphisms are closed maps in the Zariski topology, so suppose for the sake of contradiction that $\varphi$ is finite. Then consider the closed set $\mathcal{V}(z)\cap \mathcal{V}(x^2y+z-x)$. This is the algebraic set $\mathcal{V}(x(xy-1))$. Note that the image of this set under $\varphi$ is $(\C\setminus 0)\times \C$, which isn't closed (if it was, $\C^2$ would be reducible), a contradiction to the closedness of $\varphi$.
\end{solution}

\end{document}
