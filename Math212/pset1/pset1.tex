\documentclass{lkx_pset}

\title{Math 212 Problem Set 1}
\author{Lev Kruglyak}
\due{February 10, 2025}

\usepackage[T1]{fontenc}
\RequirePackage{mlmodern}

\renewcommand{\F}{\mathscr{F}}
\providecommand{\bmid}{\;\Big\lvert\;}

\ExplSyntaxOn
\NewDocumentCommand{\lkxto}{ O{} }{%
	\mathrel{\;\xrightarrow{\hphantom{xx}#1\hphantom{xx}}\;}
}
\NewDocumentCommand{\lkxmapsto}{ O{} }{%
	\mathrel{\;\xmapsto{\hphantom{xx}#1\hphantom{xx}}\;}
}
\NewDocumentCommand{\lkxisom}{ O{} }{%
	\mathrel{\;\xrightarrow{\hphantom{xx}#1\sim\hphantom{xx}}\;}
}
\NewDocumentCommand{\lkxsurj}{ O{} }{%
	\mathrel{\;\xtwoheadrightarrow{\hphantom{xx}#1\hphantom{xx}}\;}
}

\NewDocumentCommand{\lkxfunc}{ O{->} m m m g g }{
	\begin{array}{rcl}
		\IfNoValueTF {#5} {
			\tl_if_blank:nTF {#2}{}{#2 :}
			#3
			\str_case:nn {#1}
			{
				{->} {\lkxto}
					{~>} {\lkxisom}
					{->>} {\lkxsurj}
			}
			#4
		} {
			\tl_if_blank:nTF {#2}{}{#2 \;:\;}
			#3
		   &
			\str_case:nn {#1}
			{
				{->} {\lkxto}
					{~>} {\lkxisom}
					{->>} {\lkxsurj}
			}
		   & #4
		\\
		#5 & \lkxmapsto & #6
		}
	\end{array}
}

\ExplSyntaxOff



% \collaborator{AJ LaMotta}

\begin{document}
\maketitle

\begin{problem}{1}
  Find an example of a continuous function on $\R$ which goes to zero at infinity and which isn't the Fourier transform of a function in $L^1(\R)$.
\end{problem}
\begin{solution}
  Recall that the Fourier transform $\F$ maps $L^1(\R)$ functions to $C^0_0(\R)$ functions.
\end{solution}

\begin{problem}{2}
  The Schwarz space $\mathcal{S}(\R^d)$ is defined as
  \[
    \mathcal{S}(\R^d) = \left\{ f\in C^\infty(\R^d, \R) \bmid \lim_{|x|\to \infty}|(\mathcal{D} f)(x)|=0\quad\forall \mathcal{D}\in \mathrm{Diff}(\R[x_1,\ldots, x_d])\right\}
  \]
  where $\mathrm{Diff}(\R[x_1,\ldots,x_d])$ denotes the space of polynomial differential operators in $d$ variables. Prove that the Fourier transform maps the vector space $\mathcal{S}$ to itself.
\end{problem}
\begin{solution}
  Suppose $f\in \mathcal{S}(\R^d)$ is a Schwarz function. First, we claim that $f\in L^p(\R^d)$.
  Note that
  \[
    \|f\|_p = \left(\int_{\R^d}|f|^p\right)^{1/p} = 
  \]

\end{solution}

\end{document}
