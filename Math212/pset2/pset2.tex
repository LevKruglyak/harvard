\documentclass{lkx_pset}

\title{Math 212 Problem Set 2}
\author{Lev Kruglyak}
\due{February 18, 2025}

\usepackage[T1]{fontenc}
\RequirePackage{mlmodern}

\renewcommand{\F}{\mathscr{F}}
\providecommand{\bmid}{\;\Big\lvert\;}

\ExplSyntaxOn
\NewDocumentCommand{\lkxto}{ O{} }{%
	\mathrel{\;\xrightarrow{\hphantom{xx}#1\hphantom{xx}}\;}
}
\NewDocumentCommand{\lkxmapsto}{ O{} }{%
	\mathrel{\;\xmapsto{\hphantom{xx}#1\hphantom{xx}}\;}
}
\NewDocumentCommand{\lkxisom}{ O{} }{%
	\mathrel{\;\xrightarrow{\hphantom{xx}#1\sim\hphantom{xx}}\;}
}
\NewDocumentCommand{\lkxsurj}{ O{} }{%
	\mathrel{\;\xtwoheadrightarrow{\hphantom{xx}#1\hphantom{xx}}\;}
}

\NewDocumentCommand{\lkxfunc}{ O{->} m m m g g }{
	\begin{array}{rcl}
		\IfNoValueTF {#5} {
			\tl_if_blank:nTF {#2}{}{#2 :}
			#3
			\str_case:nn {#1}
			{
				{->} {\lkxto}
					{~>} {\lkxisom}
					{->>} {\lkxsurj}
			}
			#4
		} {
			\tl_if_blank:nTF {#2}{}{#2 \;:\;}
			#3
		   &
			\str_case:nn {#1}
			{
				{->} {\lkxto}
					{~>} {\lkxisom}
					{->>} {\lkxsurj}
			}
		   & #4
		\\
		#5 & \lkxmapsto & #6
		}
	\end{array}
}

\ExplSyntaxOff

\usepackage{hyperref}
\hypersetup{
	colorlinks=true,
	linkcolor=bluebord,
	urlcolor=bluebord
}


\collaborator{AJ LaMotta}

\begin{document}
\maketitle

\begin{problem}{1}
  Write a short lecture that you would give the class to define what a function of bounded variation is and how it defined a linear functional on $C^0[0,1]$. In doing that, give some examples of bounded variation functions that are not continuous.
\end{problem}
\begin{solution}
  Intuitively, a function of bounded variation (BV) doesn't oscillate too much. Formally, given any finite partition $P = \{0=x_0 < x_1 < \cdots < x_{n-1} < x_n = 1\}$ of the interval $[0,1]$, the variation of a function $f$ on $[0,1]$ is given by the sum
  \[
    \mathrm{Var}_P(f) = \sum_{1\leq k \leq n} |f(x_k)-f(x_{k-1})|.
  \]
  The total variation of a function is given by taking the supremum over all variations
  \[
    \mathrm{Var}(f) = \sup_{P} \mathrm{Var}_P(f)=\sup_{P}\sum_{1\leq k \leq n} |f(x_k)-f(x_{k-1})|.
  \]
  Thus, a function has bounded variation if $\mathrm{Var}(f)<\infty$. We denote the set of such functions by $\mathrm{BV}[0,1]$.

  Note that this condition is different from being bounded -- functions which oscillate really fast like $f(x)=x\sin(1/x)$ can have arbitrarily large variation by choosing the partition to lie along relative minima and maxima near the origin. On the other hand, a function like $f(x)=x^2\sin(1/x)$ is sufficiently dampened, and is hence of bounded variation. (At this point I would graph these functions out and show how to prove these claims). There are also many examples of functions which are BV but not continuous, for example the jump function 
  \[
    f(x) =\begin{cases} 0 & 0\leq x < c, \\ 1 & c\leq x \leq 1.\end{cases}
  \]
  In fact, any piecewise constant function will do. Another method of construction of BV functions is to look at monotonic functions which are bounded. (Explain this as well).

  Now let's recall the Riesz representation theorem for linear functions. This states that every bounded linear functional $T$ on $C^0[0,1]$ is representable by
  \[
    T(\varphi) = \int_0^1 \varphi \,d\mu
  \]
  where $\|T\|=|\mu|([0,1])$ for some signed measure $\mu$. The distributional measure of a BV function $f$ can be defined by the formula
  \[
    \mu_f((a,b]) = f(b)-f(a)
  \]
  and extending appropriately. The absolute value of the total measure is then the total variation of $f$. The linear functional corresponding to $f$ can then be defined by
  \[
    T_f(\varphi) = \int_0^1 \varphi \,d\mu_f.
  \]
  This gives $\mathrm{BV}[0,1]$ as a subset of the dual space of $C^0[0,1]$. It follows that any continuous linear functional on $C^0[0,1]$ is of this form. (Sketch proof).
\end{solution}

\begin{problem}{2}
  Let $z\in \C\setminus (-\infty,0]$ be a complex number. Use the Fourier transform to show that the differential equation
  \[
    \left[ -\left(\frac{\partial^2}{\partial x_1^2} + \cdots + \frac{\partial^2}{\partial x_n^2}\right)+z\right]f = g
  \]
  has a unique solution in $L^2(\R^d)$ when $g\in L^2(\R^d)$.
\end{problem}
\begin{solution}
  There are some technicalities here because we need functions $f$ to have second derivatives, but we'll ignore this in our solution. Let's begin by proving uniqueness. Suppose $f$ is a solution in $L^2(\R^d)$. Taking the Fourier transform of $g$, we have
  \begin{equation}
    \F(g)(k) = \F(-\nabla f + zf)(k) = (|k|^2+z)\F(f)(k) \quad\implies\quad \F(f)(k) = \frac{\F(g)(k)}{|k|^2 + z}.
  \end{equation}
  Note that the fraction have a non-zero denominator since $z\notin (-\infty, 0]$. Furthermore, the fraction is in $L^2(\R^d)$ since it is the Fourier transform of an $L^2(\R^d)$ function. The function $f$ thus can be written explicitly by 
  \begin{equation}\label{2}
    f = \F^{-1}\left(\frac{\F(g)(k)}{|k|^2+z}\right).
  \end{equation}
  This proves that if $f$ can be found in $L^2(\R^d)$, this solution exists uniquely. To prove existence, it suffices to show that $\F(g)(k)/(|k|^2+z)$ is an $L^2(\R^d)$ function whenever $g$ is. Let $\widehat{g} = \F(g)$. Then, note that
  \begin{equation}
    \int_{\R^d} \frac{|\widehat{g}(k)|^2}{\left||k|^2+ z\right|^2}\,dk \leq \|\widehat{g}\|^2_{L^2}\sup_{k\in \R^d}\frac{1}{\left||k|^2+z\right|^2}<\infty.
  \end{equation}
  This completes the proof, since we can now use (\ref{2}) to define $f$.
\end{solution}

\begin{problem}{3}
  Suppose that $p\in (1,\infty)$, that $u\in L^p(\R^d)$, and that $\{u_n\}\subset L^p(\R^d)$ is a sequence that converges weakly to $u$. Prove that
  \[
    \int |u|^p \leq \liminf_{n\to\infty} \int |u_n|^p.
  \]
  Now suppose that $\mathcal{H}$ is a Hilbert space, that $u\in \mathcal{H}$, and that $\{u_n\}\subset \mathcal{H}$ is a sequence that converges weakly to $u$. Prove that $\|u\|\leq \liminf_{n\to \infty}\|u_n\|$ with equality if and only if a subsequence of $\{u_n\}$ converges strongly to $u$.
\end{problem}
\begin{solution}
  First, we'll note that $f(t)=|t|^p$ is a convex function for $p>1$, so we have the inequality
  \begin{equation}
      |u_n(x)|^p-|u(x)|^p \geq \nabla f(u(x)) \cdot (u_n-u)(x) = p|u(x)|^{p-2}u(x)\cdot (u_n-u)(x).
  \end{equation}
  Note that this inequality still holds when $u(x)=0$, even though $f$ is not differentiable at $t=0$. Taking the integral of both sides, we get
  \begin{equation}
    \int_{\R^d} (|u_n|^p - |u|^p) \geq \int_{\R^d} p|u|^{p-2}u(u_n - u).
  \end{equation}
  Now let $g=p|u|^{p-2}u$. Notice that $|g|^q \leq p^q|u|^{(p-1)q}=p^q|u|^p$. Letting $q=p/(p-1)$, it follows that $g\in L^q(\R^d)$. This is exactly the H\"older complement to $p$, so by the duality isomorphism $(L^p(\R^d))^* = L_q(\R^d)$ and weak convergence, we have
  \[
    \liminf_{n\to \infty}\int_{\R^d} (|u_n|^p - |u|^p) \geq \lim_{n\to \infty}\int_{\R^d} g(u_n-u) = 0.
  \]
  Therefore, we have the desired inequality
  \[
    \int_{\R^d} |u|^p \leq \liminf_{n\to\infty}\int |u_n|^p.
  \]
  Next, suppose that we are in a Hilbert space $\mathcal{H}$. Weak convergence implies that $\langle u_n, u\rangle \to \langle u, u\rangle = \|u\|^2$, and by the Cauchy-Schwarz inequality $|\langle u_n, u\rangle| \leq \|u_n\|\|u\|$, we have
  \[
    \|u\|^2 = \lim_{n\to\infty}\langle u_n, u\rangle \leq \liminf_{n\to\infty} \|u_n\|\|u\| = \|u\|\liminf_{n\to\infty}\|u_n\|.
  \]
  If $\|u\|\neq 0$, dividing both sides by $\|u\|$ gives the inequality
  \[
    \|u\|\leq \liminf_{n\to\infty}\|u_n\|.
  \]
  If $\|u\|=0$, this inequality is trivially true. Now, it's clear that equality is achieved if there is a strongly convergent subsequence. Conversely, if $\|u\|=\liminf_{n\to \infty}\|u_n\|$, there must be some sequence $\{u_{n_k}\}$ such that $\lim_{k\to\infty} \|u_{n_k}\| = \|u\|$. However, we also know that $\langle u_{n_k}, u\rangle\to \|u\|^2$. Combined with the fact that
  \[
    \|u_{n_k}-u\|^2 = \|u_{n_k}\|^2 + \|u\|^2 - 2\langle u_{n_k}, u\rangle,
  \]
  it follows that $\lim_{k\to\infty}\|u_{n_k}-u\|^2 = \|u\|^2+\|u\|^2-2\|u\|^2 = 0$.
  Thus, $\{u_{n_k}\}$ converges strongly to $u$.
\end{solution}

\end{document}
