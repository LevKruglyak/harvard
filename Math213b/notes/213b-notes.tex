\documentclass{lkx_paper}

\title{\textbf{Riemann Surfaces}}
\date{}
\author{Lev Kruglyak}

\usepackage{graphicx}

\newcommand{\D}{\mathbb{D}}
\newcommand{\x}{\mathbf{x}}
\usepackage{pgfplots}
\pgfplotsset{
	compat=1.12,
}


\begin{document}
\maketitle

\section{Fundamental Theorem}

\todo{prior stuff}

\subsection{Dirichlet Norm}

\begin{definition}
	Let $X$ be a Riemann surface and $f\in \Omega^0(X)$. We define the  \defn{Dirichlet norm} of $f$ as:
	\[
		\|f\|_D^2 = 2i\int_X \dd f\wedge \ddc f.
	\]
\end{definition}

This is a \emph{semi-definite} norm, since $\|f\|_D = 0$ implies that $f$ is (locally) constant. We can, of course, work in the modulo space of smooth functions up to equivalence up to locally constant functions and obtain a definite norm on this space instead.

\begin{definition}
	The Dirichlet norm then induces the \defn{Dirichlet (semi-definite) inner product} on $\Omega^0(X)$, given by:
	\[
		\langle f, g\rangle_D = i\int_X \dd f \wedge \ddc g + \dd g\wedge \ddc f.
	\]
\end{definition}

If $X$ is a compact Riemann surface, we can integrate by parts:
\[
	\begin{aligned}
		\int \dd f \wedge \ddc g & = \int df\wedge \ddc g - \cancel{\,\int \ddc f\wedge \ddc g\,} \\
		                         & = -\int f\wedge d\ddc g =
	\end{aligned}
\]
\todo{finish this derivation}

\begin{proposition}
	If $X$ is a compact Riemann surface, then for any $f,g$ we have
	\[
		\langle f, g\rangle_D = 2i\int_X \dd f \wedge \ddc g.
	\]
	Alternatively, we could write
	\[
		\langle f, g\rangle_D = \int_X f\cdot \Delta g = \int_X g\cdot \Delta f
	\]
	where $\Delta : \Omega^0(X) \to \Omega^2(X)$ is the Laplace operator, given by $\Delta = 2i\,\ddc\dd$.
\end{proposition}

\subsection{Lebesgue Norm}

\begin{definition}
	Given an \emph{area form} $\omega\in\Omega^2(X)$, we define the \defn{Lebesgue norm} ($L^2$), given on $f\in \Omega^0(X)$ by
	\[
		\|f\|^2_{L^2} = \int_X |f|^2\,\omega.
	\]
\end{definition}

This is in some sense less canonical than the Dirichlet norm since it requires the additional data of an area form.

\subsection{Poincar\'e Inequality}

\begin{theorem}[Poincar\'e Inequality]
	Given a compact Riemann surface $X$ with area form $\omega$, then there exists a constant $C>0$ such that
	\[
		\|f\|_{L^2} \leq C\cdot \|f\|_D
	\]
	for all $f\in \Omega^0(X)$ with $\int_X f\,\omega = 0$.
\end{theorem}

We'll begin by proving the special case when $X=S^2\subset \R^3$ is the unit sphere, equipped with standard area form $\omega$. In this standard area form, we have
\[
	\begin{aligned}
		\Delta f & = 2i\,\ddc\dd f       \\
		         & = \bbDelta f\, \omega
	\end{aligned}
\]
This operator $\bbDelta :\Omega^0(X) \to \Omega^0(X)$ is called the \emph{spherical Laplacian}.

\begin{lemma}
	$\bbDelta$ admits an \emph{complete orthonormal} system of eigenfunctions $v_0, v_1,\ldots$ with eigenvalues $0=\lambda_0 < \lambda_1\leq \lambda_2\leq \cdots$, i.e. the $v_i$ are pairwise orthonormal in the Lebesgue norm, and their span is dense.
\end{lemma}

\begin{proof}
	To find the eigenfunctions of $\bbDelta$ on $S^2$, look for polynomials $F(x_1, x_2, x_3)$ on $\R^3$ which are harmonic, i.e. $\Delta_{\R^3} F = 0$, and restrict them to the sphere.
\end{proof}

\begin{proof}[Proof of the Poincar\'e Inequality for the Sphere]
	Given some $f \in \Omega^0(X)$, we can expand:
	\[
		f = \sum^\infty_{i=1} a_i v_i\quad\implies\quad \bbDelta f = \sum^\infty_{i=1} \lambda_i a_i v_i
	\]
	by convergence of partial sums in the $L^2$ metric. Note that the assumption about $\int_X f\, \omega$ is what allows us to ignore the eigenfunction $v_0$ in the expansion. (This requires more details about the construction of the eigenfunctions) In particular, we have
	\[
		\|f\|^2_{L^2} = \int_X |f|^2\,\omega =  \sum^\infty_{i=1}|a_i|^2\quad\textrm{and}\quad \|f\|^2_D = \int_X \Delta f\, \omega = \sum^\infty_{i=1} \lambda_i|a_i|^2.
	\]
	Therefore, since $\lambda_1$ is minimal among the eigenvalues, we have
	\[
		\|f\|^2_{L^2} \leq \frac{1}{\lambda_1}\cdot \|f\|^2_D
	\]
	so a constant for the Poincar\'e inequality is $C = \lambda_1^{-1/2}.$
\end{proof}

This type of proof works any time we can understand the eigenvalues of the Laplacian operator on a compact Riemann surface -- for $S^2$ it follows from spherical harmonics.

More generally, let's consider what happens on a disk $D^2$. In this case, we could prove that there is always a constant $C>0$ such that for all $f$, there is some $\overline{f}\in \R$ such that
\[
	\|f-\overline{f}\|_{L^2} \leq C\cdot \|f\|_D.
\]
Easier to work with is the case of the disk $S_+ \subset S^2$, i.e. the upper hemisphere of the unit sphere. This disk is biholomorphic to the flat disk by stereographic projection, and the volume form $\omega_{D^2}$ changes by a bounded factor from $\omega_{S^2}$, i.e.
\[
	c_0\cdot \omega_{S_+} \leq \omega_{D^2}\leq c_1\cdot \omega_{S_+},
\]
for constants $c_0, c_1>0$. Generally, for a disk with any area form, we can find a similar transformation, so it suffices to prove the inequality for the hemisphere disk.

\begin{proof}[Proof of the Poincar\'e Inequality for the Disk]
	Let $f \in \Omega^0(S_+)$, which satisfies $\int_{S_+} f\,\omega = 0$. We can form a continuous (but not necessarily smooth) function $F \in \Omega^0(S^2)$ by reflection over the equator, and it's clear that the condition $\int_{S^2} F\,\omega=0$ is still satisfied.

	Although $F$ need not be smooth at the equator, it can be approximated by a family of smooth maps $F_n$, which converge to $F$ in both Dirichlet and Lebesgue norms. Hand-waving, this involves smoothing out equator singularities using successively smaller neighborhoods of the equator. Finally, note that
	\[
		\|F\|_D = \int_{S^2} |dF|^2\cdot \omega = 2\int_{S^+} |df|^2\,\omega = 2\cdot \|f\|_D
	\]
	and similarly for the Lebesgue norm, so we can conclude by applying the Poincar\'e Inequality for the sphere.
\end{proof}

Finally, we have enough machinery to prove the full Poincar\'e Inequality.

\begin{proof}{Proof of Poincar\'e Inequality}
	Let $X$ be a compact connected Riemann surface, and cover $X$ by a finite set of closed disks $B_1,\ldots, B_k$. Assume for the sake of contradiction that there is a family of functions $f_n \in \Omega^0(X)$ satisfying
	\[
		\int_X f_n \,\omega = 0, \quad \|f_n\|_{L^2}=1,\quad\textrm{and}\quad \lim_{n\to \infty}\|f_n\|_D \to 0.
	\]
	Let $a_i(f_n)$ be the average value of $f_n$ on $B_i$, so
	\[
		\int_{B_i} \left(f_n - a_i(f_n)\right)\,\omega = 0.
	\]
	But by the Poincar\'e Inequality for the disk $B_i$, we know that
	\[
		\begin{aligned}
			\int_{B_i} \left|( f_n - a_i(f_n)\right|^2\, \omega
			\quad & \leq\quad C_i^2 \|\restr{f_n}{B_i}\|^2_D \\
			\quad & \leq\quad C_i^2 \|f_n\|_D^2 \longto 0.
		\end{aligned}
	\]
	The $a_i(f_n)$ are bounded, so we can pass to a subsequence with $a_i(f_n) \to \overline{a_i}$ as $n\to \infty$,
	i.e. $f_n-\overline{a_i}$ is a sequence converging to zero in the $L^2$ norm on $B_i$. Then we have the property that
	\[
		\lim_{n\to \infty}\int_{B_i}\| f_n - \overline{a_i}\|^2\,\omega = 0,
	\]
	Note that each $B_i$ might have different constant $\overline{a_i}$, however if any two balls overlap, they have the same constant $\overline{a_i}$. Thus, there must be a global constant $\overline{a_1}$ since the manifold is connected, and $f_n \to \overline{a_1}$ almost everywhere in the $L^2$ norm. But $\overline{a_1}$ must be $0$ since $\int_X f_n\,\omega = 0$.

	Now, integrating over the whole of $X$, we get
	\[
		\int_X |f_n|^2\,\omega \leq \sum^k_{i=1} \int_{B_i} |f_n|^2\,\omega = \int_{B_i} |f_n-\overline{a_i}|\,\omega \longto 0.
	\]
	This contradicts the assumption that $\|f_n\|_{L^2} = 1$.
\end{proof}

\subsection{Poisson Equation}

Let $X$ be a compact connected Riemann surface, and let $\rho\in \Omega^2(X)$ be an area form which satisfies $\int_X \rho = 0$.

\begin{definition}
	The \defn{Poisson equation} states that
	\[
		\Delta \varphi = \rho
	\]
	for some $\varphi\in \Omega^0(X)$.
\end{definition}

\begin{theorem}[Fundamental Theorem]
	There is a solution to $\Delta \varphi = \rho$, unique up to addition of a constant.
\end{theorem}

Let $\mathcal{H} = \Omega^0(X) / \R$ be the quotient space of smooth functions by the addition of constant terms. Consider the operator:
\[
	\mathcal{L} : \mathcal{H} \to \R\quad\textrm{given by}\quad \mathcal{L}(\varphi) = \|\varphi\|^2_D - 2\int_X \varphi\, \rho.
\]
This operator is well-defined, i.e. $\mathcal{L}(\varphi) = \mathcal{L}(\varphi + c)$ because $\int_X \rho = 0$.

This operator is motivated by the observation that if $\mathcal{L}$ achieves its infimum at $\varphi\in \mathcal{H}$, then $\Delta \varphi = \rho$. To see why, notice that for all $\psi \in \mathcal{H}$,
\[
	\begin{aligned}
		0 & = \restr{\frac{d}{dt}}{t=0} \mathcal{L}(\varphi + t\psi)                                                                                \\
		  & = \restr{\frac{d}{dt}}{t=0} \left(\left\langle \varphi + t\psi, \varphi + t\psi\right\rangle_D - 2\int_X (\varphi + t\psi)\,\rho\right) \\
		  & = 2\left\langle \varphi, \psi\right\rangle_D - 2\int_X \psi\,\rho
		= 2\int_X \psi\, \Delta \varphi - 2\int_X \psi\,\rho                                                                                        \\
		  & = 2\int_X \psi\, (\Delta \varphi -\rho)
	\end{aligned}
\]

\begin{lemma}
	$\mathcal{L}$ is bounded below.
\end{lemma}

\begin{proof}
	Writing $\rho = R\cdot \omega$ for some $R\in \Omega^0(X)$, we can write
	\[
		\begin{aligned}
			\mathcal{L}(\varphi) = \|\varphi\|_D - 2\int_X \varphi\, \rho = \|\varphi\|_D - 2\int_X (\varphi\cdot R)\, \omega.
		\end{aligned}
	\]
	Letting $\varphi_1 = \varphi - \textrm{avg}(\varphi)$ be another representative of the equivalence class of $\varphi\in \mathcal{H}$, we get the inequality
	\[
		\begin{aligned}
			\|\varphi\|_D - 2\int_X (\varphi_1\cdot R)\, \omega
			 & \geq \|\varphi\|_D^2 - 2\left(\int \varphi_2^2\, \omega\right)^{1/2}\left(\int R^2\,\omega\right)^{1/2}
			=\|\varphi\|_D^2 - 2\|\varphi_1\|_{L^2} \cdot \|R\|_{L^2}                                                  \\
			 & \geq \|\varphi\|_D^2 - 2C\|\varphi_1\|_D\|R\|_{L^2}
			= \|\varphi_D^2\|- 2C_1\|\varphi\|_D                                                                       \\
			 & = \left(\|\varphi\|_D - C_1\right)^2 - C_1^2                                                            \\
			 & \geq -C_1^2                                                                                             \\
		\end{aligned}
	\]
	by applying the Cauchy-Schwartz and Poincar\'e inequalities.
\end{proof}

Given the existence of this lower bound, let's set \[I=\inf_{\varphi\in\mathcal{H}} \mathcal{L}(\varphi)\] and choose a sequence $\varphi_i\in\mathcal{H}$ with $\mathcal{L}(\varphi_i)\to I$ as $i\to \infty$.

\begin{lemma}
	$\varphi_i$ are a Cauchy sequence in the Dirichlet norm.
\end{lemma}

\begin{proof}
	Let $\varepsilon > 0$ be given. Choose an $i_0$ so that $I \leq \mathcal{L}(\varphi_i) \leq I +\epsilon$ for all $i\geq i_0$.
	Now for any $i, j \geq i_0$, consider the function
	\[
		F(t) = \mathcal{L}(\varphi_i + t(\varphi_j - \varphi_i))
	\]
	for $t\in [0,1]$. This is shaped quadratically, i.e. $F(t) = At^2+Bt+C$ with $A = \|\varphi_j-\varphi_i\|^2_D$. But $F(t)\geq I$ for $t\in [0,1]$ since $F(0), F(1)\leq I +\epsilon$. This bound can be used to get
	\[
		\begin{aligned}
			\|\varphi_j - \varphi_i\|^2 = A & = 2\left( F(0) - 2F(1/2) + F(1)\right)                  \\
			                                & \leq 2\left((I+\varepsilon)-2I + (I+\varepsilon)\right) \\
			                                & \leq 4\varepsilon.
		\end{aligned}
	\]
	Since $\|\varphi_i - \varphi_j\|_D\leq 2\sqrt{\varepsilon}$, we are done.
\end{proof}

Just because $\varphi_i$ is a Cauchy sequence doesn't mean that it converges, since $\mathcal{H}$ is not necessarily a complete metric space. We can however form the \emph{completion}, which we denote $\widehat{\mathcal{H}}$. In this space, we can define
\[
	\int_X \varphi\,\rho = \lim_{i\to \infty}\int_X \varphi_i\,\rho \quad\textrm{for any}\quad \varphi \in \widehat{\mathcal{H}},
\]
so there is an extension $\mathcal{L} : \widehat{\mathcal{H}} \to \R$. In fact, the inner product also extends to the completion, so $\widehat{\mathcal{H}}$ is a \emph{Hilbert space}.
Hehere, there is some $\varphi\in \widehat{\mathcal{H}}$ which is tautologically the limit of the Cauchy sequence $\varphi_i$.

Generally, the convergence of some sequence $f_i\to f$ in the Dirichlet norm means that after adjusting constraints so that $\int_X f_i\,\omega=0$, the sequence $\{f_i\}$ is Cauchy in the $L^2$ norm as well, by the Poincar\'e. In our case, by Lebesgue, there is some measurable function
\[
	\Phi : X \to \R \quad\textrm{with}\quad \int_X |\Phi|^2\,\omega < \infty
\]
and a subsequence $\varphi_i \to \Phi$ almost everywhere with $\int_X |\Phi - \varphi_i|^2\,\omega \to 0$.

Now, since
\[
	2\left\langle \varphi, \psi\right\rangle_D - 2\int_X \psi\,\rho = 0\quad\textrm{for any}\quad \psi\in \mathcal{H},
\]
we can write
\[
	\lim_{i\to \infty}\left(\left\langle \varphi_i, \psi\right\rangle_D - \int_X \psi\,\rho\right) =
	\lim_{i\to \infty}\left(\int_X \varphi_i \,\Delta \psi - \int_X \psi\,\rho\right) = 0.
\]
In the $L^2$ norm, we get convergence $\varphi_i \to \Phi$, and applying Cauchy-Schwartz to first order terms gives us
\[
	\int_X \Phi\,\Delta \psi - \int_X \psi\,\rho = 0\quad\textrm{for all}\quad \psi\in \mathcal{H}.
\]
So we have a Lebesgue integrable solution, but we need a smooth solution.
% If $\Phi$ were smooth, integration by parts allows us to rewrite
% \[
% 	\int_X \Phi\,\Delta \psi - \int_X \psi\,\rho = \int_X \Delta \Phi\, \psi - \int\rho\,\psi = 0\quad\implies\quad \Delta \Phi = \rho.
% \]
% Ignoring smoothness, a $\Phi$ is called a \defn{weak solution} to $\Delta \Phi = \rho$.

In general, for a \emph{smooth} $f$ on a bounded open set $\Omega\subset \R^2$, we have
\[
	\Delta f - v = 0\quad\iff\quad \int_{\Omega} (\Delta f - v)\,\psi = 0 \quad\textrm{for all}\quad\psi\in \Omega^0_{\textrm{compact}}(\Omega),
\]
simply by integration by parts, i.e.
\[
	\int_\Omega (\Delta f - v)\,\psi = \int_{\Omega} f\,\Delta\psi - \int v\,\psi = 0.
\]
This last expression now makes sense for integrable functions, so we might call such an $f$ a \defn{weak solution}.

\begin{lemma}[Weyl's Lemma]
	If $f$ is a weak solution, then $f$ is almost everywhere equal to a smooth $f_1$ which is a strong solution.
\end{lemma}

\begin{proof}
	Let $\varphi_1$ be a smooth, non-negative function on $\R^2$ with integral $1$ which is supported in $|z|<1$ and is a function of radius only. Given such a function, we get an entire family $\varphi_t$ given by
	\[
		\varphi_t(z) = t^{-2}\varphi_1(z/t),\quad\textrm{for}\quad t < 1.
	\]
	Clearly, $\int_{\R^2} \varphi_t(z)\,dz = 1$, and $\varphi_t$ is supported in $r<t$. Consider the convolution
	\[
		(\varphi_t * f)(z) = \int_{\R^2} f(w)\varphi_t(z-w)\,d\mu_w.
	\]
	We can now deduce a few properties:
	\[
		\begin{aligned}
			(1) & \quad\lim_{t\to 0}\phantom{}^{L^1} \varphi_t * f = f,\quad\textrm{i.e.}\quad \int_{R^2}|\varphi_t * f - f|\,d\mu \to 0, \\
			(2) & \quad \varphi_t * f\quad \textrm{is smooth,}                                                                            \\
			(3) & \quad \varphi_s * f = \varphi_t * f\quad\textrm{on}\quad \Omega^{(t)}\subset \Omega \quad\textrm{for}\quad s\leq t
		\end{aligned}
	\]
	\todo{finish this proof, generalizing involving homogeneous case}
\end{proof}

\section{Riemann-Roch Theorem}

Recall that on a surface $X$ of genus $g$, at some $p\in X$, there exists a non-constant meromorphic function $f$ with poles of multiplicity $\leq g+1$ at $p$. The proof of this fact depended on the fact that $\dim H^{0,1}(X)=g$

\subsection{Divisors}

Instead of saying ``$f$ is meromorphic with poles at $p,q$ of multiplicity $\leq a,b$ respectively, and vanishing with multiplicity $\geq c,d$ at $r,s$ respectively''', we would like to use the language of \emph{divisors}.
Recall that
\[\ord_p(f) = \begin{cases}
		-k & p\textrm{ is a pole of multiplicity }k, \\
		k  & p\textrm{ is a zero of multiplicity }k, \\
		0  & p\textrm{ otherwise},
	\end{cases}\quad \implies \quad f(z) = \sum^\infty_{j=\ord_p(f)} a_j z^j\]
with $a_{\ord_p(f)}\neq 0$.

\begin{definition}
	The \defn{divisor group} of a surface $X$ is the free abelian group
	\[
		\Div(X) = \bigoplus_{p\in X}p\cdot \Z.
	\]
	We say that $D\in \Div(X)$ is \defn{non-negative}, or $D\geq 0$ if all of its coefficients are $\geq 0$.
\end{definition}

\begin{definition}
	For a divisor $D=\sum_{p\in X} m_p\cdot p$, we say that the \defn{degree} of $X$ is the sum of coefficients $\deg(D)=\sum_{p\in X} m_p$. Given two divisors $D,D'\in \Div(X)$, we say that $D\geq D'$ if $\deg(D)\geq \deg(D')$.
\end{definition}

\begin{definition}
	For $f$ a non-zero meromorphic function, the \defn{divisor} of $f$ is
	\[
		\div(f) = \sum_{p\in X}\ord_p(f)\,p\in \Div(X)
	\]
	These are called \defn{principal divisors}, and form a subgroup of $\Div(X)$.
\end{definition}
So rephrasing our earlier expression, we could now say that $\div(f)+D\geq 0$ where $D = ap + bq - cr - ds$.
\begin{definition}
	Let $X$ be a connected compact surface. For $D\in \Div(X)$, let
	\[
		\mathcal{L}(D) = \{ f\in \mathcal{M}_X : \div(f) + D \geq 0\quad\textrm{or}\quad f = 0\}.
	\]
	With the addition of $f=0$, this becomes a vector space. Let $\ell(D)=\dim \mathcal{L}(D)$.
\end{definition}

\begin{example}
	For example, if $X$ is a compact surface of genus $g$ and $D = (g+1)p$, then we've seen that $\ell(D)\geq 2$.
\end{example}

\begin{example}
	When $g=2$ and $D=-q$ for some $q\in X$, we have $\ell(D)=0$.
\end{example}

\begin{example}
	On $X=\CP^1$, any divisor of the form $D = p-q$ is principal by the function $f=(z-a)/(z-b)$. Thus, every divisor is principal.
\end{example}

\begin{definition}
	Let $\alpha$ be a meromorphic $(1,0)$-form. Then we say
	\[
		\div(\alpha)= \sum_{p\in X} \ord_p(\alpha)\,p.
	\]
	Here, $\ord_p(\alpha) = \ord_p(a)$ when $\alpha = a(z)\,dz$.
\end{definition}
\begin{lemma}
	For any meromorphic $(1,0)$-form $\alpha$, we have $\deg\div(\alpha) = 2g-2$.
\end{lemma}

\begin{definition}
	A \defn{canonical divisor} on $X$ is the divisor of a meromorphic $(1,0)$-form.
\end{definition}

\begin{lemma}
	Any two canonical divisors differ by a principal divisor.
\end{lemma}

\subsection{Residue Theorem}

\begin{theorem}
	If $X$ is a compact connected Riemann surface, and $\alpha$ is a meromorphic $(1,0)$-form on $X$ with poles $p_1,\ldots, p_n$, then
	\[
		\sum^m_{j=1}\Res_{p_i}(\alpha)=0.
	\]
\end{theorem}

\begin{proof}
	Cut out the poles $p_i$, and use Stokes' theorem, since the integral along the boundary is the sum of the residues.
\end{proof}

\subsection{Riemann-Roch Theorem}

\begin{theorem}[Riemann-Roch]
	Let $X$ be a compact connected surface of genus $g$, $D\in \Div(X)$, and $K$ some canonical divisor. Then,
	\[
		\ell(D) - \ell(K-D) = \deg(D)-(g-1).
	\]
\end{theorem}

How can we interpret this result?
\begin{itemize}
	\item $\ell(D) = \dim\mathcal{L}(D)$.
	\item For $\ell(K-D)$, consider the space
	      \[
		      \Omega(E) = \{\alpha \textrm{ is a meromorphic }(1,0)\textrm{-form}\, :\,\div(\alpha) + E \geq 0, \textrm{ or } \alpha = 0\}.
	      \]
	      Then,
\end{itemize}
\todo{idk what the fuck is going on}

\begin{example}
	When $X$ has genus $2$, and $D=2p$, we get
	\[
		\dim \mathcal{L}(2p)-\dim\Omega(-2p) = 1.
	\]
	Equivalently,
	\[
		\begin{aligned}
			\exists\, f\in \mathcal{M}_X\textrm{ with double pole at }p
			\quad & \implies\quad \dim\mathcal{L}(2p)=2                                       \\
			\quad & \implies\quad \dim\Omega(-2p)=1                                           \\
			\quad & \implies \quad \exists\, (1,0)\textrm{-form with at least double 0 at }p.
		\end{aligned}
	\]
\end{example}

Recall the cohomology groups
\[
	\begin{aligned}
		H^{0,1}(X) \quad\quad\quad & \Omega^{0,1}(X)/\textrm{Im}(\overline{\partial}) \\
		H^{1,0}(X) \quad\quad\quad & \textrm{holomorphic }(1,0)\textrm{-forms}
	\end{aligned}
\]
\begin{lemma}
	There is a well-defined, non-degenerate bilinear form
	\[
		\definefunction{\langle\cdot,\cdot \rangle}{H^{0,1}(X)\times H^{1,0}(X)}{\C}{([\beta], [\alpha])}{\int_X \beta\wedge \alpha}
	\]
\end{lemma}

\begin{proof}
	To check that it is well defined, note that if $[\beta]=[\beta + \overline{\partial}\varphi]$, then
	\[
		\int_X(\beta +\overline{\partial}\varphi)\wedge \alpha = \int_X \beta\wedge \alpha + \int_X \overline{\partial}\varphi \wedge \alpha.
	\]
	However, by Stoke's theorem, $\int_X \overline{\partial}\varphi \wedge \alpha = -\int \varphi \wedge \overline{\partial}\alpha=0$ since $\alpha$ is holomorphic, and so the term is zero as well.

	To check non-degeneracy , given some $\alpha$, set $\beta = \overline{\alpha}$. Then we have
	\[
		\int_X \overline{\alpha}\wedge \alpha = 2i\,\|\alpha\|^2_{L^2},
	\]
	which is non-zero if $\alpha\neq 0$.
\end{proof}

\begin{corollary}
	For $\beta\in \Omega^{0,1}(X),$ the is a smooth function $\varphi\in C^\infty(X; \C)=\Omega^{0,0}(X)$ with $\overline{\partial}\varphi = \beta$ if and only if for all $\alpha\in H^{1,0}(X)$, we have $\int_X \beta\wedge \alpha = 0$.
\end{corollary}

\begin{theorem}
	Given $m$ principal parts $P_i=a_{-k,i}z^{-k}+\cdots + a_{-1,i} z^i$, and points $p_i$, there exists a function $f$ with principal part $P_i$ at $p_i$ if and only if for all $\alpha\in H^{1,0}(X)$, we have
	\[
		\sum^m_{i=1} \Res_{p_i}(P_i\cdot \alpha)=0
	\]
\end{theorem}

\end{document}
