\documentclass{lkx_paper}

\title{\textbf{Riemann Surfaces}}
\date{}
\author{Lev Kruglyak}

\usepackage{graphicx}

\newcommand{\D}{\mathbb{D}}
\newcommand{\x}{\mathbf{x}}
\usepackage{pgfplots}
\pgfplotsset{
	compat=1.12,
}

\newcommand{\dd}{\partial}
\newcommand{\ddc}{\overline{\partial}}

\usepackage{bbm}
\newcommand{\bbDelta}{\bm{\Delta}}


\begin{document}
\maketitle

\section{Differential Forms}

\todo{prior stuff}

\subsection{Dirichlet Norm}

\begin{definition}
	Let $X$ be a Riemann surface and $f\in \Omega^0(X; \R)$. We define the  \defn{Dirichlet norm} of $f$ as:
	\[
		\|f\|_D^2 = 2i\int_X \dd f\wedge \ddc f.
	\]
\end{definition}

This is a \emph{semi-definite} norm, since $\|f\|_D = 0$ implies that $f$ is (locally) constant. We can, of course, work in the modulo space of smooth functions up to equivalence up to locally constant functions and obtain a definite norm on this space instead.

\begin{definition}
	The Dirichlet norm then induces the \defn{Dirichlet (semi-definite) inner product} on $\Omega^0(X; \R)$, given by:
	\[
		\langle f, g\rangle_D = i\int_X \dd f \wedge \ddc g + \dd g\wedge \ddc f.
	\]
\end{definition}

If $X$ is a compact Riemann surface, we can integrate by parts:
\[
	\begin{aligned}
		\int \dd f \wedge \ddc g & = \int df\wedge \ddc g - \cancel{\,\int \ddc f\wedge \ddc g\,} \\
		                         & = -\int f\wedge d\ddc g =
	\end{aligned}
\]
\todo{finish this derivation}

\begin{proposition}
	If $X$ is a compact Riemann surface, then for any $f,g$ we have
	\[
		\langle f, g\rangle_D = 2i\int_X \dd f \wedge \ddc g.
	\]
	Alternatively, we could write
	\[
		\langle f, g\rangle_D = \int_X f\cdot \Delta g = \int_X g\cdot \Delta f
	\]
	where $\Delta : \Omega^0(X) \to \Omega^2(X)$ is the Laplace operator, given by $\Delta = 2i\,\ddc\dd$.
\end{proposition}

\subsection{Lebesgue Norm}

\begin{definition}
	Given an \emph{area form} $\omega\in\Omega^2(X; \R)$, we define the \defn{Lebesgue norm} ($L^2$), given on $f\in \Omega^0(X; \R)$ by
	\[
		\|f\|^2_{L^2} = \int_X |f|^2\,\omega.
	\]
\end{definition}

This is in some sense less canonical than the Dirichlet norm since it requires the additional data of an area form.

\begin{proposition}[Poincar\'e Inequality]
	Given a compact Riemann surface $X$ with area form $\omega$, then there exists a constant $C>0$ such that
	\[
		\|f\|_{L^2} \leq C\cdot \|f\|_D
	\]
	for all $f\in \Omega^0(X; \R)$ with $\int_X f\,\omega = 0$.
\end{proposition}

We'll begin by proving the special case when $X=S^2\subset \R^3$ is the unit sphere, equipped with standard area form $\omega$. In this standard area form, we have
\[
	\begin{aligned}
		\Delta f & = 2i\,\ddc\dd f       \\
		         & = \bbDelta f\, \omega
	\end{aligned}
\]
This operator $\bbDelta :\Omega^0(X) \to \Omega^0(X)$ is called the \emph{spherical Laplacian}.

\begin{lemma}
	$\bbDelta$ admits an \emph{complete orthonormal} system of eigenfunctions $v_0, v_1,\ldots$ with eigenvalues $0=\lambda_0 < \lambda_1\leq \lambda_2\leq \cdots$, i.e. the $v_i$ are pairwise orthonormal in the Lebesgue norm, and their span is dense.
\end{lemma}

\begin{proof}
	To find the eigenfunctions of $\bbDelta$ on $S^2$, look for polynomials $F(x_1, x_2, x_3)$ on $\R^3$ which are harmonic, i.e. $\Delta_{\R^3} F = 0$, and restrict them to the sphere.
\end{proof}

\begin{proof}[Proof of the Poincar\'e Inequality for the Sphere]
	Given some $f \in \Omega^0(X)$, we can expand:
	\[
		f = \sum^\infty_{i=1} a_i v_i\quad\implies\quad \bbDelta f = \sum^\infty_{i=1} \lambda_i a_i v_i
	\]
	by convergence of partial sums in the $L^2$ metric. Note that the assumption about $\int_X f\, \omega$ is what allows us to ignore the eigenfunction $v_0$ in the expansion. (This requires more details about the construction of the eigenfunctions) In particular, we have
	\[
		\|f\|^2_{L^2} = \int_X |f|^2\,\omega =  \sum^\infty_{i=1}|a_i|^2\quad\textrm{and}\quad \|f\|^2_D = \int_X \Delta f\, \omega = \sum^\infty_{i=1} \lambda_i|a_i|^2.
	\]
	Therefore, since $\lambda_1$ is minimal among the eigenvalues, we have
	\[
		\|f\|^2_{L^2} \leq \frac{1}{\lambda_1}\cdot \|f\|^2_D
	\]
	so a constant for the Poincar\'e inequality is $C = \lambda_1^{-1/2}.$
\end{proof}

This type of proof works any time we can understand the eigenvalues of the Laplacian operator on a compact Riemann surface -- for $S^2$ it follows from spherical harmonics.

More generally, let's consider what happens on a flat disk $D^2\subset \R^2$. In this case, we could prove that there is always a constant $C>0$ such that for all $f$, there is some $\overline{f}\in \R$ such that
\[
	\|f-\overline{f}\|_{L^2} \leq C\cdot \|f\|_D.
\]
Easier to work with is the case of the disk $S_+ \subset S^2$, i.e. the upper hemisphere of the unit sphere. This disk is biholomorphic to the flat disk by stereographic projection, and the volume form $\omega_{D^2}$ changes by a bounded factor from $\omega_{S^2}$, i.e.
\[
	c_0\cdot \omega_{S_+} \leq \omega_{D^2}\leq c_1\cdot \omega_{S_+},
\]
for constants $c_0, c_1>0$.

\begin{proof}[Proof of the Poincar\'e Inequality for the Disk]
	Let $f \in \Omega^0(S_+; \R)$, which satisfies $\int_{S_+} f\,\omega = 0$. We can form a continuous (but not necessarily smooth) function $F \in \Omega^0(S^2; \R)$ by reflection over the equator, and it's clear that the condition $\int_{S^2} F\,\omega=0$ is still satisfied.

	Although $F$ need not be smooth at the equator, it can be approximated by a family of smooth maps $F_n$, which converge to $F$ in both Dirichlet and Lebesgue norms. Hand-waving, this involves smoothing out equator singularities using successively smaller neighborhoods of the equator. Finally, note that
	\[
		\|F\|_D = \int_{S^2} |dF|^2\cdot \omega = 2\int_{S^+} |df|^2\,\omega = 2\cdot \|f\|_D
	\]
	and similarly for the Lebesgue norm, so we can conclude by applying the Poincar\'e Inequality for the sphere.
\end{proof}

\end{document}
