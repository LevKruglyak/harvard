\documentclass{pset}

\title{Math 213b Problem Set 1}
\author{Lev Kruglyak}
\due{January 31, 2024}

\usepackage{graphicx}

\newcommand{\D}{\mathbb{D}}
\newcommand{\x}{\mathbf{x}}
\usepackage{pgfplots}
\pgfplotsset{
	compat=1.12,
}

\newcommand{\dd}{\partial}
\newcommand{\ddc}{\overline{\partial}}

\usepackage{bbm}
\newcommand{\bbDelta}{\bm{\Delta}}


\begin{document}
\maketitle

\begin{problem}{1}
  Recall from class that an end $e$ of a topogical space $X$ is an assignment of a path component $e(K)$ of the complement $X\setminus K$ for every compact subset $K\subset X$ subject to the requirement that inclusions are preserved in reverse: i.e. if $K'\subset K$, then $e(K')\subset e(K)$.
\end{problem}


\begin{parts}
  \begin{part}{a}
    Using the definition, show fairly carefully that the real line $\R$ has two ends while $\R^2$ has only one end.
  \end{part}

  \begin{part}{b}
    Suggest what you think the classification should be for connected, orientable
    2-manifolds with three ends. No proof required. Feel free to look up the classifi-
    cation of surfaces with finitely many ends.
  \end{part}
\end{parts}

\begin{problem}{2}
  \todo{[figures]} Figure 1 depicts part of a doubly-periodic surface $X$, extending in-
finitely across the plane. Figure 2 is a 2-manifold with boundary: a
genus-1 surface $Y$ with one boundary component.
\end{problem}

\begin{parts}
  \begin{part}{a}
    Exhibit a subset of $X$ that is homeomorphic to $Y$.
  \end{part}

  \begin{part}{b}
    Exhibit a subset of $X$ that is diffeomorphic to a genus-4 surface with one boundary component (i.e. a closed genus-4 surface with a disk removed).
  \end{part}
\end{parts}

\begin{problem}{3}
  Algebraic curves.
\end{problem}

\begin{parts}
  \begin{part}{a}
  Let $X \subset \mathbb{C}^2$ be the affine algebraic curve defined by $z^n + w^n = 1$. Show that $X$ is indeed a smooth curve. Let $\phi : X \rightarrow \mathbb{C}$ be the projection to the $z$ axis. Find the branch points and ramification points (aka critical points and critical values) of $ \phi $ and the order of the branch points.
  \end{part}

  \begin{part}{b}
  Let $ Y \subset \mathbb{C}^2 $ be the curve $ w^2 = z^3 - z $. Let $ \psi : Y \rightarrow \mathbb{C} $ be the projection to the $ w $ axis. Find the branch points and ramification points of $ \psi $ and the order of the branch points.
  \end{part}
\end{parts}

\begin{problem}{4}
Show that any non-constant holomorphic map $ f : \mathbb{CP}^1 \rightarrow \mathbb{CP}^1 $ arises as a rational function: a ratio of polynomials $ f(z) = p(z)/q(z) $, in the standard coordinate $ z $ on $ \mathbb{C} \subset \mathbb{CP}^1 $. You might do this by induction on the degree of $ f $, interpreted as the number of zeros counted with multiplicity.
\end{problem}

\begin{solution}
\end{solution}

\begin{problem}{5}
Let $ S \subset \mathbb{C}^2 $ be the zero set of a smooth function $ F : \mathbb{C}^2 \rightarrow \mathbb{C} $ and assume that $ F(z, w) $ is a holomorphic function of both $ z $ and $ w $. Do not assume anything else about $ F $; in particular, we do not assume that $ S $ is smooth. Show that $ S $ has no isolated points. That is, every point of $ S $ is a limit point.
\end{problem}

\begin{solution}
\end{solution}

\end{document}
