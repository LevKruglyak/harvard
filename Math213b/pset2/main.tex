\documentclass{lkx_pset}

\title{Math 213b Problem Set 2}
\author{Lev Kruglyak}
\due{February 7, 2024}

\usepackage{graphicx}

\newcommand{\D}{\mathbb{D}}
\newcommand{\x}{\mathbf{x}}
\usepackage{pgfplots}
\pgfplotsset{
	compat=1.12,
}


\collaborator{AJ LaMotta}
\collaborator{Jarell Cheong}

\begin{document}
\maketitle

\begin{problem}{1}
One of the Fischer sporadic simplegroups, $\mathrm{Fi}_{22}$, has a permutation representation as a transitive subgroup of $S_d$ for $d=3510$. As such it is generated by two elements $a$ and $b$ of orders $2$ and $13$ respectively, while the product $c=(ab)^{-1}$ has order $11$. The actual cycle structures of these elements are $2^{1408}$ (i.e. 1408 2-cycles), $13^{270}$ and $11^{319}$ respectively. Let $X \to \CP^1$ be the corresponding proper ramified holomorphic map, with three points of ramification in $\CP^1$. Calculate the genus of $X$.
\end{problem}

\begin{solution}
	Recall the Riemann-Hurwitz formula, i.e. given connected nonempty Riemann surfaces $X$ and $Y$ with genera $g_X$ and $g_Y$ respectively and a degree $d$ map $f: X \to Y$, we have the relation
	\[
		2g_X - 2 = d(2g_Y - 2) + R_f,\quad\textrm{where}\quad R_f = \sum_{x\in X} (k_x - 1)
	\]
	is the total ramification of $f$.

	Now let $f : X \to \CP^1$ be the given map. We know that this map has three critical values $y_1, y_2, y_3$. Recall that, by construction, the monodromy action is a map
	\[
		m : \pi_1(\CP^1\setminus \{y_1, y_2, y_3\}) \to \mathrm{Fi}_{22} \subset S_d
	\]
	Then given any $y\in \{y_1, y_2, y_3\}$, the number of preimages of $y$ is the number of cycles of $m([y])$ and the order of each preimage is the length of the cycle. This means that above $y_1$ there are $1408$ critical points, each of order $2$. Similarly, above $y_2$ there are $270$ critical points of order $13$, and above $y_3$ there are $319$ critical points of order $11$. Since $f$ is unramified away from $y_1, y_2, y_3$, we have total ramification
	\[
		R_f = 1408\cdot (2-1) + 270\cdot (13-1)+319\cdot (11-1) = 3190.\]
	Applying the Riemann-Hurwitz formula, we get \[2g_X - 2 = d(2g_{\CP^1}-2)+R_f\quad \implies g_X = 410.\]
\end{solution}

\begin{problem}{2}
Let $D$ be the standard open unit disk in $\C$ and let $f : D \to D$ be the map $z \mapsto z^k$. Let $\xi : D \to \C\cup \{\infty\}$ be a meromorphic function which is holomorphic except perhaps at $z=0$ where it may have a pole. Define a function
\[
	s_1 : D\setminus \{0\} \to \C\quad\textrm{by}\quad s_1(w) = \sum_{f(z)=w} \xi(z).
\]
Define functions $s_2, \ldots, s_k$ similarly using the elementary symmetric functions in place of the sum. By estimating the growth rate, or otherwise, show that each $s_i$ defines a meromorphic function on the disk.
\end{problem}

\begin{solution}
	For full generality, let's suppose $\Sigma : \C^k \to \C$ is some symmetric polynomial, i.e. it is invariant under the action of $S_k$ on $\C^k$ by permutation of coordinates. Now we want to show that the function
	\[
		s_\Sigma : D\setminus \{0\} \to 0\quad\textrm{given by}\quad s_\Sigma(w) = \Sigma(z_1,\ldots, z_k), \quad z_i\in f^{-1}(w).
	\]
	This expression is only defined since $\Sigma$ is symmetric -- the fiber $f^{-1}(w)$ does not come with any canonical ordering. We would like to show that this function is meromorphic. Recall that the restriction of $f$ to $D\setminus \{0\}$ is a holomorphic covering map, i.e. every point $p\in D\setminus \{0\}$ has an open neighborhood $\mathcal{U}$ such that
	\[
		f^{-1}(\mathcal{U}) = \bigsqcup_{j}\, \mathcal{U}_j, \quad \textrm{and}\quad \restr{f}{\mathcal{U}_j} : \mathcal{U}_j \to \mathcal{U}\quad\textrm{is a biholomorphism.}
	\]
	In particular, if we (arbitrarily) pick some branch, say $j=1$, it follows that $\mathcal{U}_j = \zeta^j_k\cdot \mathcal{U}_1$ for $j>1$ where $\zeta_k$ is a primitive $k$-th root of unity. If we let $f^{-1}$ be the inverse of the biholomorphism $\restr{f}{\mathcal{U}_1}$. This allows us to construct a map
	\[
		g : \mathcal{U} \to \mathcal{U}_1\times\cdots\times\mathcal{U}_k\quad\textrm{which sends}\quad z \mapsto (\zeta_k^0\cdot f^{-1}(z), \zeta_k^1\cdot f^{-1}(z), \ldots, \zeta_k^{k-1}\cdot f^{-1}(z)).
	\]
	In particular, this map is holomorphic in each component since it sends $z$ to some permuted tuple of its preimage under $f$. Let's now compose with $\xi$ in each component to get a function $\xi\circ g : \mathcal{U}\to (\C\cup\{\infty\})^k$ which is holomorphic in each component. Finally, composing with $\Sigma$ gives us a holomorphic map $\Sigma \circ\xi\circ g : \mathcal{U} \to \C$, since each symmetric function only consists of addition and multiplication.
	Furthermore, the fact that $\Sigma$ is symmetric removes our arbitrary branch choice, so it's clear that $\Sigma \circ \xi \circ g = \restr{s_\Sigma}{\mathcal{U}}$. Thus, $s_\Sigma$ is holomorphic on $D\setminus \{0\}$.

	It remains to extend $s_\Sigma$ to be meromorphic at $z=0$. Since $\xi$ is meromorphic at $z=0$, there is some nonnegative integer $n$ with $\varphi(z) = f(z)^n\xi(z)=z^{nk}\xi(z)$ a holomorphic function on $D$ with $\varphi(0)=0$. Now assuming that $\Sigma$ is homogeneous as a polynomial, near $0$ we have
	\[
		w^{\deg(\Sigma)\cdot n} s_\Sigma(w) = \Sigma\circ \varphi\circ g(w).
	\]
	Here the right hand side is holomorphic, so it is bounded around $z=0$. By the Riemann removable singularity theorem, this means that $w^{\deg(\Sigma)\cdot n}s_\Sigma(w)$ can be extended holomorphically on the whole disk. This in turn shows that $s_\Sigma(w)$ is meromorphic. Note that this result holds when $\Sigma$ is non-homogeneous, since a sum of meromophic functions is meromorphic.
\end{solution}

\begin{problem}{3}
As a continuation of the previous problem, let now $f : X \to Y$ be a non-constant, proper holomorphic map between connected Riemann surfaces. Let the degree of $f$ be $d$. Let $\xi : X \to \C \cup \{\infty\}$ be any meromorphic function. Show that $\xi$ satisfies a relation
\[
	\xi^d + a_1\xi^{d-1} + a_2\xi^{d-2}+\cdots+a_d = 0
\]
where each $a_i$ is a meromorphic function on $X$ that is pulled back along $f$ from the target space $Y$.
\end{problem}

\begin{solution}
	Let $S$ be the set of critical points of $f$. Then away from any critical values $v\in f(S)$, we have for every $p\not\in f(S)$ an open neighborhood $p\in \mathcal{U} \subset Y$ such that
	\[
		f^{-1}(\mathcal{U})=\bigsqcup_j\, \mathcal{U}_j, \quad \textrm{and}\quad \restr{f}{\mathcal{U}_j} : \mathcal{U}_j \to \mathcal{U}\quad\textrm{is a biholomorphism.}
	\]
	Let's set $f_j^{-1} = \restr{f}{\mathcal{U}_j}$ to be the branches of $f$. Then we can consider the function
	\[
		F(T) = \prod_{0\leq j< d} (T-\xi\circ f_j^{-1}(w)) = \sum_{0\leq j \leq d} (-1)^{j}\Sigma_j\circ \zeta(f_0^{-1}(T), \ldots, f_d^{-1}(T)) T^{d-j}
	\]
	where $\Sigma_j$ is the elementary symmetric function of degree $j$. Let's call the coefficient of $T^j$ in this expansion $a_j$; these are meromorphic functions on $Y$. Note that nowhere in the definition of these coordinates did the ordering of the branches come up, since the symmetric functions are invariant under any permutations.
	Similarly, our initial choice of point $p\in Y$ and neighborhood $\mathcal{U}$ makes no difference in the final expression for $a_j$, which is defined purely in the global function $f$. Thus, $a_j$ are meromorphic functions on $Y\setminus f(S)$, and clearly $F(\zeta)=0$. At a critical value $y\in f(S)$, a similar argument to the previous problem shows that $y^n a_j(y)$ is holomorphic for sufficiently large $n$.
\end{solution}

\begin{problem}{4}
Suppose that, in the context of the previous problem, the function $\xi$ takes $d$ distinct finite values at the points $f^{-1}(y)$ for at least one point $y\in Y$. Show that the polynomial $T^d + a_1 T^{d-1}+\cdots +a_d\in \mathcal{M}_Y[T]$ is irreducible, where $\mathcal{M}_Y$ is the field of meromorphic function on $Y$.
\end{problem}

\begin{solution}
	Let's assume $y\in Y$ is one point as in the problem statement, i.e. the values $\xi(x_1),\ldots, \xi(x_d)$ are finite and distinct. In particular, $\xi$ is holomorphic and non-critical at these values.
	Now, pulling back by $f$ gives a map $\mathcal{M}_f : \mathcal{M}_Y \to \mathcal{M}_X$, or equivalently a field extension $\mathcal{M}_X/\mathcal{M}_Y$. By the previous problem, we know that $\xi$ satisfies some relation
	\[
		\xi^d + a_1\xi^{d-1}+\cdots+a_d=\xi^d + \mathcal{M}_f(Q_1)\cdot  \xi^{d-1} + \mathcal{M}_f(Q_2)\cdot \xi^{d-2}+ \cdots + \mathcal{M}_f(Q_d) = 0.
	\]
	(In particular recall that the coefficients of this polynomial were pullbacks by $f$ of maps in $\mathcal{M}_Y$. This means that $\xi$ is algebraic over $\mathcal{M}_Y$. Let $P(T)\in \mathcal{M}_Y[T]$ now be the minimal polynomial of $\xi$, say of degree $n \leq d$ with coefficients $\mathcal{M}_f(P_j)\in \mathcal{M}_Y[T]$. If all these coefficients are holomorphic at $y$, then $P(T)$ has $d$ distinct roots, namely $\xi(x_i)$. This implies that $d=n$ and hence irreducibility of $T^d + a_1T^{d-1} + \cdots + a_d$. If any of the coefficients are not holomorphic at $y$, we can perturb $y$ by some arbitrarily small distance, keeping the values at preimages distinct but avoiding singularities. (Poles of meromorphic functions are isolated) In any case, irreducibility follows since the minimal polynomial is irreducible.

	\begin{part}{(Optional)}
		Under these circumstances, show that $\mathcal{M}_X$ is a field extension of $\mathcal{M}_Y$ of degree $d$. Use the theorem of the primitive element.
	\end{part}
\end{solution}

\begin{problem}{5}
Complex differential forms.
\end{problem}

\begin{parts}
	\begin{part}{a}
		On $\C = \R^2$ with coordinates $z=x+iy$, calculate the $(1,0)$ and $(0,1)$ parts of the $1$-form $y\,dx + x\,dy$, expressing the answer in terms of $dz$ or $d\overline{z}$ respectively. Is the $(1,0)$ part a holomorphic $1$-form?
	\end{part}

	Recall that we have the relations:
	\[
		dx = \frac{dz+d\overline{z}}{2},\quad dy = \frac{dz - d\overline{z}}{2i}.
	\]
	Expanding the given $1$-form using the relations, we get
	\[
		y\,dx + x\,dy = y\cdot\frac{dz+d\overline{z}}{2} + x\cdot\frac{dz-d\overline{z}}{2i}
		=y\cdot \frac{dz+d\overline{z}}{2} - ix\cdot\frac{dz-d\overline{z}}{2} = -\frac{iz}{2}\,dz +\frac{i\overline{z}}{2}\,d\overline{z}
	\]
	Thus, the $(1,0)$ part is a holomorphic $1$-form $-iz/2\,dz$, and the $(0,1)$ part is $-iz/2\,dz$.

	\begin{part}{b}
		Let $r$ be the radial distance function on $\C$. Let $\theta$ be the ``argument'' function (in as much it is defined). Calculate $\overline{\partial} r^2, \partial\overline{\partial} r^2, \overline{\partial}\log(r^2)$, $\partial\overline{\partial} \log(r^2)$ and $\overline{\partial}\theta$. Express the answers first in terms of $dz$ and/or $d\overline{z}$, then in terms of $dx$ and $dy$.
	\end{part}

	Using the fact that $r^2 = |z|=z\overline{z}$ and standard properties of differential operators.
	\begin{center}
		\renewcommand*{\arraystretch}{1.2}
		\begin{tabular}{|c|c|c|}
			\hline
			\textrm{form}                            & $dz,d\overline{z}$              & $dx,dy$                               \\
			\hline
			$\overline{\partial} r^2$                & $z\,dz$                         & $(x+iy)\,dx + (y-ix)\,dy$             \\
			$\partial\overline{\partial} r^2$        & $dz\wedge d\overline{z}$        & $-2i\,dx\wedge dy$                    \\
			$\overline{\partial} \log(r^2)$          & $d\overline{z}/z$               & $((x+iy)\,dx + (y-ix)\,dy)/(x^2+y^2)$ \\
			$\partial \overline{\partial} \log(r^2)$ & $0$                             & $0$                                   \\
			$\overline{\partial} \theta$             & $-d\overline{z}/2i\overline{z}$ & $((-y+ix)\,dx+(x+iy)\,dy)/2(x^2+y^2)$ \\
			\hline
		\end{tabular}
	\end{center}
	For the last computation we use the identity $2i\theta = \log(z/\overline{z})$ for a suitable branch of $\log$ as the definition of $\theta$.

\end{parts}
\end{document}
