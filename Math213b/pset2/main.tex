\documentclass{lkx_pset}

\title{Math 213b Problem Set 2}
\author{Lev Kruglyak}
\due{February 7, 2024}

\usepackage{graphicx}

\newcommand{\D}{\mathbb{D}}
\newcommand{\x}{\mathbf{x}}
\usepackage{pgfplots}
\pgfplotsset{
	compat=1.12,
}

\newcommand{\dd}{\partial}
\newcommand{\ddc}{\overline{\partial}}

\usepackage{bbm}
\newcommand{\bbDelta}{\bm{\Delta}}


\collaborator{AJ LaMotta}
\collaborator{Jarell Cheong}

\begin{document}
\maketitle

\begin{problem}{1}
  One of the Fischer sporadic simplegroups, $\mathrm{Fi}_{22}$, has a permutation representation as a transitive subgroup of $S_d$ for $d=3510$. As such it is generated by two elements $a$ and $b$ of orders $2$ and $13$ respectively, while the product $c=(ab)^{-1}$ has order $11$. The actual cycle structures of these elements are $2^{1408}$ (i.e. 1408 2-cycles), $13^{270}$ and $11^{319}$ respectively. Let $X \to \CP^1$ be the corresponding proper ramified holomorphic map, with three points of ramification in $\CP^1$. Calculate the genus of $X$.
\end{problem}

\begin{solution}
  Recall the Riemann-Hurwitz formula, i.e. given connected nonempty Riemann surfaces $X$ and $Y$ with genera $g_X$ and $g_Y$ respectively and a degree $d$ map $f: X \to Y$, we have the relation
  \[
    2g_X - 2 = d(2g_Y - 2) + R_f,\quad\textrm{where}\quad R_f = \sum_{x\in X} (k_x - 1)
  \]
  is the total ramification of $f$. 

  Now let $f : X \to \CP^1$ be the given map. We know that this map has three critical values $y_1, y_2, y_3$, and we can assume without loss of generality that $y_3=\infty$. Recall that, by construction, the monodromy action is a map 
  \[
    m : \pi_1(\CP^1\setminus \{y_1, y_2, y_3\}) \to \mathrm{Fi}_{22} \subset S_d
  \]
  Since we normalized $y_3=\infty$, this first group is just $\pi_1(\C\setminus \{y_1, y_2\})\cong F_2$, i.e. the free group on two letters $a$ and $b$.

  \todo{finish this problem}
\end{solution}

\begin{problem}{2}
  Let $D$ be the standard open unit disk in $\C$ and let $f : D \to D$ be the map $z \mapsto z^k$. Let $\xi : D \to \C\cup \{\infty\}$ be a meromorphic function which is holomorphic except perhaps at $z=0$ where it may have a pole. Define a function
  \[
    s_1 : D\setminus \{0\} \to \C\quad\textrm{by}\quad s_1(w) = \sum_{f(z)=w} \xi(z).
  \]
  Define functions $s_2, \ldots, s_k$ similarly using the elementary symmetric functions in place of the sum. By estimating the growth rate, or otherwise, show that each $s_i$ defines a meromorphic function on the disk.
\end{problem}

\begin{solution}
  For full generality, let's suppose $\Sigma : \C^k \to \C$ is some symmetric polynomial, i.e. it is invariant under the action of $S_k$ on $\C^k$ by permutation of coordinates. Now we want to show that the function
  \[
    s_\Sigma : D\setminus \{0\} \to 0\quad\textrm{given by}\quad s_\Sigma(w) = \Sigma(z_1,\ldots, z_k), \quad z_i\in f^{-1}(w).
  \]
  This expression is only defined since $\Sigma$ is symmetric -- the fiber $f^{-1}(w)$ does not come with any canonical ordering. We would like to show that this function is meromorphic. Recall that the restriction of $f$ to $D\setminus \{0\}$ is a holomorphic covering map, i.e. every point $p\in D\setminus \{0\}$ has an open neighborhood $\mathcal{U}$ such that
  \[
    f^{-1}(\mathcal{U}) = \bigsqcup_{k}\, \mathcal{U}_k, \quad \textrm{and}\quad \restr{f}{\mathcal{U}_k} : \mathcal{U}_K \to \mathcal{U}\quad\textrm{is a biholomorphism.}
  \]

\end{solution}

\begin{problem}{3}
  As a continuation of the previous problem, let now $f : X \to Y$ be a non-constant, proper holomorphic map between connected Riemann surfaces. Let the degree of $f$ be $d$. Let $\xi : X \to \C \cup \{\infty\}$ be any meromorphic function. Show that $\xi$ satisfies a relation
  \[
    \xi^d + a_1\xi^{d-1} + a_2\xi^{d-2}+\cdots+a_d = 0
  \]
  where each $a_i$ is a meromorphic function on $X$ that is pulled back along $f$ from the target space $Y$.
\end{problem}

\begin{problem}{4}
  Suppose that, in the context of the previous problem, the function $\xi$ takes $d$ distinct finite values at the points $f^{-1}(y)$ for at least one point $y\in Y$. Show that the polynomial $T^d + a_1 T^{d-1}+\cdots +a_d\in \mathcal{M}_Y[T]$ is irreducible, where $\mathcal{M}_Y$ is the field of meromorphic function on $Y$.
\end{problem}

\begin{solution}
  \begin{part}{(Optional)}
    Under these circumstances, show that $\mathcal{M}_X$ is a field extension of $\mathcal{M}_Y$ of degree $d$. Use the theorem of the primitive element.
  \end{part}
\end{solution}

\begin{problem}{5}
  Complex differential forms.
\end{problem}

\begin{parts}
  \begin{part}{a}
    On $\C = \R^2$ with coordinates $z=x+iy$, calculate the $(1,0)$ and $(0,1)$ parts of the $1$-form $y\,dx + x\,dy$, expressing the answer in terms of $dz$ or $d\overline{z}$ respectively. Is the $(1,0)$ part a holomorphic $1$-form?
  \end{part}

  Recall that we have the relations:
  \[
    dx = \frac{1}{2}(dz+d\overline{z}),\quad dy = \frac{1}{2}(dz - d\overline{z}).
  \]
  Expanding the given $1$-form using the relations, we get
  \[
    y\,dx + x\,dy = \frac{y}{2}(dz+d\overline{z}) + \frac{x}{2}(dz - d\overline{z}) = \frac{y+x}{2}\,dz + \frac{y-x}{2}\,d\overline{z}.
  \]
  Thus, the $(1,0)$ part is a holomorphic $1$-form $(y+x)/2\,dz$, and the $(0,1)$ part is $(y-x)/2\,dz$.

  \begin{part}{b}
    Let $r$ be the radial distance function on $\C$. Let $\theta$ be the ``argument'' function (in as much it is defined). Calculate $\overline{\partial} r^2, \partial\overline{\partial} r^2, \overline{\partial}\log(r^2)$, $\partial\overline{\partial} \log(r^2)$ and $\overline{\partial}\theta$. Express the answers first in terms of $dz$ and/or $d\overline{z}$, then in terms of $dx$ and $dy$.
  \end{part}

  \begin{center}
    \renewcommand*{\arraystretch}{1.2}
    \begin{tabular}{|c|c|c|} 
     \hline
     \textrm{form} & $dz/d\overline{z}$ & $dx/dy$ \\ 
     \hline
     $\overline{\partial} r^2$ & cell5 & cell6 \\ 
     $\partial\overline{\partial} r^2$ & cell8 & cell9 \\ 
     $\overline{\partial} \log(r^2)$ & cell8 & cell9 \\ 
     $\partial \overline{\partial} \log(r^2)$ & cell8 & cell9 \\ 
     $\overline{\partial} \theta$ & cell8 & cell9 \\ 
     \hline
    \end{tabular}
  \end{center}

  \todo{finish this}
\end{parts}
\end{document}
