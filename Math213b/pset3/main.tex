\documentclass{lkx_pset}

\title{Math 213b Problem Set 2}
\author{Lev Kruglyak}
\due{February 14, 2024}

\usepackage{graphicx}

\newcommand{\D}{\mathbb{D}}
\newcommand{\x}{\mathbf{x}}
\usepackage{pgfplots}
\pgfplotsset{
	compat=1.12,
}


\collaborator{AJ LaMotta}
% \collaborator{Jarell Cheong}

\begin{document}
\maketitle

\begin{problem}{1}
Let $G$ be the finite group of automorphisms of $\CP^{1}$ generated by
$z\mapsto -z$ and $z\mapsto 1/z$. Identify the quotient $\CP^{1}/G$ in
with $\CP^{1}$ in such a way that the quotient map $f: \CP^{1}\to
	\CP^{1}/G \cong \CP^{1}$ is holomorphic, and find the critical points
and critical values of $f$.
\end{problem}

\begin{problem}{2}
In class, we will prove the
following result: if $X$ is a compact connected Riemann surface of
genus $g$ and $p\in X$ is any point, then there exists a non-constant meromorphic
function $f$ on $X$ that has a pole at $p$ of order at most $g+1$ and
no other poles.
\end{problem}

\begin{parts}
	\begin{part}
		Use this to show the following: Given
		distinct points $p_{1},\dots, p_{m}$ in $X$ and any values
		$w_{1},\dots w_{m}$ in $\C$, there exists a meromorphic function $g$
		on $X$ with $f(p_{i})=w_{i}$ for all $i$.
	\end{part}
\end{parts}

\begin{problem}{3}
Use the previous problem and a result from Problem Set 2 to show that
the field of meromorphic functions $\mathcal{M}_{X}$ on a connected
Riemann surface $X$ can be described as a an algebraic extension of
the field of rational functions $\C(z)$ of degree  $d$,
whenever there is a map function $f:X\to \CP^{1}$ of degree $d$.
More generally, $\mathcal{M}_{X}$ is an algebraic extension of
$\mathcal{M}_{Y}$ whenever there is a map $X\to Y$ of degree $d$.
\end{problem}

\begin{problem}{4}
Let $S\subset \R^{3}$ be a smooth surface in Euclidean 3-space.
Let $\R^{2}$ have standard Euclidean coordinates $(s,t)$, let
$\Omega\subset \R^{2}$ be open, and let $F:\Omega\to \R^{3}$ be a
smooth injective map parametrizing a little patch $U \subset S$. This
parametrization is called \emph{isothermal} if, at each point $z =(s,t) \in \R^{2}$,
the partial derivatives $F_{s}$ and $F_{t}$ are non-zero orthogonal
vectors of equal length in $\R^{3}$. The coordinates of the inverse map $U\to
	\Omega\subset \R^{2}$ are then referred to as \emph{isothermal coordinates} on the patch $U$ of the surface.
\end{problem}

\begin{parts}
	\begin{part}{}
		Assuming the non-trivial
		result that isothermal coordinates on a surface $S$ exist in the
		neighborhood of any point of $S$, explain, briefly, how we
		can use isothermal coordinates to give $S$ the structure of a Riemann
		surface, as long as it is oriented.
	\end{part}

	\begin{part}{}
		Let $S$ be the surface of revolution obtained by rotation the curve
		$y=e^{-x}$ about the $x$ axis. Obtain isothermal
		coordinates $(s,t)$ on $S$, where $s$ is periodic of period $2\pi$.
		Leave your answer as an indefinite integral if you wish, or obtain a
		closed form. Beware that if you use Mathematica, as I did, to evaluate
		the integral then you will run into Mathematica's weakness for
		yielding answers that don't really make sense for real variables.
	\end{part}
\end{parts}

\begin{problem}{3}
Let $C\subset \C^{2}$ be the Riemann surface defined by the
equation $w^{2}=z(z+4)(z-2)$ (a non-compact Riemann surface).
Sketch this locus in the real plane (i.e. for $(z,w) \in \R^{2}$).
Show that the curve $C$ carries a meromorphic function with
\begin{itemize}
	\item a simple zero at $(z,w)=(-1,3)$,
	\item a simple pole at
	      $(z,w)=(-1,-3)$,
	\item one other simple zero somewhere,
	\item no other zeros or poles.
\end{itemize}
Find the coordinates of the other zero.

% \emph{Hint.} Thinking about the tangent line to the curve at
% $(-1,3)$ will help.
\end{problem}

\end{document}
