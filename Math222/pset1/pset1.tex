\documentclass{lkx_pset}

\title{Math 222 Problem Set 1}
\author{Lev Kruglyak}
\due{February 5, 2025}

\renewcommand{\O}{{\operatorname{O}}}
\renewcommand{\GL}{\operatorname{GL}}
\providecommand{\Det}{\operatorname{Det}}
\providecommand{\T}{\mathbb{T}}
\providecommand{\HH}{\mathbb{H}}
\providecommand{\gfr}{\mathfrak{g}}
\providecommand{\pfr}{\mathfrak{p}}
\providecommand{\gl}{\mathfrak{gl}}
\providecommand{\hfr}{\mathfrak{h}}
\providecommand{\U}{\operatorname{U}}
\providecommand{\SO}{\operatorname{SO}}
\providecommand{\SU}{\operatorname{SU}}
\providecommand{\su}{\mathfrak{su}}
\providecommand{\so}{\mathfrak{so}}
\renewcommand{\sl}{\mathfrak{sl}}
\providecommand{\Spin}{\operatorname{Spin}}
\providecommand{\Sp}{\operatorname{Sp}}
\providecommand{\scB}{\mathscr{B}}
\providecommand{\Tr}{\operatorname{Tr}}
\providecommand{\Sym}{\operatorname{Sym}}
\providecommand{\op}[1]{\operatorname{#1}}

\providecommand{\Hol}{\operatorname{Hol}}
\providecommand{\Ad}{\operatorname{Ad}}
\providecommand{\ad}{\operatorname{ad}}
\providecommand{\MC}{{\operatorname{MC}}}
\renewcommand{\T}{\operatorname{T}\!}
     \usepackage[mathscr]{euscript}

\providecommand{\sgn}{{\operatorname{sgn}}}



\ExplSyntaxOn
\NewDocumentCommand{\lkxto}{ O{} }{%
	\mathrel{\;\xrightarrow{\hphantom{xx}#1\hphantom{xx}}\;}
}
\NewDocumentCommand{\lkxmapsto}{ O{} }{%
	\mathrel{\;\xmapsto{\hphantom{xx}#1\hphantom{xx}}\;}
}
\NewDocumentCommand{\lkxisom}{ O{} }{%
	\mathrel{\;\xrightarrow{\hphantom{xx}#1\sim\hphantom{xx}}\;}
}
\NewDocumentCommand{\lkxsurj}{ O{} }{%
	\mathrel{\;\xtwoheadrightarrow{\hphantom{xx}#1\hphantom{xx}}\;}
}

\NewDocumentCommand{\lkxfunc}{ O{->} m m m g g }{
	\begin{array}{rcl}
		\IfNoValueTF {#5} {
			\tl_if_blank:nTF {#2}{}{#2 :}
			#3
			\str_case:nn {#1}
			{
				{->} {\lkxto}
					{~>} {\lkxisom}
					{->>} {\lkxsurj}
			}
			#4
		} {
			\tl_if_blank:nTF {#2}{}{#2 \;:\;}
			#3
		   &
			\str_case:nn {#1}
			{
				{->} {\lkxto}
					{~>} {\lkxisom}
					{->>} {\lkxsurj}
			}
		   & #4
		\\
		#5 & \lkxmapsto & #6
		}
	\end{array}
}

\ExplSyntaxOff

%
% \collaborator{AJ LaMotta}
% \collaborator{Leonardo Kaplan}
% \collaborator{Ignasi Vicente}

\begin{document}
\maketitle

\begin{problem}{2}
  Let $G$ have the structures of a group and a smooth manifold, and suppose that multiplication $G\times G\to G$ is smooth. Prove that inversion $G\to G$ is smooth map.
\end{problem}

\begin{solution}
  Let $m : G\times G \to G$ be the multiplication map. For any fixed element $g\in G$, the left multiplication map $L_g : G\to G$ is a diffeomorphism. Consider the map 
  \[
    \lkxfunc{f}{G\times G}{G\times G}{(g,h)}{(g,gh).}
  \]
  At the identity $(e,e)$, the differential of this map takes the form
  \[
    \lkxfunc{df_{(e,e)}}{\T_e G\oplus \T_e G}{\T_e G\oplus \T_{e} G}{\xi\oplus \eta}{\xi\oplus (\xi+\eta).}
  \]
  By the inverse function theorem, this means that $f$ is a diffeomorphism in a neighborhood of $(e,e)$. By composing with the left multiplication diffeomorphisms, we can see that $f$ is a diffeomorphism over the whole manifold. Since the inversion map $i : G\to G$ can be written as $i(g) = \pi_2 \circ f^{-1}(g,e)$, a composition of smooth maps, the inversion map must be a smooth map as well.
\end{solution}

\begin{problem}{4}
  Classify $1$-dimensional Lie groups $G$ which have two components and whose identity component is diffeomorphic to the circle.
\end{problem}

\begin{solution}
  Pick any nonzero vector $\xi\in \mathfrak{g}$ and consider the exponential map
  \[
    \lkxfunc{\exp}{\R}{G}{t}{\exp(t\xi).}
  \]
  This is a homomorphism of Lie groups. Since the identity component is diffeomorphic to $S^1$ and the exponential map is a homomorphism of Lie groups, it follows that there is a non-empty lattice $\ker(\exp) = \Lambda \subset \R$ so that we have an isomorphism $\R/\Lambda \to G$ of Lie groups. This let's us construct isomorphism with the circle group $\mu \to G_e$ so $G_e$ is not only diffeomorphic to a circle but has the group structure of the circle group. 

  Next, note that $G/G_e$ is a group of order two and so must be isomorphic to $\mu_2$. From a group theoretic perspective, the only group structures on $G$ are then $G\times \mu_2$ and $G\rtimes \mu_2$. These correspond to the Lie groups $\mu_2\times \mu$ and $\O_2$ respectively.
\end{solution}

\begin{problem}{5}
\end{problem}
\begin{parts}
  \begin{part}{(a)}
    Identify the group of rotations in $\R^3$ with the matrix group $\SO_3$.
  \end{part}

  Any rotation in $\R^3$ must preserve distances, preserve orientation, and fixes a line. Let $G\subset \GL_3(\R)$ be the group of matrices representing rotations. The requirement that rotations preserve distances restricts us to $\O_3\subset \GL_3(\R)$ and the requirement that rotations preserve orientation restricts us to $\SO_3\subset \O_3$. Therefore $G\subset \SO_3$ so it suffices to show that every special orthogonal matrix corresponds to a rotation. 

  Let $A\in \SO_3$ be a special orthogonal matrix. The characteristic polynomial of $A$ is a real polynomial is a degree $3$ real polynomial and so it must have a real root $\lambda$. We can thus write $A$ as
  \[
    A = \begin{pmatrix} 
      \lambda & 0\\
      0 & M
      \end{pmatrix}
  \]
  for some $2\times 2$ matrix $M$ in a basis $\{e_1,e_2,e_3\}$ It follows that $(\lambda)$ and $M$ must be orthogonal matrices, which in particular implies that $\lambda=\pm 1$. Therefore, $A$ represents rotation about the $e_1$ axis.

  \begin{part}{(b)}
    Embed $\O_2$ as a Lie subgroup of $\SO_3$.
  \end{part}

  For any basis, consider the inclusion map
  \[
    \lkxfunc{f}{\O_2}{\SO_3}{M}{\begin{pmatrix}\det(M) & 0 \\ 0 & M\end{pmatrix}.}
  \]
  Since the determinant of the matrix $f(M)$ is $\det(M)^2$, it follows that $f(M)$ is a special orthogonal matrix. This is injective and a Lie group homomorphism because 
  \[
    \begin{pmatrix}\det(M_1) & 0 \\ 0 & M_1\end{pmatrix}
    \begin{pmatrix}\det(M_2) & 0 \\ 0 & M_2\end{pmatrix}
    =
    \begin{pmatrix}\det(M_1M_2) & 0 \\ 0 & M_1M_2\end{pmatrix}.
  \]
\end{parts}

\end{document}
