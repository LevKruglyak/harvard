\documentclass{lkx_pset}

\title{Math 222 Problem Set 3}
\author{Lev Kruglyak}
\due{February 19, 2025}

\renewcommand{\O}{{\operatorname{O}}}
\renewcommand{\GL}{\operatorname{GL}}
\providecommand{\Det}{\operatorname{Det}}
\providecommand{\T}{\mathbb{T}}
\providecommand{\HH}{\mathbb{H}}
\providecommand{\gfr}{\mathfrak{g}}
\providecommand{\pfr}{\mathfrak{p}}
\providecommand{\gl}{\mathfrak{gl}}
\providecommand{\hfr}{\mathfrak{h}}
\providecommand{\U}{\operatorname{U}}
\providecommand{\SO}{\operatorname{SO}}
\providecommand{\SU}{\operatorname{SU}}
\providecommand{\su}{\mathfrak{su}}
\providecommand{\so}{\mathfrak{so}}
\renewcommand{\sl}{\mathfrak{sl}}
\providecommand{\Spin}{\operatorname{Spin}}
\providecommand{\Sp}{\operatorname{Sp}}
\providecommand{\scB}{\mathscr{B}}
\providecommand{\Tr}{\operatorname{Tr}}
\providecommand{\Sym}{\operatorname{Sym}}
\providecommand{\op}[1]{\operatorname{#1}}

\providecommand{\Hol}{\operatorname{Hol}}
\providecommand{\Ad}{\operatorname{Ad}}
\providecommand{\ad}{\operatorname{ad}}
\providecommand{\MC}{{\operatorname{MC}}}
\renewcommand{\T}{\operatorname{T}\!}
     \usepackage[mathscr]{euscript}




\ExplSyntaxOn
\NewDocumentCommand{\lkxto}{ O{} }{%
	\mathrel{\;\xrightarrow{\hphantom{xx}#1\hphantom{xx}}\;}
}
\NewDocumentCommand{\lkxmapsto}{ O{} }{%
	\mathrel{\;\xmapsto{\hphantom{xx}#1\hphantom{xx}}\;}
}
\NewDocumentCommand{\lkxisom}{ O{} }{%
	\mathrel{\;\xrightarrow{\hphantom{xx}#1\sim\hphantom{xx}}\;}
}
\NewDocumentCommand{\lkxsurj}{ O{} }{%
	\mathrel{\;\xtwoheadrightarrow{\hphantom{xx}#1\hphantom{xx}}\;}
}

\NewDocumentCommand{\lkxfunc}{ O{->} m m m g g }{
	\begin{array}{rcl}
		\IfNoValueTF {#5} {
			\tl_if_blank:nTF {#2}{}{#2 :}
			#3
			\str_case:nn {#1}
			{
				{->} {\lkxto}
					{~>} {\lkxisom}
					{->>} {\lkxsurj}
			}
			#4
		} {
			\tl_if_blank:nTF {#2}{}{#2 \;:\;}
			#3
		   &
			\str_case:nn {#1}
			{
				{->} {\lkxto}
					{~>} {\lkxisom}
					{->>} {\lkxsurj}
			}
		   & #4
		\\
		#5 & \lkxmapsto & #6
		}
	\end{array}
}

\ExplSyntaxOff


\collaborator{AJ LaMotta}

\begin{document}
\maketitle

\begin{problem}{2}
  Let $G$ be a connected Lie group and suppose $\pi : \widetilde{G} \to G$ is a covering map. Fix $\widetilde{e}\in \pi^{-1}(e)$. Construct a Lie group structure on $\widetilde{G}$ such that $\widetilde{e}$ is the identity element and $\pi$ is a homomorphism of Lie groups. Is this Lie group structure unique?
\end{problem}
\begin{solution}
  The Lie group structure is unique of $\widetilde{G}$ is connected. Otherwise, you could have a situation like in the case of the coverings $\U_1\times S_3 \to \U_1$ and $\U_1\times \Z/6 \to \U_1$. Topologically, these covering maps are identical, but of course the total spaces do not have the same group structure. 

  Let's now suppose that $\widetilde{G}$ is connected and $\widetilde{e}\in \pi^{-1}(e)$ is fixed. We can find some diffeomorphism $\pi^{-1}|_{\widetilde{U}} : \widetilde{U} \to U$ where $\widetilde{U}$ is a neighborhood of $\widetilde{e}$. We can define a group structure on $\widetilde{U}$ by setting $g\cdot h = \pi^{-1}(\pi(g)\cdot \pi(h))$. We can cover $\widetilde{G}$ by open sets on which $\pi$ is a diffeomorphism, and can make this a good cover so that the intersections only contain a finite number of points. Then, starting at the original $\widetilde{U}$ we can continue to extend the multiplication in this way, eventually covering the whole connected manifold. This construction necessarily requires connectedness, and a good cover to apply the glueing lemma.
\end{solution}

\begin{problem}{4}
Let $T\subset \U_3$ be the subgroup of diagonal matrices. Identify its normalizer $N(T)\subset \U_3$. Identify the quotient group $N(T)/T$. Points of $U_3/T$ parametrize a certain geometric structure on $\C^3$; what is that geometric structure? Do the same for $\U_3/N(T)$. Generalize to $\U_n$ for all $n\in \Z^{>0}$. Specialize to $\SU_2$, where again $T\subset \SU_2$ is the subgroup of diagonal matrices. Do you recognize the group $N(T)$? What is its identity component?
\end{problem}
\begin{solution}
	Let's begin by determining the normalizer $N(T)\subset \U_3$. Recall that a matrix $u\in \U_3$ is in the normalizer of $T$ if and only if $uTu^{-1} = T$. The only matrices which preserve diagonal matrices are those which permute diagonal entries and those which multiply each basis vector by a phase. It follows that every matrix in the normalizer can be written as $P_\sigma D$ with $P_\sigma$ the permutation matrix for a permutation $\sigma\in S_3$ and $D\in T$ is a diagonal matrix which shifts the phase of each vector. The quotient $N(T)/T$ is then isomorphic to the symmetry group $S_3$. These results generalize for all $\U_n$, we would then have $N(T)/T\cong S_n$.

	Next, let's identify the geometric structure induced on $\C^n$ by points of $\U_n / T$. Let's suppose we have some matrix $u\in \U_n/T$. Let $u_1,\ldots, u_n$ be the column vectors of this matrix, unique up to multiplication by a phase. Each of these defines a unique line and they must be orthogonal by the unitary condition.
	Thus, $U_n/T$ parametrizes the space of ordered sequences of orthogonal complex lines in $\C^n$. By similar logic, $\U_n/N(T)$ parametrizes unordered sequences of orthogonal complex lines in $\C^n$ since we only get a permutation class of column vectors.

	Finally, for $\SU_2$, the diagonal matrices must be be of the form
	\[
		T = \left\{\begin{pmatrix}e^{i\theta} & 0 \\ 0 & e^{-i\theta}\end{pmatrix} : \theta\in [0,2\pi)\right\} .
	\]
	The normalizer is generated by $T$ with a matrix that sends $\theta \mapsto -\theta$. This is the transposition generating $S_2$. It follows that $N(T)$ is isomorphic to $\O_2$. These results fit with the exceptional isomorphism $\SU_2 \cong \Spin_3$ which double covers $\SO_2$, the normalizer $N(T)$ which consists of two circles maps onto the single circle $\SO_2$.
\end{solution}

\begin{problem}{5}
Continuing with the previous problem, prove that every conjugacy class in $\U_3$ has nonempty intersection with $T$. What is that intersection? Are those intersections the orbits of a group action on $T$? What is the analog for $\SU_3$? Can you draw pictures for $\SU_3$? Topologize the space of conjugacy classes. What can you say about this space?
\end{problem}
\begin{solution}
\end{solution}

\begin{problem}{6}
\end{problem}
\begin{parts}
	\begin{part}{(a)}
		Let $G$ be a Lie group and let $\Ad : G \to \Aut(\gg)$ be its adjoint representation. The center $Z(G)\subset G$ is contained in the kernel of $\Ad$. Find a Lie group $G$ for which the kernel of $\Ad$ is strictly larger than $Z(G)$.
	\end{part}

	A simple set of examples occurs for non-abelian discrete Lie groups. If a group $G$ is discrete, then $\gg$ is $0$-dimensional and so $\ker \Ad = G$. Picking any non-abelian discrete group thus has $Z(G)\subset \ker \Ad$ as a proper subgroup.

	\begin{part}{(b)}
		Find an example of the following: Lie groups $G'$, $G$ with $G'$ connected, and a homomorphism $\phi : \gg' \to \gg$ between their Lie algebras such that there does \emph{not} exist a homomorphism $\psi : G' \to G$ with $\dot{\psi} = \phi$.
	\end{part}

	Let $G' = \SO_n$ and $G=\GL_m$. Letting $\phi : \so_n \to \gl_m$ be a spinor representation (arising from a map $\Spin_n \to \GL_m$) which does not arise from a representation of $\SO_n$, it follows by assumption $\phi$ does not come from a map $\psi : \SO_n \to \GL_m$. This requires a little bit of knowledge about the representation theory of $\Spin_n$ and $\SO_n$.
\end{parts}

\end{document}
