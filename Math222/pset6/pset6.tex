\documentclass{lkx_pset}

\title{Math 222 Problem Set 6}
\author{Lev Kruglyak}
\due{March 12, 2025}

\renewcommand{\O}{{\operatorname{O}}}
\renewcommand{\GL}{\operatorname{GL}}
\providecommand{\Det}{\operatorname{Det}}
\providecommand{\T}{\mathbb{T}}
\providecommand{\HH}{\mathbb{H}}
\providecommand{\gfr}{\mathfrak{g}}
\providecommand{\pfr}{\mathfrak{p}}
\providecommand{\gl}{\mathfrak{gl}}
\providecommand{\hfr}{\mathfrak{h}}
\providecommand{\U}{\operatorname{U}}
\providecommand{\SO}{\operatorname{SO}}
\providecommand{\SU}{\operatorname{SU}}
\providecommand{\su}{\mathfrak{su}}
\providecommand{\so}{\mathfrak{so}}
\renewcommand{\sl}{\mathfrak{sl}}
\providecommand{\Spin}{\operatorname{Spin}}
\providecommand{\Sp}{\operatorname{Sp}}
\providecommand{\scB}{\mathscr{B}}
\providecommand{\Tr}{\operatorname{Tr}}
\providecommand{\Sym}{\operatorname{Sym}}
\providecommand{\op}[1]{\operatorname{#1}}

\providecommand{\Hol}{\operatorname{Hol}}
\providecommand{\Ad}{\operatorname{Ad}}
\providecommand{\ad}{\operatorname{ad}}
\providecommand{\MC}{{\operatorname{MC}}}
\renewcommand{\T}{\operatorname{T}\!}
     \usepackage[mathscr]{euscript}

\providecommand{\sgn}{{\operatorname{sgn}}}



\ExplSyntaxOn
\NewDocumentCommand{\lkxto}{ O{} }{%
	\mathrel{\;\xrightarrow{\hphantom{xx}#1\hphantom{xx}}\;}
}
\NewDocumentCommand{\lkxmapsto}{ O{} }{%
	\mathrel{\;\xmapsto{\hphantom{xx}#1\hphantom{xx}}\;}
}
\NewDocumentCommand{\lkxisom}{ O{} }{%
	\mathrel{\;\xrightarrow{\hphantom{xx}#1\sim\hphantom{xx}}\;}
}
\NewDocumentCommand{\lkxsurj}{ O{} }{%
	\mathrel{\;\xtwoheadrightarrow{\hphantom{xx}#1\hphantom{xx}}\;}
}

\NewDocumentCommand{\lkxfunc}{ O{->} m m m g g }{
	\begin{array}{rcl}
		\IfNoValueTF {#5} {
			\tl_if_blank:nTF {#2}{}{#2 :}
			#3
			\str_case:nn {#1}
			{
				{->} {\lkxto}
					{~>} {\lkxisom}
					{->>} {\lkxsurj}
			}
			#4
		} {
			\tl_if_blank:nTF {#2}{}{#2 \;:\;}
			#3
		   &
			\str_case:nn {#1}
			{
				{->} {\lkxto}
					{~>} {\lkxisom}
					{->>} {\lkxsurj}
			}
		   & #4
		\\
		#5 & \lkxmapsto & #6
		}
	\end{array}
}

\ExplSyntaxOff


\collaborator{AJ LaMotta}

\begin{document}
\maketitle

\begin{problem}{1}
  A torus has a generator. We prove this in a sequence of steps.
\end{problem}

\begin{problem}{2}
  True or false. Provide a proof or example.
\end{problem}
\begin{parts}
  \begin{part}{(a)}
    There is a non-trivial homomorphism $\SO_3 \to \Sp_1$.
  \end{part}

  This is false. Since $\SO_3$ is a simple (abstract) group, any non-trivial homomorphism would have to be injective and not surjective. 
  But since $\Sp_1$ is homeomorphic to $S^3$, by removing a point not in the image of the homomorphism would give an embedding of $\SO_3$ in $\R^3$. However, $\SO_3$ is homeomorphic to $\RP^3$ which is a closed $3$-manifold and thus cannot be embedded in $3$-dimensional Euclidean space.

  \begin{part}{(b)}
    There is a non-trivial homomorphism $\Sp_1 \to \SO_3$.
  \end{part}

  Embed $\R^3\subset \HH$ by sending a basis $\{e_1,e_2,e_3\}$ for $\R^3$ to $\{i,j,k\}$ in $\HH$. Then conjugation by a unit quaternion $q\in \Sp_1$ leaves this hyperplane invariant, since $\Re(qxq^{-1})=\Re(x)$. Furthermore, conjugation by a unit norm quaternion is an isometry on $\HH$, and so is an isometry on $\R^3\subset \HH$.  We thus get a homomorphism
  \[
    \lkxfunc{}{\Sp_1}{\SO_3}{q}{(x\mapsto qxq^{-1}).}
  \]

  \begin{part}{(c)}
    There is a non-trivial homomorphism $\SO_5 \to \Sp_2$.
  \end{part}
    
  \begin{part}{(d)}
    There is a non-trivial homomorphism $\Sp_2 \to \SO_5$.
  \end{part}

\end{parts}
\end{document}

