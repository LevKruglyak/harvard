\documentclass{../../templates/lkx_pset}

\usepackage[T1]{fontenc}
\RequirePackage{mlmodern}

\title{Math 230a Problem Set 6}
\author{Lev Kruglyak}
\due{October 16, 2024}

\renewcommand{\O}{{\operatorname{O}}}
\renewcommand{\GL}{\operatorname{GL}}
\providecommand{\Det}{\operatorname{Det}}
\providecommand{\T}{\mathbb{T}}
\providecommand{\HH}{\mathbb{H}}
\providecommand{\gfr}{\mathfrak{g}}
\providecommand{\pfr}{\mathfrak{p}}
\providecommand{\gl}{\mathfrak{gl}}
\providecommand{\hfr}{\mathfrak{h}}
\providecommand{\U}{\operatorname{U}}
\providecommand{\SO}{\operatorname{SO}}
\providecommand{\SU}{\operatorname{SU}}
\providecommand{\su}{\mathfrak{su}}
\providecommand{\so}{\mathfrak{so}}
\renewcommand{\sl}{\mathfrak{sl}}
\providecommand{\Spin}{\operatorname{Spin}}
\providecommand{\Sp}{\operatorname{Sp}}
\providecommand{\scB}{\mathscr{B}}
\providecommand{\op}[1]{\operatorname{#1}}

\providecommand{\Hol}{\operatorname{Hol}}
\providecommand{\Ad}{\operatorname{Ad}}
\providecommand{\mc}{\operatorname{mc}}
     \usepackage[mathscr]{euscript}

\providecommand{\calB}{\mathcal{B}}

\providecommand{\Aff}{\operatorname{Aff}}
\providecommand{\Diff}{\operatorname{Diff}}
\providecommand{\HH}{\operatorname{H}}

\providecommand{\longto}{\;\xrightarrow{\phantom{xxx}}\;}
\providecommand{\longisom}{\;\xrightarrow{\phantom{xx}\sim\phantom{xx}}\;}
\providecommand{\longsurj}{\;\xtwoheadrightarrow{\phantom{xxx}}\;}

\providecommand{\definefunction}[5]{
	\begin{array}{rcl}
		#1 : #2 & \xrightarrow{\phantom{---}} & #3 \\
		#4      & \xmapsto{\phantom{---}}     & #5
	\end{array}
}

\providecommand{\qtq}[1]{\quad\textrm{#1}\quad}

\usepackage{adjustbox}
\newcommand{\alt}{\mathord{\adjustbox{valign=B,totalheight=.6\baselineskip}{$\bigwedge$}}}

\providecommand{\Hdr}{\operatorname{H}_{\operatorname{dR}}}

\providecommand{\pp}[2]{\frac{\partial #1}{\partial #2}}
\providecommand{\pps}[1]{\partial/\partial #1}


\ExplSyntaxOn
\NewDocumentCommand{\lkxto}{ O{} }{%
	\mathrel{\;\xrightarrow{\hphantom{xx}#1\hphantom{xx}}\;}
}
\NewDocumentCommand{\lkxmapsto}{ O{} }{%
	\mathrel{\;\xmapsto{\hphantom{xx}#1\hphantom{xx}}\;}
}
\NewDocumentCommand{\lkxisom}{ O{} }{%
	\mathrel{\;\xrightarrow{\hphantom{xx}#1\sim\hphantom{xx}}\;}
}
\NewDocumentCommand{\lkxsurj}{ O{} }{%
	\mathrel{\;\xtwoheadrightarrow{\hphantom{xx}#1\hphantom{xx}}\;}
}

\NewDocumentCommand{\lkxfunc}{ O{->} m m m g g }{
	\begin{array}{rcl}
		\IfNoValueTF {#5} {
			\tl_if_blank:nTF {#2}{}{#2 :}
			#3
			\str_case:nn {#1}
			{
				{->} {\lkxto}
					{~>} {\lkxisom}
					{->>} {\lkxsurj}
			}
			#4
		} {
			\tl_if_blank:nTF {#2}{}{#2 \;:\;}
			#3
		   &
			\str_case:nn {#1}
			{
				{->} {\lkxto}
					{~>} {\lkxisom}
					{->>} {\lkxsurj}
			}
		   & #4
		\\
		#5 & \lkxmapsto & #6
		}
	\end{array}
}

\ExplSyntaxOff


\collaborator{AJ LaMotta}
\collaborator{Ignasi Vicente}

\begin{document}
\maketitle

\begin{problem}{1}
  Let $A$ be a $3$-dimensional Euclidean space and $\Sigma \subset A$ a cooriented surface. A point $p\in \Sigma$ is \emph{umbilic} if the second fundamental form at $p$ is a multiple of the metric at $p$. Suppose $e_1,e_2,e_3$ is a moving frame on an open subset of $\Sigma$ and $\theta^1,\theta^2,\Theta^2_1, \Theta^3_1, \Theta^3_2$ the induced $1$-forms from $\scB_\O(A)$. Express the umbilic condition in terms of these forms.
\end{problem}

\begin{solution}
  A point $p\in \Sigma$ is umbilic if and only if for all $\xi,\eta\in T_p\Sigma$ we have
  \[
    -\langle D_\xi \nu, \eta \rangle_p = \lambda\langle \xi, \eta \rangle_p
  \]
  for some constant $\lambda\in \R$. If we plug in the basis vectors $e_i$, this amounts to the condition
  \[
    \langle D_{e_i}\nu, e_j\rangle_p = -\lambda\delta_{ij}.
  \]
  Rewriting this in terms of the $\Theta$ forms, we get
  \[
    (\Theta^3_1)_p(e_1) = (\Theta^3_2)_p(e_2),\quad (\Theta^3_1)_p(e_2)=0,\quad\textrm{and}\quad (\Theta^3_2)_p(e_1)=0.
  \]
  In particular, we have $h_{ij}=\lambda \delta_{ij}$.

  \begin{part}{}
    Suppose now that \emph{every} point of $\Sigma$ is umbilic. Then there is a function $\lambda : \Sigma \to \R$ such that the second fundamental form is $\lambda$ times the first fundamental form. Prove that $\lambda$ is locally constant.
  \end{part}

  This follows from the Codazzi-Mainardi equations.
\end{solution}

\begin{problem}{2}
Let $M$ be a smooth manifold of dimension $n$.
\end{problem}

\begin{solution}
	Before getting started with the constructions, let's consider the product tangent bundle $T(M^{\times n})=TM^{\times n}$, and the diagonal $\Delta_M \subset M^{\times n}$. This diagonal is of course a submanifold of $M^{\times n}$ canonically diffeomorphic to $M$ by a map
  \[
    \lkxfunc{\delta}{M}{\Delta_M\subset M^{\times n}}{x}{(x,\ldots, x).}
  \]
	We can also take the restriction of the tangent bundle to this submanifold, let's denote this $\overline{T}\Delta_M \subset TM^{\times n}$. At this point, the pullback bundle $\delta_* \overline{T}\Delta_M$ is the fiber bundle of ordered $n$-tuples of tangent vectors of $M$. Let's denote this canonical bundle $\overline{\scB}(M)$.

	\begin{part}{(a)}
		Construct the principal $\GL_n$-bundle of frames $\scB(M) \to M$.
	\end{part}

	% A frame at a point $x\in M$ of a manifold is a basis of the tangent space $T_x M$, or equivalently a linear isomorphism $\R^n \to T_x M$. The space of all frames is a right $\GL_n$-torsor, since any basis can be transformed to another by a change of coordinate matrix. We want a principal bundle over $M$ whose fiber at each point can be canonically identified with this $\GL_n$-torsor.


  % \todo{introduce determinant map.}
	Now, $TM^{\times n}$ is a $2n^2$-manifold, and can thus be embedded into some affine space $\A^k$ for $k$ large enough -- let's call this embedding $\iota : TM^{\times n} \to \A^k$. By composing the diagonal map with the zero section and the embedding, we get an embedding $\iota_0 : M \to \A^k$, where $\iota_0 = \iota\circ s_0\circ \delta$. These embeddings then give us a smooth map
	\[
    \lkxfunc{\det}{\overline{T}\Delta_M}{\R}{(x; v_1,\ldots, v_n)}{\det(\iota(v_1)-\iota_0(x),\ldots, \iota(v_n)-\iota_0(x)).}
	\]
	Note that $\det^{-1}(\R^\times)$ is an open subset of $\overline{T}\Delta_M$ and so must be a manifold. Let's finally define the total space frame bundle to be
	\[
    \scB(M) = \delta_* \textrm{det}^{-1}(\R^\times) = \left\{ (x; v_1,\ldots, v_n)\in \overline{\scB}(M) : \textrm{span}\{v_1,\ldots, v_n\} = T_xM\right\}
	\]

	\begin{part}{(b)}
		Construct the principal $\O_n$-bundle of orthonormal frames $\scB_\O(M) \to M$ given a Riemannian metric.
	\end{part}

	A Riemannian metric gives us an inner product on each tangent space. Let's consider the set
	\[
    \scB_\O(M) = \left\{ (x; v_1,\ldots, v_n) \in \scB(M) : \langle v_i, v_j\rangle = \delta_{ij} \right\}.
	\]
	This set of equations has no singularities and so forms a manifold. It's clear that the fibers are diffeomorphic to $\O_n$, and we have local trivializations by embedding in affine space as before.
\end{solution}

\begin{problem}{3}
  Does there exist an example of a Riemannian $2$-manifold $\Sigma$ and an embedded oriented circle $C\subset \Sigma$ such that parallel transport around $C$ is a reflection (rather than a rotation)?
  If so, what does this say about the lift of $C$ to the orientation double cover, which is identified with $\scB_\O(\Sigma)/\SO_2 \to \Sigma$? Does there exist an example in which $\Sigma$ is a submanifold of a Euclidean $3$-space?
\end{problem}

\begin{solution}
  Consider the M\"obius strip $M$, viewed as a nontrivial line bundle $\pi$ over $S^1$. Picking any orientation of $S^1$ and including it into $M$ by the zero section gives us an oriented subcircle $C\subset M$. Since $M$ is a line bundle, note that the tangent space of  the M\"obius strip at any point $p\in M$ is given by $T_p M = T_{\pi(p)} S^1\oplus M_{\pi(p)}$. The parallel transport of any vector in $T_{\pi(p)}S^1$ around $C$ is just given by the parallel transport in $S^1$, and parallel transports of vectors in $M_{\pi(p)}$ are given by locally constant sections of $M$. It's clear that a frame $(v_1,v_2)$ with $v_1$ a vector in the former space and $v_2$ a vector in the latter space is sent to $(v_1, -v_2)$ after a full parallel transport around $C$. This is a reflection, not a rotation.

  This implies that the orientation double cover is connected, since any frame can be reflected by a parallel transport. In particular, it implies that the lift of $C$ to $\scB_\O(\Sigma)/\SO_2$ can be identified with the double cover $S^1\to S^1$. We see that any surface satisfying the condition must be nonorientable. If we let surfaces be non-compact, then our example of the M\"obius bundle works. Otherwise, if we require compact closed manifolds, the smallest example can be embedded in Euclidean $4$-space.
\end{solution}

\end{document}
