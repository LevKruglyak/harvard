\documentclass[11pt,letterpaper]{article}

\input{../../../../.config/latex/preamble_v1.tex}

\lightmode
\usetikzlibrary{decorations.pathreplacing,decorations.markings}

\title{\textbf{Math 231a Final}}

\providecommand{\Sk}{\text{Sk}}
\providecommand{\Gr}{\text{Gr}}
\providecommand{\Tor}{\text{Tor}}
\providecommand{\Ext}{\text{Ext}}
\providecommand{\tors}{\text{tors}}

% `tikz-cd` is necessary to draw commutative diagrams.
\RequirePackage{tikz-cd}
% `amssymb` is necessary for `\lrcorner` and `\ulcorner`.
\RequirePackage{amssymb}
% `calc` is necessary to draw curved arrows.
\usetikzlibrary{calc}
% `pathmorphing` is necessary to draw squiggly arrows.
\usetikzlibrary{decorations.pathmorphing}

% A TikZ style for curved arrows of a fixed height, due to AndréC.
\tikzset{curve/.style={settings={#1},to path={(\tikztostart)
    .. controls ($(\tikztostart)!\pv{pos}!(\tikztotarget)!\pv{height}!270:(\tikztotarget)$)
    and ($(\tikztostart)!1-\pv{pos}!(\tikztotarget)!\pv{height}!270:(\tikztotarget)$)
    .. (\tikztotarget)\tikztonodes}},
    settings/.code={\tikzset{quiver/.cd,#1}
        \def\pv##1{\pgfkeysvalueof{/tikz/quiver/##1}}},
    quiver/.cd,pos/.initial=0.35,height/.initial=0}

% TikZ arrowhead/tail styles.
\tikzset{tail reversed/.code={\pgfsetarrowsstart{tikzcd to}}}
\tikzset{2tail/.code={\pgfsetarrowsstart{Implies[reversed]}}}
\tikzset{2tail reversed/.code={\pgfsetarrowsstart{Implies}}}
% TikZ arrow styles.
\tikzset{no body/.style={/tikz/dash pattern=on 0 off 1mm}}

\begin{document}
\maketitle

\begin{problem}
    Let $V_1,V_2 \subset\R^6$ denote complementary $3$-planes (i.e. $V_1$ and $V_2$ span $\R^6$ as a vector space). Compute the integral homology groups of $\R^6 - (V_1 \cup V_2)$.
\end{problem}

\begin{solution}
    \quad Let's decompose the space $X=R^6 - (V_1\cup V_2)$ into two spaces, $A_1 = \R^6 - V_1$ and $A_2 = \R^6 - V_2$. Then $X = A_1\cap A_2$ and $A_1\cup A_2 = \R^6 - \{0\}$, so we can apply the Mayer-Vietoris sequence. Before we do this, let's prove a lemma.

    \begin{claim}
        Let $V$ be a $3$-plane in $\R^6$, i.e. a 3-dimensional linear subspace of $\R^6$. Then $\R^6 - V$ is homotopy equivalent to $S^2$.
    \end{claim}

    \begin{proof}
        Let $V^\perp$ be the orthogonal complement to $V$ in $\R^6$ so that $V\oplus V^\perp = \R^6$. Let $r : \R^6 - V \to S^2$ be the map that takes $(v, v') \mapsto v' / \|v'\|$. This is continuous since the $V^\perp$ component of $\R^6 -V$ is always nonzero. We have an inverse inclusion $i : S^2 \to \R^6-V$ given by $v \mapsto (0, v)$. Then the homotopy between $r\circ i$ and the identity on $\R^6-V$ is given by
        \[
            H : (\R^6 - V)\times I \to \R^6 - V : ((v,v'), t) \mapsto ((1-t)v, tv'+(1-t)v' / \|v'\|)
        .\]
        It is easy to see that $H(-,0) = \text{id}_{(\R^6 - V)}$ and $H(-,1) = r\circ i$. 
    \end{proof}

    \quad So $A_1, A_2\simeq S^2$ and $A_1\cup A_2 = \R^6 - \{0\} \simeq S^5$. Recall that the reduced Mayer-Vietoris gives us an exact sequence:
    \begin{center}
        \begin{tikzcd}
            \widetilde{H}_{n+1}(A_1\cup A_2) \arrow[r] & \widetilde{H}_n(A_1\cap A_2) \arrow[r] & \widetilde{H}_n(A_1)\oplus \widetilde{H}_n(A_2) \arrow[r] & \widetilde{H}_n(A_1\cup A_2)
        \end{tikzcd}
    \end{center}  
    By the lemma, $\widetilde{H}_n(A_i)\cong \Z$ when $n=2$, and zero otherwise. Similarly, $\widetilde{H}_n(A_1\cup A_2)\cong \Z$ when $n=5$, and zero otherwise. Plugging these terms into the exact sequence, we get the reduced homology for $X=A_1\cap A_2$:
    \[
        \widetilde{H}_n(\R^6 - (V_1\cup V_2)) = \begin{cases}
            \Z\oplus \Z & n=2,\\
            \Z & n = 4,\\
            0 & \text{otherwise},
        \end{cases}\implies 
        \boxed{{H}_n(\R^6 - (V_1\cup V_2)) = \begin{cases}
            \Z & n = 0, 4,\\
            \Z\oplus \Z & n=2,\\
            0 & \text{otherwise}.
        \end{cases}}
    \]  
\end{solution}

\pagebreak
\begin{problem}
    Given a space $X$ with the property that $\bigoplus^\infty_{i=0} H_i(X)$ is finitely generated, define its \emph{Euler characteristic} to be $\chi(X)=\sum^\infty_{i=0}(-1)^i\text{rank } H_i(X)$. Use the universal coefficients theorem to prove that $\chi(X)=\sum^\infty_{i=0}(-1)^i\dim_k H_i(X; k)$ for $k=\Q$ or $\F_p$ where $p$ is a prime number.
\end{problem}

\begin{solution}
    \quad Recall that the universal coefficients theorem states that for $R$ a PID and $M$ an $R$-module, we have a (non-natural) isomorphism:
    \[
        H_i(X; M) \cong \left(H_i(X)\otimes_R M\right)\oplus \Tor^R_1(H_{i-1}(X), M)
    .\]  
    In our case, $R=\Z$ and $M=k=\Q$ or $\F_p$. When $k=\Q$, the Tor term vanishes since $\Q$ is a flat $\Z$-module sn so $H_i(X; \Q) \cong H_i(X)\otimes \Q$. This means that $\text{rank }H_i(X) = \dim_\Q H_i(X; \Q)$ and so clearly the result holds.
    
    \quad In the case when $k=\F_p$, the Tor term is exactly:
    \[
        \Tor^\Z_1 (H_{i-1}(X), \F_p) = \text{tors}_p\; H_{i-1}(X).
    \]   
    Thus we have a (non-natural) isomorphism:
    \[
        H_i(X; \F_p) \cong (H_i(X)\otimes \F_p) \oplus \text{tors}_p\; H_{i-1}(X)
    .\] 
    However since $H_i(X)\otimes \F_p \cong \F_p^{\oplus \text{rank }H_i(X)}\oplus \text{tors}_p H_i(X)$, we have an isomorphism:
    \[
        H_i(X; \F_p) \cong \F_p^{\text{rank }H_i(X)}\oplus \text{tors}_p\; H_i(X) \oplus \text{tors}_p\; H_{i-1}(X)
    .\]   
    Let $a_i = \text{rank } H_i(X)$ and $b_i = \dim_{\F_p} \text{tors}_p\; H_i(X)$ so that $\dim_{\F_p} H_i(X; \F_p) = a_i + b_i + b_{i-1}$. Then
    \[
        \sum^\infty_{i=0}(-1)^i \dim_{\F_p} H_i(X; \F_p) = \sum^\infty_{i = 0}(-1)^i (a_i + b_i + b_{i-1}) = \sum_{i=0}^\infty (-1)^i a_i = \chi(X)
    .\]  
\end{solution}

\begin{problem}
    Consider the following three spaces:
    \[
        A = \CP^3 \quad B = S^2\times S^4\quad \text{and}\quad C = S^2\vee S^4\vee S^6
    \] 
\end{problem}

\begin{solution}
    In this problem we'll explore their topologies.
    \begin{partproblem}{a}
        Compute the integral cohomology groups of these three spaces.
    \end{partproblem}
    
    \quad Recall that $A=\CP^3$ has a CW decomposition consisting of a single cell the dimensions $0, 2, 4, 6$. By cellular homology this trivially gives the homology:
    \[
        \boxed{H_i(\CP^3) = \begin{cases}
            \Z & i = 0,2,4,6,\\
            0 &\text{otherwise}.
        \end{cases}}
    \]
    For the product space $B=S^2\times S^4$, since the homology groups of $S^n$are torsion free, the Tor term in the K\"unneth formula vanishes and so we get an isomorphism:
    \[
        H_i(S^2\times S^4) \cong \bigoplus_{p+q=i} H_p(S^2)\otimes H_q(S^4)
    .\]
    Since $H_p(S^n) = \Z$ if and only if $p=n, 0$, and using the fact that $\Z\otimes \Z \cong \Z$ and $\Z\otimes 0 = 0$, we get the same homology for $S^2\times S^4:$ 
    \[
        \boxed{H_i(S^2\times S^4) = \begin{cases}
            \Z & i = 0,2,4,6,\\
            0 &\text{otherwise}.
        \end{cases}}
    \] 
    Finally, for the wedge sum $C = S^2\vee S^4\vee S^6$, recall that there is an isomorphism of reduced homology
    \[
        \widetilde{H}_i(X\vee Y) \cong \widetilde{H}_i(X)\oplus \widetilde{H}_i(Y).
    \] 
    Again, this gives the same homology for $C$:
    \[
        \boxed{H_i(S^2\vee S^4\vee S^6) = \begin{cases}
            \Z & i = 0,2,4,6,\\
            0 &\text{otherwise}.
        \end{cases}}
    \] 

    \begin{partproblem}{b}
        Prove that $A$ and $B$ are not homotopy equivalent.
    \end{partproblem}

    \quad Since these spaces have the same homology rings, we can look at differences in the cup product structure on the cohomology ring. Recall that the cohomology ring of a sphere is $H^\bullet(S^n)\cong \Z[x] /(x^2)$, where $|x|=n$. Then by the K\"unneth formula and properties of cup products, we have $H^\bullet(S^2\times S^4) \cong \Z[x] /(x^2) \otimes \Z[y] / (y^2) \cong \Z[x,y]/(x^2, y^2)$. 
    
    \quad Similarly, recall that $H^\bullet(\CP^3)\cong \Z[x]/(x^4)$ with $|x|=2$. (This is proved in Hatcher, or by a similar argument to the first problem of Problem Set~10) These rings are clearly not isomorphic since one has an element of multiplicative order $4$, hence $S^2\times S^4$ isn't homotopy equivalent to $\CP^3$.

    \begin{partproblem}{c}
        Prove that $C$ is not homotopy equivalent to any compact manifold.
    \end{partproblem}
    \quad Recall that Poincar\'e duality gives us a perfect pairing for any compact manifold $M$:
    \begin{center}
        \begin{tikzcd}
	        {H^p(M; \F_2)\otimes H^q(M; \F_2)} & {H^n(M; \F_2)} && {\F_2}
	        \arrow["\smile", from=1-1, to=1-2]
	        \arrow["{\langle -,[M]\rangle}", from=1-2, to=1-4]
        \end{tikzcd}
    \end{center}
    where $p+q=n$. In particular, this pairing of $H^p(M; \F_2)\otimes H^q(M; \F_2) \to \F_2$ is nontrivial. F
    
    \quad For our case, suppose for the sake of contradiction that $C$ is homotopy equivalent to a compact manifold. Now let $\delta$ be a generator for $H^2(S^2)$ and $\sigma$ a generator for $H^4(S^4)$. Let $i : S^6 \to S^2\vee S^4\vee S^6$ be the inclusion map. Now $i^*(\delta)\smile i^*(\sigma) = 0$ in $H^6(S^6)$ since $S^6$ has no 2 or 4 degree homology. Then $i^*(\delta\smile \sigma)=0$. Since $i^*$ is an isomorphism in $H^6$ (see Miller), this means that $\delta\smile \sigma=0$. By the universal coefficients theorem and naturality of the cup product, this same triviality condition holds with $\F_2$ coefficients. But this contradicts Poincar\'e duality since the pairing $H^2(C; \F_2)\otimes H^4(C; \F_2) \to H^6(C; \F_2)$ becomes trivial.
\end{solution}

\begin{problem}
    Let $M$ and $N$ denote connected $n$-manifolds. A \emph{connected sum} $M\# N$ of $M$ and $N$ is a new connected $n$-manifold defined in the following way. Choose Euclidean open balls $B_1\subset \R^n\subset M$ and $B_2\subset \R^n\subset N$. Then $M -B_1$ and $N-B_2$ contain embedded $(n-1)$-spheres which are boundaries of the balls $B_i$, and we define $M\# N$ by gluing $M-B_1$ and $N-B_2$ along any homeomorphism of these $(n-1)$-spheres. More concisely we write:
    \[
        M\# N := (M - B_1)\cup_{S^{n-1}} (N - B_2)
    .\] 
    For example, if $M_g$ and $M_h$ are oriented surfaces of genus $g$ and $h$, then $M_g\# M_h$ is an oriented surface of genus $g+h$.
\end{problem}

\begin{solution}
     In the following we let $M\#N$ denote an arbitrary choice of connected sum of $M$ and $N$.
     \begin{partproblem}{a}
         Assume that $n\geq 2$. Prove that $M\# N$ is orientable if and only if $M$ and $N$ are orientable.
     \end{partproblem}

     \quad Since Hatcher and Miller have inconsistent notations, let $\Gamma_M$ be the orientation cover of $M$, i.e. the set
     \[
         \Gamma_M = \left\{\mu : M \to \bigsqcup_{x\in M} H_n(M, M-\{x\}) \mid \mu\text{ is locally consistent, }\mu_x\text{ a generator }\forall x\in M\right\}
     \]
     with a natural topology that can be constructed from the projection map $p_M : \Gamma_M \to M$. For a manifold $M$, it is orientable if $\Gamma_M$ has two connected components and nonorientable if $\Gamma_M$ has a single connected component. In either case, it is a double sheeted covering.

     \quad For manifolds $M, N$, note that $M-B_1$ and $N-B_2$ are manifolds with boundary. Let $i_1 : M-B_1 \to M\# N$ and $i_2 : N- B_2 \to M\# N$, so $i_1(M-B_1)$ and $i_2(N-B_2)$ are submanifolds of $M\# N$. Thus we can consider $\Gamma_{i_1(M-B_1)} \cong \Gamma_{M-B_1}$ and $\Gamma_{i_2(N-B_2)}\cong \Gamma_{N-B_2}$ as submanifolds of $\Gamma_{M\# N}$. In fact, since $i_1(M-B_1)\,\cup\, i_2(N-B_2) = M\# N$, we have $\Gamma_{M\# N} = \Gamma_{i_1(M-B_1)}\cup \Gamma_{i_2(N-B_2)}$. 

    \quad Now if both $M$ and $N$ are orientable, $M-B_1$ and $N-B_2$ are also orientable so $\Gamma_{i_1(M-B_1)}$ and $\Gamma_{i_2(N-B_2)}$ both have two connected components. Since $i_1(M-B_1)\cap i_2(N-B_2)$ is homeomorphic to $S^{n-1}$, a connected, orientable manifold, it follows that $\Gamma_{M\# N}$ has two connected components. Thus $M\# N$ is orientable. (Note that $S^{n-1}$ is an $(n-1)$-manifold, so you might be suspicious of its orientability in $M\# N$, but we can always expand to a cylinder $(0,1)\times ^{n-1}$ in $M\# N$ and everything is fine.)

    \quad If one of the manifolds is nonorientable, say $M$, then $M-B_1$ is nonorientable as well so $\Gamma_{i_1(M-B_1)}$ has a single component. As before, since the intersection $i_1(M-B_1)\cap i_2(N-B_2)$ is the $(n-1)$-sphere, which is connected, it follows that regardless of how many components $\Gamma_{i_1(N-B_2)}$ has, they will both be attached to the single component in of $\Gamma_{i_1(N-B_2)}$. So $M\# N$ is nonorientable.  

     \begin{partproblem}{b}
         Assume that $M$ and $N$ are compact. If $M$ or $N$ is orientable, prove that
         \[
             H_i(M\# N)\cong H_i(M)\oplus H_i(N)
         \]
         for $0<i<n$. 
     \end{partproblem}

     \quad Consider $S^{n-1}\subset M\# N$. Relative reduced homology gives us an exact sequence:
     \begin{center}
         \begin{tikzcd}
            \widetilde{H}_i(S^{n-1}) \arrow[r] & \widetilde{H}_i(M\# N) \arrow[r]& \widetilde{H}_i(M\# N, S^{n-1}) \arrow[r] & \widetilde{H}_{i-1}(S^{n-1})
         \end{tikzcd}
     \end{center}
     For any $0<i<n-1$ we have $\widetilde{H}_i(S^{n-1})\cong \widetilde{H}_{i-1}(S^{n-1}) \cong 0$ so we have an isomorphism
     \[
         \widetilde{H}_i(M\# N) \cong \widetilde{H}_i(M\# N, S^{n-1})
     .\]  
     Since this is a good pair, by excision we get an isomorphism $\widetilde{H}_i(M\# N, S^{n-1}) \cong \widetilde{H}_i(M\# N / S^{n-1})$. However $M\# N / S^{n-1}$ is homeomorphic to $M\vee N$ and then $\widetilde{H}_i(M\vee N) \cong \widetilde{H}_i(M)\oplus \widetilde{H}_i(N)$. Since $i>0$, we thus have isomorphisms
     \[
         H_i(M\# N)\cong H_i(M)\oplus H_i(N)
     .\]   
     
     \quad When $i = n-1$ the exact sequence is
     \begin{center}
        \begin{tikzcd}
            \widetilde{H}_{n}(S^{n-1}) \arrow[r] & \widetilde{H}_{n}(M\# N) \arrow[r]& \widetilde{H}_{n}(M\# N, S^{n-1}) \arrow[dll]\\
           \widetilde{H}_{n-1}(S^{n-1}) \arrow[r] & \widetilde{H}_{n-1}(M\# N) \arrow[r]& \widetilde{H}_{n-1}(M\# N, S^{n-1}) \arrow[r] & \widetilde{H}_{n-2}(S^{n-1})
        \end{tikzcd}
    \end{center}

    Observe that $\widetilde{H}_n(S^{n-1})\cong \widetilde{H}_{n-2}(S^{n-1})\cong 0$, and we already mentioned the isomorphism $\widetilde{H}_i(M\# N, S^{n-1})\cong \widetilde{H}_i(M)\oplus \widetilde{H}_i(N)$. Then this exact sequence gives us an alternating rank formula:
    \[
        \text{rank }H_n(M\# N) - \text{rank }H_n(M) - \text{rank }H_n(N) + 1 - \text{rank }H_{n-1}(M\# N) +\text{rank }H_{n-1}(M) + \text{rank }H_{n-1}(N) =0
    .\] 
    So far we haven't assumed anything about orientability, so we can all of this for the next part as well.

    \quad We now have two cases for this part. If both manifolds are orientable, $H_n(M)\cong H_n(N) \cong \Z$ and $M\# N$ is orientable so $H_n(M\# N)\cong \Z$. Then the rank formula gives us:
    \[
        \begin{aligned}
            1 - 1 - 1 + 1 - \text{rank }H_{n-1}(M\# N) &+ \text{rank }H_{n-1}(M) +\text{rank }H_{n-1}(N) = 0 \\
            \implies \text{rank }H_{n-1}(M\# N) &= \text{rank }H_{n-1}(M) +\text{rank }H_{n-1}(N).
        \end{aligned}
    \] 
    Since $M\# N$ is orientable, $H_{n-1}(M\# N)$ has no torsion component, and neither do $H_{n-1}(M)$ or $H_{n-1}(N)$. Thus we have $H_{n-1}(M\# N) \cong H_{n-1}(M)\oplus H_{n-1}(N)$ as desired.

    \quad If $M$ is orientable but $N$ is not orientable, then $M\# N$ is not orientable. By the same reasoning as earlier, the rank formula then is:
    \[
        \begin{aligned}
            0 - 1 - 0 + 1 - \text{rank }H_{n-1}(M\# N) &+ \text{rank }H_{n-1}(M) +\text{rank }H_{n-1}(N) = 0 \\
            \implies \text{rank }H_{n-1}(M\# N) &= \text{rank }H_{n-1}(M) +\text{rank }H_{n-1}(N).
        \end{aligned}
    \] 
    Since $H_{n-1}(M)$ has no torsion part, $H_{n-1}(N)$ has torsion exactly $\Z /2$, and $H_{n-1}(M\# N)$ has torsion exactly $\Z /2$, we also get our isomorphism $H_{n-1}(M\# N) \cong H_{n-1}(M)\oplus H_{n-1}(N)$.

     \begin{partproblem}{c} 
         Assume that $M$ and $N$ are compact. If $M$ and $N$ are both nonorientable, prove that
         \[
             H_i(M\#N)\cong H_i(M)\oplus H_i(N)
         \] 
         for $0<i<n-1$, whereas $H_{n-1}(M\# N)$ is obtained from $H_{n-1}(M)\oplus H_{n-1}(N)$ by replacing a $\Z /2\Z$ by a $\Z$ summand. 
     \end{partproblem}

     \quad We can use the proof of the previous problem to get the isomorphism $H_i(M\# N)\cong H_i(M)\oplus H_i(N)$ for $0 < i < n-1$. For the case when $i=n-1$, we'll used the rank formula proved in the previous problem:
     \[
        \begin{aligned}
            0 - 0 - 0 + 1 - \text{rank }H_{n-1}(M\# N) &+ \text{rank }H_{n-1}(M) +\text{rank }H_{n-1}(N) = 0 \\
            \implies \text{rank }H_{n-1}(M\# N) &= \text{rank }H_{n-1}(M) +\text{rank }H_{n-1}(N)+1.
        \end{aligned}
    \]
    Since the torsions of $H_{n-1}(M)$, $H_{n-1}(N)$, and $H_{n-1}(M\# N)$ are exactly $\Z /2$, yet the rank of $H_{n-1}(M\# N)$ is one more than the sum of the ranks of $H_{n-1}(M)$ and $H_{n-1}(N)$. This gives exactly the effect described in the problem statement: $H_{n-1}(M\# N)$ is congruent to $H_{n-1}(M)\oplus H_{n-1}(N)$ but we take out a $\Z/2$ and replace it with a $\Z$.   
\end{solution}

\begin{problem}
    Let $M$ denote a compact connected $3$-manifold.
\end{problem}

\begin{solution}
    Suppose that $H_1(M)\cong \Z^{\oplus r}\oplus F$ for a finite abelian group $F$ and nonnegative integer $r$. 
    \begin{partproblem}{a}
        If $M$ is orientable, prove that $H_2(M)\cong \Z^{\oplus r}$.
    \end{partproblem}

    \quad Since $M$ is an orientable, compact, connected manifold, we know that $H_3(M)\cong \Z$. We can apply the Poincar\'e duality isomorphism $H_k(M) \cong H^{n-k}(M)$ to get $H_2(M)\cong H^1(M)$. In order to calculate $H^1(M)$, we'll use the universal coefficients theorem, which gives us the exact sequence:
    \begin{center}
        \begin{tikzcd}
            0 \arrow[r] & \Ext^1_\Z(H_0(M), \Z) \arrow[r]& H^1(M) \arrow[r]& \Hom_\Z (H_1(M), \Z) \arrow[r] &0
        \end{tikzcd}
    \end{center}
    Since $H_0(M)=\Z$, the Ext term vanishes and so $H^1(M)\cong \Hom_\Z(H_1(M), \Z)$. However $H_1(M)\cong \Z^{\oplus r}\oplus F$ and so $\Hom_\Z(H_1(X), \Z)\cong \Z^{\oplus r}$ as desired.

    \begin{partproblem}{b}
        If $M$ is nonorientable, prove that $H_2(M)\cong \Z^{\oplus r-1}\oplus \Z /2$. In particular, one must have $r\geq 1$ in this case. % coefficients + UCT
    \end{partproblem}

    \quad Recall that for a a nonorientable compact manifold $M$, we have $\text{tor }H_{n-1}(M) \cong \Z /2$ and $H_n(M)\cong 0$. We also know that the Euler characteristic of an odd dimensional manifold is zero by Poincar\'e duality, so we have
    \[
        \text{rank }H_0(M) - \text{rank }H_1(M) + \text{rank }H_2(M) - \text{rank }H_3(M) = 0
    .\]   
    Since $\text{rank}(H_0(M))=1$ because $M$ is connected, $\text{rank}(H_1(M)) = r$ and $\text{rank}(H_3(M)) = 0$, we have the identity $\text{rank}(H_2(M)) = r-1$. Thus,
    \[
        H_2(M) \cong \Z^{\oplus r - 1}\oplus \Z / 2
    .\] 

    % \quad In this case, we cannot apply Poincar\'e duality as we did in the previous part because $M$ is not orientable. However, like all manifolds, it is $\F_2$-orientable, so we know that there is an isomorphism $H_k(M; \F_2)\cong H^{n-k}(M; \F_2)$. This means that $H_2(M; \F_2) \cong H^1(M; \F_2)$. The universal coefficients theorem states that for cohomology there is an exact sequence:
    % \begin{center}
    %     \begin{tikzcd}
    %         0 \arrow[r] & \Ext^1_{\F_2}(H_0(M; \F_2), \F_2) \arrow[r]& H^1(M; \F_2) \arrow[r]& \Hom_{\F_2} (H_1(M; \F_2), \F_2) \arrow[r] &0
    %     \end{tikzcd}
    % \end{center}
    % Recall that $H_0(M; \F_2)\cong \F_2$ so the Ext term vanishes. Also by the universal coefficients theorem for homology, we have \[H_1(M; \F_2)\cong (H_1(M)\otimes \F_2)\oplus \Tor^{\F_2}_1(H_0(M), \F_2) \cong \F_2^{\oplus r}\oplus \text{tors}_2\, F\]
    % Since the linear dual of this group is itself, we have a isomorphism $H^1(M; \F_2)\cong \F_2^{\oplus r}\oplus \text{tors}_2\, F$, which by duality is isomorphic to $H_2(M; \F_2) \cong \F_2^{\oplus r} \oplus \text{tors}_2\; F$. 
\end{solution}

\end{document}