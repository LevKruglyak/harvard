\documentclass[11pt,letterpaper]{article}

\input{../../../../.config/latex/paper.tex}

\providecommand{\conf}{\mathrm{Conf}}

\title{Cohomology Rings of Configuration Spaces}
\author{Lev Kruglyak}
\date{May 2, 2023}

\begin{document}
\maketitle

In this paper, we prove the following theorem.
\begin{definition}
    Recall that a \emph{configuration space} of $\R^n$ is defined as:
    \[
        \conf_k(\R^n) = \{ (x_1,\ldots,x_k) : x_i\in \R^n, x_i\neq x_j \textrm{ for }i\neq j\}
    .\] 
\end{definition}
\begin{theorem}
    We have the following isomorphism of graded commutative algebras, where $1\leq a<b<c\leq k$:
    \[
        H^*(\conf_k(\R^n)) \cong \Bigg\langle \alpha_{ab}\;{\mid}\; \alpha_{ab}^2,\; \alpha_{ab}+(-1)^{n+1}\alpha_{ba},\;\alpha_{ab}\alpha_{bc}+\alpha_{bc}\alpha_{ca}+\alpha_{ca}\alpha_{ab} \Bigg\rangle
    .\] 
\end{theorem}

To prove this theorem, we'll place $\conf_k(\R^n)$ into a fiber sequence, then apply the Leray-Hisrch theorem.

\begin{theorem}[Leray-Hirsch]
    Let $\pi : E \to B$ be a fiber bundle with fiber $F$. Suppose that for each $t$, the abelian group $H^t(F)$ is free of finite rank. Assume that the restriction $H^*(E) \to H^*(F)$ is surjective. Because $H^t(F)$ is a free abelian group for each $t$, the surjection $H^*(E) \to H^*(F)$ admits a splitting; pick one, say $s : H^*(F) \to H^*(E)$. The projection map renders $H^*(E)$ a module over $H^*(B)$. The $H^*(B)$-linear extension of $s$,
    \[
        \overline{s} : H^*(B)\otimes H^*(F) \to H^*(E)
    \]
    is then an isomorphism of $H^*(B)$-modules. 
\end{theorem}

To apply this, we first need to build a fiber sequence around $\conf_k(\R^n)$. 

\begin{lemma}
    Let $p_k : \conf_k(\R^n) \to \conf_{k-1}(\R^n)$ be the map which sends $(x_1,\ldots,x_k)$ to $(x_1,\ldots,x_{k-1})$. Then $p_k$ is a fiber bundle, and the fiber at a point $(x_1,\ldots,x_{k-1})$ is the space $\R^n-\{x_1,\ldots,x_{k-1}\}$.
\end{lemma}

\begin{proof}
    We just need to prove the local triviality condition, i.e. given some point $(x_1,\ldots,x_{k-1})$, we need an open neighborhood $\cal{U}\subset \conf_{k-1}(\R^n)$, and an isomorphism $\varphi_{\cal{U}}$ such that the following diagram commutes:
    % https://q.uiver.app/?q=WzAsMyxbMCwwLCJcXHBpXnstMX0oXFxtYXRoY2Fse1V9KSJdLFswLDEsIlxcbWF0aGNhbHtVfSJdLFsxLDAsIlxcbWF0aGNhbHtVfVxcdGltZXMoXFxSXm4tXFx7eF8xLFxcbGRvdHMseF97ay0xfVxcfSkiXSxbMCwxLCJcXHBpIiwyXSxbMCwyLCJcXHZhcnBoaV97XFxtYXRoY2Fse1V9fSJdLFsyLDFdXQ==
    \[\begin{tikzcd}
        {p_k^{-1}(\mathcal{U})} & {\mathcal{U}\times(\R^n-\{x_1,\ldots,x_{k-1}\})} \\
        {\mathcal{U}}
        \arrow["p_k"', from=1-1, to=2-1]
        \arrow["{\varphi_{\mathcal{U}}}", from=1-1, to=1-2]
        \arrow[from=1-2, to=2-1]
    \end{tikzcd}\]
    Let $\epsilon>0$ be some real value such that $x_j\neq B_\epsilon(x_i)$ for $i\neq j$. Then, let $\mathcal{U}=\bigcup_i B_\epsilon(x_i)$, which is a subset of $\conf_{k-1}(\R^n)$. Note that $p_k^{-1}(\mathcal{U})$ is the set:
    \[
        p_k^{-1}(\mathcal{U}) = \{(x_1,\ldots,x_k) : \|x_i\|<\epsilon \textrm{ for }i\leq k-1\}
    .\]
    Now we can set $\varphi_{\mathcal{U}}$ to be the identity homeomorphism. This proves the local triviality condition. 
\end{proof}

Another condition which must be met to apply the Leray-Hirsch theorem is that $H^*(\conf_k(\R^n)) \to H^*(\R^n - \{x_1,\ldots,x_{k-1}\})$ is surjective. To do this, we establish some notation.

\begin{definition}
    Let $1\leq a<b\leq k$. The \emph{Gauss map} $\gamma_{ab} : \conf_k(\R^n) \to S^{n-1}$ is given by
    \[
        \gamma_{ab}(x_1,\ldots,x_k) = \frac{x_b - x_a}{\|x_b - x_a\|}
    .\] 
    Letting $\iota_{n-1}\in H^{n-1}(S^{n-1})$ be any choice of generator, set $\alpha_{ab}=\gamma^*_{ab}(\iota_{n-1})$ for some $1\leq a<b\leq k$.
\end{definition}

\begin{lemma}
    The restriction map $H^*(\conf_k(\R^n)) \to H^*(\R^n - \{x_1,\ldots,x_{k-1}\})$ is surjective.
\end{lemma}

\begin{proof}
    We prove this by showing that $H^*(\R^n - \{x_1,\ldots,x_{k-1}\})$ is generated by $\alpha_{ak}$. We will construct right inverse maps $\psi_i : S^{n-1} \to \R^n-\{x_1,\ldots,x_{k-1}\}$ to $\gamma_{ik}$. Since all $\R^n-\{x_1,\ldots,x_{k-1}\}$ are homotopic for distinct $x_i$, we can assume that $x_i = (f(i), 0,\ldots,0)$ for some real function with $f(0)=1$. Then let's set $\psi_i(v)=x_i+e^i v$ for any $v\in S^{n-1}$. Clearly 
    \[\gamma_{ik}\circ \psi_i(v)=\frac{x_i+f(i)v - x_i}{\|x_i+f(i)v - x_i\|} = v.\]
    Furthermore, note that $\psi_i : S^{n-1} \to \R^n -\{x_1,\ldots,x_{k-1}\}$ give us a homotopy equivalence between $\bigvee_{k-1}S^{n-1}$ and $\R^n-\{x_1,\ldots,x_{k-1}\}$. Thus, it follows that $H^*(\R^n-\{x_1,\ldots,x_{k-1}\})$ is generated by $\gamma^*_{ik}(\alpha)=\alpha_{ik}$. Thus the map is surjective.
\end{proof}

\begin{lemma}
    For some $k\geq 0$, let $B_k$ be the set of classes:
    \[
        B_k = \bigsqcup_{m\geq 0}\left\{\prod^m_{i\geq 1} \alpha_{a_ib_i}\,:\, 1\leq b_1<\cdots<b_m\leq k, \alpha_i<b_i \right\}
    .\] 
    Then as a $\Z$-module, $H^*(\conf_k(\R^n))\cong \Z\big\langle B_k \big\rangle$.
\end{lemma}

\begin{proof}
    This is proven by induction on $k$, via repeated applications of the Leray-Hirsch theorem. The base case of $k=1$ is trivial, since $\conf_1(\R^n) = \R^n$. Suppose the claim is true for $k-1$. By the Leray-Hirsch theorem we get an isomorphism of $\Z$-modules:
    \[
        \begin{aligned}
            H^*(\conf_k(\R^n)) &\cong H^*(\R^n-\{x_1,\ldots,x_{k-1}\})\otimes H^*(\conf_{k-1}(\R^n)) \\
            &\cong \Z\big\langle 1, \alpha_{ak}\,:\, 1\leq a\leq k-1 \big\rangle\otimes \Z\big\langle B_{k-1} \big\rangle.
        \end{aligned}
    \]
    This completes the proof, since appending $\alpha_{ak}$ to $B_{k-1}$ gives us $B_k$.
\end{proof}

Having understood the additive structure of $H^*(\conf_k(\R^n))$, let's now look at the multiplicative structure. We can first prove some basic relations.

\begin{lemma}
    $\alpha_{ab}=(-1)^n\alpha_{ba}$.
\end{lemma}

\begin{proof}
    Observe that we have a commutative diagram, where $-1$ is the antipodal map.
    % https://q.uiver.app/?q=WzAsNixbMCwxLCJTXntuLTF9Il0sWzQsMSwiXFxaW3hfe24tMX1dIl0sWzAsMCwiXFxjb25mX2soXFxSXm4pIl0sWzEsMSwiU157bi0xfSJdLFszLDAsIkheKihcXGNvbmZfayhcXFJebikpIl0sWzMsMSwiXFxaW3hfe24tMX1dIl0sWzIsMCwiXFxnYW1tYV97YWJ9IiwyXSxbMCwzLCItMSIsMl0sWzIsMywiXFxnYW1tYV97YmF9Il0sWzEsNSwiKC0xKV5uIl0sWzUsNCwiXFxnYW1tYV97YWJ9XioiXSxbMSw0LCJcXGdhbW1hX3tiYX1eKiIsMl1d
    \[\begin{tikzcd}
        {\conf_k(\R^n)} &&& {H^*(\conf_k(\R^n))} \\
        {S^{n-1}} & {S^{n-1}} && {\Z[x_{n-1}]} & {\Z[x_{n-1}]}
        \arrow["{\gamma_{ab}}"', from=1-1, to=2-1]
        \arrow["{-1}"', from=2-1, to=2-2]
        \arrow["{\gamma_{ba}}", from=1-1, to=2-2]
        \arrow["{(-1)^n}", from=2-5, to=2-4]
        \arrow["{\gamma_{ab}^*}", from=2-4, to=1-4]
        \arrow["{\gamma_{ba}^*}"', from=2-5, to=1-4]
    \end{tikzcd}\]
    The second, induced diagram follows because the antipodal map on $S^{n-1}$ has degree $(-1)^n$. Since $\gamma_{ab}^* = (-1)^n \alpha_{ba}^*$, it follows that $\gamma_{ab}^*(\iota_{n-1}) = (-1)^n \alpha_{ba}^*(\iota_{n-1})$ so we get our relation.
\end{proof}

\begin{lemma}
    $\alpha^2_{ab}=0$.
\end{lemma}

\begin{proof}
    This follows because:
    \[
        \begin{aligned}
            \alpha_{ab}^2 &= \gamma_{ab}^*(\iota_{n-1})\smile \gamma_{ab}^*(\iota_{n-1})\\
            &=\gamma_{ab}^*(\iota_{n-1}\smile \iota_{n-1})\\
            &=0.
        \end{aligned}
    \] 
    The final equality is zero because $H^*(S^{n-1})\cong \Z[x_n]/(x_n^2)$.
\end{proof}

Notice that there is a right action of the symmetric group $\Sigma_k$ on $\conf_k(\R^n)$ given by $(x_1,\ldots,x_k)\cdot \sigma = (x_{\sigma(1)},\ldots,x_{\sigma(k)})$. This gives us the following relation.

\begin{lemma}
    For any $\sigma\in \Sigma_k$, we have $\sigma^*\alpha_{ab} = \alpha_{\sigma(a)\sigma(b)}$.
\end{lemma}

\begin{proof}
    For this case, we get a similar diagram:
    % https://q.uiver.app/?q=WzAsNixbMCwwLCJcXGNvbmZfayhcXFJebikiXSxbNCwwLCJIXiooXFxjb25mX2soXFxSXm4pKSJdLFswLDEsIlNee24tMX0iXSxbMSwwLCJcXGNvbmZfayhcXFJebikiXSxbMywwLCJIXiooXFxjb25mX2soXFxSXm4pKSJdLFszLDEsIlxcWlt4X3tuLTF9XSJdLFswLDIsIlxcZ2FtbWFfe1xcc2lnbWEoYSlcXHNpZ21hKGIpfSIsMl0sWzAsMywiXFxzaWdtYSJdLFszLDIsIlxcZ2FtbWFfe2FifSJdLFsxLDQsIlxcc2lnbWFeKiIsMl0sWzUsNCwiXFxnYW1tYV97XFxzaWdtYShhKVxcc2lnbWEoYil9XioiXSxbNSwxLCJcXGdhbW1hX3thYn1eKiIsMl1d
    \[\begin{tikzcd}
        {\conf_k(\R^n)} & {\conf_k(\R^n)} & {H^*(\conf_k(\R^n))} & {H^*(\conf_k(\R^n))} \\
        {S^{n-1}} && {\Z[x_{n-1}]}
        \arrow["{\gamma_{\sigma(a)\sigma(b)}}"', from=1-1, to=2-1]
        \arrow["\sigma", from=1-1, to=1-2]
        \arrow["{\gamma_{ab}}", from=1-2, to=2-1]
        \arrow["{\sigma^*}"', from=1-4, to=1-3]
        \arrow["{\gamma_{\sigma(a)\sigma(b)}^*}", from=2-3, to=1-3]
        \arrow["{\gamma_{ab}^*}"', from=2-3, to=1-4]
    \end{tikzcd}\]
    Thus we get the desired relation.
\end{proof}

Next we prove the Arnold relation.

\begin{lemma}
    For $1\leq a<b<c\leq k$, we have $\alpha_{ab}\alpha_{bc}+\alpha_{bc}\alpha_{ca}+\alpha_{ca}\alpha_{ab}=0$.
\end{lemma}

\begin{proof}
    Recall that $H^{2n-2}(\conf_3(\R^n))$ is free, with basis $\{\alpha_{12}\alpha_{23}, \alpha_{23}\alpha_{31}\}$. Applying the relations, we also have a basis $\{\alpha_{12}\alpha_{23}, \alpha_{31}\alpha_{12}\}$. Thus, there must be some linear dependence
    \[
        x\cdot \alpha_{12}\alpha_{23}+y\cdot \alpha_{23}\alpha_{31}+z\cdot \alpha_{31}\alpha_{12} = 0 
    .\] 
    To solve for these coefficients, we apply the previous lemma. Let $\tau_{12}\in \Sigma_3$ be the transposition switching $1$ and $2$. Applying it to the linear dependence, and using the commutativity rules gives us:
    \[
        \begin{aligned}
            x\cdot \alpha_{21}\alpha_{13} + y\cdot \alpha_{13}\alpha_{32}+z\cdot \alpha_{32}\alpha_{21}&=0\\
            (-1)^nx\cdot \alpha_{12}\alpha_{13} + (-1)^ny\cdot \alpha_{13}\alpha_{23}+(-1)^{2n}z\cdot \alpha_{12}\alpha_{23}&=0
        \end{aligned}
    \] 
    We do a similar thing with $\tau_{23}$, to get another dependence. It is straightforward to check that solving the resulting system of equations gives us $x=y=z$, so by additive freeness of $H^*(\conf_3(\R^n))$ we can cancel $x$ and get our desired relation in $H^*(\conf_3(\R^n))$:
    \[
        \alpha_{ab}\alpha_{bc}+\alpha_{bc}\alpha_{ca}+\alpha_{ca}\alpha_{ab}=0
    .\]
    Now in the general case $k> 3$, given some $a,b,c$, we have a map $\varphi_{abc} : \conf_k(\R^n) \to \conf_3(\R^n)$ which sends $(x_1,\ldots,x_k)$ to $(x_a,x_b,x_c)$. Notice that $\varphi_{abc}$ induces an injective map of cohomology rings $H^*(\conf_3(\R^n)) \to H^*(\conf_k(\R^n))$. Thus, the Arnold relation lifts to the cases $k>3$.
\end{proof}

Finally, we can prove the desired theorem: the commutativity relation, the nilpotency relation, and the Arnold relation completely describe the multiplicative structure of $H^*(\conf_k(\R^n))$. Specifically, the quotient of the free graded commutative algebra on the generators $\{\alpha_{ab}\}_{1\leq a\neq  b\leq k}$ by the relations. Given some element of this free graded commutative algebra, using the nilpotency and commutativity relation, we can rewrite such an element in the form
$\alpha_{a_1b_1}\cdots\alpha_{a_mb_m}$, where $a_i<b_i$, and $1\leq b_1\leq \cdots b_m\leq k$. We can then repeatedly use the Arnold relation to force $b_1<\cdots<b_m$. Specifically, if $b_\ell=b_{\ell-1}$, we have
\[
    \begin{aligned}
        \alpha_{a_1b_1}\cdots \alpha_{a_{\ell-1}b_{\ell-1}}\alpha_{a_\ell b_\ell}\cdots \alpha_{a_m b_m} \\
        = (-1)^* \alpha_{a_1b_1}\cdots\left( \alpha_{a_\ell a_{\ell-1}}\alpha_{a_{\ell -1}b_{\ell -1} + \alpha_{b_{\ell}a_\ell}\alpha_{a_\ell a_{\ell - 1}}}\right)\cdots \alpha_{a_m b_m}   
    \end{aligned}
\] 
By repeated induction, we can thus force this element into a linear combination of elements in $B_k$, the additive spanning basis for $H^*(\conf_k(\R^n))$.



Recall that the additive basis for $H^*(\conf_k(\R^n))$ was given by $B_k$, and consists of elements of the form $\alpha_{a_1b_1}\cdots \alpha_{a_mb_m}$.

\end{document}