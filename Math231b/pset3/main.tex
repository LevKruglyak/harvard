\documentclass[11pt,letterpaper]{article}

\input{../../../../.config/latex/preamble_v1.tex}
\def\H{\mathcal{H}}
\def\L{\mathcal{L}}
\def\S{\mathcal{S}}
\def\M{\mathcal{M}}
\def\E{\mathbb{E}}
\def\Re{\mathrm{Re}}
\def\Im{\mathrm{Im}}
\def\id{\mathrm{id}}
\def\CC{\mathcal{C}}
\def\DD{\mathcal{D}}
\def\Eq{\mathrm{Eq}}
\def\ceq{\vcentcolon=}

\lightmode

\title{\textbf{Math 231b Problem Set 3}}
\date{\textbf{Due:} February 21, 2023}

\begin{document}
\maketitle

\begin{problem}
    Closure properties of cofibrations.
\end{problem}

\begin{solution}
    Prove that cofibrations are closed uner the following operations:
    \begin{partproblem}{a}
        Products with a space $Y$: if $A\to X$ is a cofibration, then the induced map $A\times Y \to X\times Y$ is as well. 
    \end{partproblem}

    \quad If we had a homotopy extension diagram for $A\times Y\to X\times Y$, we could use the cartesian closed adjunction to get a homotopy extension diagram for $A \to Y$:

    \[\begin{tikzcd}[ampersand replacement=\&]
        {A\times Y} \& {X\times Y} \\
        {(A\times Y)\times I} \& {(X\times Y)\times I} \\
        \&\& Z
        \arrow[from=1-1, to=1-2]
        \arrow[from=1-1, to=2-1]
        \arrow[from=2-1, to=2-2]
        \arrow[from=1-2, to=2-2]
        \arrow[curve={height=-18pt}, from=1-2, to=3-3]
        \arrow[curve={height=18pt}, from=2-1, to=3-3]
    \end{tikzcd}\implies\quad  \begin{tikzcd}[ampersand replacement=\&]
        A \& X \\
        {A\times I} \& {X\times I} \\
        \&\& {Z^Y}
        \arrow[from=1-1, to=1-2]
        \arrow[from=2-1, to=2-2]
        \arrow[from=1-2, to=2-2]
        \arrow[curve={height=-18pt}, from=1-2, to=3-3]
        \arrow[curve={height=18pt}, from=2-1, to=3-3]
        \arrow[from=1-1, to=2-1]
        \arrow["h", dashed, from=2-2, to=3-3]
    \end{tikzcd}\] 
    This second diagram would have a homotopy extension since $A \to X$ is a cofibration, so we could reverse the adjunction to get a homotopy extension for the original diagram.

    \begin{partproblem}{b}
        Compositions: if $A\to B$ and $B\to X$ are cofibrations, then so is $A\to B\to X$. 
    \end{partproblem}

    \quad To verify the homotopy extension property, suppose we're given a square:
    \[\begin{tikzcd}[ampersand replacement=\&]
        A \& B \& X \\
        {A\times I} \& {B\times I} \& {X\times I} \\
        \&\&\& Z
        \arrow[from=1-1, to=1-2]
        \arrow[from=1-2, to=1-3]
        \arrow[from=1-1, to=2-1]
        \arrow[from=1-2, to=2-2]
        \arrow[from=1-3, to=2-3]
        \arrow[from=2-1, to=2-2]
        \arrow[from=2-2, to=2-3]
        \arrow[curve={height=18pt}, from=2-1, to=3-4]
        \arrow[curve={height=-18pt}, from=1-3, to=3-4]
    \end{tikzcd}\]
    By precomposing $X\to Z$ with the map $B\to X$, we get a map $B\to Z$, so since the first square is a cofibration square, we get an extension $B\times I \to Z$. Then since the second square is also a cofibration, we get an extension $X\times I \to Z$ as desired.

    \begin{partproblem}{c}
        Countably transfinite compositions: if $X_i \to X_{i+1}$ are cofibrations for integers $i\geq 0$, then $X_0 \to \varinjlim_n X_n$ is a cofibration.
    \end{partproblem}

    \quad Given a maps $X_0\times I \to Z$ and $\varinjlim_n X_n \to Z$, we get maps $X_n \to Z$ and $X_n\times I \to Z$ which commute with the rest of the diagram. Thus we get a homotopy extension $X_n\times I \to Z$ given a homotopy extension $X_{n-1}\times I \to Z$. By induction, we get a compatible family of such extensions, so by the universal property we get a compatible extension $\varinjlim_n (X_n\times I) \to Z$. This limit is equal to $\varinjlim_n X_n \times I$ so we have our desired extension.    
\end{solution}

\begin{problem}
    Construct the \emph{homotopy pushout} of a triple $B \leftarrow A\to X$:
    % https://q.uiver.app/?q=WzAsMyxbMCwwLCJBIl0sWzEsMCwiWCJdLFswLDEsIkIiXSxbMCwyXSxbMCwxXV0=
\[\begin{tikzcd}[ampersand replacement=\&]
	A \& X \\
	B
	\arrow[from=1-1, to=2-1]
	\arrow[from=1-1, to=1-2]
\end{tikzcd}\quad\implies\quad \begin{tikzcd}[ampersand replacement=\&]
	A \& X \\
	B \& Y
	\arrow[from=1-1, to=2-1]
	\arrow[from=1-1, to=1-2]
	\arrow[from=2-1, to=2-2]
	\arrow[from=1-2, to=2-2]
	\arrow[Rightarrow, from=1-2, to=2-1]
\end{tikzcd}\]
\end{problem}

\begin{solution}
    \quad Suppose we're given maps $f : A\to X$ and $g: A \to B$. We need to construct a space $Y$ with maps $i_X : X \to Y$ and $i_Y : B \to Y$ as well as a homotopy $h : A\times I \to Y$ between $i_X \circ f$ and $i_Y \circ g$. We can thus construct $Y$ as the colimit of the following diagram: 

    \[\begin{tikzcd}[ampersand replacement=\&]
        \& Y \\
        B \& {A\times I} \& X \\
        \& A
        \arrow["f"', from=3-2, to=2-3]
        \arrow["g", from=3-2, to=2-1]
        \arrow["{i_1}", shift left=1, from=3-2, to=2-2]
        \arrow["{i_0}"', shift right=1, from=3-2, to=2-2]
        \arrow["{i_B}", from=2-1, to=1-2]
        \arrow["{i_X}"', from=2-3, to=1-2]
        \arrow["h", from=2-2, to=1-2]
    \end{tikzcd}\]
    Such a $Y$ can be explicitly given by the quotient:
    \[
        Y = \frac{X\sqcup B\sqcup (A\times I)}{f(a)\sim (a,0),\; g(a)\sim(a,1)}
    .\] 
    We can denote this space by $X\cup^h_A B = Y$. 

    \begin{partproblem}{a}
        Extend this construction to work for pointed spaces, and explain how the homotopy cofiber can be viewed as a homotopy pushout.
    \end{partproblem}
    \quad In the category of pointed spaces, we need to replace the disjoint union by wedge product, and the interval product with a smash product. Thus our homotopy pushout is the space
    \[
        Y = \frac{X\vee B\vee (A\wedge I_+)}{f(a)\sim (a,0),\; g(a)\sim(a,1)}
    \]
    and it's easy to check that it satisfies all the required properties as before. We can denote this space by $X\cup^{h_*}_A B=Y$. Now notice that if we have a map $f : A \to X$, the pointed homotopy pushout $X\cup_A^{h_*} * = X\wedge CA / f(a) \sim (a,0) = C(f)$, so the homotopy cofiber is an example of a homotopy pushout. 

    \begin{partproblem}{b}
        Give an example of when the homotopy pushout is not a pushout in the homotopy category.
    \end{partproblem}

    \quad For any space $B$, consider the homotopy pushout $*\cup^h_B *$. By construction, this homotopy pushout is clearly $\Sigma B$, yet the ordinary pushout is just a point, i.e. always contractible in the homotopy category. Letting $B$ be some non-contractible space such as $S^n$, it's clear that the homotopy pushout is $\Sigma S^n = S^{n+1}$ which isn't contractible whereas its ordinary pushout is.
\end{solution}

\begin{problem}
    Let $\square^n = (0,1)^n$ denote the open unit cube of dimension $n$. A map $f : \square^n \hookrightarrow \square^n$ is said to be a \emph{rectilinear embedding} if it is of the form
    \[
        f(x_1,\ldots,x_n) = (a_1x_1 + b_1, \ldots, a_n x_n + b_n)
    \]
    for some real numbers $a_i > 0$ and $b_i$. A map $\bigsqcup^k_{i=1} \square^n \to \square^n$ is said to be a \emph{rectilinear embedding} if it is an open embedding and its restriction to each $\square^n$ is in the domain is a rectilinear embedding.
    
    \medskip
    \quad We let $C_k(n)$ denote the space of rectilinear embeddings of $k$ disjoint $n$-cubes $\bigsqcup_{i=1}^k \square^n \to \square^n$, topologized as an open subspace of $(\R^{2n})^k$. There is a continuous map $C_k(n)\times (\Omega^n X)^k \to \Omega^n X$ which allows us to view $C_k(n)$ as a space parametrizing $k$-ary products on $n$-fold loop spaces.
\end{problem}

\begin{solution}
    \quad On the other hand, let $\text{Conf}_k(\R^n)$ denote the space of $k$ disjoint ordered points in $\R^n$, viewed as an open subspace of $(\R^n)^k$.

    \begin{partproblem}{a}
        Prove that there is a homotopy equivalence $C_k(n)\simeq \text{Conf}_k(\R^n)$.
    \end{partproblem}

    \quad Firstly, to make the maps nicer, lets rescale the cube by using the interval $(-1,1)$ instead of $(0,1)$. In the forward direction we have a simple map $h_0 : C_k(n) \to \text{Conf}_k(\R^n)$ which sends a map $f : \bigsqcup_{i=1}^k \square^n \to \square^n$ to the set of points $(\restr{f}{1}(0), \ldots, \restr{f}{k}(0))$ where $\restr{f}{i}$ is the restriction of the map onto the $i$-th cube. This is well defined since the map is required to be an open embedding so the resulting points will be disjoint. In the opposite direction $h_1 : \text{Conf}_k(\R^n) \to C_k(n)$, let's say we have a set of disjoint points $v=(v_1,\ldots,v_k)$. Let $$\epsilon_v = \min_{1\leq i< j\leq k} |v_i - v_j|$$ be the minimum distance between any two points in the set. Then for any set of points $v=(v_1, \ldots, v_k)$, and $s_i\in \square^n$ a point in the $i$-th square in $\bigsqcup^k_{i=1}\square^n$ we can set
    \[
        h_1(v)(s_i) = \left(\frac{\epsilon_v}{2}\cdot s_{i,1} + v_{i,1}, \ldots, \frac{\epsilon_v}{2}\cdot s_{i,k} + v_{i,k}\right) = \frac{\epsilon_v}{2}\cdot s_i + v_i
    .\]
    This is clearly continuous since $\epsilon_v : \text{Conf}_k(\R^n) \to \R$ is a continuous, and rectilinear by construction. Now notice that $h_0\circ h_1 = 1_{\text{Conf}_k(\R^n)}$, so we only need to check that $h_1\circ\ h_0 \simeq 1_{\text{Conf}_k(\R^n)}$. For some rectilinear map $f : \bigsqcup^k_{i=1}\square^n \to \square^n$, let $a_i=(a_{i,1},\ldots, a_{i,k})$ and $b_i = (b_{i,1},\ldots,v_{i,k})$ be the point sets representing each rectilinear restriction. Define the homotopy $H : C_k(n)\times I \to C_k(n)$ as
    \[
        H(f,t) = \left( s_i \mapsto \left(a_it+\frac{\epsilon_{b_i}}{2}(1-t)\right) \cdot s_i + b_i\right)
    \]    
    as before, here $s_i$ represents some element in the $i$-th cube of the disjoint union. Clearly $H(f,0)=f$ and $H(f,1)=h_1\circ h_0$, so we are done. 

    \begin{partproblem}{b}
        Prove that there is a homotopy equivalence $\text{Conf}_2(\R^n)\simeq S^{n-1}$.
    \end{partproblem}
    \quad First we claim that there is a homotopy equivalence $\text{Conf}_2(\R^n)\simeq S^{n-1}\times \R^n$. There's a clear map $h_1 : S^{n-1}\times \R^n \to \text{Conf}_2(\R^n)$ given by $h_1(s,x) = (s+x, x)$. For the other direction $h_0 : \text{Conf}_2(\R^n) \to S^{n-1}\times \R^n$ is given by
    \[
        h_0(x,y) = \left(\frac{x-y}{\|x-y\|}, y\right)
    .\]
    Now notice that $h_0 \circ h_1 = 1_{S^{n-1}\times \R^n}$. Conversely, $h_1\circ h_0$ is given by 
    \[
        (h_1\circ h_0)(x,y) = \left(\frac{x-y}{\|x-y\|}+y, y\right)
    .\]
    This can be homotoped to the identity by a simple linear interpolation:
    \[
        H(x,y,t) = \left(\left(\frac{x-y}{\|x-y\|}+y\right)t + (1-t)x, y\right)
    .\] 
    \quad Thus $\text{Conf}_2(\R^n)\simeq S^{n-1}\times \R^n$. Since $S^{n-1}\times \R^n\simeq S^{n-1}$, we thus have our desired equivalence. 
\end{solution}

\begin{problem}
    An \emph{$H$-space} is a pointed space $(X, *)$ equipped with a multiplication map $m : X \times X \to X$ which is homotopy unital in the sense that the diagram
    % https://q.uiver.app/?q=WzAsNCxbMSwwLCJYXFx0aW1lcyBYIl0sWzAsMCwiWFxcY29uZypcXHRpbWVzIFgiXSxbMiwwLCJYXFx0aW1lcyAqXFxjb25nIFgiXSxbMSwxLCJYIl0sWzEsMF0sWzIsMF0sWzEsMywiMV9YIiwyXSxbMiwzLCIxX1giXSxbMCwzLCJtIl1d
    \[\begin{tikzcd}[ampersand replacement=\&]
        {X\cong*\times X} \& {X\times X} \& {X\times *\cong X} \\
        \& X
        \arrow[from=1-1, to=1-2]
        \arrow[from=1-3, to=1-2]
        \arrow["{1_X}"', from=1-1, to=2-2]
        \arrow["{1_X}", from=1-3, to=2-2]
        \arrow["m", from=1-2, to=2-2]
    \end{tikzcd}\]
    commutes up to homotopy. Prove that if $X$ is an $H$-space, then $\pi_1(X,*)$ is an abelian group. Furthermore, prove that $\pi_1(X,*)\times \pi_1(X, *) \to \pi_1(X,*)$ induced by $m$ is equal to the group operation.
    % pure algebra since m induces a unital group homomorphism
\end{problem}

\begin{solution}
    \quad Let $f,g : I \to X$ be paths in $X$ based at $*$, and let $f\cdot g$ denote the path $(f\cdot g)(t) = m(f(t), g(t))$. This is well defined on on homotopy, since if $H_f : f\simeq f'$ and $H_g : g\simeq g'$ are homotopies then $H_f\cdot H_g : f\cdot g \simeq f'\cdot g'$. Thus $\cdot$ is a well defined operation on $\pi_1(X,*)$. The identity with respect to this operation is the constant map $c_*$, this follows by the homotopy unitality of the multiplication map. Notice that path composition and path multiplication have the same identity element.

    \quad First we'll show that path composition and path multiplication are actually the same operation:
    \[
        f\cdot g = (f * c_*) \cdot (c_* * g) = \begin{cases}
            f(2t)\cdot * & 0\leq t\leq \frac{1}{2}\\
            * \cdot g(2t-1) & \frac{1}{2}\leq t\leq 1\\
        \end{cases} = \begin{cases}
            f(2t) & 0\leq t\leq \frac{1}{2}\\
            g(2t-1) & \frac{1}{2}\leq t\leq 1\\
        \end{cases} = f * g
    .\] 
    Then by a similar argument we get commutativity:
    \[
        f * g = f\cdot g = (c_* * f)\cdot (g * c_*) = (c_* \cdot g) * (f \cdot c_*) = g * f
    .\]
\end{solution}

\end{document}