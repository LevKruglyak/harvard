\documentclass[11pt,letterpaper]{article}

\input{../../../../.config/latex/preamble_v1.tex}
\def\H{\mathcal{H}}
\def\L{\mathcal{L}}
\def\S{\mathcal{S}}
\def\M{\mathcal{M}}
\def\E{\mathbb{E}}
\def\Re{\mathrm{Re}}
\def\Im{\mathrm{Im}}
\def\id{\mathrm{id}}
\def\CC{\mathcal{C}}
\def\DD{\mathcal{D}}
\def\Eq{\mathrm{Eq}}
\def\Ext{\mathrm{Ext}}
\def\Tor{\mathrm{Tor}}
\def\ceq{\vcentcolon=}

\lightmode

\title{\textbf{Math 231b Problem Set 8}}
\date{\textbf{Due:} April 11, 2023}

\begin{document}
\maketitle

\begin{problem}
    Homology with local coefficients.
\end{problem}

\begin{solution}
    Let $X$ denote a path-connected and semilocally simply-connected space, and let $\widetilde{X} \to X$ denote its universal cover.
    \begin{partproblem}{a}
        Prove that $S_*(\widetilde{X}; R)$ is a complex of free $R[\pi_1(X)]$-modules, where $\pi_1(X)$ acts via deck transformations on $\widetilde{X}$. % basis is given by choosing a lift $\Delta^n \to \widetilde{X} for each Delta^n \to X
    \end{partproblem}

    \quad Recall that $S_n(\widetilde{X}; R) = R\textrm{Sin}_n(\widetilde{X})$ is a free $R$-module. Now recall that we also have an action of $\pi_1(X)$ on $\textrm{Sin}_n(\widetilde{X})$, and the induced map $\textrm{Sin}_n(\widetilde{X}) \to \textrm{Sin}_n(X)$ has fibers exactly the orbits of this action. Similarly, $R[\pi_1(X)]$ acts on $S_n(\widetilde{X}; R)$, and the orbits of this action are the fibers of the map $S_n(\widetilde{X}; R) \to S_n(X; R)$. Notice that since $\Delta^n$ is simply connected, we can lift any $\sigma : \Delta^n \to X$ to some $\widetilde{\sigma} : \Delta^n \to \widetilde{X}$. Then $\widetilde{\sigma}\cdot R[\pi_1(X)]$ is exactly the fiber of $R\sigma$ so we have the direct sum decomposition:
    \[
        S_n(\widetilde{X}; R) = \bigoplus_{\sigma \in \textrm{Sin}_n(X)} \widetilde{\sigma}\cdot R[\pi_1(X)]
    \]
    and thus it is a free $R[\pi_1(X)]$-module. The boundary maps are clearly seen to be $R[\pi_1(X)]$-module homomorphisms, since the inclusion of faces doesn't affect the choice of lifting. Thus $S_*(\widetilde{X}; R)$ is a complex of $R[\pi_1(X)]$-modules.

    \begin{partproblem}{b}
        In the setting of (a), prove that a short exact seuqnce of $R[\pi_1(X)]$-modules $0 \to M_1 \to M_2 \to M_3 \to 0$ gives rise to a long exact sequence:
        \[\begin{tikzcd}[column sep = scriptsize]
            \cdots & H_{n+1}(X; M_3) & H_n(X; M_1) & H_n(X; M_2) & H_n(X; M_3) & H_{n-1}(X; M_1) & \cdots
            \arrow[from=1-1, to=1-2]
            \arrow[from=1-2, to=1-3]
            \arrow[from=1-3, to=1-4]
            \arrow[from=1-4, to=1-5]
            \arrow[from=1-5, to=1-6]
            \arrow[from=1-6, to=1-7]
        \end{tikzcd}\]
    \end{partproblem}
    \quad Recall that tensoring with free modules is exact. Thus, we have an SES of chain complexes of $R[\pi_1(X)]$-modules:
    \[\begin{tikzcd}
        0 & {S_*(\widetilde{X}; R)\otimes_{R[\pi_1(X)]} M_1} & {S_*(\widetilde{X}; R)\otimes_{R[\pi_1(X)]} M_2} & {S_*(\widetilde{X}; R)\otimes_{R[\pi_1(X)]} M_3} & 0
        \arrow[from=1-1, to=1-2]
        \arrow[from=1-2, to=1-3]
        \arrow[from=1-3, to=1-4]
        \arrow[from=1-4, to=1-5]
    \end{tikzcd}\]
    This leads to a LES in homology, and the homology of these chain complexes are exactly $H_*(X; M_*)$ by construction.

    \begin{partproblem}{c}
        Prove that $H_*(K(G,1); M)\cong \Tor^{R[G]}_*(R, M)$ by noting that $S_*(\widetilde{K(G,1)}; R)$ is a resolution of $R$ by free $R[G]$-modules. This is usually called the \emph{group homology} of $G$ with cofficients in $M$ and is denoted $H_*(G; M)$.
    \end{partproblem}

    \quad Consider the canonical map $S_0(\widetilde{K(G,1)}; R) \to R$ which sends any $r\cdot \sigma$ to $r$. The kernel of this map is generated by $r\cdot (a - b)$, where $a,b : \Delta^0 \to \widetilde{K(G,1)}$. This is exactly the image of the differential $S_1(\widetilde{K(G,1)}; R) \to S_0(\widetilde{K(G,1)}; R)$, so $S_*(\widetilde{K(G,1)}; R)$ is naturally a resolution of $R$ by free $R[G]$-modules. Then, by construction of $\Tor$ and homology with local coefficients, we have:
    \[
        H_*(K(G,1);M) =H_*(S_*(\widetilde{K(G,1)}; R)\otimes_{R[G]} M) = \Tor_*^{R[G]}(R, M)
    .\] 
\end{solution}

\begin{problem}
    Let $\Z(-1)$ denote the $\Z[C_2]$-module on which the generator of $C_2$ acts by $-1$. Compute $H_*(\RP^n; \Z(-1))$.
    % Hint: there is SES 0 \to \Z \to \Z[C_2] \to \Z(-1) \to 0, Remark 62.4
\end{problem}

\begin{solution}
\quad Recall that the homology with local coefficients was defined in terms of the universal cover, as:
\[
    H_*(\RP^n; \Z(-1)) = H_*(S_*(S^n)\otimes_{\Z[C_2]} \Z(-1))
.\]
We can replace $S_*(-)$ here by cellular chains $C_*(-)$ without affecting the isomorphism class of the homology, but we need a choice of CW structure on $S^n$. Consider the cellular decomposition of $S^n$ with two cells attached in each dimension in a way that respects the antipodal map. More explicitly, we start with two $0$-cells $e^0_+, e^0_-$ on antipodal points of the sphere. Next we add two $1$-cells $e^1_+$ and $e^1_-$ as arcs on a great circle, with $de^1_+ = e^0_+ - e^0_-$ and $de^1_- = e^0_- - e^0_+$. We keep building up $S^n$, alternating the signs in each dimension so the antipodality is preserved. Notice then that the action of the only nontrivial deck automorphism of the covering $S^n \to \RP^n$ simply transposes these two cells. Thus, as a chain complex of $\Z[C_2]$-modules, $C_*(S^n)$ looks like:
\[\begin{tikzcd}[column sep=large]
	0 & {\Z[C_2]} & {\Z[C_2]} & \cdots & {\Z[C_2]} & {\Z[C_2]} & {\Z[C_2]} & 0
	\arrow[from=1-7, to=1-8]
	\arrow["{\times(1-\tau)}", from=1-6, to=1-7]
	\arrow["{\times(1+\tau)}", from=1-5, to=1-6]
	\arrow[from=1-4, to=1-5]
	\arrow[from=1-1, to=1-2]
	\arrow[from=1-3, to=1-4]
	\arrow["{\times(1+(-1)^n\tau)}", from=1-2, to=1-3]
\end{tikzcd}\]
where $\tau$ is the generator of $C_2$. When we tensor this with $\Z(-1)$ over $\Z[C_2]$, we get the chain complex:
\[\begin{tikzcd}[column sep=large]
	0 & \Z & \Z & \cdots & \Z & \Z & \Z & 0
	\arrow[from=1-7, to=1-8]
	\arrow["{\times 2}", from=1-6, to=1-7]
	\arrow["0", from=1-5, to=1-6]
	\arrow["{\times 2}", from=1-4, to=1-5]
	\arrow[from=1-1, to=1-2]
	\arrow[from=1-3, to=1-4]
	\arrow["{\times(1-(-1)^n)}", from=1-2, to=1-3]
\end{tikzcd}\]
From this, we get our desired homology:
\[
    H_k(\RP^n; \Z(-1)) = \begin{cases}
        \Z /2\Z & k < n, k\textrm{ even},\\
        0 & \textrm{otherwise}.
    \end{cases}
\] 
\end{solution}

\begin{problem}
    Using the fibrations $U(n-1) \to U(n) \to U(n) /U(n-1) \cong S^{2n-1}$, prove by induction on $n$ that $H^*(U(n); \Z)\cong \Z[x_1, x_3, \ldots, x_{2n-1}] /(x_1^2, x_3^2, \ldots, x_{2n-1}^2)$.
\end{problem}

\begin{solution}
    \quad Let's begin with the base case of $U(2)$. Recall that $U(1)\simeq S^1$, so we have a fibration $S^1 \to U(2) \to S^3$. Using the cohomological Serre spectral sequence, we have the $E_2$-page:
    \[\begin{tikzcd}[row sep=tiny]
        {E_2^{3,0}=\Z x_3} & {\Z x_1x_3} \\
        0 & 0 \\
        0 & 0 \\
        \Z & {E_2^{0,1}=\Z x_1}
    \end{tikzcd}\]
    Here note that if we let $x_1$ and $x_3$ be generators of $E_2^{0,1}$ and $E_2^{3,0}$ respectively, we get $x_1x_3$ as a generator of $E_2^{3,1}$. Notice that $x_1^2=0$ and $x_3^2=0$ follow by a simple degree check. Since there are no non-trivial differential maps, the spectral sequence collapses at $E_2$, so we get the ring structure $H^*(U(2); \Z) = \Z[x_1, x_3] / (x_1^2, x_3^2)$. Now suppose by induction that we had the desired ring structure on $H^*(U(n-1); \Z)$. The fibration $U(n-1)\to U(n) \to S^{2n-1}$ gives us the $E_2$ page of a spectral sequence:
    \[\begin{tikzcd}[column sep=scriptsize, row sep = scriptsize]
        s \\
        {} & \textcolor{rgb,255:red,214;green,92;blue,92}{\Z x_{2n-1}} & {\Z x_1x_{2n-1}} & 0 & {\Z x_3x_{2n-1}} & {\Z x_1x_3x_{2n-1}} & {\Z x_5x_{2n-1}} & {\Z x_1x_5x_{2n-1}} \\
        & 0 & 0 & 0 & 0 & 0 & 0 & 0 \\
        & \vdots & \vdots & \vdots & \vdots & \vdots & \vdots & \vdots \\
        & 0 & 0 & 0 & 0 & 0 & 0 & 0 \\
        & \Z & \textcolor{rgb,255:red,214;green,92;blue,92}{\Z x_1} & 0 & \textcolor{rgb,255:red,214;green,92;blue,92}{\Z x_3} & {\Z x_1x_3} & \textcolor{rgb,255:red,214;green,92;blue,92}{\Z x_5} & {\Z x_1x_5} \\
        {}
        \arrow[shift left=5, no head, from=1-1, to=7-1]
    \end{tikzcd}\]
    Here the bottom row has the given multiplicative structure since it's simply the cohomology of $U(n-1)$. The top row must have the same multiplicative structure, but here we list a degree $2n-1$ generator in the $E^{2n-1,0}_2$ term, which generates the rest of the multiplicative structure. Again, there are no nontrivial differentials so we get $H^*(U(n)) =H^*(U(n-1))[x_{2n-1}] / (x^2_{2n-1})$ as desired.
\end{solution}

\begin{problem}%62.6
    Let $f : S^2 \to S^2$ be a map of degree $2$. Compute the homology of its homotopy fiber.
\end{problem}

\begin{solution}
    Since $S^2$ is simply-connected, we can make a lot of useful reductions in the Serre fibration. First note that by the note from class, the fiber sequence $F \to S^2 \to S^2$ can be used in the Serre fibration, so we get a spectral sequence with $E^2$-page given by
    \[
        E^2_{s,t} = H_s(S^2; H_t(F))
    .\]
    Note also that the fiber sequence gives us a SES of the form:
    % https://q.uiver.app/?q=WzAsNSxbMCwwLCIwIl0sWzEsMCwiXFxwaV8xKFNeMikiXSxbMiwwLCJcXHBpXzEoU14yKSJdLFszLDAsIlxccGlfMShGKSJdLFs0LDAsIjAiXSxbMCwxXSxbMSwyLCIyXFx0aW1lcyJdLFsyLDNdLFszLDRdXQ==
    \[\begin{tikzcd}
        0 & {\pi_1(S^2)} & {\pi_1(S^2)} & {\pi_1(F)} & 0
        \arrow[from=1-1, to=1-2]
        \arrow["2\times", from=1-2, to=1-3]
        \arrow[from=1-3, to=1-4]
        \arrow[from=1-4, to=1-5]
    \end{tikzcd}\]
    This implies that $\pi_1(F)\cong \Z /2\Z$, and by the Hurewicz isomorphism this also tells us that $H_1(F)\cong \Z /2\Z$. Next, looking at the spectral sequence, we get:
    \[\begin{tikzcd}
        {H_0(F)} & {H_1(F)} & {H_2(F)} & {H_3(F)} & \cdots \\
        0 & 0 & 0 & 0 \\
        {H_0(F)} & {H_1(F)} & {H_2(F)} & {H_3(F)} & \cdots
        \arrow["{d^2_0}", color={rgb,255:red,214;green,92;blue,92}, from=1-1, to=3-2]
        \arrow["{d^2_1}", color={rgb,255:red,214;green,92;blue,92}, from=1-2, to=3-3]
        \arrow["{d^2_2}", color={rgb,255:red,214;green,92;blue,92}, from=1-3, to=3-4]
        \arrow["{d^2_3}", color={rgb,255:red,214;green,92;blue,92}, from=1-4, to=3-5]
    \end{tikzcd}\]
    Since each chain only has one potentitally non-trivial differential, the $E^3$-page is in fact the $E^\infty$-page, so we get the $E^\infty$-page:
    \[\begin{tikzcd}[row sep=scriptsize]
        & {\ker d^2_0} & {\ker d_1^2} & {\ker d^2_2} & {\ker d^2_3} & \cdots \\
        & 0 & 0 & 0 & 0 \\
        & {H_0(F)} & {\textrm{coker }d^2_0} & {\textrm{coker }d^2_1} & {\textrm{coker }d^2_2} & \cdots \\
        {} & \Z & 0 & \Z & 0 & 0 & \cdots
        \arrow[shift left=5, no head, from=4-1, to=4-7]
        \arrow[color={rgb,255:red,92;green,92;blue,214}, dashed, no head, from=1-2, to=3-4]
        \arrow[color={rgb,255:red,92;green,92;blue,214}, dashed, no head, from=1-3, to=3-5]
        \arrow[color={rgb,255:red,92;green,92;blue,214}, dashed, no head, from=1-4, to=3-6]
    \end{tikzcd}\]
    here the bottom row corresponds to the homology of $S^2$, the total space. Firstly, note that $H_0(F)\cong \Z$, this is expected. Next, we remember from the Hurewicz argument that $H_1(F)\cong \Z /2\Z$, so since $\textrm{coker }d^2_0=0$, it follows that the map $H_0(F) \to H_1(F)$ is the reduction mod $2$. This means that $\ker d^2_0 = 2\Z$ and so $\textrm{coker }d_1^2=0$. Next, we know that $\ker d^2_i = 0$ for all $i\geq 1$, and $\textrm{coker } d^2_i=0$ for all $i\geq 2$. Combining this with the fact that $\textrm{coker }d_1^2=0$, this means that $d^2_k : H_k(F) \to H_{k+1}(F)$ is an isomorphism for $k\geq 1$. So to conclude,
    \[
        H_k(F) \cong \begin{cases}
            \Z & k = 0,\\
            \Z /2\Z & \textrm{otherwise.}
        \end{cases}
    \] 

\end{solution}

\begin{problem}%64.6
    Induced homology isomorphisms.
\end{problem}
\begin{solution}
    Here we will prove some criteria for maps to induce homology isomorphisms.
    \begin{partproblem}{a}
        Show that if $p : E \to B$ is a fibration and each fiber has the homology of a point, then $p$ induces an isomorphism in homology.
    \end{partproblem}
    \quad Consider the Serre spectral sequence, which has $E^2_{s,t}=H_s(B; H_t(F_b))$. Since the fibers have the homology of a point, the only non-trivial groups here are $E^2_{s,0}=H_s(B)$. Note that there are no non-trivial differential maps, so the spectral sequence collapses here at the $E^2$ page. It thus follows that the edge homomorphisms $H_n(E) \to H_n(B)$, which are exactly the induced map $p_*$, are isomorphisms.

    \begin{partproblem}{b}
        Show that any weak equivalence $f : X \to Y$ induces a homology isomorphism. %consuder the homotopy fiber at a poitn in Y.
    \end{partproblem}
    \quad Notice that for any $y\in Y$, the homotopy fiber sequence $F_y(f) \to X \to Y$ extends to a long exact sequence of homotopy groups, however since $\pi_k(X) \to \pi_k(Y)$ is an isomorphism, it follows that $F_y(f)$ has trivial homotopy groups, so it is weakly contractible. This implies that it has the homology of a point by the Hurewicz homomorphism, so we can apply the previous problem to see that $f$ induces a homology isomorphism.
\end{solution}

\end{document}