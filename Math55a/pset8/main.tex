\documentclass[11pt,letterpaper]{article}

\usepackage{import}
\import{../../../../LaTeX}{basic}

\title{\textbf{Math 55a Problem Set 8}}

\begin{document}
\maketitle
\setcounter{page}{0}
\thispagestyle{empty}

\begin{itemize}
  \item How long did this assignment take you? -- 10 hours
  \item How hard was it? -- Tough
  \item What resources did you use and how much help did you need? -- Collaborated with AJ LaMotta
  \item Did you have any prior experience with this material? -- No
\end{itemize}

% PROBLEM 1
\pagebreak
\begin{problem}
Let $G$ be a finite abelian group.

\begin{enumerate} [(a)]
  \item For each prime $p$, show that the set of elements of order a power of $p$, 
    \[G_p := \{ g \in G \mid g^{p^n} = e \ \text{ for some } n \},\]
    is a subgroup of $G$.
  \item Show that there is a natural isomorphism $G\cong \prod_p G_p$.
\end{enumerate}
\end{problem}

\textit{(Hint: first construct a homomorphism $\varphi: \prod_p G_p\to G$ whose
restriction to each factor is the inclusion $G_p\hookrightarrow G$.
Next, for each prime $p$, write $|G|=p^k\,m$ with $p\! \not| m$, let
$r$ be such that $m|r$ and $r\equiv 1\mod p^k$, and consider the
homomorphism $g\mapsto g^r$ from $G$ to itself. Show that the image of
this homomorphism is $G_p$, and use this to construct an inverse to $\varphi$.)}

\textbf{(a)} First of all, $e\in G_p$ because $e$ has order $1=p^0$, so the subset contains the identity. It is closed under composition because if $a,b\in G_p$ say with $a^{p^{n_1}}=e$ and $b^{p^{n_2}}=e$, then since $G$ is abelian, $(ab)^{p^{n_1+n_2}}=e$ so $ab\in G_p$. Finally, it is closed under inverses because if  $a^{p^n}=e$, then  $(a^{-1})^{p^n}=\left(a^{p^{n}}\right)^{-1}=e$ so $a^{-1}\in G_p$.

\textbf{(b)} Let $\varphi : \prod_p G_p \to G$ be defined by $(g_2, g_3, g_5, \ldots) \mapsto \prod_p g_p$. This is clearly a homomorphism since $G$ is abelian. Furtherore, $\restr{\varphi}{G_p}$ is an inclusion.  

Now for each prime $p$, write $|G|=p^km$ for some $p\nmid m$. Using the Chinese remainder theorem, we can find some $r_p$ such that $r\equiv 1\mod p^k$ and $r\equiv 0\mod m$. Then the homomorphism $\psi_p : G \to G$ which maps $g$ to $g^r$. Clearly $(g^r)^{p^k}=g^{rp^k}=e$ so $\Ima(\phi_p)\subset G_p$ and for any $g\in G_p$, we have $g^r=g^{1+np^k}=g^1=g$ so $\Ima(\phi_p)=G_p$.

Now we'll construct an inverse for $\varphi$. Let $\varphi^{-1}(g)=(\psi_2(g), \psi_3(g), \psi_5(g),\ldots)$. Observe that $\varphi(\varphi^{-1}(g))=\prod_p \psi_p(g)=\prod_p g^{r_p}=g^{\sum_p r_p}$. Observe that $\sum_p r_p\equiv 1\mod p^k$ for all $p$ and maximal powers of $k$. Then by the Chinese remainder theorem, $\sum_p r_p\equiv 1\mod |G|$, so $g^{\sum_p r_p}=g$. The other direction follows the same way. This concludes the proof.

% PROBLEM 2
\pagebreak
\begin{problem}
  Prove that for any pair of positive integers $a,b$, the group $\Z/a \times \Z/b$ is isomorphic to $\Z/\mathrm{lcm}(a,b) \times \Z/\mathrm{gcd}(a,b)$. Using this, prove any finite product of finite cyclic groups is isomorphic to a product of the form $\Z/a_1 \times \cdots \times \Z/a_n$, where $a_1|a_2| \cdots | a_n$.
\end{problem}

By Bezout's lemma, let $u,v$ be integers satisfying $ua+vb=\gcd(a,b)$.
Consider the map
\[
  \begin{aligned}
    \varphi :& \Z/\lcm(a,b)\times \Z/\gcd(a,b) \to \Z/a\times\Z/b\\
    & (x,y) \mapsto \left(ux+\frac{a}{\gcd(a,b)}, vx-\frac{b}{\gcd(a,b)}\right).
    \end{aligned}
\]   
It can be shown with basic properties of $\lcm$ and $\gcd$ that this is an injective homomorphism, and so since the two groups are finite of the same order, they must be isomorphic. The second part follows by induction since $\gcd(a,b) | \lcm(a,b)$.   


% PROBLEM 3
\pagebreak
\begin{problem}
Show that, if an element of $\GL(2,\Z)$ (the group of invertible $2\times 2$ matrices with integer coefficients) has finite order $n$, then $n\in\{1,2,3,4,6\}$.
\end{problem}
\textit{(Hint: view the given element as a linear operator on a 2-dimensional complex vector space.
What can you say about its eigenvalues, and about its trace?)}

Suppose $T\in\GL(2,\Z)$ is some integral matrix with finite order $n$. Considered as an operator in $\GL(2,\C)$, we can find eigenvalues $\lambda_1, \lambda_2$ with $\lambda_1\lambda_2\in \Z$ and  $\lambda_1+\lambda_2\in \Z$. (This follows from trace and determinant respectively.) Since $T$ has order $n$, we also have $\lambda_1^n = \lambda_2^n = 1$, so the eigenvalues are both roots of unity, say $\lambda_1=\zeta^a$ and $\lambda_2=\zeta^b$ where $\zeta = e^{\frac{2\pi i}{n}}$.

Since $\lambda_1\lambda_2=\pm 1$, we have $\lambda_2=\pm\frac{1}{\lambda_1}$. Now recall that $\lambda_1+\lambda_2$ is an integer. Except for in the trivial cases when $\lambda_1=\lambda_2=\pm 1$, generally $\lambda_1+\lambda_2\in \{ -1, 0, 1\}$. So
\[
  \lambda_1+\frac{\pm 1}{\lambda_1}\in \{-1,0,1\} \implies \lambda_1^2+\{-1,0,1\}\lambda_1\pm 1=0 
.\]
So there is a one-to-one correspondence between finite order elements of $\GL(2,\Z)$ and complex numbers $\lambda$ which satisfies $\lambda^n=1$ and $\lambda^2+a\lambda+b=0$ for some $a\in \{-1,0,1\}$ and $b\in \{-1,1\}$. Assuming $\lambda$ has nonzero imaginary part, the quadratic polynomial is irreducible, so it must be a cyclotomic polynomial. So we are looking for $n$ such that $\deg \Phi_n(x)=\varphi(n)=2$. Basic properties of the Euler totient function imply that the only such $n$ are $3,4,$ and $6$. So including the trivial cases where the eigenvalues were $1$ or $-1$, the only possible orders are $1,2,3,4,6$.     

\pagebreak
\begin{problem}
A {\em lattice} in the Euclidean plane $\R^2$ is an additive subgroup $\Lambda=\Z u\oplus \Z v\subset \R^2$ generated by two linearly independent vectors. Let $\Lambda\subset \R^2$ be a lattice, and let $G\subset O(2)$ be the subgroup of all orthogonal transformations (rotations and/or reflections) of $\R^2$ which map $\Lambda$ to itself.

\medskip
Show that every element of $G$ has finite order (hint: $\Lambda$ has finitely many shortest vectors), and use the result of the previous problem to show that $G$ must be isomorphic to either $\Z/n$ or the dihedral group $D_n$, for some $n\in \{2,4,6\}$. (Optional: which of these possibilities actually occur?)
\end{problem}

Every element of $O(2)$ is either a reflection or a rotation by some real angle $\alpha$. The reflections clearly have order $2$, but we need to show that $\alpha$ is an rational multiple of $2\pi$. Clearly $\{u,v,-u,-v\}$ is the set of vectors in the lattice with minimum length, i.e. every other non-zero vector must have length strictly greater than a vector in this list.      

Since $T\in O(2)$ is orthogonal, letting $u$ be a shortest vector in the lattice, it follows that $T(u)\in \{u, v, -u, -v\}$. From this we can conclude that $T$ has finite order. By the results of the previous problem and using the fact that every subgroup of $O(2)$ is dihedral or cyclic, we know that $G$ can either be $\Z/2, \Z/3, \Z/4, \Z/6$ or $D_2, D_3, D_4, D_6$. However it can't be $\Z/3$ and $D_3$ since they do not contain a $180$ degree rotation. So we are done.           

\pagebreak
\begin{problem}
\leavevmode
\begin{enumerate}[(a)]
  \item Show that any group of order 6 is isomorphic to either $\Z/6$ or the symmetric group $S_3$.
  \item Classify all groups of order 8 up to isomorphism.
\end{enumerate}
\end{problem}
\textit{Hint for both parts: If $ab\neq ba$, then one of $a$, $b$, and $ab$ must have order $>2$. (Why?) Moreover, a subgroup $H\subset G$ with $|G/H|=2$ must be normal. (Why?)}

\begin{lemma}
  Suppose $ab\neq ba$ in some finite group $G$. Then $a,b$ or $ab$ must have order greater than $2$.   
\end{lemma}
\begin{proof}

\end{proof}

\textbf{(a)} Suppose $G$ is a group of order $6$. We have two cases.
\begin{itemize}
  \item Suppose 
\end{itemize}

\textbf{(b)}

\pagebreak
\begin{problem}
Let $G$ be a group, not necessarily finite, and let $H \subset G$ be a subgroup of finite index, that is, such that there are finitely many left cosets of $H$ in $G$. Prove that the number of right cosets is equal to the number of left cosets (so that we can define the index of $H$ in $G$ unambiguously).
\end{problem}
\textit{(Hint: find an operation on $G$ which maps left
cosets to right cosets)}

Consider the isomorphism $\varphi_a : G \to G$ given by $x\mapsto a^{-1}xa$. (i.e. conjugation by $a^{-1}$) This is an isomorphism by an earlier problem set, and it clearly maps left cosets to right cosets. This concludes the proof.  

\pagebreak
\begin{problem}
Let $G\subset S_n$ be any subgroup of the symmetric group. The action of $G$ on $\{1,\dots,n\}$ is said to be {\em twice transitive} if $G$ acts transitively on ordered pairs of distinct elements, i.e.\ for every $i,i',j,j'\in\{1,\dots,n\}$ with $i\neq i'$ and $j\neq j'$, there exists $\sigma\in G$ such that $\sigma(i)=j$ and $\sigma(i')=j'$. Show that if the action of $G$ on $\{1,\dots,n\}$ is twice transitive and $G$ contains a transposition then $G=S_n$.
\end{problem}

Suppose WLOG that $(12)\in G$. Then for any $i\neq j$ let $\sigma$ be an element such that $\sigma(i)=j$ and $\sigma(j)=i$. Such an element is guaranteed to exist because $G$ is twice transitive. Then $(ij)=\sigma(12)\sigma^{-1}$. Since $G$ contains all transpositions, it is the full symmetric group.   

\pagebreak
\begin{problem}
Let $V$ be a 2-dimensional vector space over the field $\F_p$, and $G = \GL(V)$ the group of automorphisms of $V$ (i.e., $G = \GL_2(\F_p)$, the group of $2 \times 2$ invertible matrices with entries in $\F_p$).

\begin{enumerate}[(a)]
  \item Show that $V$ has exactly $p+1$ 1-dimensional subspaces.
  \item Given this, we have a homomorphism $\phi:G\to S_{p+1}$, since every automorphism of $V$ must permute its 1-dimensional subspaces. Describe the kernel and the image of $\phi$ for $p=2$ and $p=3$.
\end{enumerate}
\end{problem}

\textbf{(a)} There are $p^2-1$ nonzero elements in $V$. Under the equivalence relation $x\tilde y$ iff $x=cy$ for some nonzero $c\in \F_p$, there are $p-1$ members in each equivalence class, so there are $(p^2-1)/(p-1)=p+1$ equivalence classes. These correspond to one dimensional subspaces of $V$, so there are exactly $p+1$ lines.     

\textbf{(b)} For $p=2$, the kernel is trivial, since the only matrix which acts trivially on the one-dimensional subspaces is the identity matrix. Thus the image is full since $|G|=6$ and $|S_{3}|=6$. So $\phi$ is an isomorphism. 

For $p=3$, the kernel consists of $I$ and $2I$, so the image is again full by the first isomorphism theorem and since $|G|=48$ and $|S_4|=24$.

\pagebreak
\begin{problem}
An element of the symmetric group $S_n$ is called a $k$-cycle if it permutes $k$ elements cyclically and fixes the remaining $n-k$. How many $k$-cycles are there in $S_n$?
\end{problem}

There are $\binom{n}{k}$ subsets of size $k$ in $\{1,2,3,\ldots,n\}$ which the cycle can permute, and $(k-1)!$ possible permutations. So the total number of $k$-cycles is
\[
  (k-1)!\binom{n}{k}=\frac{n!}{k(n-k)!}
.\] 

\pagebreak
\begin{problem}
Let $G$ be a group of order $p^n$ with $p$ prime, and suppose $G$ acts on a finite set $S$. Prove that if the cardinality of $S$ is not divisible by $p$, then there must be an element $s \in S$ fixed by every $g \in G$; that is, an element whose stabilizer is all of $G$.
\end{problem}

By the orbit stabilizer theorem we can write
\[
  |S|=\sum_{s\in S/G}|Gs|=\sum_{s\in S/G}\frac{|G|}{|\mathrm{Stab}(s)|}
.\]
Since $|G|/|\mathrm{Stab}(s)|$ is a power of $p$  for every $s\in S$, and $S$ is not divisible by $p$,  it follows that at least one of the $|G|/|\mathrm{Stab}(s)|=1$, so $|G|=|\mathrm{Stab}(s)|$. This concludes the proof.  

\pagebreak
\begin{problem}
How many different bracelets can you make with 4 white beads and 4 black beads?
\end{problem}
\textit{(Hint: use Burnside's formula!)}

Let $S$ be the set of bracelets with $4$ white beads and $4$ black beads. There are exactly $\binom{8}{4}=70$ elements. To find the number of bracelets up to symmetries, let $D_8$ act on $S$ in the natural way. Then to figure out the number of orbits, by Burnside's formula we must find the size of the fixed point sets for all elements of the group.
\begin{itemize}
  \item The identity element fixes all $70$ elements.
  \item The four reflections have $\binom{4}{2}=6$ fixed elements.
  \item The four reflections across diagonals have $2\cdot\binom{3}{1}=6$. 
  \item The four rotations by $\pm 45^\circ, \pm 135^\circ$ have no fixed elements.
  \item The two rotations by $\pm 90^\circ$ have $2$ fixed elements.
  \item The rotation by $180^\circ$ has $6$ fixed elements.  
\end{itemize}    

Using Burnside's formula, we have
\[
  |S/D_8|=\frac{1}{16}(70+4\cdot 6+4\cdot 6+4\cdot 0 + 2\cdot 2+6)=8
.\] 
So there are $8$ unique bracelets which can be made.

\end{document}
