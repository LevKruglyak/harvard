\documentclass[11pt,letterpaper]{article}

\usepackage{import}
\import{../../../../LaTeX}{basic}

\title{\textbf{Math 55a Problem Set 9}}

\begin{document}
\maketitle
\setcounter{page}{0}
\thispagestyle{empty}

\begin{itemize}
  \item How long did this assignment take you? -- 10 hours
  \item How hard was it? -- Tough
  \item What resources did you use and how much help did you need? -- Collaborated with AJ LaMotta
  \item Did you have any prior experience with this material? -- No
\end{itemize}

\pagebreak
\begin{problem}
Let $p$ be a prime number, and let $G$ be any group of order $p^3$.

\begin{enumerate}[(a)]
  \item What are the possible orders of the center $Z$ of $G$?
  \item Assume $G$ is not abelian, and let $g\in G$ be an element not in the
  center $Z$. What can be the order of its centralizer? (Recall that the
  centralizer of $g$ is the subgroup consisting of all elements of $G$ 
  that commute with $g$.)
  \item What are the possible class equations for $G$? 
\end{enumerate}
\end{problem}

\textbf{(a)} Since $|Z(G)|$ divides $|G|=p^3$, $|Z(G)|$ could equal $1,p,p^2,$ or $p^3$.

\textbf{Case $|Z(G)|=1$}: In this case the class equation can be written as $p^3=1+\sum_i c_i$ where $c_i$ are the sizes of the conjugacy classes. But $p|c_i$ for all $i$, which is a contradiction since $p|1+\sum_i c_i$ and $p|\sum_i c_i$. So the center must be nontrivial.

\textbf{Case $|Z(G)|=p$}: For an example of a group with such center, look at $D_4$, which has order $8$ and center of size $2$.  

\textbf{Case $|Z(G)|=p^2$}: This is impossible since $|G/Z(G)|=p$ which means that $G/Z(G)$ is cyclic. This implies that $G$ is abelian, a contradiction.   

\textbf{Case $|Z(G)|=p^3$}: This is possible, just take any abelian group of order $p^3$. 

So the only possible orders of the center are $p, p^3$. 

\textbf{(b)} Since $G$ isn't abelian, by (a) we have $|Z(G)|=p$. Then the centralizer of a non-central element must be order $p^2$ since it must divided the order of the group but cannot be equal to the order of the group.   

\textbf{(c)} The possible class equations for $G$ are 
\[
  \begin{aligned}
    p^3&=\underbrace{1+1+\cdots+1}_{p^3\textrm{ times}}\\
    &=\underbrace{1+1+\cdots+1}_{p\textrm{ times}}+\underbrace{p+p+\cdots+p}_{p^2-1\textrm{ times}}
  \end{aligned}
\]  


\pagebreak
\begin{problem}
Consider the \emph{Heisenberg group}
\[
H = H(3, \F_p) \; := \; \left\{ 
\begin{pmatrix}
1 & a & b \\
0 & 1 & c \\
0 & 0 & 1
\end{pmatrix}
\mid a, b, c \in \F_p \right\}.
\]
\begin{enumerate}[(a)]
  \item Find the commutator subgroup $H'$ of $H$ (recall that $H'$
  is the subgroup generated by all commutators $[g,h]=ghg^{-1}h^{-1}$). 
  What is the quotient $H/H'$?
  \item Describe all the conjugacy classes in $H$.
  \item Find all the normal subgroups of $H$.
\end{enumerate}
\end{problem}

First, let's understand the matrix operation better. Observe that 
\[
  \begin{pmatrix}
    1 & a & b \\
    0 & 1 & c \\
    0 & 0 & 1
    \end{pmatrix} \times \begin{pmatrix}
      1 & x & y \\
      0 & 1 & z \\
      0 & 0 & 1
      \end{pmatrix}=\begin{pmatrix}
        1 & x+a & b+y+az \\
        0 & 1 & c+z \\
        0 & 0 & 1
        \end{pmatrix}
.\]
Also note that
\[
  \begin{pmatrix}
    1 & a & b \\
    0 & 1 & c \\
    0 & 0 & 1
    \end{pmatrix}^{-1}=\begin{pmatrix}
      1 & -a & ac-b \\
      0 & 1 & -c \\
      0 & 0 & 1
      \end{pmatrix}
.\]   
To make notation a bit simpler, we use the notation
\[
  [a,b,c] := \begin{pmatrix}
    1 & a & b \\
    0 & 1 & c \\
    0 & 0 & 1
    \end{pmatrix}
.\]

\textbf{(a)} Let $a,b,c,x,y,z\in \F_p$  be some elements. Then
\[
  \begin{aligned}
    [a,b,c] [x,y,z] [a,b,c]^{-1} [x,y,z]^{-1}&=[a+x, b+y+az, c+z] [-a, ac-b, -c] [-x, xz-y, -z]\\
    &=[x,y+az-cx,z][-x,xz-y, -z]\\
    &=[0,az-cx, 0]
  \end{aligned}
.\]
Let $T$ be the subgroup of $H$ consisting of elements of the form $[0,\alpha,0]$ for any $\alpha\in \F_p$ . So clearly $H'\subset T$, but for all $\alpha\in \F_p$ we also have $[\alpha, 0, 0][0,0,1][\alpha,0,0]^{-1}[0,0,1]^{-1}=[0,\alpha, 0]$ so $H'=T$.

Thus $H'$ is exactly the set
\[
  H'=\left\{\begin{pmatrix}
    1 & 0 & \alpha \\
    0 & 1 & 0 \\
    0 & 0 & 1
    \end{pmatrix}\mid \alpha\in \F_p\right\}
\]
and is isomorphic to the additive group $\F_p$. To determine the structure of the quotient group $H/H'$, observe that for any $a,b,c\in \F_p$ the coset $[a,b,c]H'$ is
\[
  [a,b,c]H' = \{[a,b+\alpha,c] \mid \alpha\in \F_p\}
.\]
So the quotient group $H/H'$ is exactly the additive group $\F_p^2$.   

\textbf{(b)} From the computations in (a), we have the identity
\[
  [a,b,c][x,y,z][a,b,c]^{-1}=[x,y+az-cx,z]
.\]
So the conjugacy classes are 
\begin{itemize}
  \item For each $\alpha\in \F_p$ we have $\{[0,\alpha,0]\}$
  \item For each $x,y$ with $(x,y)\neq (0,0)$, we have $\{ [x,\alpha,y] \mid \alpha\in \F_p\}$   
\end{itemize} 
Thus the class equation is
\[
  p^3=\underbrace{1+1+\cdots+1}_{p}+\underbrace{p+p+\cdots+p}_{p^2-1}
.\] 

\textbf{(c)} A subgroup of a group is normal if and only if it is the union of conjugacy classes. Subgroups must also divide the order of the group. So in general, the order of a normal subgroup is $x+yp$ where $1\leq x\leq p$ and $0\leq y\leq p^2-1$. Then $x$ must equal $p$, and $1+y=p$ or $1+y=p^2$. The latter case is the full group, so the only nontrivial normal subgroups correspond to one-dimensional subspaces of $\F_p^2$, namely for some $\ell\subset \F_p^2$, there is a normal subgroup $\{\ell_x, \alpha, \ell_y \mid \alpha\in \F_p\}$. This correspondence is one-to-one.  

\pagebreak
\begin{problem}
Let $G$ be the group of affine transformations of $\F_p$ ($p$
prime), i.e.\ maps $f_{a,b}:\F_p\to \F_p$ of the form $f_{a,b}:x\mapsto
ax+b$ for $a,b\in \F_p$, $a\neq 0$.
\begin{enumerate}[(a)]
  \item Find the commutator subgroup $G'$ of $G$, and describe the quotient $G/G'$.
  \item Describe all the conjugacy classes in $G$.
  \item Show that the classification of normal subgroups of $G$ is determined
  by that of subgroups of $\F_p^\times=(\F_p-\{0\},\times)$.
\end{enumerate}
\end{problem}
\textit{(Optional: what about non-normal subgroups of $G$?)}

First, let's make a convenient relation for the group law. For any $a,b,x,y\in \F_p$, we have 
\[
  f_{a,b}f_{x,y}=f_{ax, ay+b}\quad\textrm{ and }\quad f_{a,b}^{-1}=f_{1/a,-b/a}
.\]
Here $a$ is invertible because it is nonzero and since $\F_p$ is a field. For the rest of the problem, we assume that $p\neq 2$, because if $p=2$ the results are trivial:
\begin{itemize}
  \item The commutator subgroup is trivial so $G/G'=G$.
  \item The group is abelian so every element has its own conjugacy class.
  \item The group has one nontrivial subgroup which corresponds to the single subgroup of $\F_2^\times$.  
\end{itemize}  

\textbf{(a)} To find the commutator subgroup of $G$, let $a,b,x,y\in \F_p$ with $a,x\neq 0$ . Then
\[
  \begin{aligned}
    f_{a,b}f_{x,y}f_{a,b}^{-1}f_{x,y}^{-1}&=f_{ax,ay+b}f_{1/a,-b/a}f_{1/x,-y/x}\\
    &= f_{x,ay-bx+b}f_{1/x,-y/x}\\
    &= f_{1, ay-y+b-bx}
  \end{aligned}
.\]

For any $\alpha\in \F_p$, if $a=1, x=-1, y=0, b=\alpha/2$, then this element is $f_{1,\alpha}$ so the commutator subgroup is the set 
\[
  G'=\{f_{1,\alpha}\mid \alpha\in \F_p\}
.\] 
To determine the quotient group structure of $G/G'$, let $a,b\in \F_p$ with $a\neq 0$. Then $f_{a,b}G'=\{f_{a,a\alpha+b} \mid \alpha\in \F_p\}=\{f_{a,\alpha}\mid \alpha\in \F_p\}$ since $x\mapsto ax+b$ is a bijection for $a\neq 0, x\in \F_p$. So $G/G'=\F^\times_p$.

\textbf{(b)} Let $a,b\in \F_p$ with $a\neq 0$. Then for any $x,y\in \F_p$ with $x\neq 0$, we have 
\[
  f_{a,b}f_{x,y}f_{a,b}^{-1}=f_{x,ay-bx+b}
.\]
So it follows that the conjugacy classes are
\begin{itemize}
  \item $\{f_{1,0}\}$
  \item $\{f_{1,\alpha} \mid \alpha\in \F_p^\times\}$
  \item For $\alpha\in \F_p^\times$ and $\alpha\neq 1$, the set $\{f_{\alpha, \beta}\mid \beta\in\F_p\}$  
\end{itemize} 
So the class equation is
\[
  p^2-p = 1 + p-1 + \underbrace{p+p+\cdots+p}_{p-2}
.\] 

\textbf{(c)} Lastly, (same argument as in (2)), each proper nontrivial normal subgroup corresponds to a solution to $1+a(p-1)+bp|p^2-p$ where $a=0$ or $1$ and $0\leq b < p-2$. From this we can see that every subgroup of $\F_p^\times$ corresponds to a solution to this equation, hence a normal subgroup in $G$.     

\pagebreak
\begin{problem}
Find all finite groups $G$ that have at most 3 conjugacy classes.
\end{problem}

If $G$ has only one conjugacy class, it must be the trivial group. Next, if $G$ has two conjugacy classes, either $Z(G)=2$ or $Z(G)=1$. If $Z(G)=2$, the group must be $\Z/2$ since it only has $2$ elements, and if $Z(G)=1$, the class equation becomes $n=1+c$ where $n=|G|$ and $c$ is the size of the nontrivial conjugacy class. Then $c|n$ so $0\equiv 1\mod c$, a contradiction.           

Lastly, suppose there are $3$ conjugacy classes. Write $n=1+c_1+c_2$ where $n=|G|$ and $c_1$, $c_2$ are the sizes of the two nontrivial conjugacy classes with $1\leq c_1\leq c_2$ . Note that $c_1|n$ and $c_2|n$. This implies that $c_1|1+c_2$ and $c_2|1+c_2$. So $c_1\leq 1+c_2\leq 2+c_1$. So we have either $c_2=1+c_1$ or $c_2=c_1$.

\textbf{Case 1:} If $c_2=c_1$, then $n=2c_1+1$ and $c_1|n$ so $c_1=1$ and so $n=3$. The only group of order $3$ is $\Z/3$.

\textbf{Case 2:} If $c_2=c_1+1$, then $n=2c_1+2$ and $c_1|n$. So $c_1|2$. If $c_1=1$, them $|G|=4$ which is a contradiction because this must be abelian. If $c_1=2$, then $c_2=3$ so $|G|=6$ and $G=S_3$, because $S_3$ is the only group of order $6$ which has trivial center and conjugacy classes of size $2$ and $3$.             

To summarize, the only groups with at most 3 conjugacy classes are $\{e\}, \Z/2, \Z/3, \mathrm{ and } S_3$. 

\pagebreak
\begin{problem}
Let $\sigma, \tau \in S_n$ be any two permutations. Show that,
even though the products $\sigma\tau$ and $\tau\sigma$ may not be equal, 
they have the same cycle lengths.
\end{problem}

Observe that $\sigma\tau = (\tau^{-1})\tau \sigma (\tau)$, so $\sigma\tau$ and $\tau\sigma$ are in the same conjugacy class. Thus they have the same cycle lengths.   

\pagebreak
\begin{problem}
  \leavevmode
  \begin{enumerate}[(a)]
    \item List the conjugacy classes in the alternating group $A_6$, and find the number of elements in each.
    \item Use this to prove that $A_6$ is simple.
  \end{enumerate}
\end{problem}

\textbf{(a)} Recall that every partition of $n$ corresponds to a conjugacy class in $S_n$. Using the formulas and results from class we can generate the following list: 

\setlength{\tabcolsep}{20pt}
\renewcommand{\arraystretch}{1.5}
\begin{center}
  \begin{tabular}{ |c|c|c|c| } 
    \hline
    \textbf{Partition} & \textbf{Split} & \textbf{Size} & \textbf{Representative} \\ 
    \hline
    $1+1+1+1+1+1$ & No  & $1$ & $e$  \\ 
    \hline
    $3+1+1+1$ & No & $40$ & $(123)$ \\
    \hline
    $2+2+1+1$ & No & $45$ & $(12)(34)$  \\
    \hline
    $4+2$ & No & $90$ & $(1234)(56)$ \\
    \hline
    $3+3$ & No & $40$ & $(123)(456)$ \\
    \hline
    $5+1$ & Yes & $72$ & $(12345)$ \\
    \hline 
    $5+1$ & Yes & $72$ & $(13524)$ \\
    \hline
    \end{tabular}
\end{center}

\textbf{(b)} To use this to prove that $A_6$ is simple, we use the result that normal subgroups correspond to unions of conjugacy classes. The order of $A_6$ is $360$, so the possible sizes of subgroups of $A_6$ are 
\[
  1, 2, 3, 4, 5, 6, 8, 9, 10, 12, 15, 18, 20, 24, 30, 36, 40, 45, 60, 72, 90, 120, 180, 360
.\]  
However, none of these divisors except for $1$ and $360$ can be built as a sum of sizes of conjugacy classes including $1$. So $A_6$ has no nontrivial proper normal subgroups hence it is simple. 

\pagebreak
\begin{problem}
For what integers $n$ does there exist a surjective homomorphism $\phi : S_n \to S_{n-1}$?
\end{problem}

Suppose $\phi : S_n \to S_{n-1}$ is surjective. Then $S_n/\ker(\phi)\cong S_{n-1}$, and so $\ker(\phi)$ is a normal subgroup of $S_n$ of size $n$. We claim that no such subgroup exists when $n\geq 5$.

\begin{lemma}
  For $n\geq 5$, the only nontrivial proper normal subgroup of $S_n$ is $A_n$. 
\end{lemma}
\begin{proof}
  Let $N\subset S_n$ be some normal subgroup. Then $N\cap A_n$ is a normal subgroup of $A_n$, but $A_n$ is simple so either $N\cap A_n={e}$ or $N\cap A_n=A_n$. In the second case, either $N=A_n$, or $N$ must contain an odd permutation so $N=S_n$. (Since $A_n$ is index $2$ in $S_n$) Suppose instead that $N\cap A_n=\{e\}$. Then either $N$ is trivial or $N$ contains a single odd permutation. (If $N$ contained more than one odd permutation, their product would be even, contradicting the assumption) However it is clear that such a subgroup cannot be normal. This concludes the proof.      
\end{proof}

Since $|A_n|=n!/2\neq n$ for $n\geq 5$, the only candidates for surjective homomorphisms are for $n<5$.

\textbf{Case $n=2$:} There is a canonical surjective homomorphism mapping $S_2\to S_1=\{e\}$   

\textbf{Case $n=3$:} Consider the homomorphism $\phi : S_3 \to S_2$ which maps even permutations to $e$ and odd permutations to $\tau$ where $\tau\in S_2$ is the nontrivial element. This is a homomorphism and surjective.

\textbf{Case $n=4$:} To construct a surjective homomorphism from $S_4\to S_3$, color opposing faces with the same color. Then every symmetry of the cube (corresponding to an element of $S_4$) permutes the three face pairs, so we have a mapping of $S_4\to S_3$, and it can be seen visually that it is surjective, i.e. every permutation of the three face pairs corresponds to a symmetry.

So the only $n$ satisfying the problem conditions are $2,3,4$. 

\pagebreak
\begin{problem}
Show that a nonabelian group of order 21 exists by finding
one explicitly as a subgroup of $S_7$.
\end{problem}
\textit{(Note: The shortest way to solve this problem is to observe that one of the
groups you have recently encountered contains a subgroup of order 21,
and acts on a set with 7 elements. An alternative, more systematic approach is as follows.
Sylow's theorems imply that a group of order 21
contains a unique subgroup of order 7. Taking this for granted, you can 
try to first build
an example ``by hand'', denoting by $x$ an element of order 7, by $y$ an
element not in the subgroup generated by $x$, and figuring out first the
order of $y$, then what $yxy^{-1}$ might be. 
You can then turn your example into a
subgroup of $S_7$ by finding suitable permutations that $x$ and $y$
might map to.)}

Consider the group $G$ with presentation
\[
  G=\big\langle x,y \mid x^7=y^3=e,\; yx=x^2y \big\rangle
.\] 
This group is of order $21$, since every element in the group can be expressed as $x^ay^b$ for some $0\leq a<7$ and $0\leq b< 3$ using the relations. 

Now we must find elements $X,Y\in S_7$ satisfying the relations. Consider
\[
  X=(1234567),\quad Y=(126)(475)
.\]  
Then
\[
  \begin{aligned}
    YX &= (126)(475)(1234567) = (134)(276)\\
    X^2Y &= (1357246)(126)(475) = (134)(276)
  \end{aligned}
.\] 
So the subgroup of $S_7$ generated by $X$ and $Y$ is a subgroup of order $21$.   

\pagebreak
{\bf 9*.} (Optional, extra credit) 
Let $PGL_2(\F_p)$ be the quotient of
$GL_2(\F_p)$ (the group of
$2\times 2$ invertible matrices with entries in $\F_p$) by the normal subgroup consisting of scalar multiples of the
identity. 

(a) What is the order of $PGL_2(\F_p)$? Show that $PGL_2(\F_p)$ acts on the
set of 1-dimensional subspaces of $(\F_p)^2$, and that this determines
a homomorphism $\psi:PGL_2(\F_p)\to S_{p+1}$;
what can you say about this homomorphism for $p=2$ and $p=3$?
(Cf.\ HW 8 Problem 8).

(b) We now focus on $p=5$. Show that $\psi:PGL_2(\F_5)\to S_6$ 
is an injective homomorphism, whose image $H\subset S_6$ acts transitively
on $\{1,\dots,6\}$.

(c) Show that the action of $S_6$ on the set of left cosets of $H$ (by left
multiplication) gives rise a homomorphism $f:S_6\to S_6$, and that
$f$ is an isomorphism from $S_6$ to itself. (Hint: what can you say about
$\Ker(f)$?)
Also show that $f(H)\subset S_6$ is contained in a subgroup $S_5\subset S_6$
of permutations which fix one element of $\{1,\dots,6\}$.

(d) Comparing $H$ and $f(H)$, show that the automorphism $f$ of $S_6$ 
is not an {\em inner automorphism}, i.e.\ not a conjugation $c_g:x\mapsto
gxg^{-1}$ for some $g\in S_6$.

(This is the only instance of an automorphism of $S_n$ not given by 
conjugation! For $n\neq 6$ all automorphisms of $S_n$ are inner, and for
$n\not\in \{2,6\}$, $Aut(S_n)$ is isomorphic to $S_n$ via $g\mapsto c_g$.)
\medskip

\end{document}