\documentclass[11pt,letterpaper]{article}

\input{../../../../.config/latex/preamble_v1.tex}
\lightmode

\title{\textbf{Math 55b Midterm}}

\begin{document}
\maketitle

{\bf ``I affirm my awareness of the standards of the Harvard College Honor
Code. While completing this exam, I have not consulted any external sources other than class notes
and the textbook (Munkres). I have not discussed the problems or solutions of this
exam with anyone, and will not discuss them until after the due date.''
\bigskip

Signed: \underline{\quad Lev Kruglyak\quad}}
\bigskip

\pagebreak
\begin{problem}[14 points]
    Determine whether the following statements are true or false. If true, give a proof. If false, give a counterexample.
    \begin{enumerate}[(a)]
        \item If $A_i\subset X_i$ are closed subsets for all $i\in I$, then $\prod_{i\in I} A_i$ is a closed subset of $\prod_{i\in I} X_i$ with the product topology.
        \item If $x_1,x_2,\dots\in X$ are limit points of a subset $A\subset X$, and if the sequence $x_n$ converges to a limit $x\in X$, then $x$ is a limit point of $A$.
        \item If a subspace $A$ of a topological space $X$ is connected, then its closure $\overline{A}\subset X$ is connected. 
        \item If $X$ is Hausdorff, and $A\subset X$ is compact, then its boundary $\partial A=\overline{A}-\mathrm{int}(A)$ is compact.
        \item $[0,1]\subset \R_\ell$ with the lower limit topology (generated by the basis $\{[a,b),\ a<b\}$) is compact.
        \item The addition map $f:\R_\ell \times \R_\ell \to \R_\ell$ defined by $f(x,y)=x+y$ is continuous (equipping $\R_\ell$ with the lower limit topology and $\R_\ell\times\R_\ell$ with the product topology).
        \item The set of all uniformly continuous functions $f:\R\to\R$ (i.e., such that $\forall \epsilon>0$ $\exists \delta>0$ s.t.\ $\forall p,q\in \R$, $|p-q|<\delta \Rightarrow |f(p)-f(q)|<\epsilon$) is a closed subset of the space of all functions from $\R$ to $\R$ equipped with the uniform topology.
    \end{enumerate}
\end{problem}

\begin{solution}
    \textbf{(a)} This is true. For brevity, let $X=\prod_{i\in I}X_i$ and for any $x\in X$, let $x_i\in X_i$ be the projection of $x$ to its $i$-th component. Note that \[
        \prod_{i\in I}A_i=\bigcap_{i\in I} \{x\in X \mid x_i\in A_i\} = \bigcap_{i\in I}\pi_i^{-1}(A_i)
    \]
    where $\pi_i : X \to X_i$ is the projection map onto the $i$-th component. However the product topology is the coarsest topology for which the projection maps are continuous, so $\pi_i^{-1}(A_i)$ is closed in $X$. Thus $\bigcap_{i\in I}\pi_i^{-1}(A_i)$ is closed in $X$, since it is an intersection of infinitely many closed sets.

    \textbf{(b)} This is true. Suppose $U\ni x$ is an open neighborhood. Since $x$ is a limit of the sequence $x_1,x_2,\ldots$, $U$ must contain some $x_n$. Yet since $x_n$ is a limit point of $A$, $U$ must intersect $A$ nontrivially. This implies that $x$ is a limit point of $A$, because $U$ was an arbitrary open neighborhood.
    
    \textbf{(c)} This is true. To make life easier for us, we'll use an equivalent formulation of connectedness:
    \begin{claim}
        A space $X$ is connected if and only if every continuous function $f : X \to \{0,1\}$ is constant, where $\{0,1\}$ has the discrete topology. 
    \end{claim}
    \begin{proof}
        If $X$ is connected then $f(X)$ is connected in $\{0,1\}$ so $f(X)=\{0\}$ or $\{1\}$. Conversely, if every continuous function $f : X \to \{0,1\}$ is constant then $X$ must be connected, because otherwise we could construct a continuous map which maps different connected components of $X$ to $0$ and $1$.   
    \end{proof}

    Now since $A\subset X$ is connected, an easy extension of the above claim implies every continuous map $f : X \to \{0,1\}$ must be constant when restricted to $A$. Say without loss of generality that $f(A)=0$. Then $A\subset f^{-1}(\{0\})$ and thus $f^{-1}(\{0\})$ is a closed set containing $A$. This means that $\overline{A}\subset f^{-1}(\{0\})$ so $f(\overline{A})=0$. Since $f$ was an arbitrary continuous function, it follows that $\overline{A}$ is connected.

    \textbf{(d)} This is true. Since $X$ is Hausdorff and $A$ is a compact subset, $A$ is closed so $\overline{A}=A$. Now suppose $\mathcal{U}$ is an open cover of $\partial A$. Then $\mathcal{U} \cup \{\textrm{int}(A)\}$ is an open cover of $A$ so it must contain a finite subcover $\mathcal{V}$. Removing $\textrm{int}(A)$ from $\mathcal{V}$ if needed, we get an open subcover of $\partial A$.  

    \textbf{(e)} This is false. Consider the open cover
    \[
        [0,1]\subset\left[\frac{1}{2}, 2\right) \cup \bigcup_{0<x<\frac{1}{2}}[0,x)
    .\] 
    This cannot have a finite subcover because every element in the cover contains a point which isn't in any of the other elements.

    \textbf{(f)} This is true. Let $[b,a)$ be an open set in $\R_\ell$. Then we claim that 
    \[
        f^{-1}([b,a))=\bigcup_{t\in \R}\left[t, t+\frac{a-b}{2}\right)\times\left[b-t, b-t+\frac{a-b}{2}\right)
    .\]  
    The $\supset$ direction follows because for any $t\in \R$ and $(x,y)\in \left[t, t+\frac{a-b}{2}\right)\times\left[b-t, b-t+\frac{a-b}{2}\right)$ and simply by adding inequalities together we get
    \[
        \begin{cases}
            t\leq x<t+\frac{a-b}{2}\\
            b-t\leq y<b-t+\frac{a-b}{2}
        \end{cases}\implies b\leq x+y<a
    \]
    so $(x,y)\in f^{-1}([b,a)$. Conversely for any $(x,y)\in f^{-1}([b,a))$, let $t=\frac{x-y+b}{2}$. Since $b\leq x+y<a$, we have $x\geq b-y$ and $x < a-y$ so $\frac{x-y+b}{2}\leq x < \frac{x-y+a}{2}$. This is the same as saying $x\in \left[t, t+\frac{a-b}{2}\right)$. Similarly, we get $y\in \left[b-t, b-t+\frac{a-b}{2}\right)$. So $(x,y)\in \left[t, t+\frac{a-b}{2}\right)\times\left[b-t, b-t+\frac{a-b}{2}\right)$. So $f^{-1}([b,a))$ is an arbitrary union of open sets in $\R_\ell\times \R_\ell$ and hence is open. This means that $f$ is continuous.     

    \textbf{(g)} This is false. Suppose the set of uniformly continuous functions was closed. This would mean that any convergent sequence of uniformly continuous functions must converge to a function which is also uniformly continuous. This is clearly false, consider the non-uniformly continuous function $f(x)=e^x\sin(1 /e^x)$. This is the limit of the sequence of uniformly continuous functions
    \[
        f_{n}=\begin{cases}
            f(x)&\textrm{if }x>-n\\
            f(-n)&\textrm{if }x\leq -n
        \end{cases}
    .\]  
    This gives the desired contradiction.
\end{solution}

\pagebreak
\begin{problem}[6 points]
    Let $X,Y$ be topological spaces. The graph of $f:X\to Y$ is the subset $G_f=\{(x,f(x))\,|\,x\in X\}$ of $X\times Y$.
    \begin{enumerate}[(a)]
        \item Show that if $Y$ is Hausdorff and $f:X\to Y$ is continuous then its graph $G_f$ is a closed subset of $X\times Y$ (with the product topology). 
        \item Show that if $Y$ is compact and the graph $G_f$ is closed in $X\times Y$ then $f$ is continuous.
        \item Give an example showing that the result of (b) need not hold if $Y$ is not compact.
    \end{enumerate}
\end{problem}

\begin{solution}
    \textbf{(a)} Let $(x,y)\in X\times Y-G_f$, so $y\neq f(x)$. Since $Y$ is Hausdorff, there are open sets in $Y$ such that $U\ni y$, $V\ni f(x)$ and $U\cap V=\emptyset$. Now consider the open set $f^{-1}(V)\times U\ni (x,y)$. We claim that $f^{-1}(V)\times U\cap G_f=\emptyset$. Indeed, if $(t, f(t))\in f^{-1}(V)\times U$ then $f(t)\in V$ and $f(t)\in U$, which would be a contradiction since $V$ and $U$ are disjoint. So we have found an open neighborhood of $(x,y)$ which does not intersect the graph. Since $(x,y)$ was arbitrary point not on the graph, the graph is closed. 

    \textbf{(b)} Let $V\subset Y$ be an open set, and pick an arbitrary $x\in f^{-1}(V)$. Clearly $\{x\}\times (Y-V)$ does not intersect $G_f$ so since the graph is closed, for every $(x,y)\in \{x\}\times (Y-V)$ there is some open set $U_y\ni (x,y)$ disjoint from $G_f$. So $\bigcup_{y\in Y-V}U_y$ is an open cover of $\{x\}\times (Y-V)$ disjoint from $G_f$. Note that $Y-V$ is compact since it is a closed subset of a compact space. Thus we can apply the tube lemma to find a tube of the form $U\times (Y-V)\subset \bigcup_{y\in Y-V}U_y$, where $U\subset X$ is an open neighborhood of $X$. Note that $U\times (Y-V)$ must be disjoint from $G_f$. We now claim that $U\subset f^{-1}(V)$. Indeed, for any $t\in U$, $f(t)$ must be in $V$ because otherwise $(t, f(t))\in U\times (Y-V)$ which would contradict the fact that $U\times (Y-V)$ is disjoint from $G_f$. So $U$ is an open neighborhood of $x\in f^{-1}(V)$. Since $x$ was arbitrary, it follows that $f^{-1}(V)$ is open and so $f$ is continuous. 

    \textbf{(c)} Let $X,Y=\R$ and define $f : \R \to \R$ as
    \[
        f(x)=\begin{cases}
            \frac{1}{x}&\textrm{if }x\neq 0\\ 0&\textrm{if }x=0
        \end{cases}
    .\] 
    Clearly, $f$ is not continuous at $x=0$, since $\lim_{x\to 0^-}f(x)=-\infty$ and $\lim_{x\to 0^+}f(x)=\infty$ yet $f(0)=0$. The graph of $f$ however is closed in $\R^2$ since it is a union of two curves and the origin point, which are both closed in $\R^2$.  
\end{solution}

\end{document}