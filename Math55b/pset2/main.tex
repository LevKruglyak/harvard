\documentclass[11pt,letterpaper]{article}

\input{../../../../.config/latex/preamble_v1.tex}
\lightmode

\title{\textbf{Math 55b Problem Set 2}}

\begin{document}
\maketitle

\hr
\begin{center}
    \textit{I collaborated with AJ LaMotta for this problem set, and it took me about 10 hours.}
\end{center}
\hr

% Problem 1
\begin{problem}
    The {\em Zariski topology} on $\R^2$ is the topology generated by the basis $$U_f=\{(x,y)\in \R^2\,|\,f(x,y)\neq 0\}$$ where $f$ ranges over all polynomials in $\R[x,y]$. 
    \begin{enumerate}[(a)]
        \item Show that the subsets $U_f\subset \R^2$ are indeed a basis for a
        topology.
        \item Show that this topology is not Hausdorff.
        \item What is the closure $\overline{I}$ of the line segment $I=\{(x,0)\,|\,x\in
        [0,1]\}\subset \R^2$ in the Zariski topology?
        \item If $\R\subset \R^2$ is the $x$-axis, show that the subspace topology
        on $\R$ induced by the Zariski topology is the finite complement topology.
    \end{enumerate}
\end{problem}

\begin{solution}
    \textbf{(a)} To prove that $U_f$ form a basis set for a topology, we need to show that the basis elements cover the whole space, and that the intersection of two basis elements contains a third basis element.  Clearly, the basis elements cover the whole space, just consider $U_1=\R^2$. Next, suppose $f_1, f_2\in \R[x,y]$. Then if $U_{f_1}\cap U_{f_2}\neq \emptyset$, it must contain some point $(x_0,y_0)$. We are done since $(x,y)=U_{(x-x_0)^2+(y-y_0)^2}$. 
    
    \textbf{(b)} If the topology were Hausdorff, then any subspace of it would also be Hausdorff. However by (d), $\R\subset \R^2$ has the finite complement topology, which isn't Hausdorff because any two open sets have nontrivial intersection. 
    
    \textbf{(c)} By (d), $\overline{I}\cap \R$ is equal to the closure of $I\cap \R$ in $\R$ with the finite complement topology. However in the finite complement topology, the only closed sets are collections of finite sets or the space itself. Thus $\overline{I}\cap \R$ must be $\R$ since it has an infinite number of points. However $\R$ is a closed subset of $\R^2$ because $\R = \R^2-U_y$. So $\overline{I}=\R=\{(x,0)\mid x\in R\}$.      
    
    \textbf{(d)} Clearly for any $f\in \R[x,y]$, $U_f\cap \R=\{x\in \R\mid f(x,0)\neq 0\}$. $f(x,0)$ is a polynomial $\R[x]$, so it either vanishes completely or has a finite number of roots. This is exactly an open set in the finite complement topology. Conversely, if $U=\R-\{x_1, x_2,\ldots,x_n\}$ is an open subset in the finite complement topology, note that $U_{(x-x_1)\cdots (x-x_n)}\cap \R = U$.     
\end{solution}

% Problem 2
\begin{problem}
    Let $x_1, x_2, x_3, \ldots$ be a sequence of points in a product space $\prod_\alpha X_\alpha$. Show that this sequence converges to a point $x$ if and only if the sequence $\pi_\alpha(x_1), \pi_\alpha(x_2), \ldots$ converges to $\pi_\alpha(x)$ for each $\alpha$. Is this fact true if one uses the box topology instead of the product topology? 
\end{problem}

\begin{solution}
    Suppose $x_1,x_2,x_3,\ldots$ is a sequence of points in $X=\prod_\alpha X_\alpha$ converging to some $x\in X$. Then since the projection maps $\pi_\alpha$ are continuous in both product and box topologies, so $\pi_\alpha(x_i)$ converge to $\pi_\alpha(x)$ in $X_\alpha$.
    
    Conversely, suppose $\pi_\alpha(x_i)$ converges to $\pi_\alpha(x)$ for all $\alpha$. Let $U$ be some open neighborhood of $x$, which contains some open set $\prod_\alpha U_\alpha$. By definition of the box topology, all but finitely many of the $U_\alpha$ are not equal to $X_\alpha$, say $U_{\alpha_1}, \ldots, U_{\alpha_n}$. Since $\pi_{\alpha_j}(x_i)$ converges to $\pi_{\alpha_j}(x)$, there exist $N_{\alpha_1},\ldots,N_{\alpha_n}$ such that $m\geq N_{\alpha_j}$ implies $\pi_{\alpha_j}(x_m) \in U_{\alpha_j}$. So letting $N=\max_j \{N_{\alpha_1},\ldots,N_{\alpha_n}\}$, it follows that $x_m\in \prod_\alpha U_\alpha$ for all $m\geq N$.     
    
    This argument fails in the box topology because the maximum of an infinite collection of integers need not exist, so if you had a sequence in an infinite product space which converges slower and slower in each coordinate, it may never converge in the box topology.
\end{solution}

% Problem 3
\begin{problem}
    Let $\R^\infty$ be the subset of $\R^\omega$ consisting of all sequences that are ``eventually zero,'' that is, all sequences $(x_1,x_2,\ldots)$ such that $x_i\neq 0$ for only finitely many values of $i$. What is the closure of $\R^\infty$ in $\R^\omega$ in the box and product topologies? Justify your answer.    
\end{problem}

\begin{solution}
    First, we claim that $\R^\infty$ is dense in $\R^\omega$ in the product topology, i.e. $\overline{\R^\infty}=\R^\omega$ . Recall that any open set in $\R^\omega$ is of the form $U=U_1\times U_2\times \cdots\times U_n\times \R \times\R \cdots$. Then, the sequence $x=(x_1,x_2,\cdots,x_n,0,0,\cdots)$ is in $U$ for some $x_i\in U_i$. But $x\in \R^\infty$, so every open set in $\R^\omega$ intersects $\R^\infty$ nontrivially, which implies density.

    In the box topology, $\R^\infty$ is closed, so $\overline{\R^\infty}=\R^\infty$. To show this suppose $x\not\in \R^\infty$, so there are an infinite number of terms in $x$ which are nonzero. Then
    \[
        U=\prod_i U_i\quad\textrm{where}\; U_i=\begin{cases}(x_i-|x_i| /2, x_i+|x_i| /2)&x_n\neq 0\\ (-1, 1)&x_n=0\end{cases}
    \]
    is an open set containing $x$ which is disjoint from $\R^\infty$. 
    
\end{solution}

% Problem 4
\begin{problem}
    Let $A$ be a proper subset of $X$, and let $B$ be a proper subset of $Y$. If $X$ and $Y$ are connected, show that
    \[
        (X\times Y) - (A\times B)
    \] 
    is connected. 
\end{problem}

\begin{solution}
    Suppose $F : (X\times Y) - (A\times B) \to \{0, 1\}$ is a continuous function. We claim that $F$ must be constant, since this will imply connectedness of $(X\times Y)-(A\times B)$. (If it was disconnected, then $F$ could take on different values for all of the connected components.) 
    
    Fix some $a\in X-A$ and $b\in Y-B$. Now consider an arbitrary point $(x,y)\in (X\times Y)-(A\times B)$. We have two cases. If $x\not\in A$, the set $\{x\}\times Y$ is connected so $F(x,y)=F(x,b)$. Similarly, $X\times \{b\}$ is connected, so $F(x,b)=F(a,b)$. So $F(x,y)=F(a,b)$. Otherwise, if $x\in A$ then $y\not\in B$, so we can use a similar argument to get $F(x,y)=F(a,b)$. Since $F(a,b)$ was fixed, it follows that $F$ is constant so $(X\times Y) - (A\times B)$ is connected.          
\end{solution}

% Problem 5
\begin{problem}
    Let $\R_\ell$ denote the real line with the {\em lower limit topology}, generated by the basis consisting of all intervals $[a,b)$, $a<b$. 
    \begin{enumerate}[(a)]
        \item Show that $\R_\ell$ is {\em totally disconnected}, i.e.\ its only
        (nonempty) connected subspaces are subsets consisting of a single point.
        \item Say a function $f:\R\to \R$ is {\em continuous from the right} (in the
        usual topology) if, $\forall a\in \R$, $\lim_{x\to a^+} f(x)=f(a)$, i.e.\ $\forall \epsilon>0$, $\exists \delta>0$ s.t.\ $a<x<a+\delta$ $\Rightarrow$ $|f(x)-f(a)|<\epsilon$.  Show that $f:\R\to\R$ is continuous from the right if and only if it is continuous when considered as a function from $\R_\ell$ to $\R$.
        \item What functions $f:\R\to\R$ are continuous when considered as maps from
        $\R$ to $\R_\ell$?  
        \item What can you say about functions which are continuous as maps from
        $\R_\ell$ to $\R_\ell$? 
    \end{enumerate}
\end{problem}

\begin{solution}
    \textbf{(a)} Note that we have open sets for any $b\in \R$: 
    \[
        (-\infty, b)=\bigcup_{a\leq b}[a,b),\quad [b,\infty)=\bigcup_{b\leq a}[b,a)
    .\] 
    Then any set $X$ containing more than a single element, say $a< b$ can be separated by disjoint open sets of the form $a\in(-\infty,b)$ and $b\in [b,\infty)$.
    
    \textbf{(b)} First suppose $f : \R \to \R$ is continuous from the right, and fix $a\in \R$. For each $\epsilon>0$ there exists a $\delta> 0$ such that for all $x\in \R$ with $a < x < a+\delta$, we have $|f(x)-f(a)|< \epsilon$. This implies that $f([a, a+\delta)) \subset B_\epsilon(f(p))$, so $f^{-1}((a,b))$ is open in the lower limit topology. Conversely suppose $f : \R_\ell \to \R$ is continuous. Then for any $a\in \R$ and $\epsilon > 0$, by continuity we have $f^{-1}((a - \epsilon, a + \epsilon))$ is open in $\R_\ell$. Pick some basis element $[x, y)\subset f^{-1}((a-\epsilon, a+\epsilon))$. containing $a$. Then we can find some $\delta$ such that $[a,a+\delta)\subset [x,y)$, so for any $x\in [a,a+\delta)$, $|f(x)-f(a)| < \epsilon$ by construction. This is exactly a right continuous function.        
    
    \textbf{(c)} Note that any continuous map $f : \R \to \R_\ell$ must map connected sets to connected sets, so $f(\R)$ is a connected set in $\R_\ell$. However the only connected sets in $\R_\ell$ are single points, so $f$ is a constant function.
    
    \textbf{(d)} Suppose $f : \R_\ell \to \R_\ell$ is some continuous function. By (b), since $\R_\ell$ is finer than $\R$ so any continuous function $f : \R_\ell \to \R_\ell$ is also continuous $f : \R_\ell \to \R$. So $f$ must be right continuous by (b).

    Note that if $f : \R_\ell \to \R_\ell$ is continuous, then  for any $x\in \R$ and $\epsilon > 0$, there exists a $\delta > 0$ such that for any $y\in \R$ satisfying $x\leq y<x+\delta$ we have $f(x)\leq f(y)<f(x)+\epsilon$. So $f$ must be monotonically increasing. This is a necessary but not sufficient condition.
\end{solution}

% Problem 6
\begin{problem}\noindent
    \begin{enumerate}[(a)]
        \item Show that no two of the spaces $(0, 1)$, $(0, 1]$, and $[0, 1]$ are homeomorphic.
        \item Suppose there exist embeddings $f : X \to Y$ and $g : Y \to X$. Show by means of an example that $X$ and $Y$ need not be homeomorphic.
        \item Show that $\R^n$ and $\R$ are not homeomorphic if $n>1$.  
    \end{enumerate}
\end{problem}

\begin{solution}
    We'll introduce a powerful invariant to help us answer these questions.
    \begin{claim}
        Let $X$ be a topological space. Let
        \[
            n(X)=\{ x\in X \mid X-\{x\}\textrm{ is path connected}\;\}
        .\]  
        Then if two spaces $X,Y$ are homeomorphic, their associated sets $n(X), n(Y)$ are bijective. 
    \end{claim}
    
    \begin{proof}
        We'll show that there are injections $\iota_1 : n(X) \to n(Y)$ and $\iota :  n(Y) \to n(X)$. Indeed, suppose $x\in n(X)$. Then $f(X-\{x\})=f(X)-\{f(x)\}= Y-\{f(x)\}$ since $f$ is a bijection. However $f$ is also continuous so this image is path connected. Thus $f(x)\in n(Y)$. This gives an injection $n(X)\to n(Y)$ induced by $f$. We can do the same thing for $f^{-1}$, and so there must be a bijection $n(X) \isom n(Y)$.
    \end{proof}

    \textbf{(a)} Observe that $n(0,1)=\emptyset$, $n(0,1]=\{1\}$, and $n[0,1]=\{0,1\}$, hence none of them can be homeomorphic because any homeorphism would induce bijections of their associated sets.
    
    \textbf{(b)} Consider $(0,1) \hookrightarrow [0,1]$ embedded in the natural way, and $[0,1] \hookrightarrow (0,1)$ embedded by the map sending $[0,1] \to [1 /3 ,2 /3]\subset (0,1)$. However these spaces aren't homeomorphic by (a).
    
    \textbf{(c)} Using our invariant, note that
    \[
        n(\R^n)=\begin{cases}
            \R & n=1\\
            \emptyset & n > 1
        \end{cases}
    \]  
    because any point from $\R^n$ clearly leaves it path connected for $n>1$, whereas removing a point $x$ from $\R$ leaves it split into two components, $(-\infty, x)\cup (x,\infty)$.
\end{solution}

% Problem 7
\begin{problem}
    Let $f : S^1 \to \R$ be a continuous map. Show that there exists a point $x\in S^1$ such that $f(x)=f(-x)$.  
\end{problem}

\begin{solution}
    Consider the function $g(x)=f(x)-f(-x)$. Now pick some $x\in S^1$. If $g(x)=0$, we are done since that would mean that $f(x)=f(-x)$. Otherwise, assume without loss of generality that $g(x)>0$. Then $g(-x)=f(-x)-f(x)=-g(x)<0$. 
    
    Now reparameterizing the input of $g$ so that it can be considered as a function $\overline{g}(t) : [0,1] \to \R$ with $\overline{g}(0)=g(x)$ and $\overline{g}(1)=g(-x)$. Recall that $\overline{g}(0) > 0$ and $\overline{g}(1) < 0$. Thus by the intermediate value theorem there must be a point $t\in [0,1]$ such that $\overline{g}(t)=0$. This corresponds to a point $y\in S^1$ satisfying $f(y)=f(-y)$ so we are done.             
\end{solution}

% Problem 8
\begin{problem}
    Let $f : X \to X$ be a continuous map. Show that if $X=[0,1]$, there is a point $x\in X$ such that $f(x)=x$. The point $x$ is called a \emph{fixed point} of $f$. What happens if $X=[0,1)$ or $(0,1) $?  
\end{problem}

\begin{solution}
    Let $g(x)=f(x)-x$. Then $g(0)=f(0) \geq 0$ and $g(1)=f(1)-1 \leq 0$ so by the intermediate value theorem there must be an $t\in [0,1]$ such that $g(t)=0$. Then $f(t)=t$ so we are done.    
\end{solution}

% Problem 9
\begin{problem}
    Show that every compact subspace of a metric space is bounded in that metric and is closed. Find a metric space in which not every closed bounded subspace is compact.
\end{problem}

\begin{solution}
    Let $C$ be a compact subspace of a metric space $(X,d)$. It is clearly closed because compact subspaces are closed in Hausdorff spaces, and every metric space is Hausdorff. To prove that is is bounded, write 
    \[
        C \subset \bigcup_{x\in C} B_\epsilon(x)\quad \textrm{for some }\epsilon>0
    .\] 
    Since $C$ is compact and this is an open cover, there must be some finite collection of points $x_1,x_2,\ldots,x_n\in C$ such that $C \subset B_\epsilon(x_1)\cup \cdots \cup B_\epsilon(x_n)$. Now let $x\in C$ be arbitrary. Then $C \subset B_r(x)$ where $r=\epsilon + \max_{1\leq i\leq n} d(x,x_i)$ by the triangle inequality. So $C$ is bounded by a ball of radius $r$.

    To show an example of a metric space where every closed bounded subspace is not compact, consider  any infinite set $X$ equipped with the discrete metric, so every set is open. Note that every set is closed and bounded in this space. However any infinite subset $S$ cannot be compact since $S=\bigcup_{s\in S}\{s\}$ is an open cover of $S$ with no finite subcover. 
\end{solution}

% Problem 10
\begin{problem}
    Let $A$ and $B$ be disjoint compact subspaces of the Hausdorff space $X$. Show that there exist disjoint open sets $U$ and $V$ containing $A$ and $B$, respectively.
\end{problem}

\begin{solution}
    First we'll prove that for any point $a\in A$, there exists open disjoint sets $U\ni a$ and $V\supset B$. Indeed, since $X$ is Hausdorff, we have disjoint open sets $U_b\ni a$ and $V_b\ni b$ for every $b\in B$. Considering $\{V_b\}_{b\in B}$ as an open cover of $B$, there must be some finite sub collection of the sets also covering $B$ by compactness, say $V_{b_1}, V_{b_2}, \ldots, V_{b_n}$. Then $U=\bigcap_i U_{b_i}$ and $V=\bigcup_i V_{b_i}$ satisfy the required conditions.       
    
    Now we can construct disjoint open $U\supset A$ and $V\supset B$. By the first paragraph, we have disjoint open sets $U_a\ni a$ and $V_a\supset B$ for every $a\in A$. Since $\{U_a\}_{a\in A}$ is an open cover of $A$, there must be some finite subcover, say $U_{a_1}, U_{a_2}, \ldots, U_{a_n}$. Then $U=\bigcup_i U_{a_i}$ and $V=\bigcap_i V_{a_i}$ satisfy the conditions of the problem.   
\end{solution}

% Problem 11
\begin{problem}
    Let $X$ be the union of $\R^n$ and one additional point called $\infty$. Consider the topology with basis given by open balls in $\R^n$ plus the sets $U_r=\{\infty\}\cup \{x\in \R^n\mid |x|>r\}$.  Show that $X$ is a compact Hausdorff space.
\end{problem}

\begin{solution}
    To show that the space is Hausdorff, it suffices to show that $\infty$ can be separated from any point in $\R^n$ by disjoint open sets, since any two distinct points in $\R^n$ can already be separated by disjoint open sets. Let $x\in \R^n$. Then letting $U_r(x)=\{\infty\} \cup \{x\in \R^n\mid |x|>r\}$ and choosing some $\epsilon > 0$ we have $\infty \in U_\epsilon(x)$ and $x\in B_{\epsilon}(x)$, and these sets are clearly disjoint. 
    
    To prove compactness, let $\mathcal{U}$ is some open cover of $X$. Pick some $V \in \mathcal{U}$ such that $\infty\in V$. We can assume without loss of generality that $V$ is a basis element (otherwise split it into a union of basis elements), so $V=X-C$ for some compact ball $C$. So when restricted to $\R^n$ the set $\mathcal{U}-\{V\}$ is an open cover of $C$. (Clearly any set of the form $\R^n-C_1$ is open since $\R^n$ is a Hausdorff space so any compact $C_1$ is closed) Since $C$ is compact, there is some finite subcover $\mathcal{V}\subset \mathcal{U} - \{V\}$. The $\mathcal{V}\cup \{V\}$ is an open cover of $X$.  
\end{solution}

\end{document}