\documentclass[11pt,letterpaper]{article}

\input{../../../../.config/latex/preamble_v1.tex}
\lightmode

\title{\textbf{Math 55b Problem Set 5}}

\begin{document}
\maketitle

\begin{problem}\noindent
    \begin{enumerate}[(a)]
        \item Let $A\subset X$; suppose $r : X \to A$ is a continuous map such that $r(a)=a$ for each $a\in A$. (The map $r$ is called a \emph{retraction} of $X$ onto $A$.) If $a_0\in A$, show that 
        \[
            r_* : \pi_1(X,a_0) \to \pi_1(A, a_0)
        \] 
        is surjective.
        \item Let $A$ be a subspace of $\R^n$ and let $h : (A, a_0) \to (Y, y_0)$. Show that if $h$ is extendable to a continuous map of $\R^n$ into $Y$, then $h_*$ is the trivial homomorphism. 
    \end{enumerate}
\end{problem}

\begin{solution}
    \textbf{(a)} Let $\sigma : I \to A$ be a loop in $A$ based at $a_0$, and let $\iota_A : A \to X$ be the inclusion map. Then $\iota_A \circ \sigma : I \to X$ is a loop in $X$ based at $a_0$, and it follows that $r_*(\iota_A \circ \sigma) = r\circ \iota_A\circ \sigma$. However by definition of a retraction, $r\circ \iota_A=\id_A$, so $r_*(\iota_A\circ \sigma)=\sigma$. Thus $r_*$ is surjective.

    \textbf{(b)} Suppose $h$ were extendable to a continuous map of $\R^n$ into $Y$, say $\overline{h} : \R^n \to Y$. Then we have the commutative diagrams:
    \begin{center}
        % https://tikzcd.yichuanshen.de/#N4Igdg9gJgpgziAXAbVABwnAlgFyxMJZABgBpiBdUkANwEMAbAVxiRAAoBBUgAjoH1iAShABfUuky58hFAEZyVWoxZt2ATV4BPQSPGTseAkTJyl9Zq0QcAOjYC2dHAAsARq+AAlUQD0wpAWExCRAMQxkiAGZFagtVazs0LH45LgDdYINpYxQAFhjlSzZE5NTNHSD9UKkjWWRos1iVKxASlPY7Rxd3L19-QL0lGCgAc3giUAAzACcIeyQyEBwIJAAmagY6VxgGAAUaiOtprBHnHBAmooSbfBwBTkyQGbmF6mWkBUL4kGdH5-nEOslitEJ9Nts9gcciBjqdzpdvnYIDQYNMGFgwDBgM5RH9ZgDosCkABWBEtDo3CB3ficIT8ABUFxA4J2+3C0NhZzxL0QhPeiHyXxazgZ3IBpKJArJxRsyNR6Mx2NEoo2W1ZUNkMJOXNEFFEQA
        \begin{tikzcd}
            {(A, a_0)} \arrow[d, "\iota_A"'] \arrow[r, "h"]  & {(Y, y_0)} &  & {\pi_1(A,a_0)} \arrow[d, "(\iota_A)_*"'] \arrow[r, "h_*"] & {\pi_1(Y,y_0)} \\
            {(\mathbb{R}^n,a_0)} \arrow[ru, "\overline{h}"'] &            &  & {\pi_1(\mathbb{R}^n,a_0)} \arrow[ru, "\overline{h}_*"']   &               
        \end{tikzcd}
    \end{center}
    However since $\R^n$ is contractible, $\pi_1(\R^n, a_0)=\{*\}$ so $\overline{h}_*$ is the trivial homomorphism and by extension $h_* = \overline{h}_*\circ (\iota_A)_*$ must be the trivial homomorphism.
\end{solution}

\begin{problem}
    Let $p : E \to B$ be a covering map; let $B$ be connected. Show that if $p^{-1}(b_0)$ has $k$ elements for some $b_0\in B$, then $p^{-1}(b)$ has $k$ elements for every $b\in B$. In such a case, $E$ is called a \emph{$k$-fold covering} of $B$.
\end{problem}

\begin{solution}
    Consider the map $\Omega : B \to \Z_{\geq 0}$ where $\Omega(b)=|p^{-1}(b)|$ if $p^{-1}(b)$ is finite and $0$ otherwise. We claim that $\Omega$ is continuous if $\Z_{\geq 0}$ is given the discrete topology. If suffices to prove that $\Omega^{-1}(n)$ is open for $n\in \Z_{\geq 0}$. Suppose $b\in \Omega^{-1}(n)$. By definition of a covering space, there is some evenly covered neighborhood $U\ni b$ such that $p^{-1}(U)=\bigsqcup_{i\in p^{-1}(b)}U_i$ where $\restr{p}{U_i} : U_i \to U$ is a homeomorphism for all $i\in p^{-1}(b)$. This implies that for any $b'\in U$, there is a bijection $p^{-1}(b) \to p^{-1}(b')$ so $\Omega(b)=\Omega(b')$ and we are done. Since $B$ is connected, $\Omega(B)$ must be a connected subset of $\Z_{\geq 0}$, implying that any $b\in B$ has $|p^{-1}(b)|=k$.
\end{solution}

\begin{problem}
    Let $q : X \to Y$ and $r : Y \to Z$ be covering maps; let $p=r\circ q$. Show that if $r^{-1}(z)$ is finite for each $z\in Z$, then $p$ is a covering map.
\end{problem}

\begin{solution}
    Clearly $p$ is continuous and surjective since it is a composition of two continuous surjective maps. To prove the even covering condition, let $z\in Z$ be an arbitrary point. Pick some neighborhood $U\subset Z$ of $z$ which is evenly covered by $r$, so there is some family $r^{-1}(U)=\bigsqcup_{i=0}^nU_i$ for $U_i\subset Y$ with homeomorphisms $r_i : U_i \to U$. Next, let $y_i\in U_i$ be the corresponding element in the fiber $y_i\in r^{-1}(z)$. Then since $q$ is a covering map there is a neighborhood $y_i\in V_i\subset U_i$ such that $q^{-1}(V_i)=\bigsqcup_{j\in J_i}V_{i,j}$ and homeomorphisms $q_{i,j} : V_{i,j} \to V_i$.
    
    Now let $W=\bigcap_{i=0}^nr(V_i)$. This is open an open neighborhood of $z$. (Here we use the finiteness condition) Now let $W_i=r_i^{-1}(W)$. Note that this is an open neighborhood or $y_i$ contained in $V_i$, so we have $q^{-1}(W_i)=\bigsqcup_{j\in J_i}W_{i,j}$ with homeomorphisms $q_{i,j} : W_{i,j} \to W_i$. Composing these with $r_i$ gives homomorphisms $r_i\circ q_{i,j} : W_{i,j} \to W$. So $W$ is a $p$ evenly covered neighborhood of $z$ since $p^{-1}(W) = \bigsqcup_{i=0}^n\bigsqcup_{j\in J_i}W_{i,j}$, and $p$ restricts to $q_{i,j}\circ r_i$ on the $W_{i,j}$. So $p$ is a covering map.
\end{solution}

\begin{problem}
    The M\"obius band $B$ is the quotient of $[0, 1] \times [0, 1]$ by the relation $(0, y) \sim (1, 1 - y)$. Show that the cylinder $S^1 \times [0, 1]$ is a degree 2 (that is, 2-sheeted) covering space of $B$.
\end{problem}

\begin{solution}
    Define the covering map $p : [0,1]\times [0,1] \to B$ by
    \[
        p(t,x) = \begin{cases}
            \left(2t,x\right)&0\leq t\leq \frac{1}{2}\\
            \left(2t-1,1-x\right)&\frac{1}{2}\leq t\leq 1
        \end{cases}.  
    \]  
    This can be passed to the canonical quotient map which ``closes the cylinder'' so we get a map $\widetilde{p} : S^1\times [0,1] \to B$ since $p(0,x)=(0,x)$ and $p(1,x)=(1,1-x)=(0,x)$ in the M\"{o}bius band. This clearly satisfies the conditions of a covering map:   
    
    \begin{center}
        \begin{tikzpicture}
            \begin{scope}[very thick, xshift=-3cm, yshift=-1cm]
                \draw[fill=foreground!20!background] (0,0) -- (2,0) -- (2,4) -- (0,4) -- cycle;
                \draw[red, dashed, fill=red!20] (1.3,4) -- (1.3,4) arc(180:360:0.3) --cycle; 
                \draw[red, dashed, fill=red!20] (1.6,0) -- (1.9,0) arc(0:180:0.3) --cycle; 
                \draw[red, dashed, fill=red!20] (0.4,2) circle (0.3);
                \draw[blue, dashed, fill=blue!20] (1.4,2.8) ellipse (0.3 and 0.3);
                \draw[blue, dashed, fill=blue!20] (0.6,0.8) ellipse (0.3 and 0.3);

                \draw[postaction={decorate}, decoration={
                    markings,
                    mark=at position 0.55 with {\arrow{latex}}}
                ] (0,0) -- (2,0) {};
                \draw[postaction={decorate}, decoration={
                    markings,
                    mark=at position 0.55 with {\arrow{latex}}}
                ] (0,4) -- (2,4) {};
                \draw[postaction={decorate}, decoration={
                    markings,
                    mark=at position 0.60 with {\arrow{latex}}}
                ] (2,2) -- (0,2) {};
                \draw[] (2,0) -- (2,4) {};
                \draw[] (0,0) -- (0,4) {}; 
                
                \fill[blue] (0.6,0.8) circle (0.05);
                \fill[blue] (1.4,2.8) circle (0.05);
                
                \fill[red] (1.6,0) circle (0.05);
                \fill[red] (0.4,2) circle (0.05);
                \fill[red] (1.6,4) circle (0.05);

                \node at (1,-0.5) {Cylinder $[0,1]\times S^1$};
            \end{scope}
            \draw [-latex,very thick] (0,1) -- node[above] {$\widetilde{p}$} (2,1);
            \begin{scope}[very thick, xshift=3cm]
                \draw[fill=foreground!20!background] (0,0) -- (2,0) -- (2,2) -- (0,2) -- cycle;
                \draw[blue, dashed, fill=blue!20] (0.6,0.8) ellipse (0.3 and 0.3);
                \draw[red, dashed, fill=red!20] (1.6,0) -- (1.9,0) arc(0:180:0.3) --cycle; 
                \draw[red, dashed, fill=red!20] (0.1,2) -- (0.1,2) arc(180:360:0.3) --cycle; 

                \draw[postaction={decorate}, decoration={
                    markings,
                    mark=at position 0.55 with {\arrow{latex}}}
                ] (0,0) -- (2,0) {};
                \draw[postaction={decorate}, decoration={
                    markings,
                    mark=at position 0.60 with {\arrow{latex}}}
                ] (2,2) -- (0,2) {};
                \draw[] (2,0) -- (2,2) {};
                \draw[] (0,0) -- (0,2) {};
                \fill[blue] (0.6,0.8) circle (0.05);
                \fill[red] (1.6,0) circle (0.05);
                \fill[red] (0.4,2) circle (0.05);
                \node at (1,-0.5) {M\"{o}bius strip $B$};
            \end{scope}
        \end{tikzpicture}
    \end{center}
    
    In particular, this can clearly be seen to be a two sheeted covering.
\end{solution}

\begin{problem}
    Let $p : E \to B$ be a covering map, with $E$ path connected. Show that if $B$ is simply connected, then $p$ is a homeomorphism.
\end{problem}

\begin{solution}
    Recall from Munkres that if $p : E\to B$ with $E$ path connected, then there is a surjection $\phi : \pi_1(B,b_0) \to p^{-1}(b_0)$ for every $b_0\in B$. However if $B$ is simply connected then $\pi_1(B,b_0)$ is trivial, so surjectivity of $\phi$ implies that $p^{-1}(b_0)$ consists of a single element for all $b_0\in B$. This means that $p$ is a bijection. To prove it's a homeomorphism, it then suffices to show that its inverse is continuous. So let $U\subset E$ be open, and pick some $b\in p(U)$. Then there is an evenly covered neighborhood $V\ni b$ such that $p : p^{-1}(V) \to V$ is a homemorphism. Note that $b\in p(U\cap p^{-1}(V))=p(U)\cap V \subset p(U)$ so $p(U)$ is open. This completes the proof. 
\end{solution}

\begin{problem}\noindent
    \begin{enumerate}[(a)]
        \item Show that if $A$ is a retract of $B^2$, then every continuous map $f : A \to A$ has a fixed point.
        \item Show that if $h : S^1 \to S^1$ is nullhomotopic, then $h$ has a fixed point and $h$ maps some point $x$ to its antipode $-x$. 
    \end{enumerate}
\end{problem}

\begin{solution}
    \textbf{(a)} Let $r : B^2 \to A$ be a retraction map and let $\iota_A : A \to B^2$ be the inclusion map. then $f'=\iota_A \circ f \circ r : B^2 \to B^2$ is a continuous map so it must have some fixed point $f'(x)=x$ for $x\in B^2$. However $f'(B^2)\subset A$ so $x\in A$. Thus $f'(x)=f(x)=x$ so $f$ has a fixed point. 

    \textbf{(b)} Recall that a map $h : S^1 \to S^1$ is nullhomotopic if and only if it can be extended to a continuous map $\widetilde{h} : B^2 \to S^1$. Suppose for the sake of contradiction that $h$ does not a fixed point. Clearly if $h$ has no fixed points then $\widetilde{h}$ has no fixed points, and so $\iota_{S_1}\circ \widetilde{h} : B^2 \to B^2$ has no fixed points. This is a violation of the Brouwer fixed point theorem, so $h$ must have a fixed point. For the second part, note that $-h : S^1 \to S^1$ is nullhomotopic as well, so there must be some $x\in S^1$ with $-h(x)=x$, so $h(x)=-x$.
\end{solution}

\begin{problem}
    For each of the following spaces, the fundamental group is either trivial, infinite cyclic, or isomorphic to the fundamental group of the figure eight. Determine for each space which of the three alternatives holds.
    \begin{enumerate}[(a)]
        \item The ``solid torus'', $B^2\times S^1$.
        \item The torus $T$ with a point removed.
        \item The cylinder $S^1\times I$.
        \item The infinite cylinder $S^1\times \R$.
        \item $\R^3$ with nonnegative $x,y,$ and $z$ axes deleted.
        \item $\{x\in \R^2\mid \|x\|>1\}$.
        \item $\{x\in \R^2\mid \|x\|\geq 1\}$.
        \item $\{x\in \R^2\mid \|x\|<1\}$.
        \item $S^1\cup (\R_+\times 0)$.
        \item $S^1\cup (\R_+\times \R)$.
        \item $S^1\cup (\R\times 0)$.
        \item $\R^2-(\R_+\times 0)$.
    \end{enumerate}
\end{problem}

\begin{solution}
    \textbf{(a)} Let $r : B^2 \to \{0\}$ be the deformation retraction of $B^2$ to its center. This is a homotopy equivalence since $B^2$, so $B^2\times S^1$ is homotopy equivalent to $\{0\}\times S^1=S^1$. Thus $\pi_1(B^2\times S^1)=\pi_1(S^1)=\Z$.
    
    \begin{center}
        \begin{tikzpicture}
            \begin{scope}[very thick, yscale=cos(70)]
                %\draw (-0.5,0) arc (175:315:1cm and 0cm);
                %\draw (3,-0.28) arc (-30:180:0.7cm and 0.3cm);
                \draw[double distance=5mm, double=foreground!20!background] (0:1) arc (0:180:1);
                \draw[double distance=5mm, double=foreground!20!background] (180:1) arc (180:360:1);
            \end{scope}
            \draw[-latex, very thick] (2,0) -- node[above] {$r\times \id_{S^1}$} (4.5,0); 
            \begin{scope}[very thick, xshift=6cm, yscale=cos(70)]
                \draw[] (0,0) ellipse (1 and 1);
            \end{scope}
        \end{tikzpicture}
    \end{center}
    
    \textbf{(b)} Consider the torus $T$ as the quotient of the unit square obtained by gluing opposite sides together. Let's call this quotient map $q_1 : I\times I \to T$, the corresponding homeomorphism $\widetilde{q_1} : (I\times I /\sim) \isom T$. Then removing a point $P$ from the torus gives us a quotient map $q_2 : I\times I-\{p\} \to T-\{P\}$ and homeomorphism $\widetilde{q_2} : (I\times I-\{p\} / \sim) \isom T-\{P\}$. Here $p=(0.5,0.5)$.

    \begin{center}
        \begin{tikzpicture}
            \begin{scope}[very thick, xshift=-3cm]
                \draw[fill=foreground!20!background] (0,0) -- (2,0) -- (2,2) -- (0,2) -- cycle;
                \draw[postaction={decorate}, decoration={
                    markings,
                    mark=at position 0.55 with {\arrow{latex}}}
                ] (0,0) -- (2,0) {};
                \draw[postaction={decorate}, decoration={
                    markings,
                    mark=at position 0.55 with {\arrow{latex}}}
                ] (0,2) -- (2,2) {};
                \draw[postaction={decorate}, decoration={
                    markings,
                    mark=at position 0.55 with {\arrow{latex}},
                    mark=at position 0.65 with {\arrow{latex}}}
                ] (2,0) -- (2,2) {};
                \draw[postaction={decorate}, decoration={
                    markings,
                    mark=at position 0.55 with {\arrow{latex}},
                    mark=at position 0.65 with {\arrow{latex}}}
                ] (0,0) -- (0,2) {};
                \fill[fill=background] (1,1) circle (0.1);
                \node at (1,-0.5) {Stage 0};
            \end{scope}
            \begin{scope}[very thick]
                \draw[fill=foreground!20!background] (0,0) -- (2,0) -- (2,2) -- (0,2) -- cycle;
                \draw[postaction={decorate}, decoration={
                    markings,
                    mark=at position 0.55 with {\arrow{latex}}}
                ] (0,0) -- (2,0) {};
                \draw[postaction={decorate}, decoration={
                    markings,
                    mark=at position 0.55 with {\arrow{latex}}}
                ] (0,2) -- (2,2) {};
                \draw[postaction={decorate}, decoration={
                    markings,
                    mark=at position 0.55 with {\arrow{latex}},
                    mark=at position 0.65 with {\arrow{latex}}}
                ] (2,0) -- (2,2) {};
                \draw[postaction={decorate}, decoration={
                    markings,
                    mark=at position 0.55 with {\arrow{latex}},
                    mark=at position 0.65 with {\arrow{latex}}}
                ] (0,0) -- (0,2) {};
                \draw[dotted, fill=background] (1,1) circle (0.5);
                \node at (1,-0.5) {Stage 1};
            \end{scope}
            \begin{scope}[very thick, xshift=3cm]
                \draw[postaction={decorate}, decoration={
                    markings,
                    mark=at position 0.55 with {\arrow{latex}}}
                ] (0,0) -- (2,0) {};
                \draw[postaction={decorate}, decoration={
                    markings,
                    mark=at position 0.55 with {\arrow{latex}}}
                ] (0,2) -- (2,2) {};
                \draw[postaction={decorate}, decoration={
                    markings,
                    mark=at position 0.55 with {\arrow{latex}},
                    mark=at position 0.65 with {\arrow{latex}}}
                ] (2,0) -- (2,2) {};
                \draw[postaction={decorate}, decoration={
                    markings,
                    mark=at position 0.55 with {\arrow{latex}},
                    mark=at position 0.65 with {\arrow{latex}}}
                ] (0,0) -- (0,2) {};
                \node at (1,-0.5) {Stage 2};
            \end{scope}
            \begin{scope}[very thick, xshift=6cm]
                \draw (1.5,0.2) -- (1.5,1.8) {};
                \draw[postaction={decorate}, decoration={
                    markings,
                    mark=at position 0.80 with {\arrow{latex}}}
                ] (0.75,0.2) ellipse (0.75 and 0.2);
                \draw[postaction={decorate}, decoration={
                    markings,
                    mark=at position 0.80 with {\arrow{latex}}}
                ] (0.75,1.8) ellipse (0.75 and 0.2);
                \node at (1,-0.5) {Stage 3};
            \end{scope}
            \begin{scope}[very thick, xshift=9cm]
                \draw (0.25,1) ellipse (0.75 and 0.4);
                \draw (1.75,1) ellipse (0.75 and 0.4);
                \draw[fill=foreground] (1,1) circle (0.05);
                \node at (1,-0.5) {Stage 4};
            \end{scope}
        \end{tikzpicture}
    \end{center}
    
    Then by composing the quotient maps with the deformation retraction of $I\times I-\{p\}$ onto $\partial(I\times I)$ gives us the transformation between Stage~0 and Stage~2. The induced quotient map $q_3 : \partial(I\times I) \to T-\{P\}$ then gives us an isomorphism $\widetilde{q_3} : (\partial(I\times I)/\sim)\isom T-\{P\}$. However it can be easily seen that $(\partial (I\times I) /\sim)\cong S^1\vee S^1$, the figure eight space. Since all of these stages were homotopy equivalences, it follows that $\pi_1(T-\{P\})\cong \pi_1(S^1\vee S^1)\cong \Z*\Z$.   
    
    \textbf{(c)} Note that since $I$ is contractible, there is a deformation retraction of $I$ onto a single point, so $I$ is homotopy equivalent to a point. Thus $S^1\times I$ is homotopy equivalent to $S^1\times \{0\}=S^1$  and so $\pi_1(S^1\times I)=\Z$.
    
    \textbf{(d)} The same thing happens here, $\R$ is contractible so $\pi_1(S^1\times \R)=\pi_1(S^1)=\Z$.

    \textbf{(e)} Let $r : \R^3-\{0\} \to S^2$ be the deformation retraction taking $x\mapsto \frac{x}{\|x\|}$. By composing with some obvious inclusions, this passes to a deformation retraction $r' : \R^3-P \to \S^3-\{p_1, p_2, p_3\}$ where $P$ is the union of the positive $x$, $y$, $z$ axes. So $\R^3-P$ is homotopy equivalent to the triple punctured sphere. We can then use stereographic projection to homeomorphically map this punctured sphere to a plane with two holes: (Stage 0 to Stage 1)

    \begin{center}
        \begin{tikzpicture}
            \begin{scope}[very thick]
                \draw [dashed] (-2,-0.8) -- (1.5,-0.8) -- (2.5,0.4) -- (-1,0.4) -- cycle;
                \draw [fill=background] (0,0.8) circle (0.8);

                \draw [dashed] (0,1.6) -- (-0.4,-0.4);
                \draw [dashed] (0,1.6) -- (1.6,0);
                
                \draw (-0.8,0.8) arc (180:360:0.8 and 0.2);
                \draw[dotted] (-0.8,0.8) arc (180:360:0.8 and 0.2);

                \draw [fill=background, thick] (0,1.55) circle (0.1);
                \draw [fill=background, thick] (0.8,0.75) circle (0.1);
                \draw [fill=background, thick] (-0.2,0.6) circle (0.1);
                
                \draw [fill=background, thick] (-0.4, -0.4) ellipse (0.1 and 0.05);
                \draw [fill=background, thick] (1.6, 0) ellipse (0.1 and 0.05);
                
                \node at (0,-1.5) {Stage 0};
            \end{scope}
            
            \begin{scope}[very thick, xshift=5cm]
                \draw [] (-2,-0.8) -- (1,-0.8) -- (2,0.4) -- (-1,0.4) -- cycle;
                \draw [] (-0.75,-0.2) ellipse(0.2 and 0.1);
                \draw [] (0.75,-0.2) ellipse (0.2 and 0.1);

                \node at (0,-1.5) {Stage 1};
            \end{scope}
            
            \begin{scope}[very thick, xshift=9cm]
                \draw (0.25,0) ellipse (0.75 and 0.4);
                \draw (1.75,0) ellipse (0.75 and 0.4);
                \draw[fill=foreground] (1,0) circle (0.05);
                
                \node at (1,-1.5) {Stage 2};
            \end{scope}
        \end{tikzpicture}
    \end{center}
    
    Then we can deformation retract this twice punctured plane onto the figure eight space, with each loop of the space going around one of the holes in the plane. This sequence of homotopy equivalences and homeomorphisms gives $\pi_1(\R^3-P)=\pi_1(S^1\vee S^1)=\Z*\Z$. 

    \textbf{(f)} This is the set $\R^2-\overline{B_1(x)} = \{x\in \R^2 \mid \|x\|>1\}$. Consider the deformation retraction of $\R^2-\overline{B_1(x)} \to \partial B_2(x)$ given by $x\mapsto \frac{2x}{\|x\|}$. This is clearly continuous and a deformation retraction, so $\pi_1(\R^2-\overline{B_1(x)})=\pi_1(\partial B_2(x))=\pi_1(S^1)=\Z$.
    
    \textbf{(g)} Very similarly this is the set $\R^2-B_1(x)=\{x\in \R^2 \mid \|x\|\geq 1\}$, so the same deformation retraction from (f) works here. Thus $\pi_1(\R^2-B_1(x))=\pi_1(S^1)=\Z$.
    
    \textbf{(h)} This is the set $B_1(x)=\{x\in \R^2\mid \|x\|<1\}$. This is just the standard open unit circle in $\R^2$, so it is contractible and hence $\pi_1(B_1(x))=\pi_1(\{*\})=\{0\}$.
    
    \textbf{(i)} Let $r : \R_+ \to \{1\}$ be a deformation retraction. Then we get a deformation retraction $r' : S^1\cup (\R_+\times 0) \to S_1$ defined by setting $r'(x,y)=(x,y)$ for $(x,y)\in S^1$ and $r'(x,y)=r(x)$ for $(x,0)\in \R_+\times 0$. This is clearly continuous since both functions agree on $z=(1,0)$. It's also quite easy to see that this is a deformation retraction:

    \begin{center}
        \begin{tikzpicture}
            \begin{scope}[very thick]
                \draw (0,0) circle (1.5);
                \draw [thin, -latex] (0.2,0.2) -- (1.3,0.2);
                \draw [thin, latex-] (1.7,0.2) -- (2.8,0.2);
                \draw [-latex] (0,0) -- (3,0);
                \draw[fill=foreground] (1.5,0) circle (0.05);
                \draw[fill=background] (0,0) circle (0.08);
                \node[] at (0,-2) {$S^1\cup (\R_+\times 0)$};
            \end{scope}
            
            \begin{scope}[very thick, xshift=5cm]
                \draw (0,0) circle (1.5);
                \draw[fill=foreground] (1.5,0) circle (0.05);
                \node[] at (0,-2) {$S^1$};
            \end{scope}
        \end{tikzpicture}
    \end{center}
    
    \textbf{(j)} In the space $S^1\cup (\R_+ \times \R)$, given any deformation retraction of $(\R_+\times \R)$ onto $S^1\cap (\R_+\times \R)$, we can use this to build a deformation retraction of $S^1\cup (\R_+\times \R)$ onto $S^1$. This is illustrated in the following diagram: 

    \begin{center}
        \begin{tikzpicture}
            \begin{scope}[very thick]
                \draw (0,0) circle (1.5);
                \pattern[fill=background] (0,2) -- (3,2) -- (3,-2) -- (0,-2) -- cycle;
                \fill[color=foreground!20!background] (0,2) -- (2,2) -- (2,-2) -- (0,-2) -- cycle;
                \draw[fill=foreground] (0,1.5) circle (0.05);
                \draw[fill=foreground] (0,-1.5) circle (0.05);
                \draw[dashed] (0,2) -- (0,-2);
                \draw[dashed] (0,0) circle (1.5);

                \draw [thin, -latex] (0.2,0) -- (1.3,0);
                \draw [thin, -latex] (0.2,0.2) -- (0.9,0.9);
                \draw [thin, -latex] (0.2,-0.2) -- (0.9,-0.9);
                \draw [thin, latex-] (1.2,1.2) -- (1.7,1.7);
                \draw [thin, latex-] (1.2,-1.2) -- (1.7,-1.7);
                \node[] at (0,-2.5) {$S^1\cup (\R_+\times \R)$};
            \end{scope}
            
            \begin{scope}[very thick, xshift=5cm]
                \draw[line width=3] (0,0) circle (1.5);
                \pattern[fill=background] (0,2) -- (-3,2) -- (-3,-2) -- (0,-2) -- cycle;
                \draw[] (0,0) circle (1.5);
                \node[] at (0,-2.5) {$S^1$};
            \end{scope}
        \end{tikzpicture}
    \end{center}

    Thus $\pi_1(S^1\cup (\R_+\times \R))=\pi_1(S^1)=\Z$.
    
    \textbf{(k)} If we let $r : \R \to [-1,1]$ be a deformation retraction of the reals onto a closed interval, note that it agrees with the identity map $\id_{S^1}$ so we can create a deformation retraction of $r' : S^1\cup (\R_+\times \R) \to \Theta = S^1\cup [-1,1]$ where $\Theta$ is the theta space defined in Munkres. 
    \begin{center}
        \begin{tikzpicture}
            \begin{scope}[very thick]
                \draw[fill=foreground] (1.5,0) circle (0.05);
                \draw[fill=foreground] (-1.5,0) circle (0.05);
                \draw [latex-latex] (-3, 0) -- (3, 0);
                \draw (0,0) circle (1.5);
                \draw [thin, -latex] (0.4,0.2) -- (1.3,0.2);
                \draw [thin, latex-] (1.7,0.2) -- (2.4,0.2);
                \draw [thin, -latex] (-0.4,0.2) -- (-1.3,0.2);
                \draw [thin, latex-] (-1.7,0.2) -- (-2.4,0.2);
                \node[] at (0,-2) {$S^1\cup (\R\times 0)$};
            \end{scope}

            \begin{scope}[very thick, xshift=6cm]
                \draw[fill=foreground] (1.5,0) circle (0.05);
                \draw[fill=foreground] (-1.5,0) circle (0.05);
                \draw [-] (-1.5, 0) -- (1.5, 0);
                \draw (0,0) circle (1.5);
                \node[] at (0,-2) {$\Theta=S^1\cup[-1,1]$};
            \end{scope}
        \end{tikzpicture}
    \end{center}
    This theta space is of course homotopy equivalent to $S^1\vee S^1$, so $\pi_1(S^1\cup (\R\times 0))=\Z*\Z$. 

    \textbf{(l)} This space is clearly contractible by the deformation retraction $r : \R^2-(\R_+\times 0) \to \{-1\}$ which sends $z$ to $\frac{z+1}{\|z+1\|}$. So $\pi_1(\R^2-(\R_+\times 0)) = \{0\}$.
   
\end{solution}

\begin{problem}
    Let $G$ be a {\em topological group}, i.e.\ a group (with operation $\cdot$ and identity element $x_0\in G$) which is also a topological space, in such a way that the multiplication map $\cdot:G\times G\to G$ and the map $G\to G$ taking each element to its inverse are continuous.  Examples of topological groups include: $(\R,+)$, $(S^1,\cdot)$, $GL(n,\R)$, $SO(3)$, etc. 

    \medskip
    Given two loops $f,g:I\to G$ based at $x_0$, we define a loop $f\cdot g$ by $(f\cdot g)(s) = f(s)\cdot g(s)$.
    \begin{enumerate}[(a)]
        \item Show that this operation makes the set of based loops in $(G,x_0)$ into a group, and that it induces a group operation $\cdot$ on the set of path-homotopy classes $\pi_1(G,x_0)$.
        \item Show that the two group operations $*$ and $\cdot$ on $\pi_1(G,x_0)$ are the same.
%(Hint: consider $(f*e_{x_0})\cdot (e_{x_0}*g)$).
        \item Show that $\pi_1(G, x_0)$ is abelian.
    \end{enumerate}
\end{problem}

\begin{solution}
    \textbf{(a)} This clearly makes the set of based loops in $(G,x_0)$ into a group. It clearly satisfies associativity, identity (by taking the constant loop $c_1$, where $1$ is the identity element), and inverses (given a loop $f : I\to G$ consider $f^{-1} : I \to G$ defined by taking the inverse piecewise). To show that it induces an operation on $\pi_1(G,x_0)$ it suffices to show that $\cdot$ is well defined with regards to homotopy. Indeed, suppose $f\simeq f'$ and $g\simeq g'$ are homotopic loops with path homotopies $H_f$ and $H_g$ respectively. Then $H_f\cdot H_g$ is a path homotopy between $f\cdot g$ and $f'\cdot g'$.
    
    \textbf{(b)} Let $f,g : I \to G$ be loops in $G$ based at $x_0$. Then $f\cdot g = (f*e_{x_0})\cdot (e_{x_0}*g)$. However
    \[
        (f*e_{x_0})\cdot (e_{x_0}*g)=\begin{cases}
            f(2t)\cdot x_0 &0\leq t\leq \frac{1}{2}\\
            x_0\cdot g(2t-1)& \frac{1}{2}\leq t\leq 1
        \end{cases}
        =\begin{cases}
            f(2t)&0\leq t\leq \frac{1}{2}\\
            g(2t-1)& \frac{1}{2}\leq t\leq 1
        \end{cases}=f*g
    .\] 
    \textbf{(c)} Note that by a similar argument to (b), we have for any loops $f,g : I \to G$,
    \[
        f*g=f\cdot g=(e_{x_0} * f)\cdot (g * e_{x_0}) = (e_{x_0} \cdot g)*(f\cdot e_{x_0}) = g * f
    .\] 
    Thus $\pi_1(G,x_0)$ is abelian.
\end{solution}

\end{document}