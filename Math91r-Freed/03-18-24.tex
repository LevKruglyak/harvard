\documentclass{lkx_paper}

\title{\textbf{Discussion Topics for March 18th}}
\date{}
\author{Lev Kruglyak and AJ LaMotta}

\usepackage{extpfeil}

\providecommand{\E}{\mathbb{E}}
\providecommand{\A}{\mathbb{A}}
\providecommand{\Gr}{\mathrm{Gr}}
\providecommand{\St}{\mathrm{St}}
\providecommand{\GL}{\mathrm{GL}}
\providecommand{\X}{\mathfrak{X}}
\providecommand{\id}{\mathrm{id}}
\renewcommand{\span}{\mathrm{span}}

\providecommand{\longto}{\;\xrightarrow{\phantom{xxx}}\;}
\providecommand{\longisom}{\;\xrightarrow{\phantom{xx}\sim\phantom{xx}}\;}
\providecommand{\longsurj}{\;\xtwoheadrightarrow{\phantom{xxx}}\;}


\begin{document}
\maketitle

\subsection*{1. General Theory of Vector Fields}

\begin{theorem*}
	There is a bijective correspondence:
	\[
		\Gamma(TM) \qiffq \Der(C^\infty(M)).
	\]
	We refer to either object as a \defn{vector field}, and use $\X(M)$ to refer to the set of vector fields.
\end{theorem*}
\begin{proof}
	To begin, suppose $X\in \Gamma(TM)$ is a section of the tangent bundle. We can define an action of $X$ on $C^\infty(M)$ by setting
	\[
		X(f) = df\circ X.
	\]
\end{proof}

\begin{definition*}
	The \defn{commutator} of two vector fields $X,Y\in \X(M)$ is
	\[
		[X,Y] = X\circ Y - Y\circ X,
	\]
	considering them as derivations.
\end{definition*}

In local coordinates, we can write $X=X^i\partial_i$ and $Y=Y^j\partial_j$.

\subsection*{2. Vector Fields and Flows}

\begin{theorem*}
	Let $\varphi_X^t, \varphi_Y^t$ be local flows associated to vector fields $X,Y\in \X(M)$ in some neighborhood. Then the flows commute if and only if the vector fields commute.
\end{theorem*}

\begin{proof}

\end{proof}

\end{document}
