\documentclass{lkx_paper}

\title{\textbf{Differential Operators on Manifolds}}
\date{}
\author{Lev Kruglyak}

\usepackage{extpfeil}

\providecommand{\E}{\mathbb{E}}
\providecommand{\A}{\mathbb{A}}
\providecommand{\Gr}{\mathrm{Gr}}
\providecommand{\St}{\mathrm{St}}
\providecommand{\GL}{\mathrm{GL}}
\providecommand{\X}{\mathfrak{X}}
\providecommand{\id}{\mathrm{id}}
\renewcommand{\span}{\mathrm{span}}

\providecommand{\longto}{\;\xrightarrow{\phantom{xxx}}\;}
\providecommand{\longisom}{\;\xrightarrow{\phantom{xx}\sim\phantom{xx}}\;}
\providecommand{\longsurj}{\;\xtwoheadrightarrow{\phantom{xxx}}\;}


\providecommand{\Diff}{\mathrm{Diff}}

\begin{document}
\maketitle

\subsection*{Commutative Algebra}

\begin{definition*}
	Let $A$ be a commutative, unital algebra over a ring $R$. Let's define the \defn{differential operators of order at most $k$} inductively in the following manner. When $k=0$, we set
	\[
		\Diff^0_R(A) = \Hom_A(A,A) = \{ \widehat{a} : a\in A\}
	\]
	where $\widehat{a}$ is the $A$-linear map sending $x\to a\cdot x$. When $k>0$, we set
	\[
		\Diff^k_R(A) = \{
		D\in \Hom_R(A, A) : [D, \widehat{a}]\in \Diff^{k-1}_R(A),\textrm{ for all }a\in A
		\}.
	\]
	Finally, we define the \defn{differential operators on $A$ over $R$} as the union
	\[
		\Diff_R(A) = \bigcup_{k\geq 0} \Diff^k_R(A) \subset \Hom_R(A, A).
	\]
\end{definition*}

\begin{proposition*}
	Composition of functions gives us $R$-bilinear maps
	\[
		\circ : \Diff^p_R(A)\times \Diff^q_R(A) \longto \Diff^{p+q}_R(A),
	\]
	so $\Diff_R(A)$ carries the structure of a graded, non-commutative, unital algebra over $R$.
\end{proposition*}

\begin{proposition*}
	We have an isomorphism $\Diff^1_R(A) \cong A\oplus \Der_R(A)$.
\end{proposition*}

\begin{proof}
	Let $D\in \Diff^1_R(A)$. We know that $[D, \widehat{a}] = \widehat{f(a)}$ for some element $f_D(a)\in A$ associated to each $a\in A$. Note now that
	\[
		[D,\widehat{a}](1) = D(a\cdot 1) - a\cdot D(1) = f_D(a)\cdot 1 \qiq f_D(a) = D(a)-a\cdot D(1).
	\]
	For any $a,b\in A$, we thus get
	\[
		\begin{aligned}
			[D, \widehat{a}](b) = D(ab) - a\cdot D(b) & = f_D(a)\cdot b = D(a)\cdot b-a\cdot D(1)\cdot b \\
			D(ab)                                     & = a\cdot D(b) + D(a)\cdot b - ab\cdot D(1)
		\end{aligned}
	\]
	This holds true for any differential operator of order at most $1$, so let's now consider $D'=D - \widehat{D(1)}$. Since $D'(1)=0$, it follows that
	\[
		D'(ab) = a\cdot D'(b) + D'(a)\cdot b.
	\]
	This is exactly the Leibniz rule, so $D'$ is a derivation, and $D$ can be expressed as the sum $D = D' + \widehat{D(1)}$ where $D'\in \Der_R(A)$ and $D(1)\in A$.
\end{proof}

\subsection*{Differential Geometry}

\begin{proposition*}
	On a manifold $M$, we have an isomorphism
	\[
		D_- : \X(M) \longto \Der_\R C^\infty(M)
	\]
\end{proposition*}

\begin{definition*}
	Let $T^{p,q}M$ be the \defn{tensor bundle} of $M$, i.e.
	\[
		T^{p,q}M = \underbrace{TM\otimes\cdots\otimes TM}_{p\textrm{ times}}\otimes \underbrace{T^*M\otimes \cdots \otimes T^* M}_{q\textrm{ times}},
	\]
	and let $\T^{p,q}M = \Gamma(T^{p,q} M)$ be the vector space of \defn{tensor fields}.


	% Let $\widehat{T}^k M = TM^{\otimes k}$ and $\widehat{\X}^k(M) = \Gamma(\widehat{T}^k M)$, where $\widehat{T}^0 M = M\times \R$ is the trivial bundle so that $\widehat{\X}^0(M)=C^\infty(M)$. Let's define the \defn{multitangent bundle} as the Whitney sum bundle
	% \[
	% 	\widehat{T} M = \bigoplus_{k\geq 0} \widehat{T}^k M.
	% \]
	% Equivalently this is the tensor algebra bundle of the tangent bundle. A \defn{multivector field} is a section of this bundle, and we denote the algebra of multivector fields as $\widehat{\X}(M) = \Gamma(\widehat{T}M)$.
\end{definition*}

There is a linear map
\[
	D^k_- : \T^{k, 0}(M) \longto \Diff_\R^k C^\infty(M),
\]
which sends some $(k,0)$-tensor field $X=X_1\otimes \cdots \otimes X_k$ to the $\R$-linear map
\[
	\definefunction{D^k_X}{C^\infty(M)}{C^\infty(M)}{f}{D_{X_1}\circ \cdots\circ D_{X_k}(f)}
\]
We claim that $D_X^k$ is in fact in $\Diff^k_{\R} C^\infty(M)$. We'll proceed by induction. For $k=1$.

%
% \begin{proof}
%   Let's begin by constructing surjective linear maps
% \end{proof}

\end{document}
