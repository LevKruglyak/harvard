\documentclass{lkx_paper}

\title{\textbf{Integral Manifolds}}
\date{}
\author{Lev Kruglyak}

\usepackage{extpfeil}

\providecommand{\E}{\mathbb{E}}
\providecommand{\A}{\mathbb{A}}
\providecommand{\Gr}{\mathrm{Gr}}
\providecommand{\St}{\mathrm{St}}
\providecommand{\GL}{\mathrm{GL}}
\providecommand{\X}{\mathfrak{X}}
\providecommand{\id}{\mathrm{id}}
\renewcommand{\span}{\mathrm{span}}

\providecommand{\longto}{\;\xrightarrow{\phantom{xxx}}\;}
\providecommand{\longisom}{\;\xrightarrow{\phantom{xx}\sim\phantom{xx}}\;}
\providecommand{\longsurj}{\;\xtwoheadrightarrow{\phantom{xxx}}\;}


\begin{document}
\maketitle

\subsection*{Vector Fields and Flows}

\begin{proposition*}
	There is a $C^\infty(M)$-linear isomorphism
	\[
		\X(M) \qiffq \Der(C^\infty(M)),
	\]
	where $\X(M)=\Gamma(TM)$ is the set of vector fields.
\end{proposition*}

\begin{proof}
	In the forward direction, suppose $X\in \X(M)$ is a vector field. Let's define the derivation $D_X\in \Der(C^\infty(M))$ as the directional derivative
	\[
		D_X f(p) = \nabla_{X(p)} f(p) \qiffq D_X f = df\circ X
	\]
	for any $f\in C^\infty(M)$. Note that if $f,g\in C^\infty(M)$, by properties of the differential $d$, we get
	\[
		\begin{aligned}
			D_X (f g) = d(f g)\circ X = d(f\wedge g) \circ X
			 & = (df\wedge g + f\wedge dg)\circ X \\
			 & = g df \circ X + f dg\circ X       \\
			 & = g D_X f + f D_X g.
		\end{aligned}
	\]
	This is exactly the Liebniz rule, so $D_X$ is indeed a derivation, and
	\[
		D_- : \X(M) \longto \Der(C^\infty(M))
	\]
	is a $C^\infty(M)$-linear map.

	In the reverse direction, let $D : C^\infty(M) \to C^\infty(M)$ be a derivation. Working in some local chart $U\subset M$ with coordinates $x^i : U \to \R$, let's consider the vector field
	\[
		\X_{D}|_U = D x^i  \frac{\partial}{\partial x^i}\in \X(U),
	\]
	where $\partial/\partial x^i \in \X(U)$ are vector fields satisfying
	\[dx^j \circ (\partial / \partial x^i) = \begin{cases}0 & i\neq j,\\ 1 & i = j.\end{cases}\]
	Equivalently, $\{\partial/\partial x^i\}$ is a local trivialization of $TM$ on $U$. By construction, it's clear that $\X_{-}|_U : \Der(C^\infty(U)) \to \X(U)$ is a $C^\infty(M)$-linear map. Picking some partition of unity $\{\psi_i\}_{i\in I}$ defined on a cover of open charts $\{U_i\}_{i\in I}$, we can define a global vector field.
	\[
		\X_D = \sum_{i\in I}\psi_i\, \X_{D}|_{U_i}
	\]
	This construction thus gives us a $C^\infty(M)$-linear map
	\[
		\X_- : \Der(C^\infty(M)) \to \X(M).
	\]

	We claim that $D_{\X_D}$ and $\X_{D_X} = X$ for any derivation $D\in \Der(C^\infty(M))$ and vector field $X\in \X(M)$. First, note that for any smooth function $f\in C^\infty(M)$ we have
	\[
		D_{\X_D} f = df\circ \sum_{i\in I} \psi_i\,\X_{D}|_{U_i} = \sum_{i\in I} \psi_i\,df\circ \X_{D}|_{U_i}.
	\]
	It suffices to show that $df\circ \X_{D}|_U = D|_{C^\infty(U)}$ for each open chart $U$. Working in the local coordinate system $x^i$ of this chart, note that
	\[
		df\circ \X_{D}|_U = J_f  \left(Dx^i \frac{\partial}{\partial x^i}\right) = \frac{\partial f}{\partial x^i} Dx^i
	\]
	where $J_f$ is the $n\times 1$ Jacobian matrix of $f$.
	\begin{changemargins}
		\begin{lemma*}
			For any $f\in C^\infty(\R^n)$ and $D\in \Der(C^\infty(\R^n))$ we have
			\[
				Df = \frac{\partial f}{\partial x^i}  D x^i.
			\]
		\end{lemma*}
		\begin{proof}
			For any point $p\in \R^n$, we can write
			\[
				f = f(p) + \sum_{1\leq i \leq n}(x^i-x^i(p)) R_{p,i}\qtq{where} R_{p,i}(p) = \frac{\partial f}{\partial x^i}(p)
			\]
			These functions $R_{p}$ can be obtained from the multivariate Taylor theorem by factoring out $(x-p)$ from the sum of linear and higher order terms when expanding at $x=p$.
			Applying the derivation, we get
			\[
				D f = \sum_{1\leq i \leq n}(x^i-x^i(p))  D R_p + R_p Dx^i.
			\]
			Now since this holds for any $p\in \R^n$ and derivations commute with limits,
			\[
				\begin{aligned}
					D f(p) & = \lim_{x\to p} \sum_{1\leq i \leq n} (x^i-x^i(p)) D R_p + R_p  Dx^i         \\
					       & = R_p(p)  Dx^i = \sum_{1\leq i \leq n}\frac{\partial f}{\partial x_i} D x^i.
				\end{aligned}
			\]
			This is the exact expression we were after.
		\end{proof}
	\end{changemargins}

	Consequently, such an expression should hold true on $U$ given suitable coordinate functions. Combining this with our previous expression for $df\circ X_{D}|_U$, it follows that $D_{\X_D} f = df\circ \X_{D} = Df$ so $D_{\X_D} = D$.

	In the reverse direction, suppose that $X$ is a vector field. As before, it suffices to show equality $\X_{D_X}|_U = X|_U$ on open charts $U$. Writing \[ X|_U = X^i \frac{\partial}{\partial x^i} \qtq{for} X^i\in C^\infty(U), \]
	note that by definition of the vector field $\partial/\partial x^i$, we get
	\[
		\begin{aligned}
			\X_{D_X}|_U  = D_X x^i  \frac{\partial}{\partial x^i}
			 & = \left(dx^i\circ X\right) \frac{\partial}{\partial x^i}                                               \\
			 & = \left(dx^i\circ \left(X^j  \frac{\partial}{\partial x^j}\right)\right) \frac{\partial}{\partial x^i} \\
			 & = X^i  \frac{\partial}{\partial x^i} = X|_U.
		\end{aligned}
	\]
	This completes the proof.
\end{proof}

\begin{definition*}
	Recall that the commutator is a $\R$-linear operator
	\[[-, -] : \Der(C^\infty(M))\otimes \Der(C^\infty(M)) \longto \Der(C^\infty(M)).\]
	By the preceding correspondence, we define the \defn{commutator of vector fields} as the unique $\R$-linear operator \[[-,-] : \X(M)\otimes \X(M) \longto \X(M)\qtq{such that} D_{[X,Y]} = [D_X, D_Y] \]
	for all vector fields $X, Y\in \X(M)$.
\end{definition*}

\begin{proposition*}
	In local coordinates, the commutator of $X$ and $Y$ is
	\[
		[X,Y]|_U = \left(X^j\frac{\partial Y^i}{\partial x^j} - Y^j\frac{\partial X^i}{\partial x^j}\right) \frac{\partial}{\partial x^i}
	\]
\end{proposition*}

\begin{proof}
	We claim that both sides act the same as derivations on $C^\infty(M)$. Let $f \in \C^\infty(M)$ be a smooth function. Observe that
	\[
		\begin{aligned}
			D_{[X,Y]}f = D_X\circ D_Y f - D_Y\circ D_X f
			 & = d(df\circ Y)\circ X - d(df\circ X)\circ Y                           \\
			 & = X^i\frac{\partial Y^j}{\partial x^i}\frac{\partial f}{\partial x^j}
			- Y^j\frac{\partial X^i}{\partial x^j} \frac{\partial f}{\partial x^i}.
		\end{aligned}
	\]
	This is exactly the action of the right hand side on $f$ as a derivation.
\end{proof}

\begin{theorem*}
	Let $X,Y\in \X(M)$ be vector fields. Then,
	\[
		[X,Y]=0 \qiffq \theta_X^u\circ \theta_Y^v = \theta_Y^v \circ \theta_X^u
	\]
	for any local flows $\theta_X, \theta_Y$ associated to $X$ and $Y$ respectively.
\end{theorem*}

\begin{proof}
	Let's begin with the reverse direction, i.e. suppose $\theta^u_X \circ \theta^v_Y = \theta^v_Y \circ \theta^u_X$ for some flows in a neighborhood of $p\in M$. In local coordinates, we can
	differentiate the flows with respect to $u$ and $v$, we get
	\[
		\begin{aligned}
			\frac{\partial}{\partial v}\frac{\partial}{\partial u}\restr{\theta^u_X\circ \theta^v_Y}{u,v=0}
			 & = \frac{\partial}{\partial v} \restr{\left(X \circ \theta_Y^v\right)}{v=0} = dX\circ Y =
			Y^j  \frac{\partial X^i}{\partial x^j} \frac{\partial}{\partial x^i}.
		\end{aligned}
	\]
	In the other direction, we have
	\[
		\begin{aligned}
			\frac{\partial}{\partial v}\frac{\partial}{\partial u}\restr{\theta^v_Y\circ \theta^u_X}{u,v=0}
			 & = \frac{\partial}{\partial v}\restr{\left(d\theta_Y^v \circ X\right)}{v=0}
			 & = dY\circ X = X^j\frac{\partial Y^i}{\partial x^j}\frac{\partial}{\partial x^i}
			% & = \frac{\partial}{\partial v}\restr{\left(\frac{\partial \theta^v_{Y,i}}{\partial x^j}  X^j \frac{\partial}{\partial x^j}\right)}{v=0}
			% = Y^i \frac{\partial X^j}{\partial x^i}\frac{\partial}{\partial x^j}.
		\end{aligned}
	\]
	Since both sides are equal, it follows that
	\[
		0 = \left(X^j\frac{\partial Y^i}{\partial x^j} - Y^j\frac{\partial X^i}{\partial x^j}\right)\frac{\partial}{\partial x^i}=[X,Y]|_U.
	\]
	In particular, wherever the flows commute, the vector fields commute, so $[X,Y]=0$ if the flows commute globally.

	In the reverse direction, suppose $[X,Y]=0$. For any $p\in M$ and $v$ small enough, consider the curve
	\[
		\gamma(u) = \theta_Y^v \circ \theta_X^u\circ \theta^{-v}_Y(p).
	\]
	If we can show that $\gamma$ is an an integral curve to $Y$, then by the uniqueness theorem we would have
	\[
		\theta^u_X(p) = \gamma(u) = \theta_Y^v\circ \theta^u_X\circ \theta^{-v}(p) \quad\implies\quad \theta^u_X\circ \theta^v_Y(p)= \theta^v_Y\circ \theta^u_X(p)
	\]
	which would complete the proof. The condition for $\gamma$ being an integral curve can be expanded as
	\[
		\begin{aligned}
			\frac{\partial \gamma}{\partial u} = X\circ \gamma
			 & \quad\Longleftarrow\quad d\theta^v_Y \circ X\circ \theta^u_X\circ \theta^{-v}_Y(p) = X\circ \theta^v_Y\circ \theta^u_X \circ \theta^{-v}_Y(p) \\
			 & \textrm{or more generally}\vphantom{\frac{1}{1}}\ldots                                                                                        \\
			 & \quad\Longleftarrow\quad d\theta^v_Y\circ X = X\circ \theta^v_Y                                                                               \\
			 & \quad\Longleftarrow\quad d\theta^v_Y\circ X\circ \theta^{-v}_Y = X.
		\end{aligned}
	\]

	\begin{changemargins}
		\begin{lemma*}
			We can express the commutator as
			\[
				\frac{\partial}{\partial v} \restr{\left(d\theta^v_Y \circ X \circ \theta_Y^{-v}\right)}{v=0} = [X,Y].
			\]
		\end{lemma*}
		\begin{proof}
			Differentiating the term $d\theta^v_Y \circ X \circ \theta^{-v}_Y$ with respect to $v$ in some chart $U$, we get
			\[
				\begin{aligned}
					\frac{\partial}{\partial v} \restr{\left(d\theta^v_Y \circ X \circ \theta^{-v}_Y\right)}{v=0}
					 & = \frac{\partial}{\partial v}\restr{\left(\frac{\partial \theta_{Y,i}^v}{\partial x^i}X^i\circ \theta^{-v}_{Y}\right)}{v=0}                \\
					 & = \frac{\partial \theta^0_{Y,j}}{\partial x^i}\frac{\partial}{\partial v}\restr{\left(X^i\circ \theta^{-v}_{Y}\right)}{v=0}
					+ \frac{\partial}{\partial v}\restr{\left(\frac{\partial \theta^v_{Y, j}}{\partial x^i}\right)}{v=0}X^i\circ \theta^{-0}_Y                    \\
					 & = -Y^j\frac{\partial X^i}{\partial x^j}\frac{\partial}{\partial x^j} + \frac{\partial Y^j}{\partial x^i} X^i \frac{\partial}{\partial x^i} \\
					 & = [X,Y]|_U.
				\end{aligned}
			\]
			Since this works in any chart $U$ and $d\theta^v_Y \circ X \circ \theta^{-v}_Y$ is defined globally, the result holds true globally as well.
		\end{proof}
	\end{changemargins}

	In our case, the commutator is zero so the derivative of $d\theta^v_Y \circ X \circ \theta^{-v}_Y$ is zero for $v$ in some open subset of the line. Thus, it must be constant in $v$ so equal to its value at $v=0$. At $v=0$, this is exactly $X$, so we get
	\[
		d\theta^v_Y \circ X \circ \theta^{-v}_Y = X.
	\]
	By our earlier argument, this completes the proof.
\end{proof}

\subsection*{Distributions and Integral Manifolds}

\begin{theorem*}
	Let $E$ be a rank $k$ distribution on $M$ and suppose any two vector fields $X,Y\in \Gamma(E)$ commute, i.e. $[X,Y]=0$. Then $E$ is integrable at every point.
\end{theorem*}

\begin{proof}
	At any point $p\in M$, choose a local frame $X_1,\ldots, X_k\in \Gamma(E)$ of the distribution and consider the parametrization
	\[
		\definefunction{\varphi}{(-\eps, \eps)^k}{M}{u}{\theta_{X_1}^{u_1}\circ \cdots \circ \theta_{X_k}^{u_k}(p)}
	\]
	for some $\eps>0$. Then we claim that the image of $\varphi$ is an integral manifold of $E$ passing through $p$.

	\todo{finish writing this proof}
\end{proof}

\begin{proposition*}
	Suppose $E$ is a rank $k$ distribution on $M$, and $N\subset M$ is a submanifold integral to $E$. If $X,Y\in \Gamma(TN)$ are integral vector fields, then so is their commutator, i.e. $[X,Y]\in \Gamma(TN)$.
\end{proposition*}

\begin{proof}
	Recall that the commutator can be expressed as
	\[
		[X,Y]=\frac{\partial}{\partial v}\restr{\left(d\theta^v_Y \circ X \circ \theta^{-v}_Y\right)}{v=0}=\lim_{v\to 0}\frac{d\theta^v_Y \circ X \circ \theta^{-v}_Y - X}{v}.
	\]
	Since $N$ is an integral manifold, $\theta^v_Y$ restricts to a flow on $N$. This means that $d\theta^v_Y$ also restricts to a linear isomorphism
	\[
		(d\theta^v_Y)_p : T_p N \to T_{\theta^v_Y(p)} N \implies (d\theta^v_Y)_p : E_p \to E_{\theta^v_Y(p)}.
	\]
	This means that $d\theta^v_Y$ preserves $E$, and since $X\circ \theta^{-v}_Y(p)\in E$, we know that $d\theta^v_Y\circ X \circ \theta^{-v}_Y(p)\in E_p$ for all $v$. This means that the numerator in the limit must always be in $E_p$, and so the limit must live in $E_p$ as well. This proves that $[X,Y]\in \Gamma(E)$.
\end{proof}

\end{document}
