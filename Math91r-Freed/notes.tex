\documentclass{lkx_paper}

\title{\textbf{Explorations in Geometry and Topology}}
\date{}
\author{Lev Kruglyak and AJ LaMotta}

\usepackage{extpfeil}
\usepackage{graphicx}

\providecommand{\E}{\mathbb{E}}
\providecommand{\A}{\mathbb{A}}
\providecommand{\D}{\mathfrak{D}}
\providecommand{\Gr}{\mathrm{Gr}}
\providecommand{\St}{\mathrm{St}}
\providecommand{\GL}{\mathrm{GL}}
\providecommand{\eps}{\varepsilon}
\providecommand{\X}{\mathfrak{X}}
\providecommand{\T}{\mathcal{T}}
\providecommand{\id}{\mathrm{id}}
\providecommand{\Der}{\mathrm{Der}}
\providecommand{\Tens}{\mathrm{Tens}}
\renewcommand{\span}{\mathrm{span}}

\providecommand{\longto}{\;\xrightarrow{\phantom{xxx}}\;}
\providecommand{\longisom}{\;\xrightarrow{\phantom{xx}\sim\phantom{xx}}\;}
\providecommand{\longsurj}{\;\xtwoheadrightarrow{\phantom{xxx}}\;}
\providecommand{\qtq}[1]{\quad\textrm{#1}\quad}
\providecommand{\qiq}{\quad\implies\quad}
\providecommand{\qiffq}{\quad\iff\quad}
\providecommand{\definefunction}[5]{
	\begin{array}{rcl}
		#1 : #2 & \xrightarrow{\phantom{---}} & #3 \\
		#4      & \xmapsto{\phantom{---}}     & #5
	\end{array}
}


\begin{document}
\maketitle

\section{Local and Global Solutions to PDEs}

Throughout, let $M$ be a smooth manifold of fixed dimension $n$.
Our goal is to understand how vector fields or distributions on $M$, arising naturally from the theory of PDEs on $M$ admit local or global solutions, i.e. flows and integral manifolds.

\subsection{Grassmannian and Steifel Manifolds}

Let $V$ be a vector space of dimension $n$.

\begin{definition}
	The \defn{(non-compact) Steifel manifold} is
	\[
		\St_k(V) = \{ \ell : \R^k \to V \textrm{ is a linear embedding}\} \subset \Hom(\R^k, V),
	\]
	with subspace topology from its inclusion into $\Hom(\R^k, V)$.
\end{definition}

\begin{proposition}
	$\St_k(V)$ is a smooth manifold of dimension $nk$.
\end{proposition}
\begin{proof}
	We claim that $\St_k(V)$ is actually an open subset of $\Hom(\R^k, V)$. Notice that for a map $\ell\in \Hom(\R^k, V)$ to be a linear embedding, it's sufficient for $\{\ell(e_1), \ldots, \ell(e_k)\}$ to be linearly independent for some basis $\{e_1,\ldots, e_k\}$ of $\R^k$. This is an open condition in the topology of $\Hom(\R^k, V)$ -- if we wrote $\ell$ out in matrix form, the columns being linearly independent is a condition involving a polynomial of the matrix coefficients being nonzero. Thus, $\St_k(V)$ is a smooth manifold of dimension $\dim\Hom(\R^k, V) = nk$.
\end{proof}

As a subspace of $\Hom(\R^k, V)$, the manifold $\St_k(V)$ carries a right action by $\GL(\R^k)$, given by precomposition. This action is smooth, free, and transitive.

\begin{definition}
	The \defn{Grassmannian manifold} $\Gr_k(V)$ of $k$-planes in $V$ is defined as the quotient
	\[
		\Gr_k(V) = \St_k(V) / \GL(\R^k)
	\]
\end{definition}

\begin{proposition}
	$\Gr_k(V)$ is a smooth manifold of dimension $nk-k^2$.
\end{proposition}

\begin{proof}
	\todo{quotient of a smooth manifold by a Lie group which acts smoothly, freely, properly, quotient manifold theorem}
\end{proof}

\todo{fiber bundle $\St \to \Gr$}

\subsection{Bundle Constructions}

Throughout, let $E$ be a vector bundle over $M$ of rank $m$.

\begin{definition}
	The \defn{associated Stiefel bundle} of $E$ has total space
	\[
		\St_k(E) = \{ (p, \ell) : p\in M, \ell : \R^k \to E_p \textrm{ is a linear embedding}\} \subset \Hom(M\times \R^k, E),
	\]
	with natural projection onto the base space $M$.
\end{definition}

This is not a vector bundle; rather each fiber is diffeomorphic to $\St_k(\R^m)$, the (non-compact) Steifel manifold of $k$-frames on Euclidean $m$-space. The topology on $\St_k(E)$ is the subspace topology under the natural inclusion into $\Hom(M\times \R^k, E)$.

\todo{smooth structure on $\St_k(E)$}

\todo{how does $GL(\R^k)$ act on $\St_k(E)$}

\begin{definition}
	The \defn{associated Grassmannian bundle} of $E$ has total space
	\[
		\Gr_k(E) = \{(p, V) : p\in M, V\in \Gr_k(E_p) \}
	\]
	with natural projection onto the base space $M$.
\end{definition}

\todo{smooth structure on $\Gr_k(E)$}

\begin{proposition}
	\todo{local trivializations of $E$ give local trivializations of Gr(E) and St(E)}
\end{proposition}

To give this total space a topology, we use the quotient topology given by the surjective map:
\[
	\begin{array}{rcl}
		\beta_k(E) : \St_k(E) & \xrightarrow{\phantom{---}} & \Gr_k(E)        \\
		(p,\ell)              & \xmapsto{\phantom{---}}     & (p, \ell(\R^k))
	\end{array}
\]

This quotient also gives us the map
\[
	\gamma_k(E) : \Gamma(\St_k(E)) \longto \Gamma(\Gr_k(E))
\]
which acts as post-composition by $\beta_{k}(E)$. This map need not be surjective in general; \todo{mobius bundle example}.

\begin{proposition}
	Suppose $M$ is contractible and $E=M\times \R^m$. Then $\gamma_k(E)$ is surjective.
\end{proposition}

\begin{proof}
	\todo{this proof}
\end{proof}

\begin{proposition}[Local Section Lifting]
	\label{local_section_lifting}
	For any $p\in M$, there is some open neighborhood $\mathcal{U}\subset M$ of $p$ such that the map $\gamma_k(E|_\mathcal{U})$ is surjective.
\end{proposition}

\begin{proof}
	\todo{proof}
\end{proof}

\subsection{Distributions}

\begin{definition}
	A \defn{distribution} on $M$ is a subbundle $E\subset TM$ of the tangent bundle of $M$. We'll denote the set of all distributions of rank $k$ on $M$ by $\mathscr{D}_k(M)$.
\end{definition}

\begin{proposition}
	There are a bijective correspondences:
	\[
		\begin{aligned}
			\mathscr{D}_k(M) & \longisom \Gamma(\Gr_k(TM)), \\
			\mathscr{F}_k(M) & \longisom \Gamma(\St_k(TM)), \\
		\end{aligned}
	\]
	where $\mathscr{F}_k(M)$ is the $k$-frame bundle of $TM$.
\end{proposition}

\begin{proof}
	\todo{finish}
\end{proof}

Thus, we have a map
\[
	\gamma : \mathscr{F}_k(M) \longto \mathscr{D}_k(M)
\]
which sends a $k$-frame to it's spanning distribution.

\begin{definition}
	A \defn{global basis} for a distribution $E\in \mathscr{D}_k(M)$ is a $k$-frame $\sigma\in \mathscr{F}_k(M)$ with $\gamma(\sigma)=E$. A \defn{local basis} for $E$ on some open $\mathcal{U}\subset M$ is $k$-frame $\sigma\in \mathscr{F}_k(\mathcal{U})$ with $\gamma(\sigma)=E|_\mathcal{U}$.
\end{definition}

Note that global bases need not exist for a distribution.

\begin{example}
	\todo{Torus example}
\end{example}

\begin{proposition}
	For any distribution $E\in \mathscr{D}_k(M)$ and point $p\in M$, there exists some open neighborhood $\mathcal{U}\subset M$ of $p$ on which there exists a local basis for $E$.
\end{proposition}

\begin{proof}
	Apply Proposition~\ref{local_section_lifting} to $E$, this gives us a section of $\St_k(E|_\mathcal{U})$, which naturally embeds as a section of $\St_k(T\mathcal{U})$, and can then be canonically identified with $\mathcal{F}_k(\mathcal{U})$.
\end{proof}

\subsection{Integral Manifolds}

\begin{definition}
	A submanifold $N\subset M$ is said to be an \defn{integral manifold} to a distribution $E\in \mathscr{D}_k(M)$ if for every $p\in N$, we have $T_p N = E_p$, where both are considered as subset of $T_p M$. A connected integral manifold is \defn{maximal} if it is not properly contained in any other connected integral manifold.
\end{definition}

\begin{definition}
	A distribution $E \in \mathscr{D}_k(M)$ is \defn{integrable} at a point $p\in M$ if there exists an integral manifold $N$ to $E$ which contains $p$.
\end{definition}

The condition of complete integrability is entirely separate from the condition of having a global basis.

\begin{example}[Global basis but not integrable]
	On $M= \A^3_{\langle x,y,z\rangle}$ consider the lineraly independent vector fields
	\[
		\begin{aligned}
			X & = \frac{\partial}{\partial x} + z\frac{\partial}{\partial y}, \\
			Z & = \frac{\partial}{\partial z},
		\end{aligned}
	\]
	and let $E=\langle X, Z \rangle$ be the distribution spanned by $X$ and $Z$. Then $X$ and $Z$ form a global basis for $E$, but $E$ is not integrable at any point.
	\todo{why}
\end{example}

\begin{example}[Completely integrable, but no global basis]
	\todo{Torus example}
\end{example}

% Locally however,

\end{document}
