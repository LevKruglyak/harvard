\documentclass{lkx_paper}

\title{\textbf{Discussion Topics for March 5th}}
\date{}
\author{Lev Kruglyak and AJ LaMotta}

\usepackage{extpfeil}
\usepackage{graphicx}

\providecommand{\E}{\mathbb{E}}
\providecommand{\A}{\mathbb{A}}
\providecommand{\D}{\mathfrak{D}}
\providecommand{\Gr}{\mathrm{Gr}}
\providecommand{\St}{\mathrm{St}}
\providecommand{\GL}{\mathrm{GL}}
\providecommand{\eps}{\varepsilon}
\providecommand{\X}{\mathfrak{X}}
\providecommand{\T}{\mathcal{T}}
\providecommand{\id}{\mathrm{id}}
\providecommand{\Der}{\mathrm{Der}}
\providecommand{\Tens}{\mathrm{Tens}}
\renewcommand{\span}{\mathrm{span}}

\providecommand{\longto}{\;\xrightarrow{\phantom{xxx}}\;}
\providecommand{\longisom}{\;\xrightarrow{\phantom{xx}\sim\phantom{xx}}\;}
\providecommand{\longsurj}{\;\xtwoheadrightarrow{\phantom{xxx}}\;}
\providecommand{\qtq}[1]{\quad\textrm{#1}\quad}
\providecommand{\qiq}{\quad\implies\quad}
\providecommand{\qiffq}{\quad\iff\quad}
\providecommand{\definefunction}[5]{
	\begin{array}{rcl}
		#1 : #2 & \xrightarrow{\phantom{---}} & #3 \\
		#4      & \xmapsto{\phantom{---}}     & #5
	\end{array}
}


\begin{document}
\maketitle

\subsection*{1. Integral Curves on an Integral Manifold}

Let $M$ be an $n$-dimensional manifold.
\begin{claim*}
	If $E$ is a rank $k$ distribution on $M$, and let $N\subset M$ be an integral submanifold. Suppose $\xi\in \Gamma(E)$ is some vector field, and $\gamma : (-\eps, \eps) \to M$ is an integral curve passing through a point $p$. If $p\in N$, then $\gamma(I)\subset N$ for some $I\subset (-\eps, \eps)$.
\end{claim*}

\begin{proof}
	First of all, the restriction $\xi|_N$ is a vector field on $N$. This means that there is an integral curve $\gamma' : (-\eps, \eps) \to N$ which passes through $p$. However, considered as integral curves in $M$, both $\gamma$ and $\gamma'$ have a common point in their image, so they must agree on some subinterval of their mutual domain.
\end{proof}

This implies that the distribution on $\A^3$ generated by
\[
	X = \frac{\partial}{\partial x} + y\frac{\partial}{\partial z},\quad Y = \frac{\partial}{\partial y}
\]
is not integrable. This can be shown by patching together the various integral curves of $X$ and $Y$ near a given point $p$ on some hypothetical integral manifold.

\subsection*{2. Integrating a $k$-frame}

Let $M$ be an $n$-dimensional manifold.
Let $\xi\in \X(M)$ be a vector field, and $\theta : \D(M) \to M$ be an integral local flow. This means that $\D(M)$ is an open subset of $\R\times M$ which contains $\{0\}\times M$, and $\theta$ satisfies
\[
	\frac{\partial \theta^{t}(x)}{\partial t} = \xi\circ\theta^{t}(x),\quad\theta^{t_1}\circ \theta^{t_2}(x) = \theta^{t_1+t_2}(x),\qtq{and}\theta^{0}(x)=x
\]
for all $x\in M$ and $t,t_1,t_2\in \D_x(M)$, where $\D_x(M) = \D(M)\cap (\R\times\{x\})$.

For any $(t,x)\in \D(M)$, this map $\theta$ has differential
\[
	\definefunction{d\theta^t_x}
	{T_t \R\oplus T_x M}
	{T_{\theta^{t}(x)}U}
	{(\delta t, \delta x)}
	{\xi\circ\theta^{t}(x)\cdot \delta t + \Theta^{t}(x)\cdot \delta x,}
\]
where we use $\Theta^{t}(x) : T_x M \to T_{\theta^{t}(x)} M$ to denote the ``transport'' maps for a fixed $t$.

\begin{claim*}
	The linear map $\Theta^t(x) : T_x M \to T_{\theta^t(x)} M$ is invertible.
\end{claim*}

\begin{proof}
	Note that $\theta^{-t}\circ \theta^{-t}(x)=x$. Then by the chain rule we have
	\[
		\id_{T_x M}=\frac{\partial \theta^{-t} \circ \theta^{t}(x)}{\partial x} = \frac{\partial \theta^t(\theta^{-t}(x))}{\partial \theta^{-t}(x)}\cdot \frac{\partial \theta^{-t}(x)}{\partial x}(x) = (\Theta^{-t} \circ\theta^{t}(x))\cdot\Theta^{t}(x),
	\]
	so $\Theta^{-t}\circ \theta^t(x)$ is an inverse $\Theta^t(x)$.
\end{proof}

\begin{claim*}
	We have the commutation relation $\Theta^t\cdot \xi = \xi\circ \theta^t$.
\end{claim*}

\begin{proof}
	Working in some local coordinate system $\{x_i\}^n_{i=1}$, let's write
	\[
		\xi = \alpha_1\cdot \frac{\partial}{\partial x_1} + \cdots + \alpha_n \cdot \frac{\partial}{\partial x_n}
	\]
	for smooth functions $\alpha_i$. Then, it follows that
	\[
		\xi\circ \theta^t(x_1,\ldots, x_n) = \alpha_1\cdot \frac{\partial \theta^t(x_1,\ldots, x_n)}{\partial x_1}
	\]
\end{proof}

Now suppose we have a rank $k$ distribution $E\subset TM$. Let $\sigma=\{\sigma_1,\ldots,\sigma_k\}\subset \Gamma(E|_U)$ be some local frame for $U\subset M$.
Let $\theta_i$ be a local flow corresponding to $\sigma_i$, and $\Theta_i$ the associated transport map. Let's assume that $\D(U) = (-\eps,\eps)\times M_0$ is some common domain for all of the $\theta_i$. Now consider the map
\[
	\definefunction{\varphi}{(-\eps, \eps)^k}{U}{t}{\theta^{t_1}_1\circ \theta^{t_2}_2\circ\cdots\circ \theta^{t_k}_k(x)}
\]
In ideal conditions, we want $\textrm{Im}(\varphi)$ to be an integral manifold to $E$.

To simplify the investigation, let's suppose $k=2$, so that we have $\varphi^t(x) = \theta_1^{t_1}\circ \theta_2^{t_2}(x)$. By the chain rule, the derivatives/tangent vectors of this map are
\[
	\frac{\partial \varphi^t}{\partial t_1} = \sigma_1\circ \theta^{t_1}_1\circ \theta^{t_2}_2(x),\qtq{and}\frac{\partial \varphi^t}{\partial t_2} = \Theta^{t_1}_1\circ \sigma_2\circ \theta_2^{t_2}(x).
\]

For $\textrm{Im}(\varphi)$ to be a surface at all, we would require that
\[
	\dim\textrm{span}\left\{\frac{\partial \varphi^t}{\partial t_1}, \frac{\partial \varphi^t}{\partial t_2}\right\} = 2.
\]
\begin{claim*}
	$\textrm{Im}(\varphi)$ is a submanifold, for $\eps$ small enough.
\end{claim*}

For $\textrm{Im}(\varphi)$ to be an integral surface, we would require that
\[
	\textrm{span}\left\{\frac{\partial \varphi^t}{\partial t_1}, \frac{\partial \varphi^t}{\partial t_2}\right\} = E_{\varphi^t}.
\]
for every coordinate $t$. In order for this condition to be met, we need two things: the tangent vectors are orthogonal, and they both lie in $E_{\varphi^t}$.

% For the first of these conditions, note that
% Here we drop the input to $M^{(u_1)}$ since it is implied as the base point of the tangent vector it is multiplied by. For this manifold to be integral, we thus only require that:
% \[
% 	M^{(u_1)}_1\circ \sigma_2\circ \theta_2^{(u_2)}(x)\textrm{ is orthogonal to }\sigma_1\circ \varphi(u)
% \]
% Explain why. We can drop the composition with $\theta_2^{(u_2)}(x)$, and just require that \[M^{(t)}_1\circ \sigma_2(x) \in E_{\theta_1^{(t)}(x)}\qtq{ is orthogonal to }\sigma_1\circ \theta_1^{(t)}(x)\qtq{for all}x\in U_0.\]
%

\section*{Questions:}

\begin{enumerate}
	\item AJ question things
	\item commuting flows <=> commuting vector fields
	\item Develop theory of vector fields
\end{enumerate}
\end{document}
