\documentclass{lkx_pset}

\title{Physics 212 Problem Set 1}
\author{Lev Kruglyak}
\due{February 14, 2025}

\providecommand{\pp}[2]{\frac{\partial #1}{\partial #2}}

\usepackage{siunitx}
\providecommand{\mmsi}[1]{\qty[per-mode = symbol]{#1}{\m^2}}
\providecommand{\MeVsi}[1]{\qty[per-mode = symbol]{#1}{\mathrm{MeV}}}
\providecommand{\rhosi}[1]{\qty[per-mode = symbol]{#1}{\kg\per\m^3}}
\providecommand{\nrhosi}[1]{\qty[per-mode = symbol]{#1}{\m^{-3}}}
\providecommand{\Mpcsi}[1]{\qty[per-mode = symbol]{#1}{\mathrm{Mpc}}}
\providecommand{\lysi}[1]{\qty[per-mode = symbol]{#1}{\mathrm{ly}}}
\providecommand{\yrsi}[1]{\qty[per-mode = symbol]{#1}{\mathrm{yr}}}
\providecommand{\Hyrsi}[1]{\qty[per-mode = symbol]{#1}{\mathrm{yr}\,\km \per \s\per\mathrm{Mpc}}}
\providecommand{\Hlysi}[1]{\qty[per-mode = symbol]{#1}{\mathrm{lr}\,\km \per \s\per\mathrm{Mpc}}}
\providecommand{\kmsi}[1]{\qty[per-mode = symbol]{#1}{\km}}
\providecommand{\Hsi}[1]{\qty[per-mode = symbol]{#1}{\km\per\s\per\mathrm{Mpc}}}
\providecommand{\unitsi}[1]{\qty[per-mode = symbol]{#1}{}}
\providecommand{\mpssi}[1]{\qty[per-mode = symbol]{#1}{\m\per\s}}
\providecommand{\mpsssi}[1]{\qty[per-mode = symbol]{#1}{\m\per\s^2}}
\providecommand{\Gsi}[1]{\qty[per-mode = symbol]{#1}{\m^3\per\s^2\kg}}
\providecommand{\kBsi}[1]{\qty[per-mode = symbol]{#1}{\m^2\s^{-2}\K^{-1}\kg}}
\providecommand{\hsi}[1]{\qty[per-mode = symbol]{#1}{\J\cdot s}}
\providecommand{\kgsi}[1]{\qty[per-mode = symbol]{#1}{\kg}}
\providecommand{\ssi}[1]{\qty[per-mode = symbol]{#1}{\s}}
\providecommand{\psssi}[1]{\qty[per-mode = symbol]{#1}{\s^{-2}}}
\providecommand{\msi}[1]{\qty[per-mode = symbol]{#1}{\m}}
\providecommand{\desi}[1]{\qty[per-mode = symbol]{#1}{\J\per \m^3}}

\usepackage{pgfplots}
\pgfplotsset{compat=1.17} 


\begin{document}
\maketitle

\begin{problem}{1}
Show explicitly that the $00$ and $ij$ component of the Einstein equations are given by
\[
	G^0_0 = -\frac{3}{a^2}\left[\left(\frac{\dot{a}}{a}\right)^2+ K\right]
	\quad\textrm{and}\quad
	G^i_j = -\frac{1}{a^2}\left[2\left(\frac{\ddot{a}}{a}\right)-\left(\frac{\dot{a}}{a}\right)^2+ K\right]\delta^i_j
\]
The dots here correspond to derivatives with respect to conformal time.
\end{problem}
\begin{solution}
	The Friedman-Roberston-Walker metric can be written as
	\[
		g_{\mu\nu} = -dt^2 +a^2(t)\left[\frac{dr^2}{1-Kr^2}+r^2d\Omega\right]
	\]
	where $d\Omega = d\theta^2 + \sin^2(\theta)d\phi^2$ is the metric on a $2$-dimensional sphere. The dual metric is then given by
	\[
		g^{\mu\nu} = -dt^2 +\frac{1-Kr^2}{a^2(t)}dr^2 + \frac{1}{r^2a^2(t)}d\theta^2 + \frac{1}{r^2a^2(t)\sin^2(\theta)}d\phi^2.
	\]
	We'll denote our coordinate system by $\{x^0, x^1, x^2, x^3\} = \{t, r, \theta, \phi\}$ to make some formulas simpler, but we will use these symbols interchangeable. Let's now calculate the Christoffel symbols. Let's adopt the convention that $a'$ denotes differentiation with respect to coordinate time $t$ and $\dot{a}$ denotes differentiation with respect to conformal time $\eta$.

	Firstly, let's expand the metric derivatives $\partial g_{\mu\nu}/\partial x^\kappa$. It suffices to consider $\mu=\nu$ since the metric is diagonal in our chosen coordinate system.

	Note that, $\partial g_{00}/\partial x^\nu = 0$ since $g_{00}$ is constant. The other derivatives are given in the table:
	\begin{equation}\label{derivatives}
		\everymath={\displaystyle}
		\renewcommand*{\arraystretch}{2}
		\begin{array}{lll}
			\pp{g_{11}}{x^0} = a'(t)\cdot 2a(t)\cdot \frac{1}{1-Kr^2},
			 & \pp{g_{11}}{x^1} = a^2(t)\cdot -2Kr\cdot \frac{-1}{(1-Kr)^2},
			 & \pp{g_{11}}{x^2} = 0,                                              \\
			\pp{g_{22}}{x^0} = a'(t)\cdot 2a(t)\cdot r^2,
			 & \pp{g_{22}}{x^1} = a^2(t)\cdot 2r,
			 & \pp{g_{22}}{x^2} = 0,                                              \\
			\pp{g_{33}}{x^0} = a'(t)\cdot 2a(t)\cdot r^2\sin^2(\theta),
			 & \pp{g_{33}}{x^1} = a^2(t)\cdot 2r\cdot \sin^2(\theta),
			 & \pp{g_{33}}{x^2} = \cos(\theta)\cdot 2\sin(\theta)\cdot a^2(t)r^2. \\
		\end{array}
	\end{equation}
	Next, recall that the Christoffel symbols are defined to be
	\[
		\Gamma^\mu_{\alpha\beta} = \frac{g^{\mu\nu}}{2}\left[\pp{g_{\alpha\nu}}{x^\beta} + \pp{g_{\beta\nu}}{x^\alpha} -\pp{g_{\alpha\beta}}{x^\nu}\right].
	\]
	Since the metric is diagonal, we can rewrite this formula as
	\begin{equation}\label{christoffel}
		\Gamma_{\alpha\beta}^\mu = \frac{g^{\mu\mu}}{2}\left[ \pp{g_{\alpha \mu}}{x^\beta} + \pp{g_{\beta \mu}}{x^\alpha} - \pp{g_{\alpha\beta}}{x^\mu}\right].
	\end{equation}
	By the torsion-free property of the Levi-Civita connection, we have the symmetry $\Gamma^\mu_{\alpha\beta}=\Gamma^\mu_{\beta\alpha}$. At this point, we'll drop the function dependence of $a(t)$, $\sin(\theta)$, and $\cos(\theta)$ for brevity since we aren't taking derivatives any longer and since the variables of these functions are unambiguous. We can now plug the expressions in (\ref{derivatives}) into (\ref{christoffel}). To speed things up, we note that whenever $\alpha$ or $\beta$ are zero, $\Gamma^\mu_{\alpha\beta}=0$ since the derivatives of $g_{00}$ are all zero. Also, the indices $\mu,\alpha,\beta$ must have at least one

	When $\mu=0$, the only nonzero Christoffel symbols are
	\begin{equation}\label{christoffel-calculated-1}
		\Gamma^0_{\alpha\alpha} = -\frac{1}{2}\left[-\pp{g_{\alpha\alpha}}{x^0}\right]\quad \implies \Gamma^0_{11} = \frac{a'a}{1-Kr^2}, \quad \Gamma^0_{22}=a'a r^2,\quad \Gamma^0_{33}=a'ar^2\sin^2.
	\end{equation}
	When $\mu\in \{1,2,3\}$ we have a slightly more complicated pattern. Here, the nonzero Christoffel symbols are
	\begin{equation}\label{christoffel-calculated-2}
		\everymath={\displaystyle}
		\renewcommand*{\arraystretch}{2}
		\begin{array}{ll}
			\Gamma^1_{01} = \frac{1-Kr^2}{2a^2}\cdot \frac{2a'a}{1-Kr^2} = \frac{a'}{a},
			 & \Gamma^1_{11} = \frac{1-Kr^2}{2a^2}\cdot\frac{2Kra^2}{(1-Kr)^2} = \frac{Kr}{1-Kr^2}, \\
			\Gamma^1_{22} = \frac{1-Kr^2}{2a^2}\cdot(-2ra^2) =-r(1-Kr^2),
			 & \Gamma^1_{33} = \frac{1-Kr^2}{2a^2}\cdot 2ra^2\sin^2 = \sin^2r(1-Kr^2),              \\
			\Gamma^2_{02} = \frac{1}{2r^2a^2}\cdot 2a'r^2 = \frac{a'}{a},
			 & \Gamma^2_{12} = \frac{1}{2r^2a^2}\cdot 2ra^2 = \frac{1}{r},                          \\
			\Gamma^2_{33} = \frac{1}{2r^2a^2}\cdot 2a^2r^2\cos\sin=\cos\sin,
			 & \Gamma^3_{03} = \frac{1}{2r^2a^2\sin^2}\cdot 2a'a r^2\sin^2 = \frac{a'}{a} \\
			\Gamma^3_{13} = \frac{1}{2r^2a^2\sin^2}\cdot 2r\sin^2 a^2 = \frac{1}{r},
			 & \Gamma^3_{23} = \frac{1}{2r^2a^2\sin^2}\cdot 2\sin\cos a^2r^2 = \cot.                \\
		\end{array}
	\end{equation}
	Now that we've calculated all of the Christoffel symbols, let's calculate the Ricci curvature tensor. In terms of the Christoffel symbols, the Ricci tensor can be written as
	\begin{equation}\label{ricci}
		R_{\mu\nu} = \pp{\Gamma^\alpha_{\mu\nu}}{x^\alpha} - \pp{\Gamma^\alpha_{\mu\alpha}}{x^\nu} + \Gamma^\alpha_{\beta\alpha}\Gamma^\beta_{\mu\nu} - \Gamma^\alpha_{\beta\nu} \Gamma^\beta_{\mu\alpha}.
	\end{equation}
	At this point, the computations get a bit tedious, so we omit the details and only state the results.
	If we plug the equations (\ref{christoffel-calculated-1}) and (\ref{christoffel-calculated-2}) into (\ref{ricci}), we get the following nonzero components for the Ricci tensor
	\begin{equation}
		R_{00} =  -\frac{3a''}{a}, \quad
		R_{ii} = g_{ii}\left(\frac{2K + 2(a')^2 + aa''}{a^2}\right).
	\end{equation}
	Here, we've simplified the expression $R_{ii}$ by factoring out terms which also appear in the metric tensor. Finally, let's compute the Ricci scalar.
	\[
		R = g^{00} R_{00} + g^{ii}R_{ii}=
		3\left(\frac{a''}{a}\right) + 3\left(\frac{2K+2(a')^2 + aa''}{a^2}\right) = \frac{6(K+(a')^2+aa'')}{a^2}.
	\]
	Since we have both the Ricci tensor and scalar, we can now compute the Einstein tensor
	\[
		G_{\mu\nu} = R_{\mu\nu}+\frac{1}{2}g_{\mu\nu} R
		\quad\implies\quad
		G_{00} = -\frac{3a''}{a}-\frac{3(K+(a')^2+aa''^2)}{a^2} = \frac{3(K+(a')^2)}{a^2},\\
	\]
	Contracting to get $G^0_0$, we get
	\[
	  G_0^0 = g^{0\alpha}G_{0\alpha} -G_{00}= \frac{-3(K+(a')^2)}{a^2}
	\]
	The remaining terms of the Einstein tensor are
	\[
		\begin{aligned}
			G_{ii} = g_{ii}\left[\frac{2K + 2(a')^2 + aa''}{a^2}\right] - \frac{1}{2}g_{ii}\left[\frac{6(K+(a')^2 + aa'')}{a^2}\right] &= -g_{ii}\left(\frac{K+(a')^2 + 2aa''}{a^2}\right)\\
			\implies\quad G_i^i &= -\frac{1}{a^2}(K + (a')^2 + 2aa'').
		\end{aligned}
	\]
	Putting these results together, we get the expressions
	\[
    G_0^0 = \frac{-3K}{a^2} -3\left(\frac{a'}{a}\right)^2\quad\textrm{and}\quad G_i^i = -\frac{K}{a^2} - \left(\frac{a'}{a}\right)^2 - 2\left(\frac{a''}{a}\right).
	\]
	Finally, we transform derivatives to with respect to conformal time by the transformation law
	\[
		\frac{\partial a}{\partial t} = \frac{1}{a}\frac{\partial a}{\partial \eta} \quad\textrm{and}\quad \frac{\partial^2 a}{\partial t^2} = \frac{\partial^2 a}{\partial \eta^2}\frac{1}{a^2} - \left(\frac{1}{a}\frac{\partial a}{\partial \eta}\right)^2.
	\]
	In abbreviated notation, this means $a' = \dot{a}{a}$ and $a'' = \ddot{a}/a^2 - (\dot{a}/a)^2$.
	This results in the desired formulas
	\[
		\boxed{G_0^0 = -\frac{3}{a^2}\left[K + \left(\frac{\dot{a}}{a}\right)^2\right]},
		\quad\textrm{and}\quad
		\boxed{G_i^j = -\frac{1}{a^2}\left[K  + 2\left(\frac{\ddot{a}}{a}\right)- \left(\frac{\dot{a}}{a}\right)^2\right]\delta_i^j}.
	\]
\end{solution}

\end{document}
