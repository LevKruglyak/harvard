\documentclass{lkx_pset}

\title{Physics 212 Problem Set 4}
\author{Lev Kruglyak}
\due{March 7, 2025}

\providecommand{\pp}[2]{\frac{\partial #1}{\partial #2}}


\begin{document}
\maketitle

\begin{problem}{1}
  Confirm the equation
  \[
    \mathcal{S}_{(2)} = \frac{1}{2}\int d\eta\,d^3x\left[ (v')^2-(\partial_i v)^2 + \frac{v''}{z}v^2\right],
  \]
  where $v=z\mathcal{R}$ and $z^2=a^2\dot{\phi}^2/H^2$, starting from the equation
  \[
    \mathcal{S}_{(2)} = \frac{1}{2}\int d^4 x\, a^3\frac{\dot{\phi}^2}{H^2}\left[\dot{\mathcal{R}}^2 - a^{-2}(\partial_i \mathcal{R})^2\right].
  \]
\end{problem}

\begin{problem}{2}
\end{problem}
\begin{parts}
  \begin{part}{(a)}
  Assuming slow-roll inflation, express the tensor-to-scalar ratio in terms of slow-roll parameters.
  \end{part}

  \begin{part}{(b)}
    Write down the tilt of the scalar power spectrum (given by $n_s-1=d\ln \Delta_{\mathcal{R}}^2/d\ln k$) in terms of the slow-roll parameters (going up to first order only in slow-roll parameters).
  \end{part}

  \begin{part}{(c)}
    Express the tilt of the tensor power spectrum (given by $n_t = d\ln\Delta_t^2/d\ln k$) in terms of slow-roll parameters (going also up to first order only in slow-roll parameters). 
  \end{part}

  \begin{part}{(d)}
    Using your answers in (a) and (b), write $r$ in terms of $n_t$. This relation is known as the ``single-field slow-roll consistency relation''. Any violation of this condition found in the data would violate the assumption of slow-roll single-field inflation, and would shed light on the physics of inflation!
  \end{part}
\end{parts}

\begin{problem}{3}
  Consider an inflationary potential of the form $V(\phi)=\lambda\phi^4$.
\end{problem}
\begin{parts}
  \begin{part}{(a)}
    Setting $\Delta_{\mathcal{R}}^2(k)=10^{-9}$ (consistent with observations), show what value this implies for $\lambda$. {\color{blue} see hint}
  \end{part}
\end{parts}

\end{document}
